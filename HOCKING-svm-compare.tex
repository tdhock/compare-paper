\documentclass{article}
\usepackage{listings}
\usepackage{clrscode}
\usepackage[table]{xcolor}
\usepackage{array}
\usepackage{slashbox}
\usepackage{multirow}
\usepackage{fullpage}
\usepackage{tikz}
\usepackage{stmaryrd}
\usepackage{graphicx}
\usepackage{amsmath,amssymb,amsthm}
\usepackage{stfloats}
\usepackage{float}

\renewcommand{\thefootnote}{\fnsymbol{footnote}}

\newtheorem{proposition}{Proposition}

\newtheorem{definition}{Definition}
\newtheorem{theo}{Theorem}    % numérotés par section
\newtheorem{lemma}{Lemma}

\newcommand{\RR}{\mathbb R}
\newcommand{\NN}{\mathbb N}
\newcommand{\pkg}[1]{\texttt{#1}}
\newcommand{\plausibleK}{\textit{plausibleK}}

\DeclareMathOperator*{\argmin}{arg\,min}
\DeclareMathOperator*{\argmax}{arg\,max}
\DeclareMathOperator*{\maximize}{maximize}
\DeclareMathOperator*{\minimize}{minimize}


\newfloat{Algorithm}{thp}{lop}
\floatname{Algorithm}{Algorithm}

% For citations
\usepackage{natbib}

% For algorithms
\usepackage{algorithm}
\usepackage{algorithmic}
\usepackage{hyperref}
\newcommand{\theHalgorithm}{\arabic{algorithm}}

\usepackage{icml2013} 

\icmltitlerunning{Support vector comparison machines}

\begin{document}

\renewcommand{\arraystretch}{1.5}

\definecolor{lightgray}{rgb}{0.9,0.9,0.9}
\definecolor{pastelblue}{RGB}{213,229,255}
\newcolumntype{a}{>{\columncolor{lightgray}}c}

\twocolumn[
\icmltitle{Support vector comparison machines}

% It is OKAY to include author information, even for blind
% submissions: the style file will automatically remove it for you
% unless you've provided the [accepted] option to the icml2012
% package.


%\icmlauthor{Guillem Rigaill\footnotemark[1]}{rigaill@evry.inra.fr}
%\icmladdress{Unit\'e de Recherche en G\'enomique V\'eg\'etale INRA-CNRS-Universit\'e d'Evry Val d'Essonne, Evry, France}
%\icmlauthor{Toby Dylan Hocking\footnotemark[1], 
% Francis Bach}{toby.hocking@inria.fr, francis.bach@inria.fr}
\icmlauthor{Toby Dylan Hocking}{toby@sg.cs.titech.ac.jp}
\icmlauthor{Supaporn Spanurattana}{supaporn@sg.cs.titech.ac.jp}
\icmlauthor{Masashi Sugiyama}{sugi@cs.titech.ac.jp}
\icmladdress{Department of Computer Science, Tokyo Institute of
  Technology, Tokyo 152-8552, Japan}

% You may provide any keywords that you 
% find helpful for describing your paper; these are used to populate 
% the "keywords" metadata in the PDF but will not be shown in the document
\icmlkeywords{support vector machine, ranking, comparing, convex,
  optimization, relaxation, libsvm}

\vskip 0.3in
]

\begin{abstract}
  In ranking problems, the goal is to learn a ranking function
  $f(x)\in\RR$ from labeled pairs $x,x'$ of input points. In this
  paper, we consider the related comparison problem, where the label
  $y\in\{-1,0,1\}$ indicates which element of the pair is better, or
  if there is no significant difference. We cast the learning problem
  as a margin maximization, and show that it can be solved by
  converting it to a standard SVM. We compare our algorithm to SVMrank
  using a benchmark ranking data set.
\end{abstract}

\section{Introduction}

In this paper we consider the supervised comparison problem. Assume
that we have $n$ labeled training pairs and for each pair
$i\in\{1,\dots,n\}$ we have input features $x_i,x_i'\in\RR^p$ and a
label $y_i\in\{-1,0,1\}$ that indicates which element is better:
\begin{equation}
  \label{eq:z}
  y_i =
  \begin{cases}
    -1 & \text{ if $x_i$ is better than $x'_i$}\\
    0 & \text{ if $x_i$ is as good as $x'_i$}\\
    1 & \text{ if $x'_i$ is better than $x_i$}.
  \end{cases}
\end{equation}
The goal of learning is to find a comparison function $c:\RR^p \times
\RR^p \rightarrow \{-1,0,1\}$ which generalizes to a test set of data:
\begin{equation}
  \minimize_{c} 
  \sum_{i=1}^n
  e\left[ c(x_i, x_i'), y_i \right].
\end{equation}
For evaluation, we propose the zero-one loss function
$e:\{-1,0,1\}\times\{-1,0,1\}\rightarrow\{0,1\}$, described in
Table~\ref{tab:evaluation}.

This is similar to a ranking problem, for which a number of machine
learning algorithms exist \citep{learning-to-rank}, such as RankSVM
\citep{ranksvm}. There are two key differences between ranking and
comparing, which is what we want to do:
\begin{itemize}
\item Ranking algorithms ignore the $y_i=0$ equality constraints.
\item The goal of ranking is to give an absolute order to a set of
  items, typically documents in a search engine. Comparison is simpler
  since we only need to make a decision about exactly two items.
\end{itemize}

The rest of this article is organized as follows. In
Section~\ref{sec:related} we discuss links with related work on
classification and ranking, then in Section~\ref{sec:svm-compare} we
propose a new algorithm: SVMcompare. We show results on a benchmark
dataset in Section~\ref{sec:results} and discuss future work in
Section~\ref{sec:conclusions}.

\section{Related work}
\label{sec:related}

First we discuss connections with several existing methods in the
machine learning literature, and then we discuss how ranking
algorithms can be applied to the comparison problem.

When the inputs are discrete $x_i,x_i'\in\{1,\dots,k\}$, then the
problem is known as learning a relation \citep{relations}. In this
article we only consider the case when inputs are continuous
$x_i,x_i'\in\RR^p$.

\begin{table}[b!]
  \centering
  \begin{tabular}{|a|c|c|}\hline
    \rowcolor{lightgray}
    \backslashbox{Outputs}{Inputs}
    &single items $x$&pairs $x,x'$\\ \hline
    $y\in\{-1,1\}$ &SVM  & SVMrank   	\\ \hline 
    $y\in\{-1,0,1\}$ &Reject option& this work\\ \hline
  \end{tabular}
  \caption{\label{tab:related} Comparison is similar to ranking 
    and classification with reject option.}
\end{table}

\subsection{Rank, reject, and rate}

Some related work appears in Table~\ref{tab:related}:

\begin{itemize}
\item \citet{reject-option} studied the statistical properties of the
  hinge loss for the classification with reject option.
\item \citet{ranksvm} proposed SVM for ranking.
\item \citet{rank-with-ties} proposed a boosting algorithm for ranking
  with ties, and observed that ties are more effective when there are
  more output values.
\item \citet{trueskill} proposed TrueSkill: a Bayesian skill rating
  system, a generalization of the Elo chess rating system.
\end{itemize}

\begin{figure}
  \centering
  % Created by tikzDevice version 0.10.1 on 2017-12-20 13:08:39
% !TEX encoding = UTF-8 Unicode
\begin{tikzpicture}[x=1pt,y=1pt]
\definecolor{fillColor}{RGB}{255,255,255}
\path[use as bounding box,fill=fillColor,fill opacity=0.00] (0,0) rectangle (433.62,180.67);
\begin{scope}
\path[clip] (  0.00,  0.00) rectangle (433.62,180.67);
\definecolor{drawColor}{RGB}{0,0,0}

\path[draw=drawColor,line width= 0.4pt,line join=round,line cap=round] (  0.00,  0.00) --
	(144.54,  0.00) --
	(144.54,164.83) --
	(  0.00,164.83) --
	(  0.00,  0.00);
\end{scope}
\begin{scope}
\path[clip] (  0.00,  0.00) rectangle (144.54,164.83);
\definecolor{drawColor}{RGB}{190,190,190}

\path[draw=drawColor,line width= 0.4pt,line join=round,line cap=round] (  0.00, 79.33) -- (144.54, 79.33);

\path[draw=drawColor,line width= 0.4pt,line join=round,line cap=round] ( 72.09,  0.00) -- ( 72.09,164.84);
\end{scope}
\begin{scope}
\path[clip] (  0.00,  0.00) rectangle (433.62,180.67);
\definecolor{drawColor}{RGB}{0,0,0}

\node[text=drawColor,anchor=base,inner sep=0pt, outer sep=0pt, scale=  1.00] at ( 72.27,167.47) {Original features $\mathbf x\in\mathbb R^2$};
\end{scope}
\begin{scope}
\path[clip] (  0.00,  0.00) rectangle (144.54,164.83);
\definecolor{drawColor}{gray}{0.50}

\path[draw=drawColor,line width= 0.4pt,line join=round,line cap=round] ( 46.90, 63.81) --
	( 43.63, 71.13) --
	( 42.37, 78.44) --
	( 43.14, 85.76) --
	( 45.93, 93.08) --
	( 49.26, 98.15) --
	( 51.30,100.40) --
	( 56.57,104.49) --
	( 63.30,107.71) --
	( 63.89,107.93) --
	( 71.21,108.97) --
	( 78.53,108.34) --
	( 80.50,107.71) --
	( 85.84,105.53) --
	( 93.14,100.40) --
	( 93.16,100.38) --
	( 98.30, 93.08) --
	(100.48, 87.74) --
	(101.10, 85.76) --
	(101.73, 78.44) --
	(100.70, 71.13) --
	(100.48, 70.53) --
	( 97.25, 63.81) --
	( 93.16, 58.53) --
	( 90.91, 56.49) --
	( 85.84, 53.16) --
	( 78.53, 50.37) --
	( 71.21, 49.61) --
	( 63.89, 50.86) --
	( 56.57, 54.14) --
	( 53.32, 56.49) --
	( 49.26, 60.55);

\path[draw=drawColor,line width= 0.4pt,line join=round,line cap=round] ( 49.26, 60.55) -- ( 48.82, 61.16);

\node[text=drawColor,rotate=-54.13,anchor=base west,inner sep=0pt, outer sep=0pt, scale=  0.60] at ( 45.07, 62.48) { 1 };

\path[draw=drawColor,line width= 0.4pt,line join=round,line cap=round] ( 27.31, 53.96) --
	( 25.92, 56.49);

\path[draw=drawColor,line width= 0.4pt,line join=round,line cap=round] ( 23.03, 63.81) --
	( 21.24, 71.13) --
	( 20.55, 78.44) --
	( 20.97, 85.76) --
	( 22.49, 93.08) --
	( 25.12,100.40) --
	( 27.31,104.67) --
	( 29.13,107.71) --
	( 34.62,114.76) --
	( 34.88,115.03) --
	( 41.94,121.09) --
	( 43.80,122.35) --
	( 49.26,125.44) --
	( 56.57,128.45) --
	( 61.36,129.67) --
	( 63.89,130.22) --
	( 71.21,130.84) --
	( 78.53,130.46) --
	( 82.78,129.67) --
	( 85.84,129.00) --
	( 93.16,126.27) --
	(100.48,122.40) --
	(100.55,122.35) --
	(107.80,116.45) --
	(109.22,115.03) --
	(115.11,107.79) --
	(115.16,107.71) --
	(119.04,100.40) --
	(121.77, 93.08) --
	(122.43, 90.02) --
	(123.23, 85.76) --
	(123.60, 78.44) --
	(122.99, 71.13) --
	(122.43, 68.59) --
	(121.21, 63.81) --
	(118.21, 56.49) --
	(115.11, 51.04) --
	(113.86, 49.18) --
	(107.80, 42.12) --
	(107.52, 41.86) --
	(100.48, 36.37) --
	( 97.44, 34.54) --
	( 93.16, 32.36) --
	( 85.84, 29.73) --
	( 78.53, 28.20) --
	( 71.21, 27.78) --
	( 63.89, 28.47) --
	( 56.57, 30.26) --
	( 49.26, 33.16) --
	( 46.73, 34.54) --
	( 41.94, 37.63) --
	( 36.79, 41.86) --
	( 34.62, 44.03) --
	( 30.39, 49.18) --
	( 27.31, 53.96);

\path[draw=drawColor,line width= 0.4pt,line join=round,line cap=round] ( 25.92, 56.49) -- ( 24.23, 60.77);

\node[text=drawColor,rotate=-68.44,anchor=base west,inner sep=0pt, outer sep=0pt, scale=  0.60] at ( 20.92, 62.98) { 3 };

\path[draw=drawColor,line width= 0.4pt,line join=round,line cap=round] ( 12.67, 49.26) --
	(  9.63, 56.49) --
	(  7.41, 63.81) --
	(  6.03, 71.13) --
	(  5.51, 78.44) --
	(  5.83, 85.76) --
	(  7.00, 93.08) --
	(  9.02,100.40) --
	( 11.89,107.71) --
	( 12.67,109.25);

\path[draw=drawColor,line width= 0.4pt,line join=round,line cap=round] ( 15.99,115.03) --
	( 19.99,120.69) --
	( 21.33,122.35) --
	( 27.31,128.55) --
	( 28.57,129.67) --
	( 34.62,134.27) --
	( 38.95,136.98) --
	( 41.94,138.63) --
	( 49.26,141.79) --
	( 56.57,144.08) --
	( 57.71,144.30) --
	( 63.89,145.37) --
	( 71.21,145.85) --
	( 78.53,145.56) --
	( 85.84,144.48) --
	( 86.56,144.30) --
	( 93.16,142.43) --
	(100.48,139.47) --
	(105.24,136.98) --
	(107.80,135.47) --
	(115.11,130.14) --
	(115.66,129.67) --
	(122.43,122.89) --
	(122.90,122.35) --
	(128.23,115.03) --
	(129.75,112.48) --
	(132.24,107.71) --
	(135.19,100.40) --
	(137.07, 93.79) --
	(137.25, 93.08) --
	(138.32, 85.76) --
	(138.62, 78.44) --
	(138.13, 71.13) --
	(137.07, 64.94) --
	(136.85, 63.81) --
	(134.56, 56.49) --
	(131.40, 49.18) --
	(129.75, 46.19) --
	(127.04, 41.86) --
	(122.43, 35.81) --
	(121.31, 34.54) --
	(115.11, 28.57) --
	(113.46, 27.22) --
	(107.80, 23.23) --
	(102.02, 19.91) --
	(100.48, 19.12) --
	( 93.16, 16.26) --
	( 85.84, 14.24) --
	( 78.53, 13.06) --
	( 71.21, 12.74) --
	( 63.89, 13.27) --
	( 56.57, 14.65) --
	( 49.26, 16.87) --
	( 42.03, 19.91) --
	( 41.94, 19.95) --
	( 34.62, 24.38) --
	( 30.77, 27.22) --
	( 27.31, 30.17) --
	( 22.93, 34.54) --
	( 19.99, 38.01) --
	( 17.15, 41.86) --
	( 12.71, 49.18) --
	( 12.67, 49.26);

\path[draw=drawColor,line width= 0.4pt,line join=round,line cap=round] ( 14.30,112.09) -- ( 15.99,115.03);

\node[text=drawColor,rotate= 60.15,anchor=base west,inner sep=0pt, outer sep=0pt, scale=  0.60] at ( 14.63,108.13) { 5 };

\path[draw=drawColor,line width= 0.4pt,line join=round,line cap=round] ( 27.31, 14.57) --
	( 20.36, 19.91) --
	( 19.99, 20.23) --
	( 13.00, 27.22) --
	( 12.67, 27.60) --
	(  7.34, 34.54) --
	(  5.35, 37.58);

\path[draw=drawColor,line width= 0.4pt,line join=round,line cap=round] ( 27.61, 14.37) -- ( 27.31, 14.57);

\node[text=drawColor,rotate=-33.12,anchor=base west,inner sep=0pt, outer sep=0pt, scale=  0.60] at ( 26.38, 12.48) { 7 };

\path[draw=drawColor,line width= 0.4pt,line join=round,line cap=round] (  5.35,121.14) --
	(  6.11,122.35) --
	( 11.52,129.67) --
	( 12.67,131.01);

\path[draw=drawColor,line width= 0.4pt,line join=round,line cap=round] ( 18.46,136.98) --
	( 19.99,138.37) --
	( 27.31,144.16) --
	( 27.52,144.30) --
	( 34.62,148.56) --
	( 40.84,151.62);

\path[draw=drawColor,line width= 0.4pt,line join=round,line cap=round] ( 14.94,133.35) -- ( 18.46,136.98);

\node[text=drawColor,rotate= 45.95,anchor=base west,inner sep=0pt, outer sep=0pt, scale=  0.60] at ( 14.30,129.43) { 7 };

\path[draw=drawColor,line width= 0.4pt,line join=round,line cap=round] (142.26,115.03) --
	(138.07,122.35) --
	(137.07,123.82) --
	(132.61,129.67) --
	(129.75,132.92) --
	(125.69,136.98) --
	(122.43,139.85) --
	(116.58,144.30) --
	(115.11,145.30) --
	(107.80,149.50) --
	(103.25,151.62);

\path[draw=drawColor,line width= 0.4pt,line join=round,line cap=round] (143.00,113.45) -- (142.26,115.03);

\node[text=drawColor,rotate=-65.00,anchor=base west,inner sep=0pt, outer sep=0pt, scale=  0.60] at (140.95,112.49) { 7 };

\path[draw=drawColor,line width= 0.4pt,line join=round,line cap=round] (113.90, 12.59) --
	(115.11, 13.34);

\path[draw=drawColor,line width= 0.4pt,line join=round,line cap=round] (122.43, 18.76) --
	(123.77, 19.91) --
	(129.75, 25.69) --
	(131.14, 27.22) --
	(136.92, 34.54) --
	(137.07, 34.76) --
	(141.32, 41.86) --
	(144.38, 48.07);

\path[draw=drawColor,line width= 0.4pt,line join=round,line cap=round] (117.74, 15.29) -- (122.43, 18.76);

\node[text=drawColor,rotate= 36.52,anchor=base west,inner sep=0pt, outer sep=0pt, scale=  0.60] at (116.46, 11.53) { 7 };

\path[draw=drawColor,line width= 0.4pt,line join=round,line cap=round] ( 12.68, 12.59) --
	( 12.67, 12.60) --
	(  5.36, 19.91) --
	(  5.35, 19.92);

\path[draw=drawColor,line width= 0.4pt,line join=round,line cap=round] (  5.35,138.66) --
	( 10.84,144.30) --
	( 12.67,145.99) --
	( 19.60,151.62);

\path[draw=drawColor,line width= 0.4pt,line join=round,line cap=round] (140.26,136.98) --
	(137.07,140.56) --
	(133.32,144.30) --
	(129.75,147.49) --
	(124.50,151.62);

\path[draw=drawColor,line width= 0.4pt,line join=round,line cap=round] (142.36,134.31) -- (140.26,136.98);

\node[text=drawColor,rotate=-51.83,anchor=base west,inner sep=0pt, outer sep=0pt, scale=  0.60] at (140.58,132.91) { 9 };

\path[draw=drawColor,line width= 0.4pt,line join=round,line cap=round] (137.07, 18.07) --
	(138.75, 19.91) --
	(144.38, 26.84);

\path[draw=drawColor,line width= 0.4pt,line join=round,line cap=round] (131.42, 12.59) -- (134.72, 15.80);

\node[text=drawColor,rotate= 44.21,anchor=base west,inner sep=0pt, outer sep=0pt, scale=  0.60] at (136.30, 14.17) { 9 };
\definecolor{drawColor}{RGB}{152,78,163}

\path[draw=drawColor,line width= 1.2pt,line join=round,line cap=round] ( 61.12, 83.95) -- ( 43.66, 65.31);

\path[draw=drawColor,line width= 1.2pt,line join=round,line cap=round] ( 84.09, 50.95) -- ( 92.61, 52.51);

\path[draw=drawColor,line width= 1.2pt,line join=round,line cap=round] (120.03, 30.80) -- (124.47, 35.29);

\path[draw=drawColor,line width= 1.2pt,line join=round,line cap=round] (102.12, 95.27) -- ( 87.09, 98.32);

\path[draw=drawColor,line width= 1.2pt,line join=round,line cap=round] ( 45.37, 85.82) -- ( 67.82, 63.12);

\path[draw=drawColor,line width= 1.2pt,line join=round,line cap=round] ( 77.66, 47.49) -- ( 83.75, 60.15);
\definecolor{drawColor}{RGB}{255,127,0}

\path[draw=drawColor,line width= 1.2pt,line join=round,line cap=round] ( 31.00,135.13) -- ( 29.11,151.62);

\path[draw=drawColor,line width= 1.2pt,line join=round,line cap=round] ( 33.41,145.81) --
	( 29.11,151.62) --
	( 26.23,144.99);

\path[draw=drawColor,line width= 1.2pt,line join=round,line cap=round] (105.45, 42.87) -- (101.53, 13.22);

\path[draw=drawColor,line width= 1.2pt,line join=round,line cap=round] ( 98.77, 19.90) --
	(101.53, 13.22) --
	(105.93, 18.95);

\path[draw=drawColor,line width= 1.2pt,line join=round,line cap=round] (111.97,118.65) -- (139.19,145.37);

\path[draw=drawColor,line width= 1.2pt,line join=round,line cap=round] (137.25,138.41) --
	(139.19,145.37) --
	(132.19,143.56);

\path[draw=drawColor,line width= 1.2pt,line join=round,line cap=round] (117.50, 92.13) -- (131.79,110.19);

\path[draw=drawColor,line width= 1.2pt,line join=round,line cap=round] (130.74,103.04) --
	(131.79,110.19) --
	(125.08,107.53);

\path[draw=drawColor,line width= 1.2pt,line join=round,line cap=round] (125.01, 45.85) -- (123.70, 18.73);

\path[draw=drawColor,line width= 1.2pt,line join=round,line cap=round] (120.40, 25.15) --
	(123.70, 18.73) --
	(127.62, 24.80);

\path[draw=drawColor,line width= 1.2pt,line join=round,line cap=round] (107.73, 63.64) -- (111.50, 48.56);

\path[draw=drawColor,line width= 1.2pt,line join=round,line cap=round] (106.47, 53.76) --
	(111.50, 48.56) --
	(113.49, 55.51);

\path[draw=drawColor,line width= 1.2pt,line join=round,line cap=round] (100.01,103.59) -- (114.12,121.57);

\path[draw=drawColor,line width= 1.2pt,line join=round,line cap=round] (113.10,114.42) --
	(114.12,121.57) --
	(107.41,118.88);

\path[draw=drawColor,line width= 1.2pt,line join=round,line cap=round] ( 89.83, 42.11) -- ( 74.25, 16.51);

\path[draw=drawColor,line width= 1.2pt,line join=round,line cap=round] ( 74.42, 23.74) --
	( 74.25, 16.51) --
	( 80.59, 19.98);

\path[draw=drawColor,line width= 1.2pt,line join=round,line cap=round] ( 26.76, 93.40) -- (  7.01, 72.31);

\path[draw=drawColor,line width= 1.2pt,line join=round,line cap=round] (  8.65, 79.35) --
	(  7.01, 72.31) --
	( 13.93, 74.41);

\path[draw=drawColor,line width= 1.2pt,line join=round,line cap=round] ( 35.13, 88.32) -- (  5.35, 78.10);

\path[draw=drawColor,line width= 1.2pt,line join=round,line cap=round] ( 10.10, 83.55) --
	(  5.35, 78.10) --
	( 12.45, 76.71);

\path[draw=drawColor,line width= 1.2pt,line join=round,line cap=round] ( 21.85, 91.74) -- ( 11.53,109.04);

\path[draw=drawColor,line width= 1.2pt,line join=round,line cap=round] ( 17.84,105.52) --
	( 11.53,109.04) --
	( 11.63,101.81);

\path[draw=drawColor,line width= 1.2pt,line join=round,line cap=round] ( 38.60,116.27) -- ( 20.89,116.63);

\path[draw=drawColor,line width= 1.2pt,line join=round,line cap=round] ( 27.23,120.11) --
	( 20.89,116.63) --
	( 27.08,112.89);
\definecolor{drawColor}{RGB}{77,175,74}

\path[draw=drawColor,line width= 1.2pt,line join=round,line cap=round] ( 67.47, 71.83) -- ( 40.49, 50.57);

\path[draw=drawColor,line width= 1.2pt,line join=round,line cap=round] ( 43.17, 57.28) --
	( 40.49, 50.57) --
	( 47.65, 51.61);

\path[draw=drawColor,line width= 1.2pt,line join=round,line cap=round] ( 60.78, 48.55) -- ( 74.55, 29.80);

\path[draw=drawColor,line width= 1.2pt,line join=round,line cap=round] ( 67.94, 32.70) --
	( 74.55, 29.80) --
	( 73.76, 36.98);
\definecolor{drawColor}{RGB}{0,0,0}
\definecolor{fillColor}{RGB}{255,255,255}

\path[draw=drawColor,line width= 0.4pt,line join=round,line cap=round,fill=fillColor] (  0.00, 66.00) rectangle ( 53.62,  0.00);
\definecolor{drawColor}{RGB}{255,127,0}

\path[draw=drawColor,line width= 1.2pt,line join=round,line cap=round] (  9.90, 39.60) -- ( 29.70, 39.60);
\definecolor{drawColor}{RGB}{152,78,163}

\path[draw=drawColor,line width= 1.2pt,line join=round,line cap=round] (  9.90, 26.40) -- ( 29.70, 26.40);
\definecolor{drawColor}{RGB}{77,175,74}

\path[draw=drawColor,line width= 1.2pt,line join=round,line cap=round] (  9.90, 13.20) -- ( 29.70, 13.20);
\definecolor{drawColor}{RGB}{0,0,0}

\node[text=drawColor,anchor=base,inner sep=0pt, outer sep=0pt, scale=  1.00] at ( 26.81, 52.80) {label $y_i$};

\node[text=drawColor,anchor=base west,inner sep=0pt, outer sep=0pt, scale=  1.00] at ( 39.60, 35.83) {1};

\node[text=drawColor,anchor=base west,inner sep=0pt, outer sep=0pt, scale=  1.00] at ( 39.60, 22.63) {0};

\node[text=drawColor,anchor=base west,inner sep=0pt, outer sep=0pt, scale=  1.00] at ( 39.60,  9.43) {-1};

\path[draw=drawColor,line width= 0.4pt,line join=round,line cap=round] ( 31.00,135.13) circle (  2.47);

\path[draw=drawColor,line width= 0.4pt,line join=round,line cap=round] ( 29.11,151.62) circle (  2.47);

\node[text=drawColor,anchor=base,inner sep=0pt, outer sep=0pt, scale=  1.00] at ( 45.89,132.39) {$\mathbf x_{1}$};

\node[text=drawColor,anchor=base,inner sep=0pt, outer sep=0pt, scale=  1.00] at ( 44.00,148.88) {$\mathbf x_{1}'$};
\end{scope}
\begin{scope}
\path[clip] (  0.00,  0.00) rectangle (433.62,180.67);
\definecolor{drawColor}{RGB}{0,0,0}

\path[draw=drawColor,line width= 0.4pt,line join=round,line cap=round] (144.54,  0.00) --
	(289.08,  0.00) --
	(289.08,164.83) --
	(144.54,164.83) --
	(144.54,  0.00);
\end{scope}
\begin{scope}
\path[clip] (  0.00,  0.00) rectangle (433.62,180.67);
\definecolor{drawColor}{RGB}{0,0,0}

\node[text=drawColor,anchor=base,inner sep=0pt, outer sep=0pt, scale=  1.00] at (216.81,167.47) {Enlarged features $\Phi(\mathbf x)$};
\end{scope}
\end{tikzpicture}
\begin{tikzpicture}[x=1pt,y=1pt]
\definecolor{fillColor}{RGB}{255,255,255}
\path[use as bounding box,fill=fillColor,fill opacity=0.00] (0,0) rectangle (433.62,180.67);
\begin{scope}
\path[clip] (  0.00,  0.00) rectangle (433.62,180.67);
\definecolor{drawColor}{RGB}{0,0,0}

\path[draw=drawColor,line width= 0.4pt,line join=round,line cap=round] (  0.00,  0.00) --
	(144.54,  0.00) --
	(144.54,164.83) --
	(  0.00,164.83) --
	(  0.00,  0.00);
\end{scope}
\begin{scope}
\path[clip] (  0.00,  0.00) rectangle (144.54,164.83);
\definecolor{drawColor}{RGB}{190,190,190}

\path[draw=drawColor,line width= 0.4pt,line join=round,line cap=round] (  0.00, 79.33) -- (144.54, 79.33);

\path[draw=drawColor,line width= 0.4pt,line join=round,line cap=round] ( 72.09,  0.00) -- ( 72.09,164.84);
\end{scope}
\begin{scope}
\path[clip] (  0.00,  0.00) rectangle (433.62,180.67);
\definecolor{drawColor}{RGB}{0,0,0}

\node[text=drawColor,anchor=base,inner sep=0pt, outer sep=0pt, scale=  1.00] at ( 72.27,167.47) {Original features $\mathbf x\in\mathbb R^2$};
\end{scope}
\begin{scope}
\path[clip] (  0.00,  0.00) rectangle (144.54,164.83);
\definecolor{drawColor}{gray}{0.50}

\path[draw=drawColor,line width= 0.4pt,line join=round,line cap=round] ( 46.90, 63.81) --
	( 43.63, 71.13) --
	( 42.37, 78.44) --
	( 43.14, 85.76) --
	( 45.93, 93.08) --
	( 49.26, 98.15) --
	( 51.30,100.40) --
	( 56.57,104.49) --
	( 63.30,107.71) --
	( 63.89,107.93) --
	( 71.21,108.97) --
	( 78.53,108.34) --
	( 80.50,107.71) --
	( 85.84,105.53) --
	( 93.14,100.40) --
	( 93.16,100.38) --
	( 98.30, 93.08) --
	(100.48, 87.74) --
	(101.10, 85.76) --
	(101.73, 78.44) --
	(100.70, 71.13) --
	(100.48, 70.53) --
	( 97.25, 63.81) --
	( 93.16, 58.53) --
	( 90.91, 56.49) --
	( 85.84, 53.16) --
	( 78.53, 50.37) --
	( 71.21, 49.61) --
	( 63.89, 50.86) --
	( 56.57, 54.14) --
	( 53.32, 56.49) --
	( 49.26, 60.55);

\path[draw=drawColor,line width= 0.4pt,line join=round,line cap=round] ( 49.26, 60.55) -- ( 48.82, 61.16);

\node[text=drawColor,rotate=-54.13,anchor=base west,inner sep=0pt, outer sep=0pt, scale=  0.60] at ( 45.07, 62.48) { 1 };

\path[draw=drawColor,line width= 0.4pt,line join=round,line cap=round] ( 27.31, 53.96) --
	( 25.92, 56.49);

\path[draw=drawColor,line width= 0.4pt,line join=round,line cap=round] ( 23.03, 63.81) --
	( 21.24, 71.13) --
	( 20.55, 78.44) --
	( 20.97, 85.76) --
	( 22.49, 93.08) --
	( 25.12,100.40) --
	( 27.31,104.67) --
	( 29.13,107.71) --
	( 34.62,114.76) --
	( 34.88,115.03) --
	( 41.94,121.09) --
	( 43.80,122.35) --
	( 49.26,125.44) --
	( 56.57,128.45) --
	( 61.36,129.67) --
	( 63.89,130.22) --
	( 71.21,130.84) --
	( 78.53,130.46) --
	( 82.78,129.67) --
	( 85.84,129.00) --
	( 93.16,126.27) --
	(100.48,122.40) --
	(100.55,122.35) --
	(107.80,116.45) --
	(109.22,115.03) --
	(115.11,107.79) --
	(115.16,107.71) --
	(119.04,100.40) --
	(121.77, 93.08) --
	(122.43, 90.02) --
	(123.23, 85.76) --
	(123.60, 78.44) --
	(122.99, 71.13) --
	(122.43, 68.59) --
	(121.21, 63.81) --
	(118.21, 56.49) --
	(115.11, 51.04) --
	(113.86, 49.18) --
	(107.80, 42.12) --
	(107.52, 41.86) --
	(100.48, 36.37) --
	( 97.44, 34.54) --
	( 93.16, 32.36) --
	( 85.84, 29.73) --
	( 78.53, 28.20) --
	( 71.21, 27.78) --
	( 63.89, 28.47) --
	( 56.57, 30.26) --
	( 49.26, 33.16) --
	( 46.73, 34.54) --
	( 41.94, 37.63) --
	( 36.79, 41.86) --
	( 34.62, 44.03) --
	( 30.39, 49.18) --
	( 27.31, 53.96);

\path[draw=drawColor,line width= 0.4pt,line join=round,line cap=round] ( 25.92, 56.49) -- ( 24.23, 60.77);

\node[text=drawColor,rotate=-68.44,anchor=base west,inner sep=0pt, outer sep=0pt, scale=  0.60] at ( 20.92, 62.98) { 3 };

\path[draw=drawColor,line width= 0.4pt,line join=round,line cap=round] ( 12.67, 49.26) --
	(  9.63, 56.49) --
	(  7.41, 63.81) --
	(  6.03, 71.13) --
	(  5.51, 78.44) --
	(  5.83, 85.76) --
	(  7.00, 93.08) --
	(  9.02,100.40) --
	( 11.89,107.71) --
	( 12.67,109.25);

\path[draw=drawColor,line width= 0.4pt,line join=round,line cap=round] ( 15.99,115.03) --
	( 19.99,120.69) --
	( 21.33,122.35) --
	( 27.31,128.55) --
	( 28.57,129.67) --
	( 34.62,134.27) --
	( 38.95,136.98) --
	( 41.94,138.63) --
	( 49.26,141.79) --
	( 56.57,144.08) --
	( 57.71,144.30) --
	( 63.89,145.37) --
	( 71.21,145.85) --
	( 78.53,145.56) --
	( 85.84,144.48) --
	( 86.56,144.30) --
	( 93.16,142.43) --
	(100.48,139.47) --
	(105.24,136.98) --
	(107.80,135.47) --
	(115.11,130.14) --
	(115.66,129.67) --
	(122.43,122.89) --
	(122.90,122.35) --
	(128.23,115.03) --
	(129.75,112.48) --
	(132.24,107.71) --
	(135.19,100.40) --
	(137.07, 93.79) --
	(137.25, 93.08) --
	(138.32, 85.76) --
	(138.62, 78.44) --
	(138.13, 71.13) --
	(137.07, 64.94) --
	(136.85, 63.81) --
	(134.56, 56.49) --
	(131.40, 49.18) --
	(129.75, 46.19) --
	(127.04, 41.86) --
	(122.43, 35.81) --
	(121.31, 34.54) --
	(115.11, 28.57) --
	(113.46, 27.22) --
	(107.80, 23.23) --
	(102.02, 19.91) --
	(100.48, 19.12) --
	( 93.16, 16.26) --
	( 85.84, 14.24) --
	( 78.53, 13.06) --
	( 71.21, 12.74) --
	( 63.89, 13.27) --
	( 56.57, 14.65) --
	( 49.26, 16.87) --
	( 42.03, 19.91) --
	( 41.94, 19.95) --
	( 34.62, 24.38) --
	( 30.77, 27.22) --
	( 27.31, 30.17) --
	( 22.93, 34.54) --
	( 19.99, 38.01) --
	( 17.15, 41.86) --
	( 12.71, 49.18) --
	( 12.67, 49.26);

\path[draw=drawColor,line width= 0.4pt,line join=round,line cap=round] ( 14.30,112.09) -- ( 15.99,115.03);

\node[text=drawColor,rotate= 60.15,anchor=base west,inner sep=0pt, outer sep=0pt, scale=  0.60] at ( 14.63,108.13) { 5 };

\path[draw=drawColor,line width= 0.4pt,line join=round,line cap=round] ( 27.31, 14.57) --
	( 20.36, 19.91) --
	( 19.99, 20.23) --
	( 13.00, 27.22) --
	( 12.67, 27.60) --
	(  7.34, 34.54) --
	(  5.35, 37.58);

\path[draw=drawColor,line width= 0.4pt,line join=round,line cap=round] ( 27.61, 14.37) -- ( 27.31, 14.57);

\node[text=drawColor,rotate=-33.12,anchor=base west,inner sep=0pt, outer sep=0pt, scale=  0.60] at ( 26.38, 12.48) { 7 };

\path[draw=drawColor,line width= 0.4pt,line join=round,line cap=round] (  5.35,121.14) --
	(  6.11,122.35) --
	( 11.52,129.67) --
	( 12.67,131.01);

\path[draw=drawColor,line width= 0.4pt,line join=round,line cap=round] ( 18.46,136.98) --
	( 19.99,138.37) --
	( 27.31,144.16) --
	( 27.52,144.30) --
	( 34.62,148.56) --
	( 40.84,151.62);

\path[draw=drawColor,line width= 0.4pt,line join=round,line cap=round] ( 14.94,133.35) -- ( 18.46,136.98);

\node[text=drawColor,rotate= 45.95,anchor=base west,inner sep=0pt, outer sep=0pt, scale=  0.60] at ( 14.30,129.43) { 7 };

\path[draw=drawColor,line width= 0.4pt,line join=round,line cap=round] (142.26,115.03) --
	(138.07,122.35) --
	(137.07,123.82) --
	(132.61,129.67) --
	(129.75,132.92) --
	(125.69,136.98) --
	(122.43,139.85) --
	(116.58,144.30) --
	(115.11,145.30) --
	(107.80,149.50) --
	(103.25,151.62);

\path[draw=drawColor,line width= 0.4pt,line join=round,line cap=round] (143.00,113.45) -- (142.26,115.03);

\node[text=drawColor,rotate=-65.00,anchor=base west,inner sep=0pt, outer sep=0pt, scale=  0.60] at (140.95,112.49) { 7 };

\path[draw=drawColor,line width= 0.4pt,line join=round,line cap=round] (113.90, 12.59) --
	(115.11, 13.34);

\path[draw=drawColor,line width= 0.4pt,line join=round,line cap=round] (122.43, 18.76) --
	(123.77, 19.91) --
	(129.75, 25.69) --
	(131.14, 27.22) --
	(136.92, 34.54) --
	(137.07, 34.76) --
	(141.32, 41.86) --
	(144.38, 48.07);

\path[draw=drawColor,line width= 0.4pt,line join=round,line cap=round] (117.74, 15.29) -- (122.43, 18.76);

\node[text=drawColor,rotate= 36.52,anchor=base west,inner sep=0pt, outer sep=0pt, scale=  0.60] at (116.46, 11.53) { 7 };

\path[draw=drawColor,line width= 0.4pt,line join=round,line cap=round] ( 12.68, 12.59) --
	( 12.67, 12.60) --
	(  5.36, 19.91) --
	(  5.35, 19.92);

\path[draw=drawColor,line width= 0.4pt,line join=round,line cap=round] (  5.35,138.66) --
	( 10.84,144.30) --
	( 12.67,145.99) --
	( 19.60,151.62);

\path[draw=drawColor,line width= 0.4pt,line join=round,line cap=round] (140.26,136.98) --
	(137.07,140.56) --
	(133.32,144.30) --
	(129.75,147.49) --
	(124.50,151.62);

\path[draw=drawColor,line width= 0.4pt,line join=round,line cap=round] (142.36,134.31) -- (140.26,136.98);

\node[text=drawColor,rotate=-51.83,anchor=base west,inner sep=0pt, outer sep=0pt, scale=  0.60] at (140.58,132.91) { 9 };

\path[draw=drawColor,line width= 0.4pt,line join=round,line cap=round] (137.07, 18.07) --
	(138.75, 19.91) --
	(144.38, 26.84);

\path[draw=drawColor,line width= 0.4pt,line join=round,line cap=round] (131.42, 12.59) -- (134.72, 15.80);

\node[text=drawColor,rotate= 44.21,anchor=base west,inner sep=0pt, outer sep=0pt, scale=  0.60] at (136.30, 14.17) { 9 };
\definecolor{drawColor}{RGB}{152,78,163}

\path[draw=drawColor,line width= 1.2pt,line join=round,line cap=round] ( 61.12, 83.95) -- ( 43.66, 65.31);

\path[draw=drawColor,line width= 1.2pt,line join=round,line cap=round] ( 84.09, 50.95) -- ( 92.61, 52.51);

\path[draw=drawColor,line width= 1.2pt,line join=round,line cap=round] (120.03, 30.80) -- (124.47, 35.29);

\path[draw=drawColor,line width= 1.2pt,line join=round,line cap=round] (102.12, 95.27) -- ( 87.09, 98.32);

\path[draw=drawColor,line width= 1.2pt,line join=round,line cap=round] ( 45.37, 85.82) -- ( 67.82, 63.12);

\path[draw=drawColor,line width= 1.2pt,line join=round,line cap=round] ( 77.66, 47.49) -- ( 83.75, 60.15);
\definecolor{drawColor}{RGB}{255,127,0}

\path[draw=drawColor,line width= 1.2pt,line join=round,line cap=round] ( 31.00,135.13) -- ( 29.11,151.62);

\path[draw=drawColor,line width= 1.2pt,line join=round,line cap=round] ( 33.41,145.81) --
	( 29.11,151.62) --
	( 26.23,144.99);

\path[draw=drawColor,line width= 1.2pt,line join=round,line cap=round] (105.45, 42.87) -- (101.53, 13.22);

\path[draw=drawColor,line width= 1.2pt,line join=round,line cap=round] ( 98.77, 19.90) --
	(101.53, 13.22) --
	(105.93, 18.95);

\path[draw=drawColor,line width= 1.2pt,line join=round,line cap=round] (111.97,118.65) -- (139.19,145.37);

\path[draw=drawColor,line width= 1.2pt,line join=round,line cap=round] (137.25,138.41) --
	(139.19,145.37) --
	(132.19,143.56);

\path[draw=drawColor,line width= 1.2pt,line join=round,line cap=round] (117.50, 92.13) -- (131.79,110.19);

\path[draw=drawColor,line width= 1.2pt,line join=round,line cap=round] (130.74,103.04) --
	(131.79,110.19) --
	(125.08,107.53);

\path[draw=drawColor,line width= 1.2pt,line join=round,line cap=round] (125.01, 45.85) -- (123.70, 18.73);

\path[draw=drawColor,line width= 1.2pt,line join=round,line cap=round] (120.40, 25.15) --
	(123.70, 18.73) --
	(127.62, 24.80);

\path[draw=drawColor,line width= 1.2pt,line join=round,line cap=round] (107.73, 63.64) -- (111.50, 48.56);

\path[draw=drawColor,line width= 1.2pt,line join=round,line cap=round] (106.47, 53.76) --
	(111.50, 48.56) --
	(113.49, 55.51);

\path[draw=drawColor,line width= 1.2pt,line join=round,line cap=round] (100.01,103.59) -- (114.12,121.57);

\path[draw=drawColor,line width= 1.2pt,line join=round,line cap=round] (113.10,114.42) --
	(114.12,121.57) --
	(107.41,118.88);

\path[draw=drawColor,line width= 1.2pt,line join=round,line cap=round] ( 89.83, 42.11) -- ( 74.25, 16.51);

\path[draw=drawColor,line width= 1.2pt,line join=round,line cap=round] ( 74.42, 23.74) --
	( 74.25, 16.51) --
	( 80.59, 19.98);

\path[draw=drawColor,line width= 1.2pt,line join=round,line cap=round] ( 26.76, 93.40) -- (  7.01, 72.31);

\path[draw=drawColor,line width= 1.2pt,line join=round,line cap=round] (  8.65, 79.35) --
	(  7.01, 72.31) --
	( 13.93, 74.41);

\path[draw=drawColor,line width= 1.2pt,line join=round,line cap=round] ( 35.13, 88.32) -- (  5.35, 78.10);

\path[draw=drawColor,line width= 1.2pt,line join=round,line cap=round] ( 10.10, 83.55) --
	(  5.35, 78.10) --
	( 12.45, 76.71);

\path[draw=drawColor,line width= 1.2pt,line join=round,line cap=round] ( 21.85, 91.74) -- ( 11.53,109.04);

\path[draw=drawColor,line width= 1.2pt,line join=round,line cap=round] ( 17.84,105.52) --
	( 11.53,109.04) --
	( 11.63,101.81);

\path[draw=drawColor,line width= 1.2pt,line join=round,line cap=round] ( 38.60,116.27) -- ( 20.89,116.63);

\path[draw=drawColor,line width= 1.2pt,line join=round,line cap=round] ( 27.23,120.11) --
	( 20.89,116.63) --
	( 27.08,112.89);
\definecolor{drawColor}{RGB}{77,175,74}

\path[draw=drawColor,line width= 1.2pt,line join=round,line cap=round] ( 67.47, 71.83) -- ( 40.49, 50.57);

\path[draw=drawColor,line width= 1.2pt,line join=round,line cap=round] ( 43.17, 57.28) --
	( 40.49, 50.57) --
	( 47.65, 51.61);

\path[draw=drawColor,line width= 1.2pt,line join=round,line cap=round] ( 60.78, 48.55) -- ( 74.55, 29.80);

\path[draw=drawColor,line width= 1.2pt,line join=round,line cap=round] ( 67.94, 32.70) --
	( 74.55, 29.80) --
	( 73.76, 36.98);
\definecolor{drawColor}{RGB}{0,0,0}
\definecolor{fillColor}{RGB}{255,255,255}

\path[draw=drawColor,line width= 0.4pt,line join=round,line cap=round,fill=fillColor] (  0.00, 66.00) rectangle ( 53.62,  0.00);
\definecolor{drawColor}{RGB}{255,127,0}

\path[draw=drawColor,line width= 1.2pt,line join=round,line cap=round] (  9.90, 39.60) -- ( 29.70, 39.60);
\definecolor{drawColor}{RGB}{152,78,163}

\path[draw=drawColor,line width= 1.2pt,line join=round,line cap=round] (  9.90, 26.40) -- ( 29.70, 26.40);
\definecolor{drawColor}{RGB}{77,175,74}

\path[draw=drawColor,line width= 1.2pt,line join=round,line cap=round] (  9.90, 13.20) -- ( 29.70, 13.20);
\definecolor{drawColor}{RGB}{0,0,0}

\node[text=drawColor,anchor=base,inner sep=0pt, outer sep=0pt, scale=  1.00] at ( 26.81, 52.80) {label $y_i$};

\node[text=drawColor,anchor=base west,inner sep=0pt, outer sep=0pt, scale=  1.00] at ( 39.60, 35.83) {1};

\node[text=drawColor,anchor=base west,inner sep=0pt, outer sep=0pt, scale=  1.00] at ( 39.60, 22.63) {0};

\node[text=drawColor,anchor=base west,inner sep=0pt, outer sep=0pt, scale=  1.00] at ( 39.60,  9.43) {-1};

\path[draw=drawColor,line width= 0.4pt,line join=round,line cap=round] ( 31.00,135.13) circle (  2.47);

\path[draw=drawColor,line width= 0.4pt,line join=round,line cap=round] ( 29.11,151.62) circle (  2.47);

\node[text=drawColor,anchor=base,inner sep=0pt, outer sep=0pt, scale=  1.00] at ( 45.89,132.39) {$\mathbf x_{1}$};

\node[text=drawColor,anchor=base,inner sep=0pt, outer sep=0pt, scale=  1.00] at ( 44.00,148.88) {$\mathbf x_{1}'$};
\end{scope}
\begin{scope}
\path[clip] (  0.00,  0.00) rectangle (433.62,180.67);
\definecolor{drawColor}{RGB}{0,0,0}

\path[draw=drawColor,line width= 0.4pt,line join=round,line cap=round] (144.54,  0.00) --
	(289.08,  0.00) --
	(289.08,164.83) --
	(144.54,164.83) --
	(144.54,  0.00);
\end{scope}
\begin{scope}
\path[clip] (  0.00,  0.00) rectangle (433.62,180.67);
\definecolor{drawColor}{RGB}{0,0,0}

\node[text=drawColor,anchor=base,inner sep=0pt, outer sep=0pt, scale=  1.00] at (216.81,167.47) {Enlarged features $\Phi(\mathbf x)$};
\end{scope}
\begin{scope}
\path[clip] (144.54,  0.00) rectangle (289.08,164.83);
\definecolor{drawColor}{gray}{0.50}

\node[text=drawColor,rotate=-45.00,anchor=base,inner sep=0pt, outer sep=0pt, scale=  0.80] at (259.58, 62.82) {$r(\mathbf x)=\mathbf w^\intercal \Phi(\mathbf x)$};

\path[draw=drawColor,line width= 0.4pt,line join=round,line cap=round] (176.86,  6.10) --
	(175.13,  7.83) --
	(168.82, 14.14) --
	(167.10, 15.87);

\path[draw=drawColor,line width= 0.4pt,line join=round,line cap=round] (160.79, 22.17) --
	(159.06, 23.90) --
	(152.76, 30.20) --
	(151.03, 31.93);

\path[draw=drawColor,line width= 0.4pt,line join=round,line cap=round] (164.79, 18.18) -- (160.79, 22.17);

\node[text=drawColor,rotate=-45.02,anchor=base west,inner sep=0pt, outer sep=0pt, scale=  0.60] at (163.19, 16.58) { 1 };

\path[draw=drawColor,line width= 0.4pt,line join=round,line cap=round] (199.23,  9.65) --
	(194.74, 14.14) --
	(191.19, 17.68) --
	(186.71, 22.17) --
	(183.16, 25.72) --
	(178.67, 30.20) --
	(175.13, 33.75) --
	(170.64, 38.24) --
	(167.10, 41.78) --
	(162.61, 46.27) --
	(159.06, 49.81) --
	(154.58, 54.30) --
	(151.03, 57.85);

\path[draw=drawColor,line width= 0.4pt,line join=round,line cap=round] (200.46,  8.41) -- (199.23,  9.65);

\node[text=drawColor,rotate=-45.02,anchor=base west,inner sep=0pt, outer sep=0pt, scale=  0.60] at (198.86,  6.82) { 2 };

\path[draw=drawColor,line width= 0.4pt,line join=round,line cap=round] (228.69,  6.10) --
	(223.33, 11.47);

\path[draw=drawColor,line width= 0.4pt,line join=round,line cap=round] (220.65, 14.14) --
	(215.29, 19.50) --
	(212.62, 22.17) --
	(207.26, 27.53) --
	(204.59, 30.20) --
	(199.23, 35.56) --
	(196.56, 38.24) --
	(191.19, 43.60) --
	(188.52, 46.27) --
	(183.16, 51.63) --
	(180.49, 54.30) --
	(175.13, 59.66) --
	(172.46, 62.34) --
	(167.10, 67.70) --
	(164.42, 70.37) --
	(159.06, 75.73) --
	(156.39, 78.40) --
	(151.03, 83.76);

\path[draw=drawColor,line width= 0.4pt,line join=round,line cap=round] (221.02, 13.78) -- (220.65, 14.14);

\node[text=drawColor,rotate=-45.02,anchor=base west,inner sep=0pt, outer sep=0pt, scale=  0.60] at (219.42, 12.18) { 3 };

\path[draw=drawColor,line width= 0.4pt,line join=round,line cap=round] (254.60,  6.10) --
	(247.43, 13.28) --
	(246.57, 14.14);

\path[draw=drawColor,line width= 0.4pt,line join=round,line cap=round] (239.39, 21.32) --
	(238.54, 22.17) --
	(231.36, 29.35) --
	(230.50, 30.20) --
	(223.33, 37.38) --
	(222.47, 38.24) --
	(215.29, 45.41) --
	(214.44, 46.27) --
	(207.26, 53.45) --
	(206.41, 54.30) --
	(199.23, 61.48) --
	(198.37, 62.34) --
	(191.19, 69.51) --
	(190.34, 70.37) --
	(183.16, 77.55) --
	(182.31, 78.40) --
	(175.13, 85.58) --
	(174.27, 86.43) --
	(167.10, 93.61) --
	(166.24, 94.47) --
	(159.06,101.64) --
	(158.21,102.50) --
	(151.03,109.68);

\path[draw=drawColor,line width= 0.4pt,line join=round,line cap=round] (244.26, 16.45) -- (239.39, 21.32);

\node[text=drawColor,rotate=-45.02,anchor=base west,inner sep=0pt, outer sep=0pt, scale=  0.60] at (242.66, 14.85) { 4 };

\path[draw=drawColor,line width= 0.4pt,line join=round,line cap=round] (280.52,  6.10) --
	(279.56,  7.07) --
	(272.49, 14.14) --
	(271.52, 15.10);

\path[draw=drawColor,line width= 0.4pt,line join=round,line cap=round] (264.45, 22.17) --
	(263.49, 23.13) --
	(256.42, 30.20) --
	(255.46, 31.17) --
	(248.39, 38.24) --
	(247.43, 39.20) --
	(240.35, 46.27) --
	(239.39, 47.23) --
	(232.32, 54.30) --
	(231.36, 55.26) --
	(224.29, 62.34) --
	(223.33, 63.30) --
	(216.26, 70.37) --
	(215.29, 71.33) --
	(208.22, 78.40) --
	(207.26, 79.36) --
	(200.19, 86.43) --
	(199.23, 87.40) --
	(192.16, 94.47) --
	(191.19, 95.43) --
	(184.12,102.50) --
	(183.16,103.46) --
	(176.09,110.53) --
	(175.13,111.49) --
	(168.06,118.57) --
	(167.10,119.53) --
	(160.03,126.60) --
	(159.06,127.56) --
	(151.99,134.63) --
	(151.03,135.59);

\path[draw=drawColor,line width= 0.4pt,line join=round,line cap=round] (269.21, 17.41) -- (264.45, 22.17);

\node[text=drawColor,rotate=-45.02,anchor=base west,inner sep=0pt, outer sep=0pt, scale=  0.60] at (267.61, 15.81) { 5 };

\path[draw=drawColor,line width= 0.4pt,line join=round,line cap=round] (298.40, 14.14) --
	(295.62, 16.92) --
	(290.37, 22.17) --
	(287.59, 24.95) --
	(282.34, 30.20) --
	(279.56, 32.98) --
	(274.30, 38.24) --
	(271.52, 41.01) --
	(266.27, 46.27) --
	(263.49, 49.05) --
	(258.24, 54.30) --
	(255.46, 57.08) --
	(250.20, 62.34) --
	(247.43, 65.11) --
	(242.17, 70.37) --
	(239.39, 73.15) --
	(234.14, 78.40) --
	(231.36, 81.18) --
	(226.10, 86.43) --
	(223.33, 89.21) --
	(218.07, 94.47) --
	(215.29, 97.25) --
	(210.04,102.50) --
	(207.26,105.28) --
	(202.01,110.53) --
	(199.23,113.31) --
	(193.97,118.57) --
	(191.19,121.34) --
	(185.94,126.60) --
	(183.16,129.38) --
	(177.91,134.63) --
	(175.13,137.41) --
	(169.87,142.66) --
	(167.10,145.44) --
	(161.84,150.70) --
	(159.06,153.48) --
	(153.81,158.73);

\path[draw=drawColor,line width= 0.4pt,line join=round,line cap=round] (301.35, 11.19) -- (298.40, 14.14);

\node[text=drawColor,rotate=-45.02,anchor=base west,inner sep=0pt, outer sep=0pt, scale=  0.60] at (299.75,  9.59) { 6 };

\path[draw=drawColor,line width= 0.4pt,line join=round,line cap=round] (300.22, 38.24) --
	(295.62, 42.83) --
	(292.18, 46.27) --
	(287.59, 50.86) --
	(284.15, 54.30) --
	(279.56, 58.90) --
	(276.12, 62.34) --
	(271.52, 66.93) --
	(268.09, 70.37) --
	(263.49, 74.96) --
	(260.05, 78.40) --
	(255.46, 83.00) --
	(252.02, 86.43) --
	(247.43, 91.03) --
	(243.99, 94.47) --
	(239.39, 99.06) --
	(235.95,102.50) --
	(231.36,107.09) --
	(227.92,110.53) --
	(223.33,115.13) --
	(219.89,118.57) --
	(215.29,123.16) --
	(211.86,126.60) --
	(207.26,131.19) --
	(203.82,134.63) --
	(199.23,139.23) --
	(195.79,142.66) --
	(191.19,147.26) --
	(187.76,150.70) --
	(183.16,155.29) --
	(179.72,158.73);

\path[draw=drawColor,line width= 0.4pt,line join=round,line cap=round] (301.35, 37.11) -- (300.22, 38.24);

\node[text=drawColor,rotate=-45.02,anchor=base west,inner sep=0pt, outer sep=0pt, scale=  0.60] at (299.75, 35.51) { 7 };

\path[draw=drawColor,line width= 0.4pt,line join=round,line cap=round] (303.66, 60.71) --
	(302.03, 62.34);

\path[draw=drawColor,line width= 0.4pt,line join=round,line cap=round] (295.62, 68.75) --
	(294.00, 70.37) --
	(287.59, 76.78) --
	(285.97, 78.40) --
	(279.56, 84.81) --
	(277.94, 86.43) --
	(271.52, 92.85) --
	(269.90, 94.47) --
	(263.49,100.88) --
	(261.87,102.50) --
	(255.46,108.91) --
	(253.84,110.53) --
	(247.43,116.94) --
	(245.80,118.57) --
	(239.39,124.98) --
	(237.77,126.60) --
	(231.36,133.01) --
	(229.74,134.63) --
	(223.33,141.04) --
	(221.71,142.66) --
	(215.29,149.08) --
	(213.67,150.70) --
	(207.26,157.11) --
	(205.64,158.73);

\path[draw=drawColor,line width= 0.4pt,line join=round,line cap=round] (299.72, 64.64) -- (295.62, 68.75);

\node[text=drawColor,rotate=-45.02,anchor=base west,inner sep=0pt, outer sep=0pt, scale=  0.60] at (298.12, 63.05) { 8 };

\path[draw=drawColor,line width= 0.4pt,line join=round,line cap=round] (295.82, 94.47) --
	(295.62, 94.66) --
	(287.79,102.50) --
	(287.59,102.70) --
	(279.75,110.53) --
	(279.56,110.73) --
	(271.72,118.57) --
	(271.52,118.76) --
	(263.69,126.60) --
	(263.49,126.79) --
	(255.65,134.63) --
	(255.46,134.83) --
	(247.62,142.66) --
	(247.43,142.86) --
	(239.59,150.70) --
	(239.39,150.89) --
	(231.55,158.73);

\path[draw=drawColor,line width= 0.4pt,line join=round,line cap=round] (301.35, 88.94) -- (295.82, 94.47);

\node[text=drawColor,rotate=-45.02,anchor=base west,inner sep=0pt, outer sep=0pt, scale=  0.60] at (299.75, 87.34) { 9 };

\path[draw=drawColor,line width= 0.4pt,line join=round,line cap=round] (297.63,118.57) --
	(295.62,120.58) --
	(289.60,126.60) --
	(287.59,128.61) --
	(281.57,134.63) --
	(279.56,136.64) --
	(273.54,142.66) --
	(271.52,144.68) --
	(265.50,150.70) --
	(263.49,152.71) --
	(257.47,158.73);

\path[draw=drawColor,line width= 0.4pt,line join=round,line cap=round] (299.04,117.16) -- (297.63,118.57);

\node[text=drawColor,rotate=-45.02,anchor=base west,inner sep=0pt, outer sep=0pt, scale=  0.60] at (297.44,115.56) { 10 };

\path[draw=drawColor,line width= 0.4pt,line join=round,line cap=round] (303.66,138.46) --
	(299.45,142.66);

\path[draw=drawColor,line width= 0.4pt,line join=round,line cap=round] (291.42,150.70) --
	(287.59,154.53) --
	(283.39,158.73);

\path[draw=drawColor,line width= 0.4pt,line join=round,line cap=round] (294.83,147.28) -- (291.42,150.70);

\node[text=drawColor,rotate=-45.02,anchor=base west,inner sep=0pt, outer sep=0pt, scale=  0.60] at (293.23,145.68) { 11 };
\definecolor{drawColor}{RGB}{152,78,163}

\path[draw=drawColor,line width= 1.2pt,line join=round,line cap=round] (154.50,  6.68) -- (174.61, 11.80);

\path[draw=drawColor,line width= 1.2pt,line join=round,line cap=round] (155.19, 29.59) -- (163.29, 27.07);

\path[draw=drawColor,line width= 1.2pt,line join=round,line cap=round] (218.13, 74.86) -- (231.14, 62.71);

\path[draw=drawColor,line width= 1.2pt,line join=round,line cap=round] (177.33, 13.48) -- (157.56, 16.60);

\path[draw=drawColor,line width= 1.2pt,line join=round,line cap=round] (171.85,  7.29) -- (151.52, 13.74);

\path[draw=drawColor,line width= 1.2pt,line join=round,line cap=round] (151.89, 35.67) -- (154.96, 16.81);
\definecolor{drawColor}{RGB}{255,127,0}

\path[draw=drawColor,line width= 1.2pt,line join=round,line cap=round] (200.32, 97.03) -- (204.97,158.73);

\path[draw=drawColor,line width= 1.2pt,line join=round,line cap=round] (208.10,152.22) --
	(204.97,158.73) --
	(200.90,152.76);

\path[draw=drawColor,line width= 1.2pt,line join=round,line cap=round] (183.49, 44.90) -- (176.30,133.74);

\path[draw=drawColor,line width= 1.2pt,line join=round,line cap=round] (180.41,127.80) --
	(176.30,133.74) --
	(173.20,127.21);

\path[draw=drawColor,line width= 1.2pt,line join=round,line cap=round] (197.43, 51.24) -- (282.50,133.47);

\path[draw=drawColor,line width= 1.2pt,line join=round,line cap=round] (280.51,126.53) --
	(282.50,133.47) --
	(275.49,131.72);

\path[draw=drawColor,line width= 1.2pt,line join=round,line cap=round] (211.23, 10.85) -- (255.12, 33.89);

\path[draw=drawColor,line width= 1.2pt,line join=round,line cap=round] (251.25, 27.78) --
	(255.12, 33.89) --
	(247.90, 34.18);

\path[draw=drawColor,line width= 1.2pt,line join=round,line cap=round] (232.81, 38.80) -- (228.81,113.35);

\path[draw=drawColor,line width= 1.2pt,line join=round,line cap=round] (232.75,107.30) --
	(228.81,113.35) --
	(225.53,106.91);

\path[draw=drawColor,line width= 1.2pt,line join=round,line cap=round] (188.08, 13.25) -- (196.35, 33.71);

\path[draw=drawColor,line width= 1.2pt,line join=round,line cap=round] (197.36, 26.55) --
	(196.35, 33.71) --
	(190.66, 29.26);

\path[draw=drawColor,line width= 1.2pt,line join=round,line cap=round] (173.75, 23.26) -- (202.58, 58.20);

\path[draw=drawColor,line width= 1.2pt,line join=round,line cap=round] (201.38, 51.07) --
	(202.58, 58.20) --
	(195.81, 55.67);

\path[draw=drawColor,line width= 1.2pt,line join=round,line cap=round] (160.18, 46.54) -- (151.12,121.33);

\path[draw=drawColor,line width= 1.2pt,line join=round,line cap=round] (155.46,115.55) --
	(151.12,121.33) --
	(148.29,114.68);

\path[draw=drawColor,line width= 1.2pt,line join=round,line cap=round] (211.02, 11.85) -- (274.72,  7.50);

\path[draw=drawColor,line width= 1.2pt,line join=round,line cap=round] (268.23,  4.32) --
	(274.72,  7.50) --
	(268.72, 11.53);

\path[draw=drawColor,line width= 1.2pt,line join=round,line cap=round] (190.91,  8.42) -- (281.11,  6.10);

\path[draw=drawColor,line width= 1.2pt,line join=round,line cap=round] (274.76,  2.65) --
	(281.11,  6.10) --
	(274.94,  9.88);

\path[draw=drawColor,line width= 1.2pt,line join=round,line cap=round] (224.73, 10.56) -- (258.13, 31.85);

\path[draw=drawColor,line width= 1.2pt,line join=round,line cap=round] (254.79, 25.44) --
	(258.13, 31.85) --
	(250.91, 31.53);

\path[draw=drawColor,line width= 1.2pt,line join=round,line cap=round] (183.75, 45.92) -- (227.56, 46.70);

\path[draw=drawColor,line width= 1.2pt,line join=round,line cap=round] (221.37, 42.98) --
	(227.56, 46.70) --
	(221.24, 50.20);
\definecolor{drawColor}{RGB}{77,175,74}

\path[draw=drawColor,line width= 1.2pt,line join=round,line cap=round] (151.61,  7.70) -- (180.16, 30.22);

\path[draw=drawColor,line width= 1.2pt,line join=round,line cap=round] (177.48, 23.51) --
	(180.16, 30.22) --
	(173.01, 29.18);

\path[draw=drawColor,line width= 1.2pt,line join=round,line cap=round] (154.72, 33.74) -- (151.16, 77.73);

\path[draw=drawColor,line width= 1.2pt,line join=round,line cap=round] (155.27, 71.78) --
	(151.16, 77.73) --
	(148.07, 71.20);
\definecolor{drawColor}{RGB}{0,0,0}

\path[draw=drawColor,line width= 0.4pt,line join=round,line cap=round] (200.32, 97.03) circle (  2.47);

\path[draw=drawColor,line width= 0.4pt,line join=round,line cap=round] (204.97,158.73) circle (  2.47);

\node[text=drawColor,anchor=base,inner sep=0pt, outer sep=0pt, scale=  0.80] at (215.87, 94.84) {$\Phi(\mathbf x_{1})$};

\node[text=drawColor,anchor=base,inner sep=0pt, outer sep=0pt, scale=  0.80] at (220.52,156.54) {$\Phi(\mathbf x_{1}')$};
\definecolor{drawColor}{RGB}{190,190,190}

\path[draw=drawColor,line width= 0.4pt,line join=round,line cap=round] (144.54,  6.06) -- (289.08,  6.06);

\path[draw=drawColor,line width= 0.4pt,line join=round,line cap=round] (150.99,  0.00) -- (150.99,164.84);
\end{scope}
\begin{scope}
\path[clip] (289.08,  0.00) rectangle (433.62,164.83);
\definecolor{drawColor}{RGB}{255,127,0}
\definecolor{fillColor}{RGB}{255,127,0}

\path[draw=drawColor,line width= 0.4pt,line join=round,line cap=round,fill=fillColor] (331.84,126.68) circle (  1.65);
\definecolor{drawColor}{RGB}{152,78,163}
\definecolor{fillColor}{RGB}{152,78,163}

\path[draw=drawColor,line width= 0.4pt,line join=round,line cap=round,fill=fillColor] (349.26, 62.90) circle (  1.65);
\definecolor{drawColor}{RGB}{255,127,0}
\definecolor{fillColor}{RGB}{255,127,0}

\path[draw=drawColor,line width= 0.4pt,line join=round,line cap=round,fill=fillColor] (318.51,157.27) circle (  1.65);

\path[draw=drawColor,line width= 0.4pt,line join=round,line cap=round,fill=fillColor] (422.48,149.82) circle (  1.65);
\definecolor{drawColor}{RGB}{152,78,163}
\definecolor{fillColor}{RGB}{152,78,163}

\path[draw=drawColor,line width= 0.4pt,line join=round,line cap=round,fill=fillColor] (335.73, 54.30) circle (  1.65);
\definecolor{drawColor}{RGB}{255,127,0}
\definecolor{fillColor}{RGB}{255,127,0}

\path[draw=drawColor,line width= 0.4pt,line join=round,line cap=round,fill=fillColor] (376.07, 83.10) circle (  1.65);
\definecolor{drawColor}{RGB}{152,78,163}
\definecolor{fillColor}{RGB}{152,78,163}

\path[draw=drawColor,line width= 0.4pt,line join=round,line cap=round,fill=fillColor] (341.27, 43.44) circle (  1.65);
\definecolor{drawColor}{RGB}{255,127,0}
\definecolor{fillColor}{RGB}{255,127,0}

\path[draw=drawColor,line width= 0.4pt,line join=round,line cap=round,fill=fillColor] (322.10,141.16) circle (  1.65);

\path[draw=drawColor,line width= 0.4pt,line join=round,line cap=round,fill=fillColor] (335.92, 80.19) circle (  1.65);

\path[draw=drawColor,line width= 0.4pt,line join=round,line cap=round,fill=fillColor] (359.09, 96.51) circle (  1.65);
\definecolor{drawColor}{RGB}{152,78,163}
\definecolor{fillColor}{RGB}{152,78,163}

\path[draw=drawColor,line width= 0.4pt
  \caption{geometric interpretation of the comparison problem.}
  \label{fig:geometry}
\end{figure}

\subsection{SVMrank for comparing}

In this section we explain how to apply the existing SVMrank algorithm
to a comparison data set.

First we learn a ranking function $r:\RR^p \rightarrow \RR$ and then
threshold it using TODO.

\begin{table}[b!]
  \centering
  \begin{tabular}{|a|c|c|c|}\hline
    \rowcolor{lightgray}
    \backslashbox{$\hat{y}$}{ $y$}
    &\textbf{-1}&\textbf{0}&\textbf{1}\\ \hline
    \textbf{-1}&0  & FP & Inversion   	\\ \hline 
    \textbf{0} &FN& 0& FN\\ \hline
    \textbf{1} & Inversion & FP &0	\\ \hline
  \end{tabular}
  % \cellcolor{pastelblue}
  \caption{We use the zero-one loss to evaluate a predicted label
    $\hat y$ given the true label $y$. False positives (FP) occur 
    when predicting a significant difference $\hat y\in\{-1,1\}$ 
    when there is none $y=0$, and False Negatives (FN) are the opposite.
  Inversions occur when predicting the opposite of the true label
  $\hat y = -y$.}
  \label{tab:evaluation}
\end{table}

\begin{equation}
  \begin{aligned}
    \minimize_{w\in\RR^p}\ \  & w^\intercal w \\
    \text{subject to}\ \  & w^\intercal(x_i'-x_i)y_i \geq 1,
    \ \forall i\in I_1\cup I_{-1}.
  \end{aligned}
\end{equation}

\section{Support vector comparison machines}
\label{sec:svm-compare}

In this section we propse \proc{SVMcompare}, a new learning algorithm for
comparison problems.

\subsection{LP and QP for separable data}
\label{sec:lp-qp}

To illustrate the nature of the max-margin comparison problem, in this
section we assume that we have linearly separable training data.

In the following linear program (LP), we consider learning a linear
ranking function $r(x)=w^\intercal x$ that maximizes the margin $\mu$:
\begin{equation}
  \label{eq:max-margin-lp}
  \begin{aligned}
    \maximize_{\mu\in\RR, w\in\RR^p}\ & \mu \\
    \text{subject to}\ & \mu \leq 1-|w^\intercal (x_i' - x_i)|,\
    \forall\  i\in I_0\\
    &\mu \leq -1 +  w^\intercal(x_i'-x_i)y_i,\ \forall\ i\in I_1\cup I_{-1},
  \end{aligned}
\end{equation}
where $I_y=\{i\mid y_i=y\}$ are the sets of indices for the different
labels. Note that solving this problem is a practical test for linear
separability. If the optimal $\mu>0$ then the data are linearly
separable. The geometric interpretation of the margin is shown in
Figure~TODO. It is the distance from any difference vector $x_i'-x_i$
to its nearest decision boundary $r(x)\in\{-1,1\}$.

Another way to formulate the comparison problem is by first performing
a change of variables, and then learning a binary SVM. The idea is to
maximize the margin between significant differences $y_i\in\{-1,1\}$
and equality pairs $y_i=0$. Let $X_y,X_y'$ denote the $|I_y|\times p$
matrices formed by all the pairs $i\in I_y$. We thus define a new data
set with $m=|I_1|+|I_{-1}|+2|I_0|$ pairs suitable for training a
binary SVM:
\begin{equation}
  \tilde X = \left[
    \begin{array}{c}
      X_1 \\
      X_{-1}'\\
      X_0\\
      X_0'
    \end{array}
  \right],\ 
  \tilde X' = \left[
    \begin{array}{c}
      X_1' \\
      X_{-1}\\
      X_0'\\
      X_0
    \end{array}
  \right],\ 
  \tilde y = \left[
    \begin{array}{c}
      1_{|I_1|} \\
      1_{|I_{-1}|}\\
      -1_{|I_0|}\\
      -1_{|I_0|}
    \end{array}
  \right],
\end{equation}
where $1_n$ is an $n$-vector of ones, $\tilde X,\tilde
X'\in\RR^{m\times p}$ and $\tilde y\in\{-1,1\}^m$. Note that $\tilde
y_i=-1$ implies no significant difference between $\tilde x_i$ and
$\tilde x_i'$, and $\tilde y_i=1$ implies that $\tilde x_i$ is better
than $\tilde x_i'$. We then learn an affine function
$f(x)=\beta+a^\intercal x$ using binary SVM:
\begin{equation}
  \label{eq:max-margin-qp-tilde}
  \begin{aligned}
    \minimize_{a\in\RR^p, \beta\in\RR}\ & \frac 1 2 a^\intercal a  \\
    \text{subject to}\ & 
    \tilde y_i (\beta + a^\intercal( \tilde x_i'-\tilde x_i) ) \geq 1,
    \ \forall i\in\{1,\dots,m\}.
  \end{aligned}
\end{equation}
In the next Lemma, we construct a feasible solution to the original LP
(\ref{eq:max-margin-lp}) based on this QP
(\ref{eq:max-margin-qp-tilde}).
\begin{lemma}
  Let $a\in\RR^p,\beta\in\RR$ be a solution of
  (\ref{eq:max-margin-qp-tilde}). Then $\hat \mu = -1/\beta$
  and $\hat w = -a/\beta$ are feasible for
  (\ref{eq:max-margin-lp}).
\end{lemma}
\begin{proof}
  Begin by assuming that we want to find a ranking function $r(x)=\hat
  w^\intercal x = \gamma a^\intercal x$, where $\gamma\in\RR$ is a
  scaling constant.  Then consider that for all $x$ on the decision
  boundary, we have the following two equalities
  \begin{equation}
    \label{eq:dec-boundary-rank}
    a
  \end{equation}
and TODO
\begin{equation}
  \label{eq:max-margin-qp}
  \begin{aligned}
    \minimize_{a\in\RR^p, \beta\in\RR}\ & \frac 1 2 a^\intercal a  \\
    \text{subject to}\ & \beta + |a^\intercal (x_i'-x_i)| \leq -1,\
    \forall\  i\in I_0\\
    & \beta + a^\intercal(x_i'-x_i)y_i \geq 1,\ \forall\ i\in I_1\cup I_{-1}.
  \end{aligned}
\end{equation}
\end{proof}


\subsection{Kernelized QP for non-separable data}

In this section, we assume the 

Given a positive definite kernel function $\kappa:\RR^p\times
\RR^p\rightarrow\RR$, define the $K\in\RR^{2n\times 2n}$ kernel matrix
based on the $n$ pairs of training inputs $x_i,x_i'$.
% \begin{equation}
%   \label{eq:kernel}
%   K=  \left[
%     \begin{array}{cccccc}
%     \kappa(x_1, x_1) & \cdots & \kappa(x_n, x_1) &
%     \kappa(x_1', x_1) & \cdots & \kappa(x_n', x_1) \\
%     \vdots & & \vdots & \vdots && \vdots \\
%     \kappa(x_1, x_n) & \cdots & \kappa(x_n, x_n) &
%     \kappa(x_1', x_n) & \cdots & \kappa(x_n', x_n) \\
%     \kappa(x_1, x_1') & \cdots & \kappa(x_n, x_1') &
%     \kappa(x_1', x_1') & \cdots & \kappa(x_n', x_1') \\
%     \vdots & & \vdots & \vdots && \vdots 
%     \end{array}
%     \right].
% \end{equation}

%The three rows shown are $K_1,K_n,K_1'\in\RR^{2n}$. 

The primal problem for some $C\in\RR^+$ is
\begin{equation}
  \begin{aligned}
      \minimize_{a\in\RR^{2n},\xi\in\RR^n,\beta\in\RR}\ \ & 
      \frac 1 2 a^\intercal K a + C\sum_{i=1}^n \xi_i \\
      \text{subject to}\ \ & 
      \text{for all $i\in\{1,\dots,n\}$, }
      \xi_i \geq 0,\\
      &\text{and }
      \xi_i \geq 1-y_i(\beta + a^\intercal (K_i'-K_i)).
  \end{aligned}
\end{equation}
TODO: write the Lagrangian and dual feasibility conditions.

If $v\in\RR^n$ are the dual variables corresponding to the second
constraint, then the dual problem is
\begin{equation}
  \begin{aligned}
    \label{eq:svm-dual}
    \minimize_{v\in\RR^n}\ \ &
    \frac 1 2 v^\intercal Y M^\intercal K M Y v - v^\intercal 1\\
    \text{subject to}\ \ &
    \sum_{i=1}^n v_i y_i = 0,\\
&    \text{for all $i\in\{1,\dots,n\}$, } 0\leq v_i\leq C,
  \end{aligned}
\end{equation}
where $M=[-I_n \, I_n]^\intercal\in\{-1,0,1\}^{2n\times n}$. This quadratic problem
is equivalent to a standard SVM with kernel $M^\intercal K M$.

So we can use any efficient SVM solver, such as libsvm
\citep{libsvm}. We used the R interface in the \texttt{kernlab}
package of \citet{kernlab}, and our code is available in the
\texttt{rankSVMcompare} package on Github.

After solving the SVM dual problem (\ref{eq:svm-dual}), we recover the
primal variables $a = MYv$ and the bias $\beta$ using the usual method
(TODO). The learned ranking function is $r(x) = \sum_{i\in\text{sv}}
-a_i \kappa(x, x_i)/\beta$ (TODO...Z??).

\proc{SVMcompare} is summarized as Algorithm~\ref{alg:SVMcompare}.

\begin{algorithm}[b!]
   \caption{\proc{SVMcompare}}
   \label{alg:SVMcompare}
\begin{algorithmic}
  \STATE {\bfseries Input:} equality pairs $Z,Z'\in\RR^{m \times p}$,
  inequality pairs $X,X'\in\RR^{n\times p},\, y\in\{-1,0,1\}$, cost
  parameter $C\in\RR^+$, kernel $\kappa:\RR^p\times \RR^p \rightarrow
  \RR$.

  \STATE $v \gets \proc{SVMdual}(K, y, C)$.

  \STATE $a \gets M Y v$.

  \STATE $\beta = ...$
  
  \STATE $r(x) = ...$

%  $k_c \gets \max \{ \plausibleK \}$ 

  % \STATE $\plausibleK \gets \plausibleK \setminus k_c$
  
  % \WHILE{$\plausibleK \neq \emptyset$} 

  % \STATE $\text{next}\lambda \gets + \infty, \quad \text{nextK} \gets
  % 0$ 

  % \FOR{$ k \in \plausibleK $} 
  
  % \STATE $\text{hit\_time} \gets \dfrac{ ||y_i-\hat y_i^{k_c}||_2^2 -
  %   ||y_i-\hat y_i^k||_2^2 }{ h\left(\hat y_i^k, x_i\right) -
  %   h\left(\hat y_i^{k_c}, x_i\right) }$

  % \IF{$\text{next}\lambda > \text{hit\_time}$}

  % \STATE $\text{next}\lambda \gets \text{hit\_time},\quad \text{nextK}
  % \gets k$

  %  \ENDIF
  %  \ENDFOR
  %  \STATE $k_c\gets \text{nextK},\quad$\textbf{Save} $k_c, \text{next}\lambda$ 
  %  \STATE $\plausibleK \gets
  %  \plausibleK \setminus \{k\mid k\geq k_c\}$
  %  \ENDWHILE
  %  \STATE {\bfseries Output:} $\hat k_i$ represented by breakpoints
  %  $k_c,\text{next}\lambda$.
   \end{algorithmic}
\end{algorithm}

\section{Results}
\label{sec:results}

\section{Conclusions and future work}
\label{sec:conclusions}

\bibliographystyle{abbrvnat}
\bibliography{refs}

\end{document}

