\documentclass{article}
\usepackage{aistats2014} 
\usepackage{listings}
\usepackage{clrscode}
\usepackage[table]{xcolor}
\usepackage{array}
\usepackage{slashbox}
\usepackage{multirow}
%\usepackage{fullpage}
\usepackage{tikz}
\usepackage{stmaryrd}
\usepackage{graphicx}
\usepackage{amsmath,amssymb,amsthm}
\usepackage{stfloats}
\usepackage{float}
\newcommand{\sv}{\textsc{sv}}

\renewcommand{\thefootnote}{\fnsymbol{footnote}}

\newtheorem{proposition}{Proposition}

\newtheorem{definition}{Definition}
\newtheorem{theo}{Theorem}    % numérotés par section
\newtheorem{lemma}{Lemma}

\newcommand{\RR}{\mathbb R}
\newcommand{\NN}{\mathbb N}
\newcommand{\pkg}[1]{\texttt{#1}}
\newcommand{\plausibleK}{\textit{plausibleK}}

\DeclareMathOperator*{\argmin}{arg\,min}
\DeclareMathOperator*{\Diag}{Diag}
\DeclareMathOperator*{\argmax}{arg\,max}
\DeclareMathOperator*{\maximize}{maximize}
\DeclareMathOperator*{\minimize}{minimize}


\newfloat{Algorithm}{thp}{lop}
\floatname{Algorithm}{Algorithm}

% For citations
\usepackage{natbib}

% For algorithms
\usepackage{algorithm}
\usepackage{algorithmic}
\usepackage{hyperref}
\newcommand{\theHalgorithm}{\arabic{algorithm}}


%\icmltitlerunning{Support vector comparison machines}

\begin{document}

% \begin{ekeywords}
%   support vector machine, quadratic program, linear program, ranking,
%   margin, comparison
% \end{ekeywords}

\renewcommand{\arraystretch}{1.5}

\definecolor{lightgray}{rgb}{0.9,0.9,0.9}
\definecolor{pastelblue}{RGB}{213,229,255}
\newcolumntype{a}{>{\columncolor{lightgray}}c}

\twocolumn[
\aistatstitle{Support vector comparison machines}

% It is OKAY to include author information, even for blind
% submissions: the style file will automatically remove it for you
% unless you've provided the [accepted] option to the icml2012
% package.


%\icmlauthor{Guillem Rigaill\footnotemark[1]}{rigaill@evry.inra.fr}
%\icmladdress{Unit\'e de Recherche en G\'enomique V\'eg\'etale INRA-CNRS-Universit\'e d'Evry Val d'Essonne, Evry, France}
%\icmlauthor{Toby Dylan Hocking\footnotemark[1], 
% Francis Bach}{toby.hocking@inria.fr, francis.bach@inria.fr}

%\aistatsauthor{Toby Dylan Hocking \And Supaporn Spanurattana \And Masashi Sugiyama}
%\aistatsaddress{Department of Computer Science, Tokyo Institute of Technology, Tokyo 152-8552, Japan}

\aistatsauthor{Anonymous authors}
\aistatsaddress{Some school}

% You may provide any keywords that you 
% find helpful for describing your paper; these are used to populate 
% the "keywords" metadata in the PDF but will not be shown in the document

\vskip 0.3in
]

\begin{abstract}
  In ranking problems, the goal is to learn a ranking function
  $r(\mathbf x)\in\RR$ from labeled pairs $\mathbf x,\mathbf x'$ of
  input points. In this paper, we consider the related comparison
  problem, where the label $y\in\{-1,0,1\}$ indicates which element of
  the pair is better, or if there is no significant difference. We
  cast the learning problem as a margin maximization, and show that it
  can be solved by converting it to a standard SVM. We use simulated
  nonlinear patterns and a real learning to rank sushi data set to
  show that our proposed SVMcompare algorithm outperforms SVMrank when
  there are equality $y=0$ pairs.
\end{abstract}

\section{Introduction}

In this paper we consider the supervised comparison problem. Assume
that we have $n$ labeled training pairs and for each pair
$i\in\{1,\dots,n\}$ we have input features $\mathbf x_i,\mathbf
x_i'\in\RR^p$ and a label $y_i\in\{-1,0,1\}$ that indicates which
element is better:
\begin{equation}
  \label{eq:z}
  y_i =
  \begin{cases}
    -1 &  r(\mathbf x_i)>r(\mathbf x_i')
    \text{, $\mathbf x_i$ is better than $\mathbf x'_i$},\\
    0 & r(\mathbf x_i) = r(\mathbf x_i')
    \text{, $\mathbf x_i$ is as good as $\mathbf x'_i$},\\
    1 & r(\mathbf x_i)<r(\mathbf x_i')
    \text{, $\mathbf x'_i$ is better than $\mathbf x_i$}.
  \end{cases}
\end{equation}
These data are geometrically represented by segments and arrows in the
top panel of Figure~\ref{fig:norm-data}. 

Comparison data naturally arise when considering subjective human
evaluations of pairs of items. For example, if I were to compare some
pairs of movies I have watched, I would say \textit{Les Mis\'erables}
is better than \textit{Star Wars}, and \textit{The Empire Strikes
  Back} is as good as \textit{Star Wars}. Features $\mathbf
x_i,\mathbf x_i'$ of the movies can be length in minutes, year of
theatrical release, indicators for genre, actors/actresses, directors,
etc.

The goal of learning is to find a comparison function $c:\RR^p \times
\RR^p \rightarrow \{-1,0,1\}$ which generalizes to a test set of data,
as measured by the zero-one loss:
\begin{equation}
  \label{eq:min_c}
  \minimize_{c} 
  \sum_{i\in\text{test}}
  I\left[ c(\mathbf x_i, \mathbf x_i')\neq y_i \right],
\end{equation}
where $I$ is the indicator function. If there are no equality $y_i=0$
pairs, then this problem is equivalent to learning to rank with a
pairwise zero-one loss function \citep{learning-to-rank}. Learning to
rank has been extensively studied, resulting in state-of-the-art
algorithms such as SVMrank \citep{ranksvm}. However, we are interested
in learning to compare with equality $y_i=0$ pairs, which to our
knowledge has only been studied by \citet{rank-with-ties}. In this
article we propose SVMcompare, a support vector algorithm for these
data.

The notation and organization of this article is as follows. We use
bold uppercase letters for matrices such as $\mathbf X, \mathbf K$,
and bold lowercase letters for their row vectors $\mathbf x_i, \mathbf
k_i$. In Section~\ref{sec:related} we discuss links with related work
on classification and ranking, then in Section~\ref{sec:svm-compare}
we propose a new algorithm: SVMcompare. We show results on 3
illustrative simulated data sets and one real learning to rank sushi
data set in Section~\ref{sec:results}, then discuss future work in
Section~\ref{sec:conclusions}.

\begin{table}[b!]
  \centering
  \begin{tabular}{|a|c|c|}\hline
    \rowcolor{lightgray}
    \backslashbox{Outputs}{Inputs}
    &single items $\mathbf x$&pairs $\mathbf x,\mathbf x'$\\ \hline
    $y\in\{-1,1\}$ &SVM  & SVMrank   	\\ \hline 
    $y\in\{-1,0,1\}$ &Reject option& this work\\ \hline
  \end{tabular}
  \caption{\label{tab:related} Comparison is similar to ranking 
    and classification with reject option.}
\end{table}

\section{Related work}
\label{sec:related}

First we discuss connections with several existing methods, and then
we discuss how ranking algorithms can be applied to the comparison
problem.

% $\mathbf x_i,\mathbf x_i'\in\RR^p$.

\begin{figure}
  \centering
  % Created by tikzDevice version 0.7.0 on 2014-01-30 15:30:08
% !TEX encoding = UTF-8 Unicode
\begin{tikzpicture}[x=1pt,y=1pt]
\definecolor[named]{fillColor}{rgb}{1.00,1.00,1.00}
\path[use as bounding box,fill=fillColor,fill opacity=0.00] (0,0) rectangle (224.04,578.16);
\begin{scope}
\path[clip] (  0.00,  0.00) rectangle (224.04,578.16);
\definecolor[named]{drawColor}{rgb}{0.00,0.00,0.00}

\path[draw=drawColor,line width= 0.4pt,line join=round,line cap=round] ( 14.40,409.44) --
	(212.04,409.44) --
	(212.04,578.16) --
	( 14.40,578.16) --
	( 14.40,409.44);
\end{scope}
\begin{scope}
\path[clip] (  0.00,385.44) rectangle (224.04,578.16);
\definecolor[named]{drawColor}{rgb}{0.00,0.00,0.00}

\node[text=drawColor,anchor=base,inner sep=0pt, outer sep=0pt, scale=  1.00] at (113.22,398.64) {input feature $ x_{i,1}$};

\node[text=drawColor,rotate= 90.00,anchor=base,inner sep=0pt, outer sep=0pt, scale=  1.00] at ( 10.80,493.80) {input feature $ x_{i,2}$};
\end{scope}
\begin{scope}
\path[clip] (  0.00,  0.00) rectangle (224.04,578.16);
\definecolor[named]{drawColor}{rgb}{0.00,0.00,0.00}

\node[text=drawColor,rotate= 90.00,anchor=base,inner sep=0pt, outer sep=0pt, scale=  1.00] at (221.64,493.80) {Original feature space $\mathbf x\in\mathbb R^p$};
\end{scope}
\begin{scope}
\path[clip] ( 14.40,409.44) rectangle (212.04,578.16);
\definecolor[named]{drawColor}{rgb}{0.50,0.50,0.50}

\path[draw=drawColor,line width= 0.4pt,line join=round,line cap=round] ( 84.58,472.80) --
	( 80.89,481.06) --
	( 79.47,489.32) --
	( 80.34,497.58) --
	( 83.48,505.83) --
	( 87.24,511.56) --
	( 89.55,514.09) --
	( 95.50,518.71) --
	(103.09,522.35) --
	(103.76,522.60) --
	(112.02,523.77) --
	(120.28,523.06) --
	(122.51,522.35) --
	(128.54,519.89) --
	(136.78,514.09) --
	(136.80,514.07) --
	(142.60,505.83) --
	(145.06,499.81) --
	(145.76,497.58) --
	(146.48,489.32) --
	(145.31,481.06) --
	(145.06,480.39) --
	(141.42,472.80) --
	(136.80,466.84) --
	(134.26,464.54) --
	(128.54,460.78) --
	(120.28,457.63) --
	(112.02,456.76) --
	(103.76,458.18) --
	( 95.50,461.88) --
	( 91.83,464.54) --
	( 87.24,469.12);

\path[draw=drawColor,line width= 0.4pt,line join=round,line cap=round] ( 85.48,471.55) -- ( 84.58,472.80);

\node[text=drawColor,rotate=-54.13,anchor=base west,inner sep=0pt, outer sep=0pt, scale=  0.60] at ( 83.81,470.34) { 1 };

\path[draw=drawColor,line width= 0.4pt,line join=round,line cap=round] ( 68.95,472.80) --
	( 66.57,481.06) --
	( 65.65,489.32) --
	( 66.21,497.58) --
	( 68.24,505.83) --
	( 70.72,511.70) --
	( 71.95,514.09) --
	( 77.99,522.35) --
	( 78.98,523.40) --
	( 87.24,530.23) --
	( 87.89,530.61) --
	( 95.50,534.32) --
	(103.76,536.81) --
	(112.02,537.76) --
	(120.28,537.18) --
	(128.54,535.06) --
	(136.80,531.41) --
	(138.07,530.61) --
	(145.06,525.22) --
	(147.92,522.35) --
	(153.32,515.36) --
	(154.11,514.09) --
	(157.77,505.83) --
	(159.89,497.58) --
	(160.47,489.32) --
	(159.52,481.06) --
	(157.03,472.80) --
	(153.32,465.19) --
	(152.93,464.54) --
	(146.10,456.28) --
	(145.06,455.29) --
	(136.80,449.25) --
	(134.41,448.02) --
	(128.54,445.53) --
	(120.28,443.51) --
	(112.02,442.95) --
	(103.76,443.86) --
	( 95.50,446.24) --
	( 91.69,448.02) --
	( 87.24,450.54) --
	( 79.92,456.28) --
	( 78.98,457.22) --
	( 73.25,464.54) --
	( 70.72,468.99);

\path[draw=drawColor,line width= 0.4pt,line join=round,line cap=round] ( 70.72,468.99) -- ( 70.22,470.08);

\node[text=drawColor,rotate=-65.04,anchor=base west,inner sep=0pt, outer sep=0pt, scale=  0.60] at ( 67.07,471.92) { 2 };

\path[draw=drawColor,line width= 0.4pt,line join=round,line cap=round] ( 60.90,464.54) --
	( 57.63,472.80) --
	( 55.61,481.06) --
	( 54.84,489.32) --
	( 55.31,497.58) --
	( 57.03,505.83) --
	( 60.00,514.09) --
	( 62.46,518.92) --
	( 64.53,522.35) --
	( 70.72,530.31) --
	( 71.02,530.61) --
	( 78.98,537.46) --
	( 81.09,538.87) --
	( 87.24,542.37) --
	( 95.50,545.76) --
	(100.90,547.13) --
	(103.76,547.76) --
	(112.02,548.45) --
	(120.28,548.03) --
	(125.08,547.13) --
	(128.54,546.39) --
	(136.80,543.30) --
	(145.06,538.93) --
	(145.14,538.87) --
	(153.32,532.22) --
	(154.92,530.61) --
	(161.58,522.44) --
	(161.63,522.35) --
	(166.01,514.09) --
	(169.09,505.83) --
	(169.84,502.38) --
	(170.74,497.58) --
	(171.16,489.32) --
	(170.47,481.06) --
	(169.84,478.20) --
	(168.47,472.80) --
	(165.07,464.54) --
	(161.58,458.38) --
	(160.16,456.28) --
	(153.32,448.31) --
	(153.01,448.02) --
	(145.06,441.82) --
	(141.63,439.76) --
	(136.80,437.29) --
	(128.54,434.33) --
	(120.28,432.61) --
	(112.02,432.13) --
	(103.76,432.91) --
	( 95.50,434.93) --
	( 87.24,438.20) --
	( 84.39,439.76) --
	( 78.98,443.24) --
	( 73.17,448.02) --
	( 70.72,450.47) --
	( 65.95,456.28) --
	( 62.46,461.68);

\path[draw=drawColor,line width= 0.4pt,line join=round,line cap=round] ( 62.46,461.68) -- ( 62.34,461.91);

\node[text=drawColor,rotate=-61.36,anchor=base west,inner sep=0pt, outer sep=0pt, scale=  0.60] at ( 59.09,463.55) { 3 };

\path[draw=drawColor,line width= 0.4pt,line join=round,line cap=round] ( 45.94,487.25) --
	( 45.80,489.32);

\path[draw=drawColor,line width= 0.4pt,line join=round,line cap=round] ( 45.94,492.70) --
	( 46.19,497.58) --
	( 47.68,505.83) --
	( 50.26,514.09) --
	( 53.92,522.35) --
	( 54.20,522.84) --
	( 59.35,530.61) --
	( 62.46,534.45) --
	( 66.70,538.87) --
	( 70.72,542.42) --
	( 77.23,547.13) --
	( 78.98,548.23) --
	( 87.24,552.28) --
	( 95.50,555.22) --
	( 96.29,555.39) --
	(103.76,556.84) --
	(112.02,557.45) --
	(120.28,557.07) --
	(128.54,555.71) --
	(129.68,555.39) --
	(136.80,553.10) --
	(145.06,549.31) --
	(148.73,547.13) --
	(153.32,543.98) --
	(159.38,538.87) --
	(161.58,536.68) --
	(166.69,530.61) --
	(169.84,526.03) --
	(172.02,522.35) --
	(175.80,514.09) --
	(178.10,506.97) --
	(178.42,505.83) --
	(179.78,497.58) --
	(180.15,489.32) --
	(179.54,481.06) --
	(178.10,473.59) --
	(177.92,472.80) --
	(174.99,464.54) --
	(170.94,456.28) --
	(169.84,454.52) --
	(165.13,448.02) --
	(161.58,443.99) --
	(157.16,439.76) --
	(153.32,436.64) --
	(145.55,431.50) --
	(145.06,431.22) --
	(136.80,427.56) --
	(128.54,424.98) --
	(120.28,423.48) --
	(115.40,423.24) --
	(112.02,423.09) --
	(109.96,423.24) --
	(103.76,423.74) --
	( 95.50,425.50) --
	( 87.24,428.34) --
	( 80.60,431.50) --
	( 78.98,432.38) --
	( 70.72,438.15) --
	( 68.82,439.76) --
	( 62.46,446.12) --
	( 60.85,448.02) --
	( 55.09,456.28) --
	( 54.20,457.89) --
	( 51.05,464.54) --
	( 48.21,472.80) --
	( 46.45,481.06) --
	( 45.94,487.25);

\path[draw=drawColor,line width= 0.4pt,line join=round,line cap=round] ( 45.93,492.31) -- ( 45.94,492.70);

\node[text=drawColor,rotate= 87.52,anchor=base west,inner sep=0pt, outer sep=0pt, scale=  0.60] at ( 47.86,489.23) { 4 };

\path[draw=drawColor,line width= 0.4pt,line join=round,line cap=round] ( 42.52,464.54) --
	( 40.01,472.80) --
	( 38.45,481.06) --
	( 37.86,489.32) --
	( 38.22,497.58) --
	( 39.54,505.83) --
	( 41.82,514.09) --
	( 45.06,522.35) --
	( 45.94,524.09) --
	( 49.69,530.61) --
	( 54.20,537.01) --
	( 55.72,538.87) --
	( 62.46,545.87) --
	( 63.90,547.13) --
	( 70.72,552.33) --
	( 75.61,555.39) --
	( 78.98,557.25) --
	( 87.24,560.82) --
	( 95.50,563.40) --
	( 96.78,563.65) --
	(103.76,564.86) --
	(112.02,565.40) --
	(120.28,565.07) --
	(128.54,563.86) --
	(129.35,563.65) --
	(136.80,561.53) --
	(145.06,558.20) --
	(150.44,555.39) --
	(153.32,553.68) --
	(161.58,547.66) --
	(162.19,547.13) --
	(169.84,539.49) --
	(170.37,538.87) --
	(176.39,530.61) --
	(178.10,527.73) --
	(180.91,522.35) --
	(184.24,514.09) --
	(186.36,506.64) --
	(186.56,505.83) --
	(187.78,497.58) --
	(188.11,489.32) --
	(187.56,481.06) --
	(186.36,474.07) --
	(186.11,472.80) --
	(183.53,464.54) --
	(179.96,456.28) --
	(178.10,452.90) --
	(175.04,448.02) --
	(169.84,441.19) --
	(168.58,439.76) --
	(161.58,433.02) --
	(159.71,431.50) --
	(153.32,426.99) --
	(146.80,423.24) --
	(145.06,422.36) --
	(136.80,419.12) --
	(128.54,416.84) --
	(120.28,415.52) --
	(112.02,415.15) --
	(103.76,415.75) --
	( 95.50,417.30) --
	( 87.24,419.81) --
	( 79.08,423.24) --
	( 78.98,423.29) --
	( 70.72,428.29) --
	( 66.38,431.50) --
	( 62.46,434.82) --
	( 57.53,439.76) --
	( 54.20,443.67) --
	( 51.00,448.02) --
	( 45.99,456.28) --
	( 45.94,456.38);

\path[draw=drawColor,line width= 0.4pt,line join=round,line cap=round] ( 45.94,456.38) -- ( 43.68,461.77);

\node[text=drawColor,rotate=-67.25,anchor=base west,inner sep=0pt, outer sep=0pt, scale=  0.60] at ( 40.61,463.74) { 5 };

\path[draw=drawColor,line width= 0.4pt,line join=round,line cap=round] ( 79.85,414.98) --
	( 78.98,415.34);

\path[draw=drawColor,line width= 0.4pt,line join=round,line cap=round] ( 70.72,419.77) --
	( 65.40,423.24) --
	( 62.46,425.41) --
	( 55.45,431.50) --
	( 54.20,432.75) --
	( 48.11,439.76) --
	( 45.94,442.70) --
	( 42.48,448.02) --
	( 38.05,456.28) --
	( 37.69,457.15);

\path[draw=drawColor,line width= 0.4pt,line join=round,line cap=round] ( 76.34,416.76) -- ( 70.72,419.77);

\node[text=drawColor,rotate=-28.21,anchor=base west,inner sep=0pt, outer sep=0pt, scale=  0.60] at ( 75.36,414.94) { 6 };

\path[draw=drawColor,line width= 0.4pt,line join=round,line cap=round] ( 41.32,530.61) --
	( 45.94,538.02) --
	( 46.54,538.87) --
	( 53.46,547.13) --
	( 54.20,547.90) --
	( 62.46,555.31) --
	( 62.57,555.39) --
	( 70.72,560.86) --
	( 75.78,563.65) --
	( 78.98,565.23) --
	( 87.24,568.42) --
	( 95.50,570.73) --
	(102.36,571.91);

\path[draw=drawColor,line width= 0.4pt,line join=round,line cap=round] ( 39.04,526.13) -- ( 41.32,530.61);

\node[text=drawColor,rotate= 63.10,anchor=base west,inner sep=0pt, outer sep=0pt, scale=  0.60] at ( 39.53,522.52) { 6 };

\path[draw=drawColor,line width= 0.4pt,line join=round,line cap=round] (193.86,505.83) --
	(191.76,514.09) --
	(188.78,522.35) --
	(186.36,527.55) --
	(184.76,530.61) --
	(179.46,538.87) --
	(178.10,540.66) --
	(172.50,547.13) --
	(169.84,549.79) --
	(163.36,555.39) --
	(161.58,556.75) --
	(153.32,562.05) --
	(150.26,563.65) --
	(145.06,566.08) --
	(136.80,569.06) --
	(128.54,571.15) --
	(123.37,571.91);

\path[draw=drawColor,line width= 0.4pt,line join=round,line cap=round] (194.18,503.63) -- (193.86,505.83);

\node[text=drawColor,rotate=-81.68,anchor=base west,inner sep=0pt, outer sep=0pt, scale=  0.60] at (192.14,503.33) { 6 };

\path[draw=drawColor,line width= 0.4pt,line join=round,line cap=round] (153.32,418.62) --
	(160.73,423.24) --
	(161.58,423.84) --
	(169.84,430.75) --
	(170.61,431.50) --
	(178.02,439.76) --
	(178.10,439.86) --
	(183.57,448.02) --
	(186.36,453.07) --
	(187.93,456.28) --
	(191.12,464.54) --
	(193.43,472.80) --
	(194.62,479.66);

\path[draw=drawColor,line width= 0.4pt,line join=round,line cap=round] (148.84,416.34) -- (153.32,418.62);

\node[text=drawColor,rotate= 26.94,anchor=base west,inner sep=0pt, outer sep=0pt, scale=  0.60] at (147.10,413.14) { 6 };

\path[draw=drawColor,line width= 0.4pt,line join=round,line cap=round] ( 62.46,417.22) --
	( 54.63,423.24) --
	( 54.20,423.61) --
	( 46.31,431.50) --
	( 45.94,431.92) --
	( 39.93,439.76) --
	( 37.69,443.19);

\path[draw=drawColor,line width= 0.4pt,line join=round,line cap=round] ( 63.39,416.62) -- ( 62.46,417.22);

\node[text=drawColor,rotate=-33.12,anchor=base west,inner sep=0pt, outer sep=0pt, scale=  0.60] at ( 62.26,414.89) { 7 };

\path[draw=drawColor,line width= 0.4pt,line join=round,line cap=round] ( 37.69,537.51) --
	( 38.54,538.87);

\path[draw=drawColor,line width= 0.4pt,line join=round,line cap=round] ( 44.65,547.13) --
	( 45.94,548.65) --
	( 52.48,555.39) --
	( 54.20,556.96) --
	( 62.46,563.49) --
	( 62.71,563.65) --
	( 70.72,568.46) --
	( 77.74,571.91);

\path[draw=drawColor,line width= 0.4pt,line join=round,line cap=round] ( 40.32,541.28) -- ( 44.65,547.13);

\node[text=drawColor,rotate= 53.53,anchor=base west,inner sep=0pt, outer sep=0pt, scale=  0.60] at ( 40.20,537.64) { 7 };

\path[draw=drawColor,line width= 0.4pt,line join=round,line cap=round] (192.22,530.61) --
	(187.49,538.87) --
	(186.36,540.53) --
	(181.33,547.13) --
	(178.10,550.81) --
	(173.51,555.39) --
	(169.84,558.63) --
	(163.24,563.65) --
	(161.58,564.78) --
	(153.32,569.52) --
	(148.19,571.91);

\path[draw=drawColor,line width= 0.4pt,line join=round,line cap=round] (193.35,528.21) -- (192.22,530.61);

\node[text=drawColor,rotate=-65.00,anchor=base west,inner sep=0pt, outer sep=0pt, scale=  0.60] at (191.48,527.33) { 7 };

\path[draw=drawColor,line width= 0.4pt,line join=round,line cap=round] (160.21,414.98) --
	(161.58,415.83);

\path[draw=drawColor,line width= 0.4pt,line join=round,line cap=round] (169.84,421.94) --
	(171.35,423.24) --
	(178.10,429.77) --
	(179.67,431.50) --
	(186.19,439.76) --
	(186.36,440.00) --
	(191.16,448.02) --
	(194.62,455.03);

\path[draw=drawColor,line width= 0.4pt,line join=round,line cap=round] (163.99,417.62) -- (169.84,421.94);

\node[text=drawColor,rotate= 36.52,anchor=base west,inner sep=0pt, outer sep=0pt, scale=  0.60] at (162.81,414.17) { 7 };

\path[draw=drawColor,line width= 0.4pt,line join=round,line cap=round] ( 55.05,414.98) --
	( 54.20,415.63);

\path[draw=drawColor,line width= 0.4pt,line join=round,line cap=round] ( 45.94,422.93) --
	( 45.63,423.24) --
	( 38.33,431.50) --
	( 37.69,432.34);

\path[draw=drawColor,line width= 0.4pt,line join=round,line cap=round] ( 51.96,417.61) -- ( 45.94,422.93);

\node[text=drawColor,rotate=-41.50,anchor=base west,inner sep=0pt, outer sep=0pt, scale=  0.60] at ( 50.59,416.07) { 8 };

\path[draw=drawColor,line width= 0.4pt,line join=round,line cap=round] ( 37.69,548.27) --
	( 43.78,555.39);

\path[draw=drawColor,line width= 0.4pt,line join=round,line cap=round] ( 45.94,557.62) --
	( 52.58,563.65) --
	( 54.20,564.97) --
	( 62.46,570.80) --
	( 64.32,571.91);

\path[draw=drawColor,line width= 0.4pt,line join=round,line cap=round] ( 43.78,555.39) -- ( 43.85,555.46);

\node[text=drawColor,rotate= 45.84,anchor=base west,inner sep=0pt, outer sep=0pt, scale=  0.60] at ( 45.34,554.03) { 8 };

\path[draw=drawColor,line width= 0.4pt,line join=round,line cap=round] (189.16,547.13) --
	(186.36,550.70) --
	(182.22,555.39) --
	(178.10,559.52) --
	(173.40,563.65) --
	(169.84,566.46) --
	(161.82,571.91);

\path[draw=drawColor,line width= 0.4pt,line join=round,line cap=round] (192.93,541.59) -- (189.16,547.13);

\node[text=drawColor,rotate=-55.81,anchor=base west,inner sep=0pt, outer sep=0pt, scale=  0.60] at (191.22,540.43) { 8 };

\path[draw=drawColor,line width= 0.4pt,line join=round,line cap=round] (170.97,414.98) --
	(178.10,421.08);

\path[draw=drawColor,line width= 0.4pt,line join=round,line cap=round] (180.32,423.24) --
	(186.36,429.88) --
	(187.68,431.50) --
	(193.51,439.76) --
	(194.62,441.61);

\path[draw=drawColor,line width= 0.4pt,line join=round,line cap=round] (180.25,423.17) -- (180.32,423.24);

\node[text=drawColor,rotate= 44.21,anchor=base west,inner sep=0pt, outer sep=0pt, scale=  0.60] at (179.54,419.60) { 8 };

\path[draw=drawColor,line width= 0.4pt,line join=round,line cap=round] ( 45.96,414.98) --
	( 45.94,414.99) --
	( 37.70,423.24) --
	( 37.69,423.25);

\path[draw=drawColor,line width= 0.4pt,line join=round,line cap=round] ( 43.87,563.65) --
	( 45.94,565.55) --
	( 53.77,571.91);

\path[draw=drawColor,line width= 0.4pt,line join=round,line cap=round] ( 37.69,557.28) -- ( 41.78,561.50);

\node[text=drawColor,rotate= 45.84,anchor=base west,inner sep=0pt, outer sep=0pt, scale=  0.60] at ( 43.27,560.06) { 9 };

\path[draw=drawColor,line width= 0.4pt,line join=round,line cap=round] (194.62,549.47) --
	(189.96,555.39);

\path[draw=drawColor,line width= 0.4pt,line join=round,line cap=round] (186.36,559.42) --
	(182.13,563.65) --
	(178.10,567.25) --
	(172.18,571.91);

\path[draw=drawColor,line width= 0.4pt,line join=round,line cap=round] (189.96,555.39) -- (188.36,557.19);

\node[text=drawColor,rotate=-48.25,anchor=base west,inner sep=0pt, outer sep=0pt, scale=  0.60] at (184.82,558.05) { 9 };

\path[draw=drawColor,line width= 0.4pt,line join=round,line cap=round] (186.36,421.17) --
	(188.26,423.24) --
	(194.62,431.06);

\path[draw=drawColor,line width= 0.4pt,line join=round,line cap=round] (182.14,417.07) -- (186.36,421.17);

\node[text=drawColor,rotate= 44.21,anchor=base west,inner sep=0pt, outer sep=0pt, scale=  0.60] at (181.43,413.50) { 9 };

\path[draw=drawColor,line width= 0.4pt,line join=round,line cap=round] ( 38.02,414.98) --
	( 37.69,415.31);

\path[draw=drawColor,line width= 0.4pt,line join=round,line cap=round] ( 37.69,565.26) --
	( 44.92,571.91);

\path[draw=drawColor,line width= 0.4pt,line join=round,line cap=round] (194.62,558.34) --
	(189.88,563.65) --
	(186.36,567.17) --
	(181.05,571.91);

\path[draw=drawColor,line width= 0.4pt,line join=round,line cap=round] (187.96,414.98) --
	(194.62,422.22);
\definecolor[named]{drawColor}{rgb}{0.60,0.31,0.64}

\path[draw=drawColor,line width= 1.2pt,line join=round,line cap=round] (100.63,495.53) -- ( 80.92,474.49);

\path[draw=drawColor,line width= 1.2pt,line join=round,line cap=round] (126.56,458.28) -- (136.18,460.04);

\path[draw=drawColor,line width= 1.2pt,line join=round,line cap=round] (167.13,435.53) -- (172.14,440.61);

\path[draw=drawColor,line width= 1.2pt,line join=round,line cap=round] (146.92,508.30) -- (129.95,511.75);

\path[draw=drawColor,line width= 1.2pt,line join=round,line cap=round] ( 82.85,497.64) -- (108.20,472.01);

\path[draw=drawColor,line width= 1.2pt,line join=round,line cap=round] (119.30,454.37) -- (126.18,468.66);
\definecolor[named]{drawColor}{rgb}{1.00,0.50,0.00}

\path[draw=drawColor,line width= 1.2pt,line join=round,line cap=round] ( 66.63,553.30) -- ( 64.50,571.91);

\path[draw=drawColor,line width= 1.2pt,line join=round,line cap=round] ( 68.80,566.10) --
	( 64.50,571.91) --
	( 61.62,565.28);

\path[draw=drawColor,line width= 1.2pt,line join=round,line cap=round] (150.67,449.16) -- (146.25,415.69);

\path[draw=drawColor,line width= 1.2pt,line join=round,line cap=round] (143.48,422.37) --
	(146.25,415.69) --
	(150.65,421.42);

\path[draw=drawColor,line width= 1.2pt,line join=round,line cap=round] (158.03,534.70) -- (188.75,564.86);

\path[draw=drawColor,line width= 1.2pt,line join=round,line cap=round] (186.82,557.89) --
	(188.75,564.86) --
	(181.75,563.05);

\path[draw=drawColor,line width= 1.2pt,line join=round,line cap=round] (164.27,504.77) -- (180.41,525.15);

\path[draw=drawColor,line width= 1.2pt,line join=round,line cap=round] (179.36,518.00) --
	(180.41,525.15) --
	(173.69,522.48);

\path[draw=drawColor,line width= 1.2pt,line join=round,line cap=round] (172.75,452.52) -- (171.28,421.91);

\path[draw=drawColor,line width= 1.2pt,line join=round,line cap=round] (167.97,428.33) --
	(171.28,421.91) --
	(175.19,427.98);

\path[draw=drawColor,line width= 1.2pt,line join=round,line cap=round] (153.24,472.60) -- (157.50,455.59);

\path[draw=drawColor,line width= 1.2pt,line join=round,line cap=round] (152.47,460.78) --
	(157.50,455.59) --
	(159.49,462.54);

\path[draw=drawColor,line width= 1.2pt,line join=round,line cap=round] (144.53,517.70) -- (160.45,538.00);

\path[draw=drawColor,line width= 1.2pt,line join=round,line cap=round] (159.43,530.84) --
	(160.45,538.00) --
	(153.75,535.30);

\path[draw=drawColor,line width= 1.2pt,line join=round,line cap=round] (133.04,448.30) -- (115.45,419.41);

\path[draw=drawColor,line width= 1.2pt,line join=round,line cap=round] (115.62,426.64) --
	(115.45,419.41) --
	(121.79,422.88);

\path[draw=drawColor,line width= 1.2pt,line join=round,line cap=round] ( 61.85,506.20) -- ( 39.56,482.39);

\path[draw=drawColor,line width= 1.2pt,line join=round,line cap=round] ( 41.20,489.43) --
	( 39.56,482.39) --
	( 46.47,484.49);

\path[draw=drawColor,line width= 1.2pt,line join=round,line cap=round] ( 71.29,500.46) -- ( 37.69,488.92);

\path[draw=drawColor,line width= 1.2pt,line join=round,line cap=round] ( 42.43,494.37) --
	( 37.69,488.92) --
	( 44.78,487.54);

\path[draw=drawColor,line width= 1.2pt,line join=round,line cap=round] ( 56.31,504.32) -- ( 44.66,523.85);

\path[draw=drawColor,line width= 1.2pt,line join=round,line cap=round] ( 50.97,520.33) --
	( 44.66,523.85) --
	( 44.76,516.62);

\path[draw=drawColor,line width= 1.2pt,line join=round,line cap=round] ( 75.21,532.01) -- ( 55.23,532.41);

\path[draw=drawColor,line width= 1.2pt,line join=round,line cap=round] ( 61.56,535.90) --
	( 55.23,532.41) --
	( 61.41,528.67);
\definecolor[named]{drawColor}{rgb}{0.30,0.69,0.29}

\path[draw=drawColor,line width= 1.2pt,line join=round,line cap=round] (107.80,481.85) -- ( 77.35,457.85);

\path[draw=drawColor,line width= 1.2pt,line join=round,line cap=round] ( 80.03,464.56) --
	( 77.35,457.85) --
	( 84.50,458.89);

\path[draw=drawColor,line width= 1.2pt,line join=round,line cap=round] (100.25,455.57) -- (115.80,434.41);

\path[draw=drawColor,line width= 1.2pt,line join=round,line cap=round] (109.18,437.31) --
	(115.80,434.41) --
	(115.00,441.59);
\definecolor[named]{drawColor}{rgb}{0.00,0.00,0.00}
\definecolor[named]{fillColor}{rgb}{1.00,1.00,1.00}

\path[draw=drawColor,line width= 0.4pt,line join=round,line cap=round,fill=fillColor] ( 14.40,469.44) rectangle ( 63.23,409.44);
\definecolor[named]{drawColor}{rgb}{1.00,0.50,0.00}

\path[draw=drawColor,line width= 1.2pt,line join=round,line cap=round] ( 23.40,445.44) -- ( 41.40,445.44);
\definecolor[named]{drawColor}{rgb}{0.60,0.31,0.64}

\path[draw=drawColor,line width= 1.2pt,line join=round,line cap=round] ( 23.40,433.44) -- ( 41.40,433.44);
\definecolor[named]{drawColor}{rgb}{0.30,0.69,0.29}

\path[draw=drawColor,line width= 1.2pt,line join=round,line cap=round] ( 23.40,421.44) -- ( 41.40,421.44);
\definecolor[named]{drawColor}{rgb}{0.00,0.00,0.00}

\node[text=drawColor,anchor=base,inner sep=0pt, outer sep=0pt, scale=  1.00] at ( 38.82,457.44) {label $y_i$};

\node[text=drawColor,anchor=base west,inner sep=0pt, outer sep=0pt, scale=  1.00] at ( 50.40,442.00) {1};

\node[text=drawColor,anchor=base west,inner sep=0pt, outer sep=0pt, scale=  1.00] at ( 50.40,430.00) {0};

\node[text=drawColor,anchor=base west,inner sep=0pt, outer sep=0pt, scale=  1.00] at ( 50.40,418.00) {-1};

\path[draw=drawColor,line width= 0.4pt,line join=round,line cap=round] ( 66.63,553.30) circle (  2.25);

\path[draw=drawColor,line width= 0.4pt,line join=round,line cap=round] (146.92,508.30) circle (  2.25);

\path[draw=drawColor,line width= 0.4pt,line join=round,line cap=round] (115.80,434.41) circle (  2.25);

\path[draw=drawColor,line width= 0.4pt,line join=round,line cap=round] ( 64.50,571.91) circle (  2.25);

\path[draw=drawColor,line width= 0.4pt,line join=round,line cap=round] (129.95,511.75) circle (  2.25);

\path[draw=drawColor,line width= 0.4pt,line join=round,line cap=round] (100.25,455.57) circle (  2.25);

\node[text=drawColor,anchor=base,inner sep=0pt, outer sep=0pt, scale=  1.00] at (107.39,427.70) {$\mathbf x_{20}$};

\node[text=drawColor,anchor=base,inner sep=0pt, outer sep=0pt, scale=  1.00] at ( 91.84,448.87) {$\mathbf x_{20}'$};

\node[text=drawColor,anchor=base,inner sep=0pt, outer sep=0pt, scale=  1.00] at ( 75.04,550.80) {$\mathbf x_{1}$};

\node[text=drawColor,anchor=base,inner sep=0pt, outer sep=0pt, scale=  1.00] at ( 72.90,569.41) {$\mathbf x_{1}'$};

\node[text=drawColor,anchor=base,inner sep=0pt, outer sep=0pt, scale=  1.00] at (146.92,497.40) {$\mathbf x_{11}$};

\node[text=drawColor,anchor=base,inner sep=0pt, outer sep=0pt, scale=  1.00] at (129.95,500.85) {$\mathbf x_{11}'$};
\end{scope}
\begin{scope}
\path[clip] (  0.00,  0.00) rectangle (224.04,578.16);
\definecolor[named]{drawColor}{rgb}{0.00,0.00,0.00}

\path[draw=drawColor,line width= 0.4pt,line join=round,line cap=round] ( 14.40,216.72) --
	(212.04,216.72) --
	(212.04,385.44) --
	( 14.40,385.44) --
	( 14.40,216.72);
\end{scope}
\begin{scope}
\path[clip] (  0.00,192.72) rectangle (224.04,385.44);
\definecolor[named]{drawColor}{rgb}{0.00,0.00,0.00}

\node[text=drawColor,anchor=base,inner sep=0pt, outer sep=0pt, scale=  1.00] at (113.22,205.92) {additional feature $ x_{i,1}^2$};

\node[text=drawColor,rotate= 90.00,anchor=base,inner sep=0pt, outer sep=0pt, scale=  1.00] at ( 10.80,301.08) {additional feature $ x_{i,2}^2$};
\end{scope}
\begin{scope}
\path[clip] (  0.00,  0.00) rectangle (224.04,578.16);
\definecolor[named]{drawColor}{rgb}{0.00,0.00,0.00}

\node[text=drawColor,rotate= 90.00,anchor=base,inner sep=0pt, outer sep=0pt, scale=  1.00] at (221.64,301.08) {Enlarged feature space $\Phi(\mathbf x)$};
\end{scope}
\begin{scope}
\path[clip] ( 14.40,216.72) rectangle (212.04,385.44);
\definecolor[named]{drawColor}{rgb}{0.50,0.50,0.50}

\node[text=drawColor,rotate=-45.00,anchor=base,inner sep=0pt, outer sep=0pt, scale=  1.00] at (176.71,260.95) {$r(\mathbf x)=\mathbf w^\intercal \Phi(\mathbf x)$};

\path[draw=drawColor,line width= 0.4pt,line join=round,line cap=round] ( 72.32,222.97) --
	( 70.56,224.74);

\path[draw=drawColor,line width= 0.4pt,line join=round,line cap=round] ( 64.10,231.19) --
	( 62.33,232.96) --
	( 55.88,239.41) --
	( 54.11,241.18) --
	( 47.66,247.64) --
	( 45.89,249.40);

\path[draw=drawColor,line width= 0.4pt,line join=round,line cap=round] ( 68.43,226.86) -- ( 64.10,231.19);

\node[text=drawColor,rotate=-45.02,anchor=base west,inner sep=0pt, outer sep=0pt, scale=  0.60] at ( 66.97,225.40) { 1 };

\path[draw=drawColor,line width= 0.4pt,line join=round,line cap=round] ( 95.22,226.60) --
	( 90.63,231.19) --
	( 87.00,234.82) --
	( 82.41,239.41) --
	( 78.78,243.04) --
	( 74.18,247.64) --
	( 70.56,251.26) --
	( 65.96,255.86) --
	( 62.33,259.49) --
	( 57.74,264.08) --
	( 54.11,267.71) --
	( 49.52,272.30) --
	( 45.89,275.93);

\path[draw=drawColor,line width= 0.4pt,line join=round,line cap=round] ( 96.73,225.09) -- ( 95.22,226.60);

\node[text=drawColor,rotate=-45.02,anchor=base west,inner sep=0pt, outer sep=0pt, scale=  0.60] at ( 95.27,223.63) { 2 };

\path[draw=drawColor,line width= 0.4pt,line join=round,line cap=round] (119.89,228.46) --
	(117.15,231.19) --
	(111.67,236.68) --
	(108.93,239.41) --
	(103.44,244.90) --
	(100.71,247.64) --
	( 95.22,253.12) --
	( 92.49,255.86) --
	( 87.00,261.35) --
	( 84.27,264.08) --
	( 78.78,269.57) --
	( 76.04,272.30) --
	( 70.56,277.79) --
	( 67.82,280.52) --
	( 62.33,286.01) --
	( 59.60,288.75) --
	( 54.11,294.23) --
	( 51.38,296.97) --
	( 45.89,302.46);

\path[draw=drawColor,line width= 0.4pt,line join=round,line cap=round] (123.26,225.09) -- (119.89,228.46);

\node[text=drawColor,rotate=-45.02,anchor=base west,inner sep=0pt, outer sep=0pt, scale=  0.60] at (121.79,223.63) { 3 };

\path[draw=drawColor,line width= 0.4pt,line join=round,line cap=round] (144.56,230.32) --
	(143.68,231.19) --
	(136.33,238.54) --
	(135.46,239.41) --
	(128.11,246.76) --
	(127.24,247.64) --
	(119.89,254.98) --
	(119.01,255.86) --
	(111.67,263.20) --
	(110.79,264.08) --
	(103.44,271.43) --
	(102.57,272.30) --
	( 95.22,279.65) --
	( 94.35,280.52) --
	( 87.00,287.87) --
	( 86.12,288.75) --
	( 78.78,296.09) --
	( 77.90,296.97) --
	( 70.56,304.32) --
	( 69.68,305.19) --
	( 62.33,312.54) --
	( 61.46,313.41) --
	( 54.11,320.76) --
	( 53.24,321.64) --
	( 45.89,328.98);

\path[draw=drawColor,line width= 0.4pt,line join=round,line cap=round] (149.78,225.09) -- (144.56,230.32);

\node[text=drawColor,rotate=-45.02,anchor=base west,inner sep=0pt, outer sep=0pt, scale=  0.60] at (148.32,223.63) { 4 };

\path[draw=drawColor,line width= 0.4pt,line join=round,line cap=round] (178.43,222.97) --
	(177.44,223.95);

\path[draw=drawColor,line width= 0.4pt,line join=round,line cap=round] (170.21,231.19) --
	(169.22,232.18) --
	(161.98,239.41) --
	(161.00,240.40) --
	(153.76,247.64) --
	(152.78,248.62) --
	(145.54,255.86) --
	(144.56,256.84) --
	(137.32,264.08) --
	(136.33,265.06) --
	(129.10,272.30) --
	(128.11,273.29) --
	(120.87,280.52) --
	(119.89,281.51) --
	(112.65,288.75) --
	(111.67,289.73) --
	(104.43,296.97) --
	(103.44,297.95) --
	( 96.21,305.19) --
	( 95.22,306.18) --
	( 87.98,313.41) --
	( 87.00,314.40) --
	( 79.76,321.64) --
	( 78.78,322.62) --
	( 71.54,329.86) --
	( 70.56,330.84) --
	( 63.32,338.08) --
	( 62.33,339.06) --
	( 55.10,346.30) --
	( 54.11,347.29) --
	( 46.87,354.52) --
	( 45.89,355.51);

\path[draw=drawColor,line width= 0.4pt,line join=round,line cap=round] (175.32,226.07) -- (170.21,231.19);

\node[text=drawColor,rotate=-45.02,anchor=base west,inner sep=0pt, outer sep=0pt, scale=  0.60] at (173.86,224.61) { 5 };

\path[draw=drawColor,line width= 0.4pt,line join=round,line cap=round] (196.73,231.19) --
	(193.89,234.03) --
	(188.51,239.41) --
	(185.67,242.26) --
	(180.29,247.64) --
	(177.44,250.48) --
	(172.07,255.86) --
	(169.22,258.70) --
	(163.84,264.08) --
	(161.00,266.92) --
	(155.62,272.30) --
	(152.78,275.15) --
	(147.40,280.52) --
	(144.56,283.37) --
	(139.18,288.75) --
	(136.33,291.59) --
	(130.95,296.97) --
	(128.11,299.81) --
	(122.73,305.19) --
	(119.89,308.03) --
	(114.51,313.41) --
	(111.67,316.26) --
	(106.29,321.64) --
	(103.44,324.48) --
	( 98.07,329.86) --
	( 95.22,332.70) --
	( 89.84,338.08) --
	( 87.00,340.92) --
	( 81.62,346.30) --
	( 78.78,349.15) --
	( 73.40,354.52) --
	( 70.56,357.37) --
	( 65.18,362.75) --
	( 62.33,365.59) --
	( 56.95,370.97) --
	( 54.11,373.81) --
	( 48.73,379.19);

\path[draw=drawColor,line width= 0.4pt,line join=round,line cap=round] (199.99,227.93) -- (196.73,231.19);

\node[text=drawColor,rotate=-45.02,anchor=base west,inner sep=0pt, outer sep=0pt, scale=  0.60] at (198.53,226.47) { 6 };

\path[draw=drawColor,line width= 0.4pt,line join=round,line cap=round] (198.59,255.86) --
	(193.89,260.56) --
	(190.37,264.08) --
	(185.67,268.78) --
	(182.15,272.30) --
	(177.44,277.01) --
	(173.93,280.52) --
	(169.22,285.23) --
	(165.70,288.75) --
	(161.00,293.45) --
	(157.48,296.97) --
	(152.78,301.67) --
	(149.26,305.19) --
	(144.56,309.89) --
	(141.04,313.41) --
	(136.33,318.12) --
	(132.81,321.64) --
	(128.11,326.34) --
	(124.59,329.86) --
	(119.89,334.56) --
	(116.37,338.08) --
	(111.67,342.78) --
	(108.15,346.30) --
	(103.44,351.01) --
	( 99.93,354.52) --
	( 95.22,359.23) --
	( 91.70,362.75) --
	( 87.00,367.45) --
	( 83.48,370.97) --
	( 78.78,375.67) --
	( 75.26,379.19);

\path[draw=drawColor,line width= 0.4pt,line join=round,line cap=round] (199.99,254.46) -- (198.59,255.86);

\node[text=drawColor,rotate=-45.02,anchor=base west,inner sep=0pt, outer sep=0pt, scale=  0.60] at (198.53,253.00) { 7 };

\path[draw=drawColor,line width= 0.4pt,line join=round,line cap=round] (202.11,278.87) --
	(200.45,280.52);

\path[draw=drawColor,line width= 0.4pt,line join=round,line cap=round] (193.89,287.09) --
	(192.23,288.75) --
	(185.67,295.31) --
	(184.01,296.97) --
	(177.44,303.53) --
	(175.78,305.19) --
	(169.22,311.75) --
	(167.56,313.41) --
	(161.00,319.98) --
	(159.34,321.64) --
	(152.78,328.20) --
	(151.12,329.86) --
	(144.56,336.42) --
	(142.90,338.08) --
	(136.33,344.64) --
	(134.67,346.30) --
	(128.11,352.87) --
	(126.45,354.52) --
	(119.89,361.09) --
	(118.23,362.75) --
	(111.67,369.31) --
	(110.01,370.97) --
	(103.44,377.53) --
	(101.78,379.19);

\path[draw=drawColor,line width= 0.4pt,line join=round,line cap=round] (198.33,282.65) -- (193.89,287.09);

\node[text=drawColor,rotate=-45.02,anchor=base west,inner sep=0pt, outer sep=0pt, scale=  0.60] at (196.87,281.18) { 8 };

\path[draw=drawColor,line width= 0.4pt,line join=round,line cap=round] (194.09,313.41) --
	(193.89,313.61) --
	(185.87,321.64) --
	(185.67,321.84) --
	(177.64,329.86) --
	(177.44,330.06) --
	(169.42,338.08) --
	(169.22,338.28) --
	(161.20,346.30) --
	(161.00,346.50) --
	(152.98,354.52) --
	(152.78,354.72) --
	(144.76,362.75) --
	(144.56,362.95) --
	(136.53,370.97) --
	(136.33,371.17) --
	(128.31,379.19);

\path[draw=drawColor,line width= 0.4pt,line join=round,line cap=round] (199.99,307.51) -- (194.09,313.41);

\node[text=drawColor,rotate=-45.02,anchor=base west,inner sep=0pt, outer sep=0pt, scale=  0.60] at (198.53,306.05) { 9 };

\path[draw=drawColor,line width= 0.4pt,line join=round,line cap=round] (195.95,338.08) --
	(193.89,340.14) --
	(187.73,346.30) --
	(185.67,348.36) --
	(179.50,354.52) --
	(177.44,356.58) --
	(171.28,362.75) --
	(169.22,364.81) --
	(163.06,370.97) --
	(161.00,373.03) --
	(154.84,379.19);

\path[draw=drawColor,line width= 0.4pt,line join=round,line cap=round] (197.87,336.16) -- (195.95,338.08);

\node[text=drawColor,rotate=-45.02,anchor=base west,inner sep=0pt, outer sep=0pt, scale=  0.60] at (196.41,334.70) { 10 };

\path[draw=drawColor,line width= 0.4pt,line join=round,line cap=round] (197.81,362.75) --
	(193.89,366.67) --
	(189.59,370.97) --
	(185.67,374.89) --
	(181.36,379.19);

\path[draw=drawColor,line width= 0.4pt,line join=round,line cap=round] (197.87,362.69) -- (197.81,362.75);

\node[text=drawColor,rotate=-45.02,anchor=base west,inner sep=0pt, outer sep=0pt, scale=  0.60] at (196.41,361.22) { 11 };
\definecolor[named]{drawColor}{rgb}{0.60,0.31,0.64}

\path[draw=drawColor,line width= 1.2pt,line join=round,line cap=round] ( 49.44,223.56) -- ( 70.02,228.80);

\path[draw=drawColor,line width= 1.2pt,line join=round,line cap=round] ( 50.15,247.00) -- ( 58.44,244.42);

\path[draw=drawColor,line width= 1.2pt,line join=round,line cap=round] (114.57,293.35) -- (127.88,280.91);

\path[draw=drawColor,line width= 1.2pt,line join=round,line cap=round] ( 72.81,230.52) -- ( 52.57,233.71);

\path[draw=drawColor,line width= 1.2pt,line join=round,line cap=round] ( 67.20,224.19) -- ( 46.39,230.78);

\path[draw=drawColor,line width= 1.2pt,line join=round,line cap=round] ( 46.77,253.24) -- ( 49.91,233.92);
\definecolor[named]{drawColor}{rgb}{1.00,0.50,0.00}

\path[draw=drawColor,line width= 1.2pt,line join=round,line cap=round] ( 96.34,316.03) -- (101.10,379.19);

\path[draw=drawColor,line width= 1.2pt,line join=round,line cap=round] (104.23,372.68) --
	(101.10,379.19) --
	( 97.02,373.22);

\path[draw=drawColor,line width= 1.2pt,line join=round,line cap=round] ( 79.11,262.68) -- ( 71.75,353.62);

\path[draw=drawColor,line width= 1.2pt,line join=round,line cap=round] ( 75.86,347.67) --
	( 71.75,353.62) --
	( 68.66,347.09);

\path[draw=drawColor,line width= 1.2pt,line join=round,line cap=round] ( 93.39,269.17) -- (180.45,353.34);

\path[draw=drawColor,line width= 1.2pt,line join=round,line cap=round] (178.47,346.39) --
	(180.45,353.34) --
	(173.44,351.59);

\path[draw=drawColor,line width= 1.2pt,line join=round,line cap=round] (107.50,227.83) -- (152.43,251.41);

\path[draw=drawColor,line width= 1.2pt,line join=round,line cap=round] (148.57,245.30) --
	(152.43,251.41) --
	(145.21,251.70);

\path[draw=drawColor,line width= 1.2pt,line join=round,line cap=round] (129.59,256.44) -- (125.50,332.74);

\path[draw=drawColor,line width= 1.2pt,line join=round,line cap=round] (129.44,326.69) --
	(125.50,332.74) --
	(122.22,326.30);

\path[draw=drawColor,line width= 1.2pt,line join=round,line cap=round] ( 83.82,230.29) -- ( 92.28,251.22);

\path[draw=drawColor,line width= 1.2pt,line join=round,line cap=round] ( 93.28,244.07) --
	( 92.28,251.22) --
	( 86.58,246.77);

\path[draw=drawColor,line width= 1.2pt,line join=round,line cap=round] ( 69.15,240.53) -- ( 98.65,276.29);

\path[draw=drawColor,line width= 1.2pt,line join=round,line cap=round] ( 97.46,269.16) --
	( 98.65,276.29) --
	( 91.88,273.76);

\path[draw=drawColor,line width= 1.2pt,line join=round,line cap=round] ( 55.25,264.35) -- ( 45.98,340.91);

\path[draw=drawColor,line width= 1.2pt,line join=round,line cap=round] ( 50.32,335.13) --
	( 45.98,340.91) --
	( 43.15,334.26);

\path[draw=drawColor,line width= 1.2pt,line join=round,line cap=round] (107.30,228.85) -- (172.50,224.40);

\path[draw=drawColor,line width= 1.2pt,line join=round,line cap=round] (166.01,221.22) --
	(172.50,224.40) --
	(166.50,228.43);

\path[draw=drawColor,line width= 1.2pt,line join=round,line cap=round] ( 86.71,225.34) -- (179.03,222.97);

\path[draw=drawColor,line width= 1.2pt,line join=round,line cap=round] (172.68,219.52) --
	(179.03,222.97) --
	(172.87,226.74);

\path[draw=drawColor,line width= 1.2pt,line join=round,line cap=round] (121.33,227.53) -- (155.51,249.32);

\path[draw=drawColor,line width= 1.2pt,line join=round,line cap=round] (152.17,242.91) --
	(155.51,249.32) --
	(148.29,249.00);

\path[draw=drawColor,line width= 1.2pt,line join=round,line cap=round] ( 79.38,263.72) -- (124.22,264.52);

\path[draw=drawColor,line width= 1.2pt,line join=round,line cap=round] (118.03,260.80) --
	(124.22,264.52) --
	(117.90,268.02);
\definecolor[named]{drawColor}{rgb}{0.30,0.69,0.29}

\path[draw=drawColor,line width= 1.2pt,line join=round,line cap=round] ( 46.48,224.61) -- ( 75.70,247.65);

\path[draw=drawColor,line width= 1.2pt,line join=round,line cap=round] ( 73.03,240.94) --
	( 75.70,247.65) --
	( 68.55,246.61);

\path[draw=drawColor,line width= 1.2pt,line join=round,line cap=round] ( 49.67,251.26) -- ( 46.02,296.28);

\path[draw=drawColor,line width= 1.2pt,line join=round,line cap=round] ( 50.13,290.33) --
	( 46.02,296.28) --
	( 42.93,289.75);
\definecolor[named]{drawColor}{rgb}{0.00,0.00,0.00}

\path[draw=drawColor,line width= 0.4pt,line join=round,line cap=round] ( 96.34,316.03) circle (  2.25);

\path[draw=drawColor,line width= 0.4pt,line join=round,line cap=round] ( 72.81,230.52) circle (  2.25);

\path[draw=drawColor,line width= 0.4pt,line join=round,line cap=round] ( 46.02,296.28) circle (  2.25);

\path[draw=drawColor,line width= 0.4pt,line join=round,line cap=round] (101.10,379.19) circle (  2.25);

\path[draw=drawColor,line width= 0.4pt,line join=round,line cap=round] ( 52.57,233.71) circle (  2.25);

\path[draw=drawColor,line width= 0.4pt,line join=round,line cap=round] ( 49.67,251.26) circle (  2.25);

\node[text=drawColor,anchor=base,inner sep=0pt, outer sep=0pt, scale=  0.80] at ( 30.11,300.91) {$\Phi(\mathbf x_{20})$};

\node[text=drawColor,anchor=base,inner sep=0pt, outer sep=0pt, scale=  0.80] at ( 33.75,255.89) {$\Phi(\mathbf x_{20}')$};

\node[text=drawColor,anchor=base,inner sep=0pt, outer sep=0pt, scale=  0.80] at (112.26,314.03) {$\Phi(\mathbf x_{1})$};

\node[text=drawColor,anchor=base,inner sep=0pt, outer sep=0pt, scale=  0.80] at (117.01,377.19) {$\Phi(\mathbf x_{1}')$};

\node[text=drawColor,anchor=base,inner sep=0pt, outer sep=0pt, scale=  0.80] at ( 32.58,234.19) {$\Phi(\mathbf x_{11}')$};

\node[text=drawColor,anchor=base,inner sep=0pt, outer sep=0pt, scale=  0.80] at ( 77.67,219.60) {$\Phi(\mathbf x_{11})$};
\end{scope}
\begin{scope}
\path[clip] ( 14.40, 24.00) rectangle (212.04,192.72);
\definecolor[named]{drawColor}{rgb}{1.00,0.50,0.00}
\definecolor[named]{fillColor}{rgb}{1.00,0.50,0.00}

\path[draw=drawColor,line width= 0.4pt,line join=round,line cap=round,fill=fillColor] ( 82.42,154.55) circle (  1.50);
\definecolor[named]{drawColor}{rgb}{0.60,0.31,0.64}
\definecolor[named]{fillColor}{rgb}{0.60,0.31,0.64}

\path[draw=drawColor,line width= 0.4pt,line join=round,line cap=round,fill=fillColor] (100.60, 88.00) circle (  1.50);
\definecolor[named]{drawColor}{rgb}{1.00,0.50,0.00}
\definecolor[named]{fillColor}{rgb}{1.00,0.50,0.00}

\path[draw=drawColor,line width= 0.4pt,line join=round,line cap=round,fill=fillColor] ( 68.51,186.47) circle (  1.50);

\path[draw=drawColor,line width= 0.4pt,line join=round,line cap=round,fill=fillColor] (177.01,178.70) circle (  1.50);
\definecolor[named]{drawColor}{rgb}{0.60,0.31,0.64}
\definecolor[named]{fillColor}{rgb}{0.60,0.31,0.64}

\path[draw=drawColor,line width= 0.4pt,line join=round,line cap=round,fill=fillColor] ( 86.48, 79.02) circle (  1.50);
\definecolor[named]{drawColor}{rgb}{1.00,0.50,0.00}
\definecolor[named]{fillColor}{rgb}{1.00,0.50,0.00}

\path[draw=drawColor,line width= 0.4pt,line join=round,line cap=round,fill=fillColor] (128.58,109.07) circle (  1.50);
\definecolor[named]{drawColor}{rgb}{0.60,0.31,0.64}
\definecolor[named]{fillColor}{rgb}{0.60,0.31,0.64}

\path[draw=drawColor,line width= 0.4pt,line join=round,line cap=round,fill=fillColor] ( 92.26, 67.68) circle (  1.50);
\definecolor[named]{drawColor}{rgb}{1.00,0.50,0.00}
\definecolor[named]{fillColor}{rgb}{1.00,0.50,0.00}

\path[draw=drawColor,line width= 0.4pt,line join=round,line cap=round,fill=fillColor] ( 72.25,169.66) circle (  1.50);

\path[draw=drawColor,line width= 0.4pt,line join=round,line cap=round,fill=fillColor] ( 86.69,106.04) circle (  1.50);

\path[draw=drawColor,line width= 0.4pt,line join=round,line cap=round,fill=fillColor] (110.86,123.07) circle (  1.50);
\definecolor[named]{drawColor}{rgb}{0.60,0.31,0.64}
\definecolor[named]{fillColor}{rgb}{0.60,0.31,0.64}

\path[draw=drawColor,line width= 0.4pt,line join=round,line cap=round,fill=fillColor] ( 53.70, 85.65) circle (  1.50);
\definecolor[named]{drawColor}{rgb}{0.30,0.69,0.29}
\definecolor[named]{fillColor}{rgb}{0.30,0.69,0.29}

\path[draw=drawColor,line width= 0.4pt,line join=round,line cap=round,fill=fillColor] ( 43.39, 55.50) circle (  1.50);
\definecolor[named]{drawColor}{rgb}{0.60,0.31,0.64}
\definecolor[named]{fillColor}{rgb}{0.60,0.31,0.64}

\path[draw=drawColor,line width= 0.4pt,line join=round,line cap=round,fill=fillColor] ( 53.05, 89.56) circle (  1.50);

\path[draw=drawColor,line width= 0.4pt,line join=round,line cap=round,fill=fillColor] ( 80.57, 59.79) circle (  1.50);
\definecolor[named]{drawColor}{rgb}{1.00,0.50,0.00}
\definecolor[named]{fillColor}{rgb}{1.00,0.50,0.00}

\path[draw=drawColor,line width= 0.4pt,line join=round,line cap=round,fill=fillColor] ( 66.31,169.94) circle (  1.50);

\path[draw=drawColor,line width= 0.4pt,line join=round,line cap=round,fill=fillColor] (151.88, 76.87) circle (  1.50);

\path[draw=drawColor,line width= 0.4pt,line join=round,line cap=round,fill=fillColor] (183.05, 79.25) circle (  1.50);

\path[draw=drawColor,line width= 0.4pt,line join=round,line cap=round,fill=fillColor] (116.24,107.02) circle (  1.50);

\path[draw=drawColor,line width= 0.4pt,line join=round,line cap=round,fill=fillColor] (128.48, 82.90) circle (  1.50);
\definecolor[named]{drawColor}{rgb}{0.30,0.69,0.29}
\definecolor[named]{fillColor}{rgb}{0.30,0.69,0.29}

\path[draw=drawColor,line width= 0.4pt,line join=round,line cap=round,fill=fillColor] ( 81.15, 30.25) circle (  1.50);
\end{scope}
\begin{scope}
\path[clip] (  0.00,  0.00) rectangle (224.04,578.16);
\definecolor[named]{drawColor}{rgb}{0.00,0.00,0.00}

\path[draw=drawColor,line width= 0.4pt,line join=round,line cap=round] ( 14.40, 24.00) --
	(212.04, 24.00) --
	(212.04,192.72) --
	( 14.40,192.72) --
	( 14.40, 24.00);
\end{scope}
\begin{scope}
\path[clip] (  0.00,  0.00) rectangle (224.04,192.72);
\definecolor[named]{drawColor}{rgb}{0.00,0.00,0.00}

\node[text=drawColor,anchor=base,inner sep=0pt, outer sep=0pt, scale=  1.00] at (113.22, 13.20) {difference feature ${x'}_{i,1}^2-x_{i,1}^2$};

\node[text=drawColor,rotate= 90.00,anchor=base,inner sep=0pt, outer sep=0pt, scale=  1.00] at ( 10.80,108.36) {difference feature ${x'}_{i,2}^2-x_{i,2}^2$};
\end{scope}
\begin{scope}
\path[clip] (  0.00,  0.00) rectangle (224.04,578.16);
\definecolor[named]{drawColor}{rgb}{0.00,0.00,0.00}

\node[text=drawColor,rotate= 90.00,anchor=base,inner sep=0pt, outer sep=0pt, scale=  1.00] at (221.64,108.36) {Difference of enlarged features $\Phi(\mathbf x')-\Phi(\mathbf x)$};
\end{scope}
\begin{scope}
\path[clip] ( 14.40, 24.00) rectangle (212.04,192.72);
\definecolor[named]{drawColor}{rgb}{0.75,0.75,0.75}

\path[draw=drawColor,line width= 0.4pt,line join=round,line cap=round] ( 14.40, 81.98) -- (212.04, 81.98);

\path[draw=drawColor,line width= 0.4pt,line join=round,line cap=round] ( 76.96, 24.00) -- ( 76.96,192.72);
\definecolor[named]{drawColor}{rgb}{0.00,0.00,0.00}

\path[draw=drawColor,line width= 0.4pt,line join=round,line cap=round] ( 14.40,175.02) -- (189.42,  0.00);

\path[draw=drawColor,line width= 0.4pt,line join=round,line cap=round] ( 14.40,114.06) -- (128.46,  0.00);

\path[draw=drawColor,line width= 0.4pt,line join=round,line cap=round] ( 82.42,154.55) circle (  2.25);

\path[draw=drawColor,line width= 0.4pt,line join=round,line cap=round] ( 53.70, 85.65) circle (  2.25);

\path[draw=drawColor,line width= 0.4pt,line join=round,line cap=round] ( 81.15, 30.25) circle (  2.25);

\node[text=drawColor,anchor=base,inner sep=0pt, outer sep=0pt, scale=  0.80] at ( 52.58, 37.31) {$\Phi(\mathbf x_{20}')-\Phi(\mathbf x_{20})$};
\definecolor[named]{drawColor}{rgb}{0.60,0.31,0.64}

\node[text=drawColor,anchor=base,inner sep=0pt, outer sep=0pt, scale=  1.00] at (107.44, 55.01) {$y_i=0$};
\definecolor[named]{drawColor}{rgb}{1.00,0.50,0.00}

\node[text=drawColor,anchor=base,inner sep=0pt, outer sep=0pt, scale=  1.00] at (137.92,125.11) {$y_i=1$};
\definecolor[named]{drawColor}{rgb}{0.30,0.69,0.29}

\node[text=drawColor,anchor=base,inner sep=0pt, outer sep=0pt, scale=  1.00] at ( 37.34, 64.15) {$y_i=-1$};
\definecolor[named]{drawColor}{rgb}{0.00,0.00,0.00}

\node[text=drawColor,anchor=base,inner sep=0pt, outer sep=0pt, scale=  0.80] at ( 52.58,104.36) {$\Phi(\mathbf x_{11}')-\Phi(\mathbf x_{11})$};

\node[text=drawColor,anchor=base,inner sep=0pt, outer sep=0pt, scale=  0.80] at (113.54,156.18) {$\Phi(\mathbf x_{1}')-\Phi(\mathbf x_{1})$};

\node[text=drawColor,rotate=-45.00,anchor=base,inner sep=0pt, outer sep=0pt, scale=  0.70] at ( 81.82, 50.26) {$\mathbf w^\intercal [ \Phi(\mathbf x') - \Phi(\mathbf x) ] = -1$};

\node[text=drawColor,rotate=-45.00,anchor=base,inner sep=0pt, outer sep=0pt, scale=  0.70] at (142.78, 50.26) {$\mathbf w^\intercal [ \Phi(\mathbf x') - \Phi(\mathbf x) ] = 1$};
\end{scope}
\end{tikzpicture}

  \vskip -0.5cm
  \caption{Geometric interpretation. \textbf{Top}: input feature pairs
    $\mathbf x_i,\mathbf x_i'\in\RR^p$ are segments for $y_i=0$ and
    arrows for $y_i\in\{-1,1\}$. The level curves of the ranking
    function $r(\mathbf x)=||\mathbf x||_2^2$ are grey, and
    differences $|r(\mathbf x)-r(\mathbf x')|\leq 1$ are considered
    insignificant ($y_i=0$). \textbf{Middle}: in the enlarged feature
    space, the ranking function is linear: $r(\mathbf x)=\mathbf
    w^\intercal \Phi(\mathbf x)$. \textbf{Bottom}: two symmetric
    hyperplanes $\mathbf w^\intercal[\Phi(\mathbf x_i')-\Phi(\mathbf
    x_i)]\in\{-1,1\}$ are used to classify the difference vectors.}
  \label{fig:norm-data}
\end{figure}

\subsection{Compare, reject, and rank}

Comparison is similar to classification with a reject option and
ranking (Table~\ref{tab:related}). The defining features of the
comparison problem that we study in this paper are that inputs are
pairs $\mathbf x,\mathbf x'\in\RR^p$, and outputs $y\in\{-1,0,1\}$
include the $y=0$ equality pairs or ties.  The statistics literature
contains many probabilistic models for paired comparison experiments,
some of which directly model ties \citep{davidson-ties}. Such models
are concerned with accurately ranking a finite number of inputs
$x\in\{1,\dots,t\}$, so are not directly applicable to the real-valued
inputs $\mathbf x\in\RR^p$ we consider in this paper. Another related
problem is classification with rejection, a version of
binary classification with outputs $y\in\{-1,0,1\}$, where 0 signifies
``rejection'' or ``no guess'' \citep{reject-option}.

The supervised learning to rank problem has been extensively studied
in the machine learning literature \citep{learning-to-rank,
  object-ranking-methods}, and is similar to the supervised comparison
problem we consider in this paper. A Bayesian model which can be
applied to learning to rank is TrueSkill \citep{trueskill}, a
generalization of the Elo chess rating system. The SVMrank algorithm
was proposed for learning to rank \citep{ranksvm}, and the
large-margin learning formulation we propose in this article is
similar. The difference is that we also consider the case where both
items/documents are judged to be equally good ($y_i=0$).  A boosting
algorithm for this ``ranking with ties'' problem was proposed by
\citet{rank-with-ties}, who observed that modeling ties is
more effective when there are more output values. Indeed, ranking data
sets are often described not in terms of labeled pairs of inputs
$(\mathbf x_i, \mathbf x_i', y_i)$ but instead single inputs $\mathbf
x_i$ with ordinal labels $y_i\in\{1,\dots,k\}$, where $k$ is the
number of integral output values. In that case, one can use Support
Vector Ordinal Regression \citep{ordinal}, which has a large-margin
learning formulation similar to the one we propose in this paper.

% When the inputs are discrete $x_i,x_i'\in\{1,\dots,k\}$, then the
% problem is known as learning a relation \citep{relations}. In this
% article we consider the case when inputs are real-valued vectors

\subsection{SVMrank for comparing}
\label{sec:svmrank}
In this section we explain how to apply the existing SVMrank algorithm
to a comparison data set.  The goal of SVMrank is to learn a ranking
function $r:\RR^p \rightarrow \RR$. When $r(\mathbf x)=\mathbf
w^\intercal \mathbf x$ is linear, the primal problem for some cost
parameter $C\in\RR^+$ is the following quadratic program (QP):
\begin{equation}
  \begin{aligned}
    \minimize_{\mathbf w, \mathbf \xi}\ \ & \frac 1 2
    \mathbf w^\intercal \mathbf w
    + C \sum_{i\in \mathcal I_1\cup \mathcal I_{-1}} \xi_i \\
    \text{subject to}\ \ &
    \forall i\in \mathcal I_1\cup \mathcal I_{-1},\ \xi_i \geq 0,\\
    & \text{and }\xi_i \geq 1-\mathbf w^\intercal(\mathbf x_i'-\mathbf
    x_i)y_i,
  \end{aligned}
  \label{eq:svmrank}
\end{equation}
where $\mathcal I_y=\{i\mid y_i=y\}$ are the sets of indices for the different
labels. Note that (\ref{eq:svmrank}) is the same as Optimization
Problem 1 (Ranking SVM), in the paper of \citet{ranksvm}. Note also
that the equality $y_i=0$ pairs are not used in the optimization
problem.

After obtaining a weight vector $\mathbf w\in\RR^p$ by solving SVMrank
(\ref{eq:svmrank}), we get a ranking function $r(\mathbf x)\in\RR$,
but we are not yet able to predict equality $y_i=0$ pairs. To do so,
we extend SVMrank by defining a threshold $\tau\in\RR^+$ and a
thresholding function $t_\tau:\RR\rightarrow\{-1,0,1\}$
\begin{equation}
  \label{eq:threshold}
  t_\tau(x) = 
  \begin{cases}
    -1 & \text{ if } x < -\tau, \\
    0 & \text{ if } |x| \leq \tau, \\
    1 & \text{ if } x > \tau. \\
  \end{cases}
\end{equation}
A comparison function $c_\tau:\RR^p\times \RR^p\rightarrow \{-1, 0,
1\}$ is defined as the thresholded difference of predicted ranks
\begin{equation}
  \label{eq:compare_general}
  c_\tau(\mathbf x, \mathbf x') = 
  t_\tau\big(
  r(\mathbf x') - r(\mathbf x)
  \big).
\end{equation}
We can then use grid search to estimate an optimal threshold $\hat
\tau$, by minimizing the zero-one loss with respect to all the
training pairs:
\begin{equation}
  \hat \tau = \argmin_{\tau}
  \sum_{i=1}^n
  I\left[ c_\tau(\mathbf x_i, \mathbf x_i') \neq y_i \right].
\end{equation}
However, there are two potential problems with the learned comparison
function $c_{\hat\tau}$. First, the equality pairs $i\in \mathcal I_0$ are not
used to learn the weight vector $\mathbf w$ in (\ref{eq:svmrank}). Second, the
threshold $\hat \tau$ is learned in a separate optimization step,
which may be suboptimal. In the next section, we propose a new
algorithm that fixes these issues by directly using all the
training pairs in a single learning problem.
%Supplementary File for Online Publication
\section{Support vector comparison machines}
\label{sec:svm-compare}

\begin{figure*}[b!]
  \centering
  % Created by tikzDevice version 0.10.1 on 2017-12-13 15:58:48
% !TEX encoding = UTF-8 Unicode
\begin{tikzpicture}[x=1pt,y=1pt]
\definecolor{fillColor}{RGB}{255,255,255}
\path[use as bounding box,fill=fillColor,fill opacity=0.00] (0,0) rectangle (433.62,274.63);
\begin{scope}
\path[clip] (  0.00,  0.00) rectangle (433.62,274.63);
\definecolor{drawColor}{RGB}{255,255,255}
\definecolor{fillColor}{RGB}{255,255,255}

\path[draw=drawColor,line width= 0.6pt,line join=round,line cap=round,fill=fillColor] (  0.00,  0.00) rectangle (433.62,274.63);
\end{scope}
\begin{scope}
\path[clip] ( 34.75,250.04) rectangle (165.70,268.63);
\definecolor{drawColor}{gray}{0.50}
\definecolor{fillColor}{gray}{0.80}

\path[draw=drawColor,line width= 0.2pt,line join=round,line cap=round,fill=fillColor] ( 34.75,250.04) rectangle (165.70,268.63);
\definecolor{drawColor}{gray}{0.10}

\node[text=drawColor,anchor=base,inner sep=0pt, outer sep=0pt, scale=  0.87] at (100.23,256.04) {original $\mathbf x'-\mathbf x$};
\end{scope}
\begin{scope}
\path[clip] (165.70,250.04) rectangle (296.66,268.63);
\definecolor{drawColor}{gray}{0.50}
\definecolor{fillColor}{gray}{0.80}

\path[draw=drawColor,line width= 0.2pt,line join=round,line cap=round,fill=fillColor] (165.70,250.04) rectangle (296.66,268.63);
\definecolor{drawColor}{gray}{0.10}

\node[text=drawColor,anchor=base,inner sep=0pt, outer sep=0pt, scale=  0.87] at (231.18,256.04) {flipped $\mathbf{\tilde x'}-\mathbf{\tilde x}$};
\end{scope}
\begin{scope}
\path[clip] (296.66,250.04) rectangle (427.62,268.63);
\definecolor{drawColor}{gray}{0.50}
\definecolor{fillColor}{gray}{0.80}

\path[draw=drawColor,line width= 0.2pt,line join=round,line cap=round,fill=fillColor] (296.66,250.04) rectangle (427.62,268.63);
\definecolor{drawColor}{gray}{0.10}

\node[text=drawColor,anchor=base,inner sep=0pt, outer sep=0pt, scale=  0.87] at (362.14,256.04) {scaled and flipped $\mathbf{\tilde x'}-\mathbf{\tilde x}$};
\end{scope}
\begin{scope}
\path[clip] ( 34.75,105.04) rectangle (165.70,250.04);
\definecolor{fillColor}{RGB}{255,255,255}

\path[fill=fillColor] ( 34.75,105.04) rectangle (165.70,250.04);
\definecolor{drawColor}{gray}{0.60}

\path[draw=drawColor,line width= 1.1pt,line join=round] ( 44.25,274.63) -- (125.52,  0.00);

\path[draw=drawColor,line width= 1.1pt,line join=round] ( 98.39,274.63) -- (165.70, 47.12);

\path[draw=drawColor,line width= 1.1pt,dash pattern=on 1pt off 3pt ,line join=round] ( 86.29,274.63) -- (165.70,  6.24);

\path[draw=drawColor,line width= 1.1pt,dash pattern=on 1pt off 3pt ,line join=round] (110.48,274.63) -- (165.70, 88.01);

\path[draw=drawColor,line width= 1.1pt,dash pattern=on 1pt off 3pt ,line join=round] ( 34.75,265.87) -- (113.42,  0.00);

\path[draw=drawColor,line width= 1.1pt,dash pattern=on 1pt off 3pt ,line join=round] ( 56.35,274.63) -- (137.62,  0.00);
\definecolor{drawColor}{RGB}{152,78,163}

\path[draw=drawColor,line width= 0.4pt,line join=round,line cap=round] (105.90,149.19) circle (  2.28);

\path[draw=drawColor,line width= 0.4pt,line join=round,line cap=round] (118.10,134.98) circle (  2.28);

\path[draw=drawColor,line width= 0.4pt,line join=round,line cap=round] (120.55,111.63) circle (  2.28);

\path[draw=drawColor,line width= 0.4pt,line join=round,line cap=round] (128.40,123.74) circle (  2.28);

\path[draw=drawColor,line width= 0.4pt,line join=round,line cap=round] (103.16,148.80) circle (  2.28);

\path[draw=drawColor,line width= 0.4pt,line join=round,line cap=round] (127.24,113.29) circle (  2.28);

\path[draw=drawColor,line width= 0.4pt,line join=round,line cap=round] (113.73,159.23) circle (  2.28);

\path[draw=drawColor,line width= 0.4pt,line join=round,line cap=round] (103.54,152.23) circle (  2.28);

\path[draw=drawColor,line width= 0.4pt,line join=round,line cap=round] (107.06,121.71) circle (  2.28);
\definecolor{drawColor}{RGB}{255,127,0}

\path[draw=drawColor,line width= 0.4pt,line join=round,line cap=round] (139.89,214.37) circle (  2.28);

\path[draw=drawColor,line width= 0.4pt,line join=round,line cap=round] (141.88,221.51) circle (  2.28);
\definecolor{drawColor}{RGB}{77,175,74}

\path[draw=drawColor,line width= 0.4pt,line join=round,line cap=round] ( 62.62,135.64) circle (  2.28);

\path[draw=drawColor,line width= 0.4pt,line join=round,line cap=round] ( 63.38,160.55) circle (  2.28);

\path[draw=drawColor,line width= 0.4pt,line join=round,line cap=round] ( 61.92,133.12) circle (  2.28);
\definecolor{drawColor}{RGB}{255,127,0}

\path[draw=drawColor,line width= 0.4pt,line join=round,line cap=round] (138.91,222.04) circle (  2.28);

\path[draw=drawColor,line width= 0.4pt,line join=round,line cap=round] (141.64,185.30) circle (  2.28);
\definecolor{drawColor}{RGB}{77,175,74}

\path[draw=drawColor,line width= 0.4pt,line join=round,line cap=round] ( 64.43,155.18) circle (  2.28);

\path[draw=drawColor,line width= 0.4pt,line join=round,line cap=round] ( 55.18,124.74) circle (  2.28);
\definecolor{drawColor}{RGB}{255,127,0}

\path[draw=drawColor,line width= 0.4pt,line join=round,line cap=round] (140.38,196.34) circle (  2.28);
\definecolor{drawColor}{RGB}{152,78,163}

\path[draw=drawColor,line width= 0.4pt,line join=round,line cap=round] (112.42,162.72) circle (  2.28);

\path[draw=drawColor,line width= 0.4pt,line join=round,line cap=round] (114.87,139.38) circle (  2.28);

\path[draw=drawColor,line width= 0.4pt,line join=round,line cap=round] (120.44,149.20) -- (125.01,153.77);

\path[draw=drawColor,line width= 0.4pt,line join=round,line cap=round] (120.44,153.77) -- (125.01,149.20);

\path[draw=drawColor,line width= 0.4pt,line join=round,line cap=round] (119.50,151.49) -- (125.95,151.49);

\path[draw=drawColor,line width= 0.4pt,line join=round,line cap=round] (122.72,148.26) -- (122.72,154.72);

\path[draw=drawColor,line width= 0.4pt,line join=round,line cap=round] ( 97.48,176.55) circle (  2.28);

\path[draw=drawColor,line width= 0.4pt,line join=round,line cap=round] (121.57,141.04) circle (  2.28);

\path[draw=drawColor,line width= 0.4pt,line join=round,line cap=round] (108.05,186.98) circle (  2.28);

\path[draw=drawColor,line width= 0.4pt,line join=round,line cap=round] ( 97.86,179.97) circle (  2.28);

\path[draw=drawColor,line width= 0.4pt,line join=round,line cap=round] (101.38,149.45) circle (  2.28);
\definecolor{drawColor}{RGB}{77,175,74}

\path[draw=drawColor,line width= 0.4pt,line join=round,line cap=round] ( 54.88,167.25) circle (  2.28);

\path[draw=drawColor,line width= 0.4pt,line join=round,line cap=round] ( 52.90,160.11) circle (  2.28);
\definecolor{drawColor}{RGB}{255,127,0}

\path[draw=drawColor,line width= 0.4pt,line join=round,line cap=round] (143.51,190.49) circle (  2.28);
\definecolor{drawColor}{RGB}{77,175,74}

\path[draw=drawColor,line width= 0.4pt,line join=round,line cap=round] ( 55.42,186.01) -- ( 59.98,190.58);

\path[draw=drawColor,line width= 0.4pt,line join=round,line cap=round] ( 55.42,190.58) -- ( 59.98,186.01);

\path[draw=drawColor,line width= 0.4pt,line join=round,line cap=round] ( 54.47,188.30) -- ( 60.93,188.30);

\path[draw=drawColor,line width= 0.4pt,line join=round,line cap=round] ( 57.70,185.07) -- ( 57.70,191.52);

\path[draw=drawColor,line width= 0.4pt,line join=round,line cap=round] ( 56.25,160.87) circle (  2.28);
\definecolor{drawColor}{RGB}{255,127,0}

\path[draw=drawColor,line width= 0.4pt,line join=round,line cap=round] (144.59,194.30) circle (  2.28);

\path[draw=drawColor,line width= 0.4pt,line join=round,line cap=round] (147.32,157.55) circle (  2.28);
\definecolor{drawColor}{RGB}{77,175,74}

\path[draw=drawColor,line width= 0.4pt,line join=round,line cap=round] ( 58.75,182.93) circle (  2.28);
\definecolor{drawColor}{RGB}{255,127,0}

\path[draw=drawColor,line width= 0.4pt,line join=round,line cap=round] (150.95,201.39) circle (  2.28);

\path[draw=drawColor,line width= 0.4pt,line join=round,line cap=round] (146.06,168.59) circle (  2.28);
\definecolor{drawColor}{RGB}{152,78,163}

\path[draw=drawColor,line width= 0.4pt,line join=round,line cap=round] (102.68,153.59) circle (  2.28);

\path[draw=drawColor,line width= 0.4pt,line join=round,line cap=round] (110.53,165.70) circle (  2.28);

\path[draw=drawColor,line width= 0.4pt,line join=round,line cap=round] ( 85.28,190.76) circle (  2.28);

\path[draw=drawColor,line width= 0.4pt,line join=round,line cap=round] (109.37,155.25) circle (  2.28);

\path[draw=drawColor,line width= 0.4pt,line join=round,line cap=round] ( 95.85,201.19) circle (  2.28);

\path[draw=drawColor,line width= 0.4pt,line join=round,line cap=round] ( 85.67,194.19) circle (  2.28);

\path[draw=drawColor,line width= 0.4pt,line join=round,line cap=round] ( 86.90,161.38) -- ( 91.47,165.95);

\path[draw=drawColor,line width= 0.4pt,line join=round,line cap=round] ( 86.90,165.95) -- ( 91.47,161.38);

\path[draw=drawColor,line width= 0.4pt,line join=round,line cap=round] ( 85.96,163.67) -- ( 92.41,163.67);

\path[draw=drawColor,line width= 0.4pt,line join=round,line cap=round] ( 89.18,160.44) -- ( 89.18,166.90);
\definecolor{drawColor}{RGB}{255,127,0}

\path[draw=drawColor,line width= 0.4pt,line join=round,line cap=round] (157.76,172.41) circle (  2.28);
\definecolor{drawColor}{RGB}{77,175,74}

\path[draw=drawColor,line width= 0.4pt,line join=round,line cap=round] ( 40.70,174.33) circle (  2.28);

\path[draw=drawColor,line width= 0.4pt,line join=round,line cap=round] ( 44.75,177.60) circle (  2.28);

\path[draw=drawColor,line width= 0.4pt,line join=round,line cap=round] ( 45.50,202.51) circle (  2.28);
\definecolor{drawColor}{RGB}{255,127,0}

\path[draw=drawColor,line width= 0.4pt,line join=round,line cap=round] (156.40,178.79) circle (  2.28);

\path[draw=drawColor,line width= 0.4pt,line join=round,line cap=round] (156.78,180.08) circle (  2.28);
\definecolor{drawColor}{RGB}{77,175,74}

\node[text=drawColor,anchor=base,inner sep=0pt, outer sep=0pt, scale=  0.78] at ( 53.30,140.75) {$y_i=-1$};
\definecolor{drawColor}{RGB}{152,78,163}

\node[text=drawColor,anchor=base,inner sep=0pt, outer sep=0pt, scale=  0.78] at ( 87.88,216.14) {$y_i=0$};
\definecolor{drawColor}{RGB}{255,127,0}

\node[text=drawColor,anchor=base,inner sep=0pt, outer sep=0pt, scale=  0.78] at (144.68,227.22) {$y_i=1$};
\definecolor{drawColor}{gray}{0.50}

\path[draw=drawColor,line width= 0.6pt,line join=round,line cap=round] ( 34.75,105.04) rectangle (165.70,250.04);
\end{scope}
\begin{scope}
\path[clip] (165.70,105.04) rectangle (296.66,250.04);
\definecolor{fillColor}{RGB}{255,255,255}

\path[fill=fillColor] (165.70,105.04) rectangle (296.66,250.04);
\definecolor{drawColor}{gray}{0.60}

\path[draw=drawColor,line width= 1.1pt,line join=round] (175.21,274.63) -- (256.47,  0.00);

\path[draw=drawColor,line width= 1.1pt,line join=round] (229.34,274.63) -- (296.66, 47.12);

\path[draw=drawColor,line width= 1.1pt,dash pattern=on 1pt off 3pt ,line join=round] (217.24,274.63) -- (296.66,  6.24);

\path[draw=drawColor,line width= 1.1pt,dash pattern=on 1pt off 3pt ,line join=round] (241.44,274.63) -- (296.66, 88.01);

\path[draw=drawColor,line width= 1.1pt,dash pattern=on 1pt off 3pt ,line join=round] (165.70,265.87) -- (244.38,  0.00);

\path[draw=drawColor,line width= 1.1pt,dash pattern=on 1pt off 3pt ,line join=round] (187.31,274.63) -- (268.57,  0.00);
\definecolor{drawColor}{RGB}{0,0,0}

\path[draw=drawColor,line width= 0.9pt,dash pattern=on 1pt off 3pt ,line join=round] (259.36,111.63) -- (259.20,242.24);

\path[draw=drawColor,line width= 0.9pt,dash pattern=on 1pt off 3pt ,line join=round] (267.10,111.63) -- (266.94,242.24);

\path[draw=drawColor,line width= 0.9pt,line join=round] (263.23,111.63) -- (263.07,242.24);

\node[text=drawColor,anchor=base,inner sep=0pt, outer sep=0pt, scale=  0.78] at (280.58,240.53) {QP};
\definecolor{drawColor}{gray}{0.60}

\node[text=drawColor,anchor=base,inner sep=0pt, outer sep=0pt, scale=  0.78] at (181.79,174.01) {LP};
\definecolor{drawColor}{RGB}{152,78,163}

\path[draw=drawColor,line width= 0.4pt,line join=round,line cap=round] (236.86,149.19) circle (  2.28);

\path[draw=drawColor,line width= 0.4pt,line join=round,line cap=round] (249.06,134.98) circle (  2.28);

\path[draw=drawColor,line width= 0.4pt,line join=round,line cap=round] (251.51,111.63) circle (  2.28);

\path[draw=drawColor,line width= 0.4pt,line join=round,line cap=round] (259.36,123.74) circle (  2.28);

\path[draw=drawColor,line width= 0.4pt,line join=round,line cap=round] (234.12,148.80) circle (  2.28);

\path[draw=drawColor,line width= 0.4pt,line join=round,line cap=round] (258.20,113.29) circle (  2.28);

\path[draw=drawColor,line width= 0.4pt,line join=round,line cap=round] (244.68,159.23) circle (  2.28);

\path[draw=drawColor,line width= 0.4pt,line join=round,line cap=round] (234.50,152.23) circle (  2.28);

\path[draw=drawColor,line width= 0.4pt,line join=round,line cap=round] (238.02,121.71) circle (  2.28);
\definecolor{drawColor}{RGB}{255,127,0}

\path[draw=drawColor,line width= 0.4pt,line join=round,line cap=round] (270.85,214.37) circle (  2.28);

\path[draw=drawColor,line width= 0.4pt,line join=round,line cap=round] (272.84,221.51) circle (  2.28);

\path[draw=drawColor,line width= 0.4pt,line join=round,line cap=round] (268.79,218.24) circle (  2.28);

\path[draw=drawColor,line width= 0.4pt,line join=round,line cap=round] (268.03,193.33) circle (  2.28);

\path[draw=drawColor,line width= 0.4pt,line join=round,line cap=round] (269.48,220.75) circle (  2.28);

\path[draw=drawColor,line width= 0.4pt,line join=round,line cap=round] (269.87,222.04) circle (  2.28);

\path[draw=drawColor,line width= 0.4pt,line join=round,line cap=round] (272.60,185.30) circle (  2.28);

\path[draw=drawColor,line width= 0.4pt,line join=round,line cap=round] (266.98,198.69) circle (  2.28);

\path[draw=drawColor,line width= 0.4pt,line join=round,line cap=round] (276.23,229.13) circle (  2.28);

\path[draw=drawColor,line width= 0.4pt,line join=round,line cap=round] (271.34,196.34) circle (  2.28);
\definecolor{drawColor}{RGB}{152,78,163}

\path[draw=drawColor,line width= 0.4pt,line join=round,line cap=round] (243.38,162.72) circle (  2.28);

\path[draw=drawColor,line width= 0.4pt,line join=round,line cap=round] (245.83,139.38) circle (  2.28);

\path[draw=drawColor,line width= 0.4pt,line join=round,line cap=round] (251.40,149.20) -- (255.97,153.77);

\path[draw=drawColor,line width= 0.4pt,line join=round,line cap=round] (251.40,153.77) -- (255.97,149.20);

\path[draw=drawColor,line width= 0.4pt,line join=round,line cap=round] (250.45,151.49) -- (256.91,151.49);

\path[draw=drawColor,line width= 0.4pt,line join=round,line cap=round] (253.68,148.26) -- (253.68,154.72);

\path[draw=drawColor,line width= 0.4pt,line join=round,line cap=round] (228.44,176.55) circle (  2.28);

\path[draw=drawColor,line width= 0.4pt,line join=round,line cap=round] (252.52,141.04) circle (  2.28);

\path[draw=drawColor,line width= 0.4pt,line join=round,line cap=round] (239.01,186.98) circle (  2.28);

\path[draw=drawColor,line width= 0.4pt,line join=round,line cap=round] (228.82,179.97) circle (  2.28);

\path[draw=drawColor,line width= 0.4pt,line join=round,line cap=round] (232.34,149.45) circle (  2.28);
\definecolor{drawColor}{RGB}{255,127,0}

\path[draw=drawColor,line width= 0.4pt,line join=round,line cap=round] (276.53,186.63) circle (  2.28);

\path[draw=drawColor,line width= 0.4pt,line join=round,line cap=round] (278.51,193.76) circle (  2.28);

\path[draw=drawColor,line width= 0.4pt,line join=round,line cap=round] (274.46,190.49) circle (  2.28);

\path[draw=drawColor,line width= 0.4pt,line join=round,line cap=round] (271.42,163.30) -- (275.99,167.86);

\path[draw=drawColor,line width= 0.4pt,line join=round,line cap=round] (271.42,167.86) -- (275.99,163.30);

\path[draw=drawColor,line width= 0.4pt,line join=round,line cap=round] (270.48,165.58) -- (276.94,165.58);

\path[draw=drawColor,line width= 0.4pt,line join=round,line cap=round] (273.71,162.35) -- (273.71,168.81);

\path[draw=drawColor,line width= 0.4pt,line join=round,line cap=round] (275.16,193.01) circle (  2.28);

\path[draw=drawColor,line width= 0.4pt,line join=round,line cap=round] (275.54,194.30) circle (  2.28);

\path[draw=drawColor,line width= 0.4pt,line join=round,line cap=round] (278.28,157.55) circle (  2.28);

\path[draw=drawColor,line width= 0.4pt,line join=round,line cap=round] (272.66,170.95) circle (  2.28);

\path[draw=drawColor,line width= 0.4pt,line join=round,line cap=round] (281.91,201.39) circle (  2.28);

\path[draw=drawColor,line width= 0.4pt,line join=round,line cap=round] (277.01,168.59) circle (  2.28);
\definecolor{drawColor}{RGB}{152,78,163}

\path[draw=drawColor,line width= 0.4pt,line join=round,line cap=round] (233.63,153.59) circle (  2.28);

\path[draw=drawColor,line width= 0.4pt,line join=round,line cap=round] (241.49,165.70) circle (  2.28);

\path[draw=drawColor,line width= 0.4pt,line join=round,line cap=round] (216.24,190.76) circle (  2.28);

\path[draw=drawColor,line width= 0.4pt,line join=round,line cap=round] (240.33,155.25) circle (  2.28);

\path[draw=drawColor,line width= 0.4pt,line join=round,line cap=round] (226.81,201.19) circle (  2.28);

\path[draw=drawColor,line width= 0.4pt,line join=round,line cap=round] (216.62,194.19) circle (  2.28);

\path[draw=drawColor,line width= 0.4pt,line join=round,line cap=round] (217.86,161.38) -- (222.43,165.95);

\path[draw=drawColor,line width= 0.4pt,line join=round,line cap=round] (217.86,165.95) -- (222.43,161.38);

\path[draw=drawColor,line width= 0.4pt,line join=round,line cap=round] (216.91,163.67) -- (223.37,163.67);

\path[draw=drawColor,line width= 0.4pt,line join=round,line cap=round] (220.14,160.44) -- (220.14,166.90);
\definecolor{drawColor}{RGB}{255,127,0}

\path[draw=drawColor,line width= 0.4pt,line join=round,line cap=round] (288.72,172.41) circle (  2.28);

\path[draw=drawColor,line width= 0.4pt,line join=round,line cap=round] (290.71,179.55) circle (  2.28);

\path[draw=drawColor,line width= 0.4pt,line join=round,line cap=round] (286.66,176.28) circle (  2.28);

\path[draw=drawColor,line width= 0.4pt,line join=round,line cap=round] (285.90,151.37) circle (  2.28);

\path[draw=drawColor,line width= 0.4pt,line join=round,line cap=round] (287.36,178.79) circle (  2.28);

\path[draw=drawColor,line width= 0.4pt,line join=round,line cap=round] (287.74,180.08) circle (  2.28);
\definecolor{drawColor}{RGB}{152,78,163}

\path[draw=drawColor,line width= 0.4pt,line join=round,line cap=round] (225.51,204.68) circle (  2.28);

\path[draw=drawColor,line width= 0.4pt,line join=round,line cap=round] (213.31,218.90) circle (  2.28);

\path[draw=drawColor,line width= 0.4pt,line join=round,line cap=round] (210.86,242.24) circle (  2.28);

\path[draw=drawColor,line width= 0.4pt,line join=round,line cap=round] (203.01,230.14) circle (  2.28);

\path[draw=drawColor,line width= 0.4pt,line join=round,line cap=round] (228.25,205.07) circle (  2.28);

\path[draw=drawColor,line width= 0.4pt,line join=round,line cap=round] (204.17,240.59) circle (  2.28);

\path[draw=drawColor,line width= 0.4pt,line join=round,line cap=round] (217.68,194.64) circle (  2.28);

\path[draw=drawColor,line width= 0.4pt,line join=round,line cap=round] (227.87,201.65) circle (  2.28);

\path[draw=drawColor,line width= 0.4pt,line join=round,line cap=round] (224.35,232.17) circle (  2.28);

\path[draw=drawColor,line width= 0.4pt,line join=round,line cap=round] (218.99,191.15) circle (  2.28);

\path[draw=drawColor,line width= 0.4pt,line join=round,line cap=round] (216.54,214.50) circle (  2.28);

\path[draw=drawColor,line width= 0.4pt,line join=round,line cap=round] (206.40,200.11) -- (210.97,204.67);

\path[draw=drawColor,line width= 0.4pt,line join=round,line cap=round] (206.40,204.67) -- (210.97,200.11);

\path[draw=drawColor,line width= 0.4pt,line join=round,line cap=round] (205.46,202.39) -- (211.91,202.39);

\path[draw=drawColor,line width= 0.4pt,line join=round,line cap=round] (208.68,199.16) -- (208.68,205.62);

\path[draw=drawColor,line width= 0.4pt,line join=round,line cap=round] (233.93,177.33) circle (  2.28);

\path[draw=drawColor,line width= 0.4pt,line join=round,line cap=round] (209.84,212.84) circle (  2.28);

\path[draw=drawColor,line width= 0.4pt,line join=round,line cap=round] (223.36,166.90) circle (  2.28);

\path[draw=drawColor,line width= 0.4pt,line join=round,line cap=round] (233.55,173.90) circle (  2.28);

\path[draw=drawColor,line width= 0.4pt,line join=round,line cap=round] (230.03,204.42) circle (  2.28);

\path[draw=drawColor,line width= 0.4pt,line join=round,line cap=round] (228.73,200.28) circle (  2.28);

\path[draw=drawColor,line width= 0.4pt,line join=round,line cap=round] (220.88,188.17) circle (  2.28);

\path[draw=drawColor,line width= 0.4pt,line join=round,line cap=round] (246.12,163.11) circle (  2.28);

\path[draw=drawColor,line width= 0.4pt,line join=round,line cap=round] (222.04,198.62) circle (  2.28);

\path[draw=drawColor,line width= 0.4pt,line join=round,line cap=round] (235.56,152.68) circle (  2.28);

\path[draw=drawColor,line width= 0.4pt,line join=round,line cap=round] (245.74,159.69) circle (  2.28);

\path[draw=drawColor,line width= 0.4pt,line join=round,line cap=round] (239.94,187.93) -- (244.51,192.49);

\path[draw=drawColor,line width= 0.4pt,line join=round,line cap=round] (239.94,192.49) -- (244.51,187.93);

\path[draw=drawColor,line width= 0.4pt,line join=round,line cap=round] (239.00,190.21) -- (245.45,190.21);

\path[draw=drawColor,line width= 0.4pt,line join=round,line cap=round] (242.22,186.98) -- (242.22,193.44);
\definecolor{drawColor}{RGB}{0,0,0}
\definecolor{fillColor}{RGB}{0,0,0}

\path[draw=drawColor,line width= 0.4pt,line join=round,line cap=round,fill=fillColor] (259.36,123.74) circle (  1.52);

\path[draw=drawColor,line width= 0.4pt,line join=round,line cap=round,fill=fillColor] (266.98,198.69) circle (  1.52);
\definecolor{drawColor}{gray}{0.50}

\path[draw=drawColor,line width= 0.6pt,line join=round,line cap=round] (165.70,105.04) rectangle (296.66,250.04);
\end{scope}
\begin{scope}
\path[clip] (296.66,105.04) rectangle (427.62,250.04);
\definecolor{fillColor}{RGB}{255,255,255}

\path[fill=fillColor] (296.66,105.04) rectangle (427.62,250.04);
\definecolor{drawColor}{gray}{0.60}

\path[draw=drawColor,line width= 1.1pt,line join=round] (306.17,274.63) -- (387.43,  0.00);

\path[draw=drawColor,line width= 1.1pt,line join=round] (360.30,274.63) -- (427.62, 47.12);

\path[draw=drawColor,line width= 1.1pt,dash pattern=on 1pt off 3pt ,line join=round] (348.20,274.63) -- (427.62,  6.24);

\path[draw=drawColor,line width= 1.1pt,dash pattern=on 1pt off 3pt ,line join=round] (372.40,274.63) -- (427.62, 88.01);

\path[draw=drawColor,line width= 1.1pt,dash pattern=on 1pt off 3pt ,line join=round] (296.66,265.87) -- (375.33,  0.00);

\path[draw=drawColor,line width= 1.1pt,dash pattern=on 1pt off 3pt ,line join=round] (318.27,274.63) -- (399.53,  0.00);
\definecolor{drawColor}{RGB}{0,0,0}

\path[draw=drawColor,line width= 0.9pt,dash pattern=on 1pt off 3pt ,line join=round] (396.43,111.63) -- (357.78,242.24);

\path[draw=drawColor,line width= 0.9pt,dash pattern=on 1pt off 3pt ,line join=round] (420.63,111.63) -- (381.98,242.24);

\path[draw=drawColor,line width= 0.9pt,line join=round] (408.53,111.63) -- (369.88,242.24);
\definecolor{drawColor}{RGB}{152,78,163}

\path[draw=drawColor,line width= 0.4pt,line join=round,line cap=round] (367.82,149.19) circle (  2.28);

\path[draw=drawColor,line width= 0.4pt,line join=round,line cap=round] (380.01,134.98) circle (  2.28);

\path[draw=drawColor,line width= 0.4pt,line join=round,line cap=round] (382.46,111.63) circle (  2.28);

\path[draw=drawColor,line width= 0.4pt,line join=round,line cap=round] (390.32,123.74) circle (  2.28);

\path[draw=drawColor,line width= 0.4pt,line join=round,line cap=round] (365.07,148.80) circle (  2.28);

\path[draw=drawColor,line width= 0.4pt,line join=round,line cap=round] (389.16,113.29) circle (  2.28);

\path[draw=drawColor,line width= 0.4pt,line join=round,line cap=round] (375.64,159.23) circle (  2.28);

\path[draw=drawColor,line width= 0.4pt,line join=round,line cap=round] (365.45,152.23) circle (  2.28);

\path[draw=drawColor,line width= 0.4pt,line join=round,line cap=round] (368.97,121.71) circle (  2.28);
\definecolor{drawColor}{RGB}{255,127,0}

\path[draw=drawColor,line width= 0.4pt,line join=round,line cap=round] (401.81,214.37) circle (  2.28);

\path[draw=drawColor,line width= 0.4pt,line join=round,line cap=round] (403.79,221.51) circle (  2.28);

\path[draw=drawColor,line width= 0.4pt,line join=round,line cap=round] (399.75,218.24) circle (  2.28);

\path[draw=drawColor,line width= 0.4pt,line join=round,line cap=round] (398.99,193.33) circle (  2.28);

\path[draw=drawColor,line width= 0.4pt,line join=round,line cap=round] (400.44,220.75) circle (  2.28);

\path[draw=drawColor,line width= 0.4pt,line join=round,line cap=round] (400.83,222.04) circle (  2.28);

\path[draw=drawColor,line width= 0.4pt,line join=round,line cap=round] (403.56,185.30) circle (  2.28);

\path[draw=drawColor,line width= 0.4pt,line join=round,line cap=round] (397.94,198.69) circle (  2.28);

\path[draw=drawColor,line width= 0.4pt,line join=round,line cap=round] (407.19,229.13) circle (  2.28);

\path[draw=drawColor,line width= 0.4pt,line join=round,line cap=round] (402.29,196.34) circle (  2.28);
\definecolor{drawColor}{RGB}{152,78,163}

\path[draw=drawColor,line width= 0.4pt,line join=round,line cap=round] (374.34,162.72) circle (  2.28);

\path[draw=drawColor,line width= 0.4pt,line join=round,line cap=round] (376.79,139.38) circle (  2.28);

\path[draw=drawColor,line width= 0.4pt,line join=round,line cap=round] (382.36,149.20) -- (386.92,153.77);

\path[draw=drawColor,line width= 0.4pt,line join=round,line cap=round] (382.36,153.77) -- (386.92,149.20);

\path[draw=drawColor,line width= 0.4pt,line join=round,line cap=round] (381.41,151.49) -- (387.87,151.49);

\path[draw=drawColor,line width= 0.4pt,line join=round,line cap=round] (384.64,148.26) -- (384.64,154.72);

\path[draw=drawColor,line width= 0.4pt,line join=round,line cap=round] (359.40,176.55) circle (  2.28);

\path[draw=drawColor,line width= 0.4pt,line join=round,line cap=round] (383.48,141.04) circle (  2.28);

\path[draw=drawColor,line width= 0.4pt,line join=round,line cap=round] (369.97,186.98) circle (  2.28);

\path[draw=drawColor,line width= 0.4pt,line join=round,line cap=round] (359.78,179.97) circle (  2.28);

\path[draw=drawColor,line width= 0.4pt,line join=round,line cap=round] (363.30,149.45) circle (  2.28);
\definecolor{drawColor}{RGB}{255,127,0}

\path[draw=drawColor,line width= 0.4pt,line join=round,line cap=round] (407.48,186.63) circle (  2.28);

\path[draw=drawColor,line width= 0.4pt,line join=round,line cap=round] (409.47,193.76) circle (  2.28);

\path[draw=drawColor,line width= 0.4pt,line join=round,line cap=round] (405.42,190.49) circle (  2.28);

\path[draw=drawColor,line width= 0.4pt,line join=round,line cap=round] (402.38,163.30) -- (406.95,167.86);

\path[draw=drawColor,line width= 0.4pt,line join=round,line cap=round] (402.38,167.86) -- (406.95,163.30);

\path[draw=drawColor,line width= 0.4pt,line join=round,line cap=round] (401.44,165.58) -- (407.89,165.58);

\path[draw=drawColor,line width= 0.4pt,line join=round,line cap=round] (404.67,162.35) -- (404.67,168.81);

\path[draw=drawColor,line width= 0.4pt,line join=round,line cap=round] (406.12,193.01) circle (  2.28);

\path[draw=drawColor,line width= 0.4pt,line join=round,line cap=round] (406.50,194.30) circle (  2.28);

\path[draw=drawColor,line width= 0.4pt,line join=round,line cap=round] (409.23,157.55) circle (  2.28);

\path[draw=drawColor,line width= 0.4pt,line join=round,line cap=round] (403.61,170.95) circle (  2.28);

\path[draw=drawColor,line width= 0.4pt,line join=round,line cap=round] (412.87,201.39) circle (  2.28);

\path[draw=drawColor,line width= 0.4pt,line join=round,line cap=round] (407.97,168.59) circle (  2.28);
\definecolor{drawColor}{RGB}{152,78,163}

\path[draw=drawColor,line width= 0.4pt,line join=round,line cap=round] (364.59,153.59) circle (  2.28);

\path[draw=drawColor,line width= 0.4pt,line join=round,line cap=round] (372.44,165.70) circle (  2.28);

\path[draw=drawColor,line width= 0.4pt,line join=round,line cap=round] (347.20,190.76) circle (  2.28);

\path[draw=drawColor,line width= 0.4pt,line join=round,line cap=round] (371.29,155.25) circle (  2.28);

\path[draw=drawColor,line width= 0.4pt,line join=round,line cap=round] (357.77,201.19) circle (  2.28);

\path[draw=drawColor,line width= 0.4pt,line join=round,line cap=round] (347.58,194.19) circle (  2.28);

\path[draw=drawColor,line width= 0.4pt,line join=round,line cap=round] (348.82,161.38) -- (353.38,165.95);

\path[draw=drawColor,line width= 0.4pt,line join=round,line cap=round] (348.82,165.95) -- (353.38,161.38);

\path[draw=drawColor,line width= 0.4pt,line join=round,line cap=round] (347.87,163.67) -- (354.33,163.67);

\path[draw=drawColor,line width= 0.4pt,line join=round,line cap=round] (351.10,160.44) -- (351.10,166.90);
\definecolor{drawColor}{RGB}{255,127,0}

\path[draw=drawColor,line width= 0.4pt,line join=round,line cap=round] (419.68,172.41) circle (  2.28);

\path[draw=drawColor,line width= 0.4pt,line join=round,line cap=round] (421.67,179.55) circle (  2.28);

\path[draw=drawColor,line width= 0.4pt,line join=round,line cap=round] (417.62,176.28) circle (  2.28);

\path[draw=drawColor,line width= 0.4pt,line join=round,line cap=round] (416.86,151.37) circle (  2.28);

\path[draw=drawColor,line width= 0.4pt,line join=round,line cap=round] (418.31,178.79) circle (  2.28);

\path[draw=drawColor,line width= 0.4pt,line join=round,line cap=round] (418.70,180.08) circle (  2.28);
\definecolor{drawColor}{RGB}{152,78,163}

\path[draw=drawColor,line width= 0.4pt,line join=round,line cap=round] (356.47,204.68) circle (  2.28);

\path[draw=drawColor,line width= 0.4pt,line join=round,line cap=round] (344.27,218.90) circle (  2.28);

\path[draw=drawColor,line width= 0.4pt,line join=round,line cap=round] (341.82,242.24) circle (  2.28);

\path[draw=drawColor,line width= 0.4pt,line join=round,line cap=round] (333.97,230.14) circle (  2.28);

\path[draw=drawColor,line width= 0.4pt,line join=round,line cap=round] (359.21,205.07) circle (  2.28);

\path[draw=drawColor,line width= 0.4pt,line join=round,line cap=round] (335.12,240.59) circle (  2.28);

\path[draw=drawColor,line width= 0.4pt,line join=round,line cap=round] (348.64,194.64) circle (  2.28);

\path[draw=drawColor,line width= 0.4pt,line join=round,line cap=round] (358.83,201.65) circle (  2.28);

\path[draw=drawColor,line width= 0.4pt,line join=round,line cap=round] (355.31,232.17) circle (  2.28);

\path[draw=drawColor,line width= 0.4pt,line join=round,line cap=round] (349.94,191.15) circle (  2.28);

\path[draw=drawColor,line width= 0.4pt,line join=round,line cap=round] (347.50,214.50) circle (  2.28);

\path[draw=drawColor,line width= 0.4pt,line join=round,line cap=round] (337.36,200.11) -- (341.93,204.67);

\path[draw=drawColor,line width= 0.4pt,line join=round,line cap=round] (337.36,204.67) -- (341.93,200.11);

\path[draw=drawColor,line width= 0.4pt,line join=round,line cap=round] (336.41,202.39) -- (342.87,202.39);

\path[draw=drawColor,line width= 0.4pt,line join=round,line cap=round] (339.64,199.16) -- (339.64,205.62);

\path[draw=drawColor,line width= 0.4pt,line join=round,line cap=round] (364.89,177.33) circle (  2.28);

\path[draw=drawColor,line width= 0.4pt,line join=round,line cap=round] (340.80,212.84) circle (  2.28);

\path[draw=drawColor,line width= 0.4pt,line join=round,line cap=round] (354.32,166.90) circle (  2.28);

\path[draw=drawColor,line width= 0.4pt,line join=round,line cap=round] (364.50,173.90) circle (  2.28);

\path[draw=drawColor,line width= 0.4pt,line join=round,line cap=round] (360.99,204.42) circle (  2.28);

\path[draw=drawColor,line width= 0.4pt,line join=round,line cap=round] (359.69,200.28) circle (  2.28);

\path[draw=drawColor,line width= 0.4pt,line join=round,line cap=round] (351.84,188.17) circle (  2.28);

\path[draw=drawColor,line width= 0.4pt,line join=round,line cap=round] (377.08,163.11) circle (  2.28);

\path[draw=drawColor,line width= 0.4pt,line join=round,line cap=round] (353.00,198.62) circle (  2.28);

\path[draw=drawColor,line width= 0.4pt,line join=round,line cap=round] (366.51,152.68) circle (  2.28);

\path[draw=drawColor,line width= 0.4pt,line join=round,line cap=round] (376.70,159.69) circle (  2.28);

\path[draw=drawColor,line width= 0.4pt,line join=round,line cap=round] (370.90,187.93) -- (375.47,192.49);

\path[draw=drawColor,line width= 0.4pt,line join=round,line cap=round] (370.90,192.49) -- (375.47,187.93);

\path[draw=drawColor,line width= 0.4pt,line join=round,line cap=round] (369.95,190.21) -- (376.41,190.21);

\path[draw=drawColor,line width= 0.4pt,line join=round,line cap=round] (373.18,186.98) -- (373.18,193.44);
\definecolor{drawColor}{RGB}{0,0,0}
\definecolor{fillColor}{RGB}{0,0,0}

\path[draw=drawColor,line width= 0.4pt,line join=round,line cap=round,fill=fillColor] (384.64,151.49) circle (  1.52);

\path[draw=drawColor,line width= 0.4pt,line join=round,line cap=round,fill=fillColor] (404.67,165.58) circle (  1.52);

\path[draw=drawColor,line width= 0.4pt,line join=round,line cap=round,fill=fillColor] (373.18,190.21) circle (  1.52);
\definecolor{drawColor}{RGB}{152,78,163}

\node[text=drawColor,anchor=base,inner sep=0pt, outer sep=0pt, scale=  0.78] at (354.73,134.10) {$\tilde y_i=-1$};
\definecolor{drawColor}{RGB}{255,127,0}

\node[text=drawColor,anchor=base,inner sep=0pt, outer sep=0pt, scale=  0.78] at (406.59,240.53) {$\tilde y_i=1$};
\definecolor{drawColor}{gray}{0.50}

\path[draw=drawColor,line width= 0.6pt,line join=round,line cap=round] (296.66,105.04) rectangle (427.62,250.04);
\end{scope}
\begin{scope}
\path[clip] (  0.00,  0.00) rectangle (433.62,274.63);
\definecolor{drawColor}{RGB}{0,0,0}

\node[text=drawColor,anchor=base east,inner sep=0pt, outer sep=0pt, scale=  0.87] at ( 29.35,129.30) {-2};

\node[text=drawColor,anchor=base east,inner sep=0pt, outer sep=0pt, scale=  0.87] at ( 29.35,173.65) {0};

\node[text=drawColor,anchor=base east,inner sep=0pt, outer sep=0pt, scale=  0.87] at ( 29.35,217.99) {2};
\end{scope}
\begin{scope}
\path[clip] (  0.00,  0.00) rectangle (433.62,274.63);
\definecolor{drawColor}{RGB}{0,0,0}

\path[draw=drawColor,line width= 0.6pt,line join=round] ( 31.75,132.59) --
	( 34.75,132.59);

\path[draw=drawColor,line width= 0.6pt,line join=round] ( 31.75,176.94) --
	( 34.75,176.94);

\path[draw=drawColor,line width= 0.6pt,line join=round] ( 31.75,221.28) --
	( 34.75,221.28);
\end{scope}
\begin{scope}
\path[clip] (  0.00,  0.00) rectangle (433.62,274.63);
\definecolor{drawColor}{RGB}{0,0,0}

\path[draw=drawColor,line width= 0.6pt,line join=round] ( 50.83,102.04) --
	( 50.83,105.04);

\path[draw=drawColor,line width= 0.6pt,line join=round] ( 75.53,102.04) --
	( 75.53,105.04);

\path[draw=drawColor,line width= 0.6pt,line join=round] (100.23,102.04) --
	(100.23,105.04);

\path[draw=drawColor,line width= 0.6pt,line join=round] (124.92,102.04) --
	(124.92,105.04);

\path[draw=drawColor,line width= 0.6pt,line join=round] (149.62,102.04) --
	(149.62,105.04);
\end{scope}
\begin{scope}
\path[clip] (  0.00,  0.00) rectangle (433.62,274.63);
\definecolor{drawColor}{RGB}{0,0,0}

\node[text=drawColor,anchor=base,inner sep=0pt, outer sep=0pt, scale=  0.87] at ( 50.83, 93.06) {-200};

\node[text=drawColor,anchor=base,inner sep=0pt, outer sep=0pt, scale=  0.87] at ( 75.53, 93.06) {-100};

\node[text=drawColor,anchor=base,inner sep=0pt, outer sep=0pt, scale=  0.87] at (100.23, 93.06) {0};

\node[text=drawColor,anchor=base,inner sep=0pt, outer sep=0pt, scale=  0.87] at (124.92, 93.06) {100};

\node[text=drawColor,anchor=base,inner sep=0pt, outer sep=0pt, scale=  0.87] at (149.62, 93.06) {200};
\end{scope}
\begin{scope}
\path[clip] (  0.00,  0.00) rectangle (433.62,274.63);
\definecolor{drawColor}{RGB}{0,0,0}

\path[draw=drawColor,line width= 0.6pt,line join=round] (181.79,102.04) --
	(181.79,105.04);

\path[draw=drawColor,line width= 0.6pt,line join=round] (206.49,102.04) --
	(206.49,105.04);

\path[draw=drawColor,line width= 0.6pt,line join=round] (231.18,102.04) --
	(231.18,105.04);

\path[draw=drawColor,line width= 0.6pt,line join=round] (255.88,102.04) --
	(255.88,105.04);

\path[draw=drawColor,line width= 0.6pt,line join=round] (280.58,102.04) --
	(280.58,105.04);
\end{scope}
\begin{scope}
\path[clip] (  0.00,  0.00) rectangle (433.62,274.63);
\definecolor{drawColor}{RGB}{0,0,0}

\node[text=drawColor,anchor=base,inner sep=0pt, outer sep=0pt, scale=  0.87] at (181.79, 93.06) {-200};

\node[text=drawColor,anchor=base,inner sep=0pt, outer sep=0pt, scale=  0.87] at (206.49, 93.06) {-100};

\node[text=drawColor,anchor=base,inner sep=0pt, outer sep=0pt, scale=  0.87] at (231.18, 93.06) {0};

\node[text=drawColor,anchor=base,inner sep=0pt, outer sep=0pt, scale=  0.87] at (255.88, 93.06) {100};

\node[text=drawColor,anchor=base,inner sep=0pt, outer sep=0pt, scale=  0.87] at (280.58, 93.06) {200};
\end{scope}
\begin{scope}
\path[clip] (  0.00,  0.00) rectangle (433.62,274.63);
\definecolor{drawColor}{RGB}{0,0,0}

\path[draw=drawColor,line width= 0.6pt,line join=round] (312.75,102.04) --
	(312.75,105.04);

\path[draw=drawColor,line width= 0.6pt,line join=round] (337.45,102.04) --
	(337.45,105.04);

\path[draw=drawColor,line width= 0.6pt,line join=round] (362.14,102.04) --
	(362.14,105.04);

\path[draw=drawColor,line width= 0.6pt,line join=round] (386.84,102.04) --
	(386.84,105.04);

\path[draw=drawColor,line width= 0.6pt,line join=round] (411.53,102.04) --
	(411.53,105.04);
\end{scope}
\begin{scope}
\path[clip] (  0.00,  0.00) rectangle (433.62,274.63);
\definecolor{drawColor}{RGB}{0,0,0}

\node[text=drawColor,anchor=base,inner sep=0pt, outer sep=0pt, scale=  0.87] at (312.75, 93.06) {-2};

\node[text=drawColor,anchor=base,inner sep=0pt, outer sep=0pt, scale=  0.87] at (337.45, 93.06) {-1};

\node[text=drawColor,anchor=base,inner sep=0pt, outer sep=0pt, scale=  0.87] at (362.14, 93.06) {0};

\node[text=drawColor,anchor=base,inner sep=0pt, outer sep=0pt, scale=  0.87] at (386.84, 93.06) {1};

\node[text=drawColor,anchor=base,inner sep=0pt, outer sep=0pt, scale=  0.87] at (411.53, 93.06) {2};
\end{scope}
\begin{scope}
\path[clip] (  0.00,  0.00) rectangle (433.62,274.63);
\definecolor{drawColor}{RGB}{0,0,0}

\node[text=drawColor,anchor=base,inner sep=0pt, outer sep=0pt, scale=  1.09] at (231.18, 80.03) {difference feature 1};
\end{scope}
\begin{scope}
\path[clip] (  0.00,  0.00) rectangle (433.62,274.63);
\definecolor{drawColor}{RGB}{0,0,0}

\node[text=drawColor,rotate= 90.00,anchor=base,inner sep=0pt, outer sep=0pt, scale=  1.09] at ( 16.63,177.54) {difference feature 2};
\end{scope}
\begin{scope}
\path[clip] (  0.00,  0.00) rectangle (433.62,274.63);
\definecolor{fillColor}{RGB}{255,255,255}

\path[fill=fillColor] ( 95.74, 44.66) rectangle (366.62, 69.10);
\end{scope}
\begin{scope}
\path[clip] (  0.00,  0.00) rectangle (433.62,274.63);
\definecolor{drawColor}{RGB}{0,0,0}

\node[text=drawColor,anchor=base west,inner sep=0pt, outer sep=0pt, scale=  1.09] at (100.01, 52.77) {boundary};
\end{scope}
\begin{scope}
\path[clip] (  0.00,  0.00) rectangle (433.62,274.63);
\definecolor{drawColor}{gray}{0.80}
\definecolor{fillColor}{RGB}{255,255,255}

\path[draw=drawColor,line width= 0.6pt,line join=round,line cap=round,fill=fillColor] (153.48, 48.93) rectangle (169.38, 64.83);
\end{scope}
\begin{scope}
\path[clip] (  0.00,  0.00) rectangle (433.62,274.63);
\definecolor{drawColor}{gray}{0.60}

\path[draw=drawColor,line width= 1.1pt,line join=round] (153.48, 48.93) -- (169.38, 64.83);
\end{scope}
\begin{scope}
\path[clip] (  0.00,  0.00) rectangle (433.62,274.63);
\definecolor{drawColor}{RGB}{0,0,0}

\path[draw=drawColor,line width= 0.9pt,line join=round] (155.07, 56.88) -- (167.79, 56.88);
\end{scope}
\begin{scope}
\path[clip] (  0.00,  0.00) rectangle (433.62,274.63);
\definecolor{drawColor}{gray}{0.80}
\definecolor{fillColor}{RGB}{255,255,255}

\path[draw=drawColor,line width= 0.6pt,line join=round,line cap=round,fill=fillColor] (252.07, 48.93) rectangle (267.97, 64.83);
\end{scope}
\begin{scope}
\path[clip] (  0.00,  0.00) rectangle (433.62,274.63);
\definecolor{drawColor}{gray}{0.60}

\path[draw=drawColor,line width= 1.1pt,dash pattern=on 1pt off 3pt ,line join=round] (252.07, 48.93) -- (267.97, 64.83);
\end{scope}
\begin{scope}
\path[clip] (  0.00,  0.00) rectangle (433.62,274.63);
\definecolor{drawColor}{RGB}{0,0,0}

\path[draw=drawColor,line width= 0.9pt,dash pattern=on 1pt off 3pt ,line join=round] (253.66, 56.88) -- (266.38, 56.88);
\end{scope}
\begin{scope}
\path[clip] (  0.00,  0.00) rectangle (433.62,274.63);
\definecolor{drawColor}{RGB}{0,0,0}

\node[text=drawColor,anchor=base west,inner sep=0pt, outer sep=0pt, scale=  0.87] at (171.37, 53.59) {decision $r(\mathbf x)=\pm 1$};
\end{scope}
\begin{scope}
\path[clip] (  0.00,  0.00) rectangle (433.62,274.63);
\definecolor{drawColor}{RGB}{0,0,0}

\node[text=drawColor,anchor=base west,inner sep=0pt, outer sep=0pt, scale=  0.87] at (269.96, 53.59) {margin $r(\mathbf x)=\pm 1\pm\mu$};
\end{scope}
\begin{scope}
\path[clip] (  0.00,  0.00) rectangle (433.62,274.63);
\definecolor{fillColor}{RGB}{255,255,255}

\path[fill=fillColor] ( 55.34, 14.54) rectangle (407.02, 38.97);
\end{scope}
\begin{scope}
\path[clip] (  0.00,  0.00) rectangle (433.62,274.63);
\definecolor{drawColor}{RGB}{0,0,0}

\node[text=drawColor,anchor=base west,inner sep=0pt, outer sep=0pt, scale=  1.09] at ( 59.61, 22.64) {point};
\end{scope}
\begin{scope}
\path[clip] (  0.00,  0.00) rectangle (433.62,274.63);
\definecolor{drawColor}{gray}{0.80}
\definecolor{fillColor}{RGB}{255,255,255}

\path[draw=drawColor,line width= 0.6pt,line join=round,line cap=round,fill=fillColor] ( 90.65, 18.80) rectangle (106.54, 34.70);
\end{scope}
\begin{scope}
\path[clip] (  0.00,  0.00) rectangle (433.62,274.63);
\definecolor{drawColor}{RGB}{0,0,0}

\path[draw=drawColor,line width= 0.4pt,line join=round,line cap=round] ( 96.31, 24.47) -- (100.88, 29.04);

\path[draw=drawColor,line width= 0.4pt,line join=round,line cap=round] ( 96.31, 29.04) -- (100.88, 24.47);

\path[draw=drawColor,line width= 0.4pt,line join=round,line cap=round] ( 95.37, 26.75) -- (101.82, 26.75);

\path[draw=drawColor,line width= 0.4pt,line join=round,line cap=round] ( 98.60, 23.52) -- ( 98.60, 29.98);
\end{scope}
\begin{scope}
\path[clip] (  0.00,  0.00) rectangle (433.62,274.63);
\definecolor{drawColor}{gray}{0.80}
\definecolor{fillColor}{RGB}{255,255,255}

\path[draw=drawColor,line width= 0.6pt,line join=round,line cap=round,fill=fillColor] (195.04, 18.80) rectangle (210.94, 34.70);
\end{scope}
\begin{scope}
\path[clip] (  0.00,  0.00) rectangle (433.62,274.63);
\definecolor{drawColor}{RGB}{0,0,0}

\path[draw=drawColor,line width= 0.4pt,line join=round,line cap=round] (202.99, 26.75) circle (  2.28);
\end{scope}
\begin{scope}
\path[clip] (  0.00,  0.00) rectangle (433.62,274.63);
\definecolor{drawColor}{gray}{0.80}
\definecolor{fillColor}{RGB}{255,255,255}

\path[draw=drawColor,line width= 0.6pt,line join=round,line cap=round,fill=fillColor] (307.35, 18.80) rectangle (323.25, 34.70);
\end{scope}
\begin{scope}
\path[clip] (  0.00,  0.00) rectangle (433.62,274.63);
\definecolor{drawColor}{RGB}{0,0,0}
\definecolor{fillColor}{RGB}{0,0,0}

\path[draw=drawColor,line width= 0.4pt,line join=round,line cap=round,fill=fillColor] (315.30, 26.75) circle (  1.52);
\end{scope}
\begin{scope}
\path[clip] (  0.00,  0.00) rectangle (433.62,274.63);
\definecolor{drawColor}{RGB}{0,0,0}

\node[text=drawColor,anchor=base west,inner sep=0pt, outer sep=0pt, scale=  0.87] at (108.53, 23.46) {LP constraint active};
\end{scope}
\begin{scope}
\path[clip] (  0.00,  0.00) rectangle (433.62,274.63);
\definecolor{drawColor}{RGB}{0,0,0}

\node[text=drawColor,anchor=base west,inner sep=0pt, outer sep=0pt, scale=  0.87] at (212.92, 23.46) {LP constraint inactive};
\end{scope}
\begin{scope}
\path[clip] (  0.00,  0.00) rectangle (433.62,274.63);
\definecolor{drawColor}{RGB}{0,0,0}

\node[text=drawColor,anchor=base west,inner sep=0pt, outer sep=0pt, scale=  0.87] at (325.24, 23.46) {QP support vector};
\end{scope}
\end{tikzpicture}

  \vskip -1cm
  \caption{The separable LP and QP comparison problems. \textbf{Left}:
    the difference vectors $\mathbf x'-\mathbf x$ of the original data
    and the optimal solution to the LP
    (\ref{eq:max-margin-lp}). \textbf{Middle}: for the unscaled
    flipped data $\mathbf{\tilde x'}-\mathbf{ \tilde x}$
    (\ref{eq:tilde}), the LP is not the same as the QP
    (\ref{eq:max-margin-qp-tilde}). \textbf{Right}: in these scaled data,
    the QP is equivalent to the LP.}
  \label{fig:hard-margin}
\end{figure*}

In this section we discuss new learning algorithms for comparison
problems. In all cases, we will first learn a ranking function
$r:\RR^p\rightarrow\RR$ and then a comparison function
$c_1:\RR^p\times \RR^p\rightarrow\{-1,0,1\}$
(\ref{eq:compare_general}). In other words, a small rank difference
$|r(\mathbf x')-r(\mathbf x)|\leq 1$ is considered insignificant, and
there are two decision boundaries $r(\mathbf x')-r(\mathbf
x)\in\{-1,1\}$.

\subsection{LP and QP for separable data}
\label{sec:lp-qp}

In our learning setup, the best comparison function is the one with
maximum margin. We will define the margin in two different ways, which
correspond to the linear program (LP) and quadratic program (QP)
discussed below. To illustrate the differences between these
max-margin comparison problems, in this section we assume that the
training data are linearly separable. Later in
Section~\ref{sec:kernelized-qp}, we propose an algorithm for learning
a nonlinear function from non-separable data.

In the following linear program, we learn a linear ranking function
$r(\mathbf x)=\mathbf w^\intercal \mathbf x$ that maximizes the margin
$\mu$, defined in terms of ranking function values. The margin $\mu$
is the smallest rank difference between a decision boundary $r(\mathbf
x)\in\{-1,1\}$ and a difference vector $r(\mathbf x_i'-\mathbf
x_i)$. The max margin LP is
\begin{eqnarray}
  \label{eq:max-margin-lp}
  \maximize_{\mu\in\RR^+, \mathbf w\in\RR^p}\ &&\hskip -0.5cm \mu \\
  \nonumber
  \text{subject to}\ && \hskip -0.5cm \mu \leq
  1-|\mathbf w^\intercal (\mathbf x_i' - \mathbf x_i)|,\ 
  \forall\  i\in \mathcal I_0,\\
  \nonumber
  &&\hskip -0.5cm
  \mu \leq -1 +  \mathbf w^\intercal(\mathbf x_i'-\mathbf x_i)y_i,
  \ \forall\ i\in \mathcal I_1\cup \mathcal I_{-1}.
\end{eqnarray}
The optimal decision boundaries $r(\mathbf x)\in\{-1,1\}$ and margin
boundaries $r(\mathbf x)\in\{-1\pm \mu, 1 \pm \mu\}$ are drawn 
in Figure~\ref{fig:hard-margin}. Note that finding a feasible point
for this LP is a test of linear separability. If there are no feasible
points then the data are not linearly separable.

Another way to formulate the comparison problem is by first performing
a change of variables, and then solving a binary SVM QP. The idea is
to maximize the margin between significant differences
$y_i\in\{-1,1\}$ and equality pairs $y_i=0$. Let $\mathbf X_y,\mathbf
X_y'$ be the $|\mathcal I_y|\times p$ matrices formed by all the pairs
$i\in \mathcal I_y$. We define a new ``flipped'' data set with
$m=|\mathcal I_1|+|\mathcal I_{-1}|+2|\mathcal I_0|$ pairs suitable
for training a binary SVM:
\begin{equation}
\label{eq:tilde}
\mathbf{  \tilde X} = \left[
    \begin{array}{c}
      \mathbf X_1 \\
      \mathbf X_{-1}'\\
      \mathbf X_0\\
      \mathbf X_0'
    \end{array}
  \right],\ 
  \mathbf{\tilde X'} = \left[
    \begin{array}{c}
      \mathbf X_1' \\
      \mathbf X_{-1}\\
      \mathbf X_0'\\
      \mathbf X_0
    \end{array}
  \right],\ 
  \mathbf{\tilde y} = \left[
    \begin{array}{c}
      \mathbf 1_{|\mathcal I_1|} \\
      \mathbf 1_{|\mathcal I_{-1}|}\\
      \mathbf{-1}_{|\mathcal I_0|}\\
      \mathbf{-1}_{|\mathcal I_0|}
    \end{array}
  \right],
\end{equation}
where $\mathbf 1_n$ is an $n$-vector of ones, $\mathbf{\tilde
  X},\mathbf{\tilde X'}\in\RR^{m\times p}$ and $\mathbf{\tilde
  y}\in\{-1,1\}^m$. Note that $\tilde y_i=-1$ implies no significant
difference between $\mathbf{\tilde x}_i$ and $\mathbf{\tilde x}_i'$,
and $\tilde y_i=1$ implies that $\mathbf{\tilde x}_i'$ is better than
$\mathbf{\tilde x}_i$. We then learn an affine function $f(\mathbf
x)=\beta+\mathbf u^\intercal \mathbf x$ using a binary SVM QP:
\begin{eqnarray}
  \label{eq:max-margin-qp-tilde}
  \minimize_{\mathbf u\in\RR^p, \beta\in\RR}\ &&\hskip -0.5cm
  \mathbf u^\intercal \mathbf u  \\
\nonumber    \text{subject to}\ &&\hskip -0.5cm 
    \tilde y_i (\beta + 
    \mathbf u^\intercal( \mathbf{\tilde x}_i'-\mathbf{\tilde x}_i) ) \geq 1,
    \ \forall i\in\{1,\dots,m\}.
\end{eqnarray}
This SVM QP learns a separator $f(\mathbf x)=0$ between significant
difference pairs $\tilde y_i=1$ and insignificant difference pairs
$\tilde y_i=-1$ (middle and right panels of
Figure~\ref{fig:hard-margin}). However, we want a comparison function
that predicts $c(\mathbf x,\mathbf x')\in\{-1,0,1\}$. So we use the
next Lemma to construct a ranking function $r(\mathbf x)= \mathbf{\hat
  w}^\intercal \mathbf x$ that is feasible for the original max margin
comparison LP (\ref{eq:max-margin-lp}), and can be used with the
comparison function~$c_1$ (\ref{eq:compare_general}).
\begin{lemma}
  Let $\mathbf u\in\RR^p,\beta\in\RR$ be a solution of
  (\ref{eq:max-margin-qp-tilde}). Then $\hat \mu = -1/\beta$ and
  $\mathbf{\hat w} = -\mathbf u/\beta$ are feasible for
  (\ref{eq:max-margin-lp}).
  \label{lemma:feasible}
\end{lemma}
\begin{proof}
  Begin by assuming that we want to find a ranking function $r(\mathbf
  x)=\mathbf{\hat w}^\intercal \mathbf x = \gamma \mathbf u^\intercal
  \mathbf x$, where $\gamma\in\RR$ is a scaling constant. Then
  consider that for all $\mathbf x$ on the decision boundary, we have
  \begin{equation}
    \label{eq:dec-boundary-rank}
    r(\mathbf x) = \mathbf{\hat w}^\intercal \mathbf x = 1\text{ and } 
    f(\mathbf x) = \mathbf u^\intercal \mathbf x + \beta = 0.
  \end{equation}
  Taken together, it is clear that $\gamma=-1/\beta$ and thus
  $\mathbf{\hat w} = -\mathbf u/\beta$. Consider for all $\mathbf x$ on the
  margin we have
  \begin{equation}
    \label{eq:margin-rank}
    r(\mathbf x) = \mathbf{\hat w}^\intercal \mathbf x = 1+\hat\mu\text{ and } 
    f(\mathbf x) = \mathbf u^\intercal \mathbf x + \beta= 1.
  \end{equation}
  Taken together, these imply $\hat \mu=-1/\beta$. Now, by definition
  of the flipped data (\ref{eq:tilde}), we can re-write the max margin
  QP (\ref{eq:max-margin-qp-tilde}) as
\begin{eqnarray}
  \label{eq:max-margin-qp}
    \minimize_{\mathbf u\in\RR^p, \beta\in\RR}\ &&
    \hskip -0.5cm \mathbf u^\intercal \mathbf u  \\
    \text{subject to}\ &&\hskip -0.5cm
    \nonumber \beta + |\mathbf u^\intercal (\mathbf x_i'-\mathbf x_i)| \leq -1,\
    \forall\  i\in \mathcal I_0,\\
    &&\hskip -0.5cm
\nonumber \beta + \mathbf u^\intercal(\mathbf x_i'-\mathbf x_i)y_i \geq 1,
\ \forall\ i \in \mathcal I_1\cup \mathcal I_{-1}.
\end{eqnarray}
By re-writing the constraints of (\ref{eq:max-margin-qp}) in terms of
$\hat \mu$ and $\mathbf{\hat w}$, we recover the same constraints as
the max margin comparison LP (\ref{eq:max-margin-lp}). Thus $\hat \mu,
\mathbf{\hat w}$ are feasible for (\ref{eq:max-margin-lp}).
\end{proof}

One may also wonder: are $\hat \mu,\mathbf{\hat w}$ optimal for the
max margin comparison LP? In general, the answer is no, and we give
one counterexample in the middle panel of
Figure~\ref{fig:hard-margin}. This is because the LP defines the
margin in terms of ranking function values $r(\mathbf x)=\mathbf
w^\intercal \mathbf x$, but the QP defines the margin in terms of the
size of the normal vector $||\mathbf u||$, which depends on the scale
of the inputs $\mathbf x,\mathbf x'$. However, when the input
variables are scaled in a pre-processing step, we have observed that
the solutions to the LP and QP are equivalent (right panel of
Figure~\ref{fig:hard-margin}).

Lemma~\ref{lemma:feasible} is very useful in practice. It means that a
ranking function $r$ can be learned on comparison data by first
solving a standard binary SVM, and then transforming the solution. We
use this result in the next section to build a general support vector
algorithm for comparison data.

\subsection{Kernelized QP for non-separable data}
\label{sec:kernelized-qp}
In this section, we assume the data are not separable, and want to
learn a nonlinear ranking function. We define a positive definite
kernel $\kappa:\RR^p\times \RR^p\rightarrow\RR$, which implicitly
defines an enlarged set of features $\Phi(\mathbf x)$ (middle panel of
Figure~\ref{fig:norm-data}). As in (\ref{eq:max-margin-qp-tilde}), we
learn a function $f(\mathbf x)=\beta + \mathbf u^\intercal
\Phi(\mathbf x)$ which is affine in the feature space. Let $\mathbf
\alpha,\mathbf \alpha'\in\RR^m$ be coefficients such that $\mathbf
u=\sum_{i=1}^m
\alpha_i \Phi(\mathbf{\tilde x}_i) + 
\alpha_i' \Phi(\mathbf{\tilde x}_i')$, and so we have
 $f(\mathbf x) =\beta + \sum_{i=1}^m 
 \alpha_i \kappa(\mathbf{\tilde x}_i, \mathbf x) +
 \alpha_i' \kappa(\mathbf{\tilde x}_i', \mathbf x)$. 
 We then use Lemma~\ref{lemma:feasible} to
define the ranking function
\begin{equation}
  \label{eq:kernelized_r}
  r(\mathbf x)= \frac{\mathbf u^\intercal \Phi(\mathbf x)}{-\beta} = 
  \sum_{i=1}^m
  \frac{
    \alpha_i \kappa(\mathbf{\tilde x}_i, \mathbf x) +
    \alpha_i'  \kappa(\mathbf{\tilde x}_i', \mathbf x)}
{-\beta}.
\end{equation}



\begin{algorithm}[b!]
   \caption{SVMcompare}
   \label{alg:SVMcompare}
\begin{algorithmic}
  \STATE {\bfseries Input:} cost $C\in\RR^+$, kernel
  $\kappa:\RR^p\times \RR^p \rightarrow \RR$, features $\mathbf
  X,\mathbf X'\in\RR^{n \times p}$, labels $\mathbf y\in\{-1,0,1\}^n$.

  \STATE \makebox[0.5cm]{$\mathbf{\tilde X}$} $\gets [$
  \makebox[1cm]{$\mathbf X_1^\intercal$}
  \makebox[1cm]{$\mathbf X_{-1}'^\intercal$}
  \makebox[1cm]{$\mathbf X_0^\intercal$}
  \makebox[1cm]{$\mathbf X_0'^\intercal$}
  $]^\intercal$.

  \STATE \makebox[0.5cm]{$\mathbf{\tilde X}'$} $\gets [$
  \makebox[1cm]{$\mathbf X_1'^\intercal$}
  \makebox[1cm]{$\mathbf X_{-1}^\intercal$}
  \makebox[1cm]{$\mathbf X_0'^\intercal$}
  \makebox[1cm]{$\mathbf X_0^\intercal$}
  $]^\intercal$.

  \STATE \makebox[0.5cm]{$\mathbf{\tilde y}$} $\gets [$
  \makebox[1cm]{$\mathbf 1_{|\mathcal I_1|}^\intercal$}
  \makebox[1cm]{$\mathbf 1_{|\mathcal I_{-1}|}^\intercal$}
  \makebox[1cm]{$-\mathbf 1_{|\mathcal I_0|}^\intercal$}
  \makebox[1cm]{$-\mathbf 1_{|\mathcal I_0|}^\intercal$}
  $]^\intercal$.

  \STATE $\mathbf K \gets \proc{KernelMatrix}(
  \mathbf{\tilde X}, \mathbf{\tilde X'}, \kappa)$.

  \STATE $\mathbf M \gets [ -\mathbf I_m\ \mathbf I_m ]^\intercal$.

  \STATE $\mathbf{\tilde K} \gets \mathbf M^\intercal \mathbf  K \mathbf M$.

  \STATE $\mathbf v,\beta \gets \proc{SVMdual}(
  \mathbf{\tilde K}, \mathbf{\tilde y}, C)$.

  \STATE $\sv \gets\{i: v_i>0\}$.
  
  \STATE {\bfseries Output:} Support vectors $\mathbf{\tilde
    X}_{\sv },\mathbf{\tilde X}_{\sv }'$, labels
  $\mathbf{\tilde y}_{\sv }$, bias~$\beta$, dual variables $\mathbf v$.

   \end{algorithmic}
\end{algorithm}

Let $\mathbf K=[
\mathbf k_1\cdots \mathbf k_m
\ \mathbf k_1'\cdots \mathbf k_m']\in\RR^{2m\times 2m}$ be the
kernel matrix, where for all pairs $i\in\{1, \dots, m\}$, the kernel
vectors $\mathbf k_i,\mathbf k_i'\in\RR^{2m}$ are defined as
\begin{equation}
  \mathbf k_i = \left[
    \begin{array}{c}
      \kappa(\mathbf{\tilde x}_1, \mathbf{\tilde x}_i)\\
      \vdots\\
      \kappa(\mathbf{\tilde x}_m, \mathbf{\tilde x}_i)\\
      \kappa(\mathbf{\tilde x}_1', \mathbf{\tilde x}_i)\\
      \vdots\\
      \kappa(\mathbf{\tilde x}_m', \mathbf{\tilde x}_i)
    \end{array}
  \right],\ 
  \mathbf k_i' = \left[
    \begin{array}{c}
      \kappa(\mathbf{\tilde x}_1, \mathbf{\tilde x}_i')\\
      \vdots\\
      \kappa(\mathbf{\tilde x}_m, \mathbf{\tilde x}_i')\\
      \kappa(\mathbf{\tilde x}_1', \mathbf{\tilde x}_i')\\
      \vdots\\
      \kappa(\mathbf{\tilde x}_m', \mathbf{\tilde x}_i')
    \end{array}
  \right].
\end{equation}
Letting $\mathbf a=[\alpha^\intercal\
\alpha'^\intercal]^\intercal\in\RR^{2m}$, the norm of the affine
function $f$ in the feature space is $\mathbf u^\intercal \mathbf u =
\mathbf a^\intercal \mathbf K \mathbf a$, and we can write the primal soft-margin
comparison QP for some $C\in\RR^+$ as
\begin{eqnarray}
  \minimize_{\mathbf a\in\RR^{2m},\mathbf \xi\in\RR^m,\beta\in\RR}\ \ &&\hskip -0.5cm 
  \frac 1 2 \mathbf a^\intercal \mathbf K \mathbf a + C\sum_{i=1}^m \xi_i \\
  \text{subject to}\ \ &&\hskip -0.5cm \nonumber
  \text{for all $i\in\{1,\dots,m\}$, }
  \xi_i \geq 0,\\
  &&\hskip -0.5cm \nonumber \text{and }
  \xi_i \geq 1-\tilde y_i(\beta + \mathbf a^\intercal (\mathbf k_i'-\mathbf k_i)).
\end{eqnarray}
Let $\mathbf z, \mathbf v\in\RR^m$ be the dual variables, and
$\mathbf Y=\Diag(\mathbf{\tilde y})$ be the diagonal matrix of $m$
labels. Then the Lagrangian can be written as
\begin{equation}
  \label{eq:lagrangian}
  \mathcal L = \frac 1 2 \mathbf a^\intercal \mathbf K \mathbf a + 
  C\mathbf \xi^\intercal\mathbf  1_{m}\\
  -\mathbf z^\intercal \mathbf \xi
  + \mathbf v^\intercal(\mathbf 1_m - \mathbf{\tilde y}\beta
  - \mathbf Y \mathbf M^\intercal \mathbf K\mathbf  a - \xi),
\end{equation}
where $\mathbf M=[-\mathbf I_m \, \mathbf
I_m]^\intercal\in\{-1,0,1\}^{2m\times m}$ and $\mathbf I_m$ is the
identity matrix. Solving $\nabla_{\mathbf a} \mathcal L=0$ results in
the following stationary condition:
\begin{equation}
  \label{eq:stationarity}
  \mathbf a = \mathbf M \mathbf Y \mathbf v.
\end{equation}
The rest of the derivation of the dual comparison problem is the same
as for the standard binary SVM. The resulting dual QP is
\begin{equation}
  \begin{aligned}
    \label{eq:svm-dual}
    \minimize_{\mathbf v\in\RR^m}\ \ &
    \frac 1 2 \mathbf v^\intercal \mathbf Y \mathbf M^\intercal 
    \mathbf K \mathbf M \mathbf Y \mathbf v - \mathbf v^\intercal \mathbf 1_m\\
    \text{subject to}\ \ &
    \sum_{i=1}^m v_i \tilde y_i = 0,\\
    & \text{for all $i\in\{1,\dots,m\}$, } 0\leq v_i\leq C,
  \end{aligned}
\end{equation}
which is equivalent to the dual problem of a standard binary SVM with
kernel $\mathbf{\tilde K} = \mathbf M^\intercal \mathbf K \mathbf
M\in\RR^{m\times m}$ and labels $\mathbf{\tilde y}\in\{-1,1\}^m$.

So we can solve the dual comparison problem (\ref{eq:svm-dual}) using
any efficient SVM solver, such as libsvm \citep{libsvm}. We used the R
interface in the \texttt{kernlab} package \citep{kernlab}, and our
code is available in the \texttt{rankSVMcompare} package on Github.

After obtaining optimal dual variables $\mathbf v\in\RR^m$ as the solution of
(\ref{eq:svm-dual}), the SVM solver also gives us the optimal bias
$\beta$ by analyzing the complementary slackness conditions.
The learned ranking function can be quickly evaluated since the
optimal $\mathbf v$ is sparse. Let $\sv =\{i: v_i > 0\}$ be the indices
of the support vectors. Since we need only $2|\sv |$ kernel
evaluations, the ranking function (\ref{eq:kernelized_r}) becomes
\begin{equation}
  \label{eq:r_sv}
  r(\mathbf x)= 
  \sum_{i\in \sv }
  \tilde y_i v_i\left[ 
    \kappa(\mathbf{\tilde x}_i, \mathbf x)
    - \kappa(\mathbf{\tilde x}_i', \mathbf x)
  \right]/\beta.
\end{equation}
Note that for all $i\in\{1,\dots,m\}$, the optimal primal variables
$\alpha_i=-\tilde y_i v_i$ and $\alpha_i'=\tilde y_i v_i$ are
recovered using the stationary condition (\ref{eq:stationarity}). The
learned comparison function $c_1$ remains the same (\ref{eq:compare_general}).

The training procedure is summarized as
Algorithm~\ref{alg:SVMcompare}, SVMcompare.
There are two sub-routines: \proc{KernelMatrix} computes
the $2m\times 2m$ kernel matrix, and \proc{SVMdual} solves the SVM
dual QP (\ref{eq:svm-dual}). There are two hyper-parameters to tune:
the cost $C$ and the kernel $\kappa$. As with standard SVM for binary
classification, these parameters can be tuned by minimizing the
prediction error on a held-out validation set.

\section{Results}
\label{sec:results}

\begin{table}[b!]
  \centering
  \begin{tabular}{r|cc|cc|}
Input:&    \multicolumn{2}{c|}{equality pairs}
&    \multicolumn{2}{c|}{inequality pairs}\\
    & $|\mathcal I_0|$ %$\tilde y_i= -1$
    & --- 
    & $|\mathcal I_1|+|\mathcal I_{-1}|$ %$\tilde y_i=1$
    & $\rightarrow$
    \\
    \hline
    rank 
    & 0 
    & 
    & $|\mathcal I_1|+|\mathcal I_{-1}|$ 
    & $\rightarrow$ 
    \\
    \hline
    rank2 
    & $2|\mathcal I_0|$ 
    & $\leftarrow \rightarrow$
    & $2(|\mathcal I_1|+|\mathcal I_{-1}|)$ 
    & $\rightarrow \rightarrow$
    \\
    \hline
    compare 
    & $2|\mathcal I_0|$ 
    & --- --- 
    & $|\mathcal I_1|+|\mathcal I_{-1}|$ 
    & $\rightarrow$\\
    \hline
  \end{tabular}
  \caption{\label{tab:models}
    Summary of how the different algorithms 
    use the input pairs to learn the ranking 
    function $r$. Equality $y_i=0$ pairs are shown as ---  
    segments and inequality $y_i\in\{-1,1\}$ pairs 
    are shown as $\rightarrow$  arrows. For example, 
    the rank2 algorithm converts each input equality pair
    to two opposite-facing inequality pairs.}
\end{table}

The goal of learning to compare is to accurately predict a test set of
labeled pairs (\ref{eq:min_c}), which includes equality $y_i=0$
pairs. We test the SVMcompare algorithm alongside two baseline models
that use SVMrank \citep{ranksvm}. We chose SVMrank as a baseline
because of its similar large-margin learning formulation, to
demonstrate the importance of directly modeling the equality $y_i=0$
pairs. SVMrank does not directly model the equality $y_i=0$ pairs, so
we expect that the proposed SVMcompare algorithm makes better
predictions when these data are present. The differences between the
algorithms are summarized in Table~\ref{tab:models}:

\begin{description}
\item[rank] is described in Section~\ref{sec:svmrank}: first we use
  $|\mathcal I_1|+|\mathcal I_{-1}|$ inequality pairs to learn
  SVMrank, then we use all $n$ pairs to learn a threshold $\hat \tau$
  for when to predict $c(\mathbf x,\mathbf x')=0$.
\item[rank2] is another variant of SVMrank that treats each input pair
  as 2 inequality pairs. Since SVMrank can only use inequality pairs,
  we transform each equality pair $(\mathbf x_i,\mathbf x_i',0)$ into two
  opposite-facing inequality pairs $(\mathbf x_i',\mathbf x_i,1)$ and
  $(\mathbf x_i,\mathbf x_i',1)$. 
  To ensure equal weight for all input pairs in the
  cost function, we also duplicate each inequality pair, resulting in
  $2n$ pairs used to train SVMrank.
\item[compare] is the SVMcompare model proposed in this paper, which
  uses $m=n+|\mathcal I_0|$ input pairs.
\end{description}

We use two evaluation metrics to judge the performance of the models
on the test pairs: zero-one loss, and area under the ROC curve. Note
that the ROC curves are calculated by first evaluating the learned
ranking function $r(\mathbf x)$ at each test point $\mathbf x$, and
then varying the threshold $\tau$ of the comparison function $c_\tau$
(\ref{eq:compare_general}). For $\tau=0$ we have 100\% false positive
rate and for $\tau=\infty$ we have 100\% false negative rate
(Table~\ref{tab:evaluation}).

\begin{table}[b!]
  \centering
  \begin{tabular}{|a|c|c|c|}\hline
    \rowcolor{lightgray}
    \backslashbox{$\hat{y}$}{ $y$}
    &\textbf{-1}&\textbf{0}&\textbf{1}\\ \hline
    \textbf{-1}&0  & FP & Inversion   	\\ \hline 
    \textbf{0} &FN& 0& FN\\ \hline
    \textbf{1} & Inversion & FP &0	\\ \hline
  \end{tabular}
  % \cellcolor{pastelblue}
  \caption{We use area under the ROC curve to evaluate predictions
    $\hat y$ given the true label $y$. False positives (FP) occur 
    when predicting a significant difference $\hat y\in\{-1,1\}$ 
    when there is none $y=0$, and False Negatives (FN) are the opposite.
    Inversions occur when predicting the opposite of the true label
    $\hat y = -y$, 
    % are infrequent in practice, 
    but are not used for ROC analysis.}
  \label{tab:evaluation}
\end{table}

\subsection{Simulation: norms in 2D}
\label{sec:simulations}

We used a simulation to visualize the learned nonlinear ranking
functions in 2D, and to demonstrate that our model can achieve lower
test error than the baseline SVMrank model, by learning from the
equality $y_i=0$ pairs.  We generated pairs $\mathbf x_i,\mathbf
x_i'\in[-3,3]^2$ and noisy labels $y_i=t_1[r(\mathbf x'_i)-r(\mathbf
x_i)+\epsilon_i]$, where $t_1$ is the threshold function
(\ref{eq:threshold}), $r$ is the latent ranking function,
$\epsilon_i\sim N(0, \sigma)$ is noise, and $\sigma=1/4$ is the
standard deviation. We picked train, validation, and test sets, each
with the same number of pairs $n$ and the same proportion $\rho$ of
equality pairs. We fit a $10\times 10$ grid of models to the training
set (cost parameter $C=10^{-3},\dots,10^3$, Gaussian kernel width
$2^{-7},\dots,2^4$), select the model with minimal zero-one loss on
the validation set, and then use the test set to estimate the
generalization ability of the selected model.

In Figure~\ref{fig:norm-level-curves} we fixed $n=100$ pairs, with
$\rho=1/2$ equality and inequality pairs. We show the training set and
the level curves of the ranking functions learned by the SVMrank and
SVMcompare models. It is clear that the rank model does not accurately
recover the true ranking function $r(\mathbf x)=||\mathbf x||_1^2$,
since it does not use the equality $y_i=0$ pairs. In contrast, the
compare and rank2 methods which exploit the equality $y_i=0$ pairs are
able to recover a ranking function that is closer to the true $r$.

In Figure~\ref{fig:simulation-samples} we fixed the proportion of
equality pairs $\rho=1/2$, varied the number of training pairs
$n\in\{50,\dots, 800\}$, and tested three simulated ranking functions
$r(\mathbf x)=||\mathbf x||^2_j$ for $j\in\{1,2,\infty\}$. In general,
the test error of all models decreases as training set size $n$
increases. The model with the highest test error is the rank model,
which does not use the equality $y_i=0$ pairs. The next best model is
the rank2 model, which converts the equality $y_i=0$ pairs to
inequality pairs and then trains SVMrank. The best model is the
proposed SVMcompare model, which achieves test error as good as the
true ranking function in the case of $r(\mathbf x)=||\mathbf x||^2_2$.

\begin{figure*}[b!]
  \centering
  % Created by tikzDevice version 0.7.0 on 2014-01-30 09:59:34
% !TEX encoding = UTF-8 Unicode
\begin{tikzpicture}[x=1pt,y=1pt]
\definecolor[named]{fillColor}{rgb}{1.00,1.00,1.00}
\path[use as bounding box,fill=fillColor,fill opacity=0.00] (0,0) rectangle (505.89,144.54);
\begin{scope}
\path[clip] (  4.52,  0.00) rectangle (501.37,144.54);
\definecolor[named]{drawColor}{rgb}{1.00,1.00,1.00}
\definecolor[named]{fillColor}{rgb}{1.00,1.00,1.00}

\path[draw=drawColor,line width= 0.6pt,line join=round,line cap=round,fill=fillColor] (  4.52,  0.00) rectangle (501.37,144.54);
\end{scope}
\begin{scope}
\path[clip] ( 39.94,119.86) rectangle (125.76,132.50);
\definecolor[named]{drawColor}{rgb}{0.50,0.50,0.50}
\definecolor[named]{fillColor}{rgb}{0.80,0.80,0.80}

\path[draw=drawColor,line width= 0.6pt,line join=round,line cap=round,fill=fillColor] ( 39.94,119.86) rectangle (125.76,132.50);
\definecolor[named]{drawColor}{rgb}{0.00,0.00,0.00}

\node[text=drawColor,anchor=base,inner sep=0pt, outer sep=0pt, scale=  0.96] at ( 82.85,122.87) {training data};
\end{scope}
\begin{scope}
\path[clip] (125.76,119.86) rectangle (211.59,132.50);
\definecolor[named]{drawColor}{rgb}{0.50,0.50,0.50}
\definecolor[named]{fillColor}{rgb}{0.80,0.80,0.80}

\path[draw=drawColor,line width= 0.6pt,line join=round,line cap=round,fill=fillColor] (125.76,119.86) rectangle (211.59,132.50);
\definecolor[named]{drawColor}{rgb}{0.00,0.00,0.00}

\node[text=drawColor,anchor=base,inner sep=0pt, outer sep=0pt, scale=  0.96] at (168.68,122.87) {rank model};
\end{scope}
\begin{scope}
\path[clip] (211.59,119.86) rectangle (297.42,132.50);
\definecolor[named]{drawColor}{rgb}{0.50,0.50,0.50}
\definecolor[named]{fillColor}{rgb}{0.80,0.80,0.80}

\path[draw=drawColor,line width= 0.6pt,line join=round,line cap=round,fill=fillColor] (211.59,119.86) rectangle (297.42,132.50);
\definecolor[named]{drawColor}{rgb}{0.00,0.00,0.00}

\node[text=drawColor,anchor=base,inner sep=0pt, outer sep=0pt, scale=  0.96] at (254.50,122.87) {rank2 model};
\end{scope}
\begin{scope}
\path[clip] (297.42,119.86) rectangle (383.24,132.50);
\definecolor[named]{drawColor}{rgb}{0.50,0.50,0.50}
\definecolor[named]{fillColor}{rgb}{0.80,0.80,0.80}

\path[draw=drawColor,line width= 0.6pt,line join=round,line cap=round,fill=fillColor] (297.42,119.86) rectangle (383.24,132.50);
\definecolor[named]{drawColor}{rgb}{0.00,0.00,0.00}

\node[text=drawColor,anchor=base,inner sep=0pt, outer sep=0pt, scale=  0.96] at (340.33,122.87) {compare model};
\end{scope}
\begin{scope}
\path[clip] ( 39.94, 34.03) rectangle (125.76,119.86);
\definecolor[named]{fillColor}{rgb}{1.00,1.00,1.00}

\path[fill=fillColor] ( 39.94, 34.03) rectangle (125.76,119.86);
\definecolor[named]{drawColor}{rgb}{0.60,0.31,0.64}
\definecolor[named]{fillColor}{rgb}{0.60,0.31,0.64}

\path[draw=drawColor,line width= 1.1pt,line join=round,fill=fillColor] ( 72.04, 73.49) -- ( 72.06, 78.16);

\path[draw=drawColor,line width= 1.1pt,line join=round,fill=fillColor] ( 84.79, 83.17) -- ( 88.11, 77.62);

\path[draw=drawColor,line width= 1.1pt,line join=round,fill=fillColor] ( 89.99, 73.70) -- ( 93.36, 76.52);

\path[draw=drawColor,line width= 1.1pt,line join=round,fill=fillColor] ( 76.10, 83.81) -- ( 75.89, 79.90);

\path[draw=drawColor,line width= 1.1pt,line join=round,fill=fillColor] (100.47, 75.81) -- ( 98.00, 80.05);

\path[draw=drawColor,line width= 1.1pt,line join=round,fill=fillColor] ( 87.44, 82.02) -- ( 81.46, 87.33);

\path[draw=drawColor,line width= 1.1pt,line join=round,fill=fillColor] ( 83.52, 60.77) -- ( 87.63, 61.35);

\path[draw=drawColor,line width= 1.1pt,line join=round,fill=fillColor] (104.78, 51.48) -- (108.30, 56.95);

\path[draw=drawColor,line width= 1.1pt,line join=round,fill=fillColor] ( 76.95, 84.66) -- ( 82.21, 86.29);

\path[draw=drawColor,line width= 1.1pt,line join=round,fill=fillColor] (101.95, 72.25) -- (103.51, 78.46);

\path[draw=drawColor,line width= 1.1pt,line join=round,fill=fillColor] ( 94.60, 81.07) -- ( 92.96, 87.50);

\path[draw=drawColor,line width= 1.1pt,line join=round,fill=fillColor] ( 67.91, 68.15) -- ( 61.67, 73.04);

\path[draw=drawColor,line width= 1.1pt,line join=round,fill=fillColor] ( 58.24, 90.18) -- ( 57.29, 88.42);

\path[draw=drawColor,line width= 1.1pt,line join=round,fill=fillColor] ( 62.83, 75.38) -- ( 59.93, 76.51);

\path[draw=drawColor,line width= 1.1pt,line join=round,fill=fillColor] ( 87.27, 60.36) -- ( 91.60, 65.63);

\path[draw=drawColor,line width= 1.1pt,line join=round,fill=fillColor] ( 75.77, 80.19) -- ( 79.83, 75.59);

\path[draw=drawColor,line width= 1.1pt,line join=round,fill=fillColor] ( 69.82, 80.99) -- ( 75.20, 84.99);

\path[draw=drawColor,line width= 1.1pt,line join=round,fill=fillColor] ( 85.98, 68.13) -- ( 90.15, 71.14);

\path[draw=drawColor,line width= 1.1pt,line join=round,fill=fillColor] ( 57.42,100.86) -- ( 60.88,103.07);

\path[draw=drawColor,line width= 1.1pt,line join=round,fill=fillColor] ( 81.26, 75.84) -- ( 82.56, 79.35);

\path[draw=drawColor,line width= 1.1pt,line join=round,fill=fillColor] ( 96.63, 61.83) -- ( 95.83, 64.48);

\path[draw=drawColor,line width= 1.1pt,line join=round,fill=fillColor] ( 77.44, 60.28) -- ( 83.24, 55.12);

\path[draw=drawColor,line width= 1.1pt,line join=round,fill=fillColor] ( 72.91, 69.57) -- ( 76.26, 69.99);

\path[draw=drawColor,line width= 1.1pt,line join=round,fill=fillColor] ( 93.88, 75.24) -- ( 90.11, 76.87);

\path[draw=drawColor,line width= 1.1pt,line join=round,fill=fillColor] ( 89.12, 69.32) -- ( 87.63, 74.65);

\path[draw=drawColor,line width= 1.1pt,line join=round,fill=fillColor] ( 73.56, 89.81) -- ( 75.73, 94.80);

\path[draw=drawColor,line width= 1.1pt,line join=round,fill=fillColor] ( 86.49, 59.09) -- ( 84.40, 55.90);

\path[draw=drawColor,line width= 1.1pt,line join=round,fill=fillColor] (108.06, 69.99) -- (109.38, 73.66);

\path[draw=drawColor,line width= 1.1pt,line join=round,fill=fillColor] ( 72.27, 82.01) -- ( 75.08, 84.64);

\path[draw=drawColor,line width= 1.1pt,line join=round,fill=fillColor] ( 73.58, 91.22) -- ( 73.30, 89.65);

\path[draw=drawColor,line width= 1.1pt,line join=round,fill=fillColor] ( 61.69, 97.18) -- ( 61.26, 98.16);

\path[draw=drawColor,line width= 1.1pt,line join=round,fill=fillColor] ( 82.32, 65.28) -- ( 80.22, 71.24);

\path[draw=drawColor,line width= 1.1pt,line join=round,fill=fillColor] ( 82.47, 69.62) -- ( 78.95, 74.58);

\path[draw=drawColor,line width= 1.1pt,line join=round,fill=fillColor] (108.52, 67.40) -- (110.71, 71.38);

\path[draw=drawColor,line width= 1.1pt,line join=round,fill=fillColor] ( 79.24, 52.57) -- ( 72.93, 56.48);

\path[draw=drawColor,line width= 1.1pt,line join=round,fill=fillColor] (107.20, 72.49) -- (110.08, 76.83);

\path[draw=drawColor,line width= 1.1pt,line join=round,fill=fillColor] ( 66.51, 91.28) -- ( 69.72, 96.88);

\path[draw=drawColor,line width= 1.1pt,line join=round,fill=fillColor] (107.35, 52.75) -- (102.15, 46.57);

\path[draw=drawColor,line width= 1.1pt,line join=round,fill=fillColor] ( 83.61, 88.85) -- ( 77.66, 84.73);

\path[draw=drawColor,line width= 1.1pt,line join=round,fill=fillColor] ( 84.15, 73.78) -- ( 83.29, 77.74);

\path[draw=drawColor,line width= 1.1pt,line join=round,fill=fillColor] ( 88.19, 84.15) -- ( 90.82, 83.96);

\path[draw=drawColor,line width= 1.1pt,line join=round,fill=fillColor] ( 77.28, 76.82) -- ( 79.45, 75.43);

\path[draw=drawColor,line width= 1.1pt,line join=round,fill=fillColor] ( 80.62, 99.86) -- ( 80.11, 98.55);

\path[draw=drawColor,line width= 1.1pt,line join=round,fill=fillColor] ( 89.07, 88.60) -- ( 86.21, 93.26);

\path[draw=drawColor,line width= 1.1pt,line join=round,fill=fillColor] ( 81.70, 81.38) -- ( 77.19, 75.59);

\path[draw=drawColor,line width= 1.1pt,line join=round,fill=fillColor] ( 75.97, 67.96) -- ( 81.35, 63.99);

\path[draw=drawColor,line width= 1.1pt,line join=round,fill=fillColor] ( 70.71, 88.84) -- ( 70.70, 89.24);

\path[draw=drawColor,line width= 1.1pt,line join=round,fill=fillColor] (106.78, 79.79) -- (102.31, 83.85);

\path[draw=drawColor,line width= 1.1pt,line join=round,fill=fillColor] ( 96.30, 82.58) -- ( 94.90, 82.31);

\path[draw=drawColor,line width= 1.1pt,line join=round,fill=fillColor] (108.72, 68.20) -- (111.51, 69.27);
\definecolor[named]{drawColor}{rgb}{1.00,0.50,0.00}
\definecolor[named]{fillColor}{rgb}{1.00,0.50,0.00}

\path[draw=drawColor,line width= 1.1pt,line join=round,fill=fillColor] ( 97.64, 76.47) -- (102.46, 73.80);

\path[draw=drawColor,line width= 1.1pt,line join=round] ( 98.85, 73.74) --
	(102.46, 73.80) --
	(100.61, 76.89);

\path[draw=drawColor,line width= 1.1pt,line join=round,fill=fillColor] (106.76, 72.65) -- (104.97, 66.58);

\path[draw=drawColor,line width= 1.1pt,line join=round] (104.12, 70.09) --
	(104.97, 66.58) --
	(107.59, 69.07);

\path[draw=drawColor,line width= 1.1pt,line join=round,fill=fillColor] (101.45, 92.16) -- (107.75, 90.83);

\path[draw=drawColor,line width= 1.1pt,line join=round] (104.31, 89.71) --
	(107.75, 90.83) --
	(105.06, 93.24);

\path[draw=drawColor,line width= 1.1pt,line join=round,fill=fillColor] ( 92.88, 86.35) -- ( 94.18, 89.34);

\path[draw=drawColor,line width= 1.1pt,line join=round] ( 94.59, 85.75) --
	( 94.18, 89.34) --
	( 91.27, 87.19);

\path[draw=drawColor,line width= 1.1pt,line join=round,fill=fillColor] ( 78.07, 62.71) -- ( 72.04, 58.20);

\path[draw=drawColor,line width= 1.1pt,line join=round] ( 73.46, 61.52) --
	( 72.04, 58.20) --
	( 75.62, 58.63);

\path[draw=drawColor,line width= 1.1pt,line join=round,fill=fillColor] ( 68.97,100.06) -- ( 63.35,102.00);

\path[draw=drawColor,line width= 1.1pt,line join=round] ( 66.90,102.68) --
	( 63.35,102.00) --
	( 65.72, 99.27);

\path[draw=drawColor,line width= 1.1pt,line join=round,fill=fillColor] (110.73, 93.38) -- (106.32, 99.44);

\path[draw=drawColor,line width= 1.1pt,line join=round] (109.62, 97.97) --
	(106.32, 99.44) --
	(106.70, 95.84);

\path[draw=drawColor,line width= 1.1pt,line join=round,fill=fillColor] ( 87.58, 90.82) -- ( 86.81, 96.20);

\path[draw=drawColor,line width= 1.1pt,line join=round] ( 89.04, 93.36) --
	( 86.81, 96.20) --
	( 85.46, 92.85);

\path[draw=drawColor,line width= 1.1pt,line join=round,fill=fillColor] (100.49, 97.53) -- (100.60,101.38);

\path[draw=drawColor,line width= 1.1pt,line join=round] (102.32, 98.20) --
	(100.60,101.38) --
	( 98.71, 98.30);

\path[draw=drawColor,line width= 1.1pt,line join=round,fill=fillColor] ( 57.75, 65.17) -- ( 61.04, 58.80);

\path[draw=drawColor,line width= 1.1pt,line join=round] ( 58.00, 60.75) --
	( 61.04, 58.80) --
	( 61.21, 62.40);

\path[draw=drawColor,line width= 1.1pt,line join=round,fill=fillColor] ( 90.50, 65.58) -- ( 96.80, 59.56);

\path[draw=drawColor,line width= 1.1pt,line join=round] ( 93.29, 60.42) --
	( 96.80, 59.56) --
	( 95.78, 63.03);

\path[draw=drawColor,line width= 1.1pt,line join=round,fill=fillColor] ( 63.06, 95.44) -- ( 58.27, 99.98);

\path[draw=drawColor,line width= 1.1pt,line join=round] ( 61.79, 99.14) --
	( 58.27, 99.98) --
	( 59.30, 96.52);

\path[draw=drawColor,line width= 1.1pt,line join=round,fill=fillColor] ( 61.02, 61.72) -- ( 57.90, 60.30);

\path[draw=drawColor,line width= 1.1pt,line join=round] ( 59.99, 63.24) --
	( 57.90, 60.30) --
	( 61.49, 59.95);

\path[draw=drawColor,line width= 1.1pt,line join=round,fill=fillColor] ( 62.29, 60.99) -- ( 58.92, 56.15);

\path[draw=drawColor,line width= 1.1pt,line join=round] ( 59.22, 59.75) --
	( 58.92, 56.15) --
	( 62.19, 57.69);

\path[draw=drawColor,line width= 1.1pt,line join=round,fill=fillColor] (103.34, 90.34) -- (104.69, 93.40);

\path[draw=drawColor,line width= 1.1pt,line join=round] (105.08, 89.81) --
	(104.69, 93.40) --
	(101.77, 91.27);

\path[draw=drawColor,line width= 1.1pt,line join=round,fill=fillColor] ( 99.99, 70.87) -- ( 97.22, 65.58);

\path[draw=drawColor,line width= 1.1pt,line join=round] ( 97.07, 69.19) --
	( 97.22, 65.58) --
	(100.27, 67.52);

\path[draw=drawColor,line width= 1.1pt,line join=round,fill=fillColor] (103.75, 77.24) -- (107.73, 74.61);

\path[draw=drawColor,line width= 1.1pt,line join=round] (104.12, 74.83) --
	(107.73, 74.61) --
	(106.11, 77.84);

\path[draw=drawColor,line width= 1.1pt,line join=round,fill=fillColor] ( 92.29, 68.75) -- ( 94.38, 62.91);

\path[draw=drawColor,line width= 1.1pt,line join=round] ( 91.62, 65.25) --
	( 94.38, 62.91) --
	( 95.03, 66.47);

\path[draw=drawColor,line width= 1.1pt,line join=round,fill=fillColor] ( 72.05, 67.57) -- ( 67.96, 66.14);

\path[draw=drawColor,line width= 1.1pt,line join=round] ( 70.31, 68.87) --
	( 67.96, 66.14) --
	( 71.51, 65.46);

\path[draw=drawColor,line width= 1.1pt,line join=round,fill=fillColor] ( 76.61, 58.44) -- ( 72.45, 57.20);

\path[draw=drawColor,line width= 1.1pt,line join=round] ( 74.93, 59.83) --
	( 72.45, 57.20) --
	( 75.96, 56.36);

\path[draw=drawColor,line width= 1.1pt,line join=round,fill=fillColor] ( 73.63, 88.77) -- ( 67.30, 89.81);

\path[draw=drawColor,line width= 1.1pt,line join=round] ( 70.69, 91.09) --
	( 67.30, 89.81) --
	( 70.10, 87.52);

\path[draw=drawColor,line width= 1.1pt,line join=round,fill=fillColor] ( 72.82, 49.81) -- ( 66.82, 52.75);

\path[draw=drawColor,line width= 1.1pt,line join=round] ( 70.43, 53.00) --
	( 66.82, 52.75) --
	( 68.84, 49.75);

\path[draw=drawColor,line width= 1.1pt,line join=round,fill=fillColor] ( 58.45, 92.17) -- ( 58.43, 95.63);

\path[draw=drawColor,line width= 1.1pt,line join=round] ( 60.26, 92.51) --
	( 58.43, 95.63) --
	( 56.65, 92.49);

\path[draw=drawColor,line width= 1.1pt,line join=round,fill=fillColor] ( 80.53,101.44) -- ( 74.33,100.68);

\path[draw=drawColor,line width= 1.1pt,line join=round] ( 77.22,102.86) --
	( 74.33,100.68) --
	( 77.66, 99.27);

\path[draw=drawColor,line width= 1.1pt,line join=round,fill=fillColor] ( 67.31, 90.43) -- ( 65.55, 92.02);

\path[draw=drawColor,line width= 1.1pt,line join=round] ( 69.08, 91.27) --
	( 65.55, 92.02) --
	( 66.67, 88.58);

\path[draw=drawColor,line width= 1.1pt,line join=round,fill=fillColor] (104.67, 96.64) -- (103.38,101.40);

\path[draw=drawColor,line width= 1.1pt,line join=round] (105.94, 98.85) --
	(103.38,101.40) --
	(102.46, 97.91);

\path[draw=drawColor,line width= 1.1pt,line join=round,fill=fillColor] (104.12, 88.46) -- (108.92, 94.84);

\path[draw=drawColor,line width= 1.1pt,line join=round] (108.48, 91.25) --
	(108.92, 94.84) --
	(105.60, 93.42);

\path[draw=drawColor,line width= 1.1pt,line join=round,fill=fillColor] ( 59.06, 88.42) -- ( 58.04, 91.44);

\path[draw=drawColor,line width= 1.1pt,line join=round] ( 60.75, 89.05) --
	( 58.04, 91.44) --
	( 57.33, 87.89);

\path[draw=drawColor,line width= 1.1pt,line join=round,fill=fillColor] ( 91.75, 61.02) -- ( 97.89, 57.24);

\path[draw=drawColor,line width= 1.1pt,line join=round] ( 94.28, 57.34) --
	( 97.89, 57.24) --
	( 96.17, 60.42);

\path[draw=drawColor,line width= 1.1pt,line join=round,fill=fillColor] ( 59.03, 58.59) -- ( 55.83, 53.60);

\path[draw=drawColor,line width= 1.1pt,line join=round] ( 56.00, 57.20) --
	( 55.83, 53.60) --
	( 59.04, 55.25);

\path[draw=drawColor,line width= 1.1pt,line join=round,fill=fillColor] ( 77.67, 59.95) -- ( 72.14, 61.13);

\path[draw=drawColor,line width= 1.1pt,line join=round] ( 75.58, 62.24) --
	( 72.14, 61.13) --
	( 74.83, 58.71);

\path[draw=drawColor,line width= 1.1pt,line join=round,fill=fillColor] ( 67.75, 61.00) -- ( 66.07, 54.85);

\path[draw=drawColor,line width= 1.1pt,line join=round] ( 65.15, 58.34) --
	( 66.07, 54.85) --
	( 68.64, 57.39);

\path[draw=drawColor,line width= 1.1pt,line join=round,fill=fillColor] ( 77.44, 53.64) -- ( 72.20, 48.31);

\path[draw=drawColor,line width= 1.1pt,line join=round] ( 73.10, 51.80) --
	( 72.20, 48.31) --
	( 75.68, 49.27);

\path[draw=drawColor,line width= 1.1pt,line join=round,fill=fillColor] ( 63.74, 73.73) -- ( 57.40, 76.04);

\path[draw=drawColor,line width= 1.1pt,line join=round] ( 60.96, 76.67) --
	( 57.40, 76.04) --
	( 59.72, 73.27);

\path[draw=drawColor,line width= 1.1pt,line join=round,fill=fillColor] ( 61.42, 63.63) -- ( 56.17, 57.33);

\path[draw=drawColor,line width= 1.1pt,line join=round] ( 56.78, 60.89) --
	( 56.17, 57.33) --
	( 59.56, 58.57);

\path[draw=drawColor,line width= 1.1pt,line join=round,fill=fillColor] ( 68.09, 84.08) -- ( 67.20, 89.86);

\path[draw=drawColor,line width= 1.1pt,line join=round] ( 69.46, 87.04) --
	( 67.20, 89.86) --
	( 65.89, 86.49);

\path[draw=drawColor,line width= 1.1pt,line join=round,fill=fillColor] ( 63.33, 57.40) -- ( 57.02, 52.79);

\path[draw=drawColor,line width= 1.1pt,line join=round] ( 58.48, 56.09) --
	( 57.02, 52.79) --
	( 60.61, 53.18);

\path[draw=drawColor,line width= 1.1pt,line join=round,fill=fillColor] ( 76.67, 61.42) -- ( 70.59, 63.07);

\path[draw=drawColor,line width= 1.1pt,line join=round] ( 74.08, 63.99) --
	( 70.59, 63.07) --
	( 73.14, 60.50);

\path[draw=drawColor,line width= 1.1pt,line join=round,fill=fillColor] ( 72.77, 69.03) -- ( 66.99, 65.63);

\path[draw=drawColor,line width= 1.1pt,line join=round] ( 68.77, 68.77) --
	( 66.99, 65.63) --
	( 70.61, 65.66);

\path[draw=drawColor,line width= 1.1pt,line join=round,fill=fillColor] ( 75.29, 76.54) -- ( 71.16, 73.76);

\path[draw=drawColor,line width= 1.1pt,line join=round] ( 72.75, 77.01) --
	( 71.16, 73.76) --
	( 74.77, 74.01);

\path[draw=drawColor,line width= 1.1pt,line join=round,fill=fillColor] (108.25, 75.69) -- (109.40, 72.55);

\path[draw=drawColor,line width= 1.1pt,line join=round] (106.63, 74.87) --
	(109.40, 72.55) --
	(110.02, 76.11);

\path[draw=drawColor,line width= 1.1pt,line join=round,fill=fillColor] ( 93.15, 76.40) -- ( 97.21, 82.56);

\path[draw=drawColor,line width= 1.1pt,line join=round] ( 97.00, 78.96) --
	( 97.21, 82.56) --
	( 93.98, 80.94);

\path[draw=drawColor,line width= 1.1pt,line join=round,fill=fillColor] (105.05, 61.27) -- (109.59, 56.32);

\path[draw=drawColor,line width= 1.1pt,line join=round] (106.14, 57.41) --
	(109.59, 56.32) --
	(108.81, 59.85);

\path[draw=drawColor,line width= 1.1pt,line join=round,fill=fillColor] ( 99.53, 71.92) -- (104.63, 71.14);

\path[draw=drawColor,line width= 1.1pt,line join=round] (101.27, 69.82) --
	(104.63, 71.14) --
	(101.81, 73.40);

\path[draw=drawColor,line width= 1.1pt,line join=round,fill=fillColor] ( 91.80, 98.53) -- ( 95.27,101.71);

\path[draw=drawColor,line width= 1.1pt,line join=round] ( 94.19, 98.27) --
	( 95.27,101.71) --
	( 91.74,100.93);

\path[draw=drawColor,line width= 1.1pt,line join=round,fill=fillColor] ( 69.74, 76.19) -- ( 64.80, 82.18);

\path[draw=drawColor,line width= 1.1pt,line join=round] ( 68.18, 80.92) --
	( 64.80, 82.18) --
	( 65.39, 78.62);

\path[draw=drawColor,line width= 1.1pt,line join=round,fill=fillColor] ( 65.47, 59.28) -- ( 65.18, 53.32);

\path[draw=drawColor,line width= 1.1pt,line join=round] ( 63.53, 56.53) --
	( 65.18, 53.32) --
	( 67.14, 56.36);

\path[draw=drawColor,line width= 1.1pt,line join=round,fill=fillColor] ( 90.09, 53.19) -- ( 91.04, 48.02);

\path[draw=drawColor,line width= 1.1pt,line join=round] ( 88.70, 50.77) --
	( 91.04, 48.02) --
	( 92.25, 51.43);

\path[draw=drawColor,line width= 1.1pt,line join=round,fill=fillColor] (104.97, 65.91) -- (107.55, 60.84);

\path[draw=drawColor,line width= 1.1pt,line join=round] (104.52, 62.81) --
	(107.55, 60.84) --
	(107.74, 64.45);

\path[draw=drawColor,line width= 1.1pt,line join=round,fill=fillColor] ( 69.45, 89.28) -- ( 63.28, 92.20);

\path[draw=drawColor,line width= 1.1pt,line join=round] ( 66.88, 92.49) --
	( 63.28, 92.20) --
	( 65.33, 89.23);
\definecolor[named]{drawColor}{rgb}{0.50,0.50,0.50}

\path[draw=drawColor,line width= 0.6pt,line join=round,line cap=round] ( 39.94, 34.03) rectangle (125.76,119.86);
\end{scope}
\begin{scope}
\path[clip] (125.76, 34.03) rectangle (211.59,119.86);
\definecolor[named]{fillColor}{rgb}{1.00,1.00,1.00}

\path[fill=fillColor] (125.76, 34.03) rectangle (211.59,119.86);
\definecolor[named]{drawColor}{rgb}{0.00,0.00,0.00}

\path[draw=drawColor,line width= 1.1pt,line join=round] (162.83, 71.74) --
	(162.00, 73.05) --
	(161.34, 75.00) --
	(161.23, 76.95) --
	(161.66, 78.90) --
	(162.64, 80.85) --
	(162.83, 81.09) --
	(164.42, 82.80) --
	(164.78, 83.09) --
	(166.73, 84.32) --
	(167.69, 84.75) --
	(168.68, 85.11) --
	(170.63, 85.53) --
	(172.58, 85.66) --
	(174.53, 85.48) --
	(176.48, 84.98) --
	(177.02, 84.75) --
	(178.43, 83.99) --
	(179.99, 82.80) --
	(180.38, 82.36) --
	(181.44, 80.85) --
	(182.18, 78.90) --
	(182.32, 76.95) --
	(181.93, 75.00) --
	(181.03, 73.05) --
	(180.38, 72.14) --
	(179.48, 71.10) --
	(178.43, 70.18) --
	(176.89, 69.15) --
	(176.48, 68.92) --
	(174.53, 68.18) --
	(172.58, 67.78) --
	(170.63, 67.73) --
	(168.68, 68.02) --
	(166.73, 68.66) --
	(165.78, 69.15) --
	(164.78, 69.81) --
	(163.32, 71.10) --
	(162.83, 71.74);

\path[draw=drawColor,line width= 1.1pt,line join=round] (156.97, 73.60) --
	(156.65, 75.00) --
	(156.57, 76.95) --
	(156.86, 78.90) --
	(156.97, 79.25) --
	(157.56, 80.85) --
	(158.67, 82.80) --
	(158.92, 83.14) --
	(160.32, 84.75) --
	(160.87, 85.27) --
	(162.67, 86.70) --
	(162.83, 86.81) --
	(164.78, 87.91) --
	(166.46, 88.65) --
	(166.73, 88.76) --
	(168.68, 89.32) --
	(170.63, 89.66) --
	(172.58, 89.80) --
	(174.53, 89.72) --
	(176.48, 89.43) --
	(178.43, 88.91) --
	(179.10, 88.65) --
	(180.38, 88.08) --
	(182.33, 86.94) --
	(182.67, 86.70) --
	(184.28, 85.25) --
	(184.75, 84.75) --
	(186.10, 82.80) --
	(186.23, 82.50) --
	(186.90, 80.85) --
	(187.27, 78.90) --
	(187.27, 76.95) --
	(186.92, 75.00) --
	(186.23, 73.09) --
	(186.22, 73.05) --
	(185.11, 71.10) --
	(184.28, 70.00) --
	(183.57, 69.15) --
	(182.33, 67.95) --
	(181.43, 67.19) --
	(180.38, 66.45) --
	(178.43, 65.33) --
	(178.24, 65.24) --
	(176.48, 64.55) --
	(174.53, 64.02) --
	(172.58, 63.74) --
	(170.63, 63.71) --
	(168.68, 63.92) --
	(166.73, 64.38) --
	(164.78, 65.09) --
	(164.47, 65.24) --
	(162.83, 66.16) --
	(161.36, 67.19) --
	(160.87, 67.61) --
	(159.36, 69.15) --
	(158.92, 69.72) --
	(158.01, 71.10) --
	(157.11, 73.05) --
	(156.97, 73.60);

\path[draw=drawColor,line width= 1.1pt,line join=round] (153.07, 73.39) --
	(152.75, 75.00) --
	(152.66, 76.95) --
	(152.89, 78.90) --
	(153.07, 79.58) --
	(153.44, 80.85) --
	(154.32, 82.80) --
	(155.02, 83.93) --
	(155.58, 84.75) --
	(156.97, 86.41) --
	(157.24, 86.70) --
	(158.92, 88.22) --
	(159.47, 88.65) --
	(160.87, 89.64) --
	(162.50, 90.60) --
	(162.83, 90.78) --
	(164.78, 91.66) --
	(166.73, 92.35) --
	(167.50, 92.55) --
	(168.68, 92.85) --
	(170.63, 93.16) --
	(172.58, 93.30) --
	(174.53, 93.27) --
	(176.48, 93.08) --
	(178.43, 92.70) --
	(178.97, 92.55) --
	(180.38, 92.13) --
	(182.33, 91.35) --
	(183.80, 90.60) --
	(184.28, 90.32) --
	(186.23, 88.97) --
	(186.63, 88.65) --
	(188.18, 87.14) --
	(188.58, 86.70) --
	(189.96, 84.75) --
	(190.13, 84.40) --
	(190.88, 82.80) --
	(191.44, 80.85) --
	(191.66, 78.90) --
	(191.58, 76.95) --
	(191.21, 75.00) --
	(190.55, 73.05) --
	(190.13, 72.18) --
	(189.59, 71.10) --
	(188.33, 69.15) --
	(188.18, 68.95) --
	(186.70, 67.19) --
	(186.23, 66.72) --
	(184.63, 65.24) --
	(184.28, 64.97) --
	(182.33, 63.58) --
	(181.86, 63.29) --
	(180.38, 62.48) --
	(178.43, 61.62) --
	(177.61, 61.34) --
	(176.48, 60.99) --
	(174.53, 60.58) --
	(172.58, 60.37) --
	(170.63, 60.36) --
	(168.68, 60.54) --
	(166.73, 60.93) --
	(165.37, 61.34) --
	(164.78, 61.53) --
	(162.83, 62.37) --
	(161.13, 63.29) --
	(160.87, 63.45) --
	(158.92, 64.84) --
	(158.43, 65.24) --
	(156.97, 66.64) --
	(156.45, 67.19) --
	(155.02, 69.05) --
	(154.96, 69.15) --
	(153.88, 71.10) --
	(153.14, 73.05) --
	(153.07, 73.39);

\path[draw=drawColor,line width= 1.1pt,line join=round] (149.17, 74.59) --
	(149.09, 75.00) --
	(148.99, 76.95) --
	(149.17, 78.86) --
	(149.17, 78.90) --
	(149.64, 80.85) --
	(150.39, 82.80) --
	(151.12, 84.19) --
	(151.43, 84.75) --
	(152.81, 86.70) --
	(153.07, 87.02) --
	(154.54, 88.65) --
	(155.02, 89.12) --
	(156.71, 90.60) --
	(156.97, 90.81) --
	(158.92, 92.20) --
	(159.49, 92.55) --
	(160.87, 93.36) --
	(162.83, 94.32) --
	(163.27, 94.50) --
	(164.78, 95.10) --
	(166.73, 95.71) --
	(168.68, 96.16) --
	(170.60, 96.45) --
	(170.63, 96.46) --
	(172.58, 96.61) --
	(174.53, 96.61) --
	(176.48, 96.47) --
	(176.60, 96.45) --
	(178.43, 96.18) --
	(180.38, 95.74) --
	(182.33, 95.14) --
	(183.98, 94.50) --
	(184.28, 94.38) --
	(186.23, 93.40) --
	(187.66, 92.55) --
	(188.18, 92.21) --
	(190.13, 90.71) --
	(190.27, 90.60) --
	(192.08, 88.78) --
	(192.20, 88.65) --
	(193.64, 86.70) --
	(194.03, 85.99) --
	(194.69, 84.75) --
	(195.39, 82.80) --
	(195.79, 80.85) --
	(195.90, 78.90) --
	(195.75, 76.95) --
	(195.35, 75.00) --
	(194.70, 73.05) --
	(194.03, 71.60) --
	(193.80, 71.10) --
	(192.66, 69.15) --
	(192.08, 68.35) --
	(191.24, 67.19) --
	(190.13, 65.92) --
	(189.51, 65.24) --
	(188.18, 63.97) --
	(187.42, 63.29) --
	(186.23, 62.34) --
	(184.83, 61.34) --
	(184.28, 60.98) --
	(182.33, 59.86) --
	(181.35, 59.39) --
	(180.38, 58.95) --
	(178.43, 58.24) --
	(176.48, 57.71) --
	(174.96, 57.44) --
	(174.53, 57.37) --
	(172.58, 57.21) --
	(170.63, 57.22) --
	(168.68, 57.41) --
	(168.52, 57.44) --
	(166.73, 57.78) --
	(164.78, 58.33) --
	(162.83, 59.06) --
	(162.12, 59.39) --
	(160.87, 60.00) --
	(158.92, 61.13) --
	(158.61, 61.34) --
	(156.97, 62.53) --
	(156.03, 63.29) --
	(155.02, 64.21) --
	(153.99, 65.24) --
	(153.07, 66.32) --
	(152.37, 67.19) --
	(151.12, 69.10) --
	(151.09, 69.15) --
	(150.14, 71.10) --
	(149.47, 73.05) --
	(149.17, 74.59);

\path[draw=drawColor,line width= 1.1pt,line join=round] (147.22, 69.44) --
	(146.45, 71.10) --
	(145.81, 73.05) --
	(145.44, 75.00) --
	(145.32, 76.95) --
	(145.45, 78.90) --
	(145.85, 80.85) --
	(146.50, 82.80) --
	(147.22, 84.31) --
	(147.43, 84.75) --
	(148.62, 86.70) --
	(149.17, 87.44) --
	(150.11, 88.65) --
	(151.12, 89.79) --
	(151.90, 90.60) --
	(153.07, 91.70) --
	(154.05, 92.55) --
	(155.02, 93.33) --
	(156.65, 94.50) --
	(156.97, 94.72) --
	(158.92, 95.92) --
	(159.93, 96.45) --
	(160.87, 96.94) --
	(162.83, 97.80) --
	(164.48, 98.40) --
	(164.78, 98.51) --
	(166.73, 99.08) --
	(168.68, 99.51) --
	(170.63, 99.80) --
	(172.58, 99.96) --
	(174.53, 99.98) --
	(176.48, 99.88) --
	(178.43, 99.66) --
	(180.38, 99.30) --
	(182.33, 98.81) --
	(183.62, 98.40) --
	(184.28, 98.19) --
	(186.23, 97.43) --
	(188.18, 96.52) --
	(188.30, 96.45) --
	(190.13, 95.42) --
	(191.55, 94.50) --
	(192.08, 94.12) --
	(194.03, 92.56) --
	(194.05, 92.55) --
	(195.99, 90.61) --
	(196.00, 90.60) --
	(197.52, 88.65) --
	(197.94, 87.97) --
	(198.68, 86.70) --
	(199.51, 84.75) --
	(199.89, 83.40) --
	(200.05, 82.80) --
	(200.33, 80.85) --
	(200.35, 78.90) --
	(200.13, 76.95) --
	(199.89, 75.91) --
	(199.68, 75.00) --
	(199.02, 73.05) --
	(198.12, 71.10) --
	(197.94, 70.77) --
	(197.02, 69.15) --
	(195.99, 67.62) --
	(195.69, 67.19) --
	(194.13, 65.24) --
	(194.03, 65.14) --
	(192.32, 63.29) --
	(192.08, 63.06) --
	(190.21, 61.34) --
	(190.13, 61.28) --
	(188.18, 59.74) --
	(187.70, 59.39) --
	(186.23, 58.40) --
	(184.60, 57.44) --
	(184.28, 57.26) --
	(182.33, 56.30) --
	(180.38, 55.52) --
	(180.30, 55.49) --
	(178.43, 54.90) --
	(176.48, 54.45) --
	(174.53, 54.17) --
	(172.58, 54.06) --
	(170.63, 54.10) --
	(168.68, 54.31) --
	(166.73, 54.68) --
	(164.78, 55.21) --
	(163.98, 55.49) --
	(162.83, 55.90) --
	(160.87, 56.75) --
	(159.55, 57.44) --
	(158.92, 57.78) --
	(156.97, 58.98) --
	(156.38, 59.39) --
	(155.02, 60.39) --
	(153.84, 61.34) --
	(153.07, 62.02) --
	(151.73, 63.29) --
	(151.12, 63.95) --
	(149.98, 65.24) --
	(149.17, 66.31) --
	(148.53, 67.19) --
	(147.36, 69.15) --
	(147.22, 69.44);

\path[draw=drawColor,line width= 1.1pt,line join=round] (143.32, 69.52) --
	(142.59, 71.10) --
	(141.94, 73.05) --
	(141.55, 75.00) --
	(141.40, 76.95) --
	(141.49, 78.90) --
	(141.82, 80.85) --
	(142.41, 82.80) --
	(143.26, 84.75) --
	(143.32, 84.86) --
	(144.33, 86.70) --
	(145.27, 88.08) --
	(145.66, 88.65) --
	(147.22, 90.58) --
	(147.24, 90.60) --
	(149.08, 92.55) --
	(149.17, 92.64) --
	(151.12, 94.43) --
	(151.21, 94.50) --
	(153.07, 96.01) --
	(153.68, 96.45) --
	(155.02, 97.41) --
	(156.60, 98.40) --
	(156.97, 98.64) --
	(158.92, 99.73) --
	(160.23,100.35) --
	(160.87,100.67) --
	(162.83,101.49) --
	(164.78,102.15) --
	(165.35,102.31) --
	(166.73,102.70) --
	(168.68,103.12) --
	(170.63,103.41) --
	(172.58,103.58) --
	(174.53,103.62) --
	(176.48,103.55) --
	(178.43,103.37) --
	(180.38,103.07) --
	(182.33,102.66) --
	(183.63,102.31) --
	(184.28,102.13) --
	(186.23,101.52) --
	(188.18,100.77) --
	(189.13,100.35) --
	(190.13, 99.91) --
	(192.08, 98.92) --
	(192.99, 98.40) --
	(194.03, 97.78) --
	(195.99, 96.47) --
	(196.02, 96.45) --
	(197.94, 94.95) --
	(198.47, 94.50) --
	(199.89, 93.14) --
	(200.46, 92.55) --
	(201.84, 90.88) --
	(202.06, 90.60) --
	(203.33, 88.65) --
	(203.79, 87.71) --
	(204.29, 86.70) --
	(204.96, 84.75) --
	(205.36, 82.80) --
	(205.51, 80.85) --
	(205.43, 78.90) --
	(205.13, 76.95) --
	(204.61, 75.00) --
	(203.86, 73.05) --
	(203.79, 72.89) --
	(202.95, 71.10) --
	(201.84, 69.19) --
	(201.81, 69.15) --
	(200.52, 67.19) --
	(199.89, 66.38) --
	(199.02, 65.24) --
	(197.94, 64.00) --
	(197.32, 63.29) --
	(195.99, 61.92) --
	(195.41, 61.34) --
	(194.03, 60.07) --
	(193.27, 59.39) --
	(192.08, 58.41) --
	(190.83, 57.44) --
	(190.13, 56.93) --
	(188.18, 55.62) --
	(187.98, 55.49) --
	(186.23, 54.45) --
	(184.44, 53.54) --
	(184.28, 53.46) --
	(182.33, 52.60) --
	(180.38, 51.92) --
	(179.15, 51.59) --
	(178.43, 51.39) --
	(176.48, 51.01) --
	(174.53, 50.79) --
	(172.58, 50.72) --
	(170.63, 50.81) --
	(168.68, 51.05) --
	(166.73, 51.45) --
	(166.24, 51.59) --
	(164.78, 51.98) --
	(162.83, 52.66) --
	(160.87, 53.50) --
	(160.79, 53.54) --
	(158.92, 54.46) --
	(157.13, 55.49) --
	(156.97, 55.58) --
	(155.02, 56.84) --
	(154.18, 57.44) --
	(153.07, 58.26) --
	(151.67, 59.39) --
	(151.12, 59.87) --
	(149.50, 61.34) --
	(149.17, 61.68) --
	(147.62, 63.29) --
	(147.22, 63.77) --
	(145.99, 65.24) --
	(145.27, 66.27) --
	(144.61, 67.19) --
	(143.49, 69.15) --
	(143.32, 69.52);

\path[draw=drawColor,line width= 1.1pt,line join=round] (207.69, 69.00) --
	(206.41, 67.19) --
	(205.74, 66.38) --
	(204.85, 65.24) --
	(203.79, 64.06) --
	(203.12, 63.29) --
	(201.84, 61.96) --
	(201.25, 61.34) --
	(199.89, 60.02) --
	(199.23, 59.39) --
	(197.94, 58.23) --
	(197.04, 57.44) --
	(195.99, 56.56) --
	(194.64, 55.49) --
	(194.03, 55.02) --
	(192.08, 53.61) --
	(191.98, 53.54) --
	(190.13, 52.30) --
	(188.95, 51.59) --
	(188.18, 51.13) --
	(186.23, 50.10) --
	(185.22, 49.64) --
	(184.28, 49.20) --
	(182.33, 48.44) --
	(180.38, 47.86) --
	(179.63, 47.69) --
	(178.43, 47.40) --
	(176.48, 47.09) --
	(174.53, 46.94) --
	(172.58, 46.94) --
	(170.63, 47.10) --
	(168.68, 47.41) --
	(167.49, 47.69) --
	(166.73, 47.85) --
	(164.78, 48.42) --
	(162.83, 49.14) --
	(161.71, 49.64) --
	(160.87, 49.99) --
	(158.92, 50.95) --
	(157.80, 51.59) --
	(156.97, 52.04) --
	(155.02, 53.25) --
	(154.60, 53.54) --
	(153.07, 54.57) --
	(151.83, 55.49) --
	(151.12, 56.02) --
	(149.35, 57.44) --
	(149.17, 57.59) --
	(147.22, 59.31) --
	(147.13, 59.39) --
	(145.27, 61.19) --
	(145.11, 61.34) --
	(143.32, 63.29) --
	(143.32, 63.29) --
	(141.71, 65.24) --
	(141.37, 65.74) --
	(140.32, 67.19) --
	(139.42, 68.74) --
	(139.16, 69.15) --
	(138.22, 71.10) --
	(137.53, 73.05) --
	(137.47, 73.34) --
	(137.06, 75.00) --
	(136.85, 76.95) --
	(136.89, 78.90) --
	(137.19, 80.85) --
	(137.47, 81.83) --
	(137.73, 82.80) --
	(138.50, 84.75) --
	(139.42, 86.49) --
	(139.53, 86.70) --
	(140.76, 88.65) --
	(141.37, 89.47) --
	(142.22, 90.60) --
	(143.32, 91.89) --
	(143.89, 92.55) --
	(145.27, 93.99) --
	(145.78, 94.50) --
	(147.22, 95.86) --
	(147.88, 96.45) --
	(149.17, 97.56) --
	(150.23, 98.40) --
	(151.12, 99.11) --
	(152.86,100.35) --
	(153.07,100.51) --
	(155.02,101.80) --
	(155.87,102.31) --
	(156.97,102.97) --
	(158.92,104.00) --
	(159.49,104.26) --
	(160.87,104.93) --
	(162.83,105.71) --
	(164.32,106.21) --
	(164.78,106.37) --
	(166.73,106.93) --
	(168.68,107.35) --
	(170.63,107.64) --
	(172.58,107.81) --
	(174.53,107.88) --
	(176.48,107.83) --
	(178.43,107.69) --
	(180.38,107.44) --
	(182.33,107.09) --
	(184.28,106.64) --
	(185.79,106.21) --
	(186.23,106.09) --
	(188.18,105.48) --
	(190.13,104.77) --
	(191.39,104.26) --
	(192.08,103.98) --
	(194.03,103.10) --
	(195.62,102.31) --
	(195.99,102.12) --
	(197.94,101.06) --
	(199.11,100.35) --
	(199.89, 99.86) --
	(201.84, 98.54) --
	(202.03, 98.40) --
	(203.79, 97.05) --
	(204.52, 96.45) --
	(205.74, 95.33) --
	(206.61, 94.50) --
	(207.69, 93.28);

\path[draw=drawColor,line width= 1.1pt,line join=round] (207.69, 59.74) --
	(207.29, 59.39) --
	(205.74, 58.10) --
	(204.97, 57.44) --
	(203.79, 56.47) --
	(202.59, 55.49) --
	(201.84, 54.89) --
	(200.13, 53.54) --
	(199.89, 53.35) --
	(197.94, 51.85) --
	(197.59, 51.59) --
	(195.99, 50.39) --
	(194.91, 49.64) --
	(194.03, 49.02) --
	(192.08, 47.76) --
	(191.97, 47.69) --
	(190.13, 46.54) --
	(188.62, 45.74) --
	(188.18, 45.49) --
	(186.23, 44.52) --
	(184.38, 43.79) --
	(184.28, 43.75) --
	(182.33, 43.05) --
	(180.38, 42.54) --
	(178.43, 42.20) --
	(176.48, 42.02) --
	(174.53, 41.99) --
	(172.58, 42.11) --
	(170.63, 42.38) --
	(168.68, 42.80) --
	(166.73, 43.39) --
	(165.69, 43.79) --
	(164.78, 44.09) --
	(162.83, 44.90) --
	(161.16, 45.74) --
	(160.87, 45.87) --
	(158.92, 46.89) --
	(157.61, 47.69) --
	(156.97, 48.05) --
	(155.02, 49.28) --
	(154.50, 49.64) --
	(153.07, 50.59) --
	(151.69, 51.59) --
	(151.12, 51.99) --
	(149.17, 53.47) --
	(149.08, 53.54) --
	(147.22, 55.01) --
	(146.63, 55.49) --
	(145.27, 56.64) --
	(144.33, 57.44) --
	(143.32, 58.35) --
	(142.17, 59.39) --
	(141.37, 60.17) --
	(140.15, 61.34) --
	(139.42, 62.12) --
	(138.27, 63.29) --
	(137.47, 64.24) --
	(136.56, 65.24) --
	(135.52, 66.61) --
	(135.04, 67.19) --
	(133.72, 69.15) --
	(133.57, 69.44) --
	(132.58, 71.10) --
	(131.74, 73.05) --
	(131.62, 73.48) --
	(131.11, 75.00) --
	(130.76, 76.95) --
	(130.70, 78.90) --
	(130.93, 80.85) --
	(131.46, 82.80) --
	(131.62, 83.16) --
	(132.21, 84.75) --
	(133.25, 86.70) --
	(133.57, 87.17) --
	(134.50, 88.65) --
	(135.52, 89.98) --
	(135.97, 90.60) --
	(137.47, 92.35) --
	(137.64, 92.55) --
	(139.42, 94.46) --
	(139.46, 94.50) --
	(141.37, 96.41) --
	(141.41, 96.45) --
	(143.32, 98.23) --
	(143.51, 98.40) --
	(145.27, 99.97) --
	(145.73,100.35) --
	(147.22,101.61) --
	(148.10,102.31) --
	(149.17,103.17) --
	(150.63,104.26) --
	(151.12,104.63) --
	(153.07,106.01) --
	(153.38,106.21) --
	(155.02,107.33) --
	(156.43,108.16) --
	(156.97,108.51) --
	(158.92,109.60) --
	(159.99,110.11) --
	(160.87,110.58) --
	(162.83,111.43) --
	(164.61,112.06) --
	(164.78,112.13) --
	(166.73,112.76) --
	(168.68,113.23) --
	(170.63,113.55) --
	(172.58,113.75) --
	(174.53,113.83) --
	(176.48,113.80) --
	(178.43,113.68) --
	(180.38,113.46) --
	(182.33,113.14) --
	(184.28,112.73) --
	(186.23,112.23) --
	(186.80,112.06) --
	(188.18,111.70) --
	(190.13,111.11) --
	(192.08,110.45) --
	(193.00,110.11) --
	(194.03,109.75) --
	(195.99,109.03) --
	(197.94,108.23) --
	(198.12,108.16) --
	(199.89,107.45) --
	(201.84,106.61) --
	(202.75,106.21) --
	(203.79,105.75) --
	(205.74,104.86) --
	(207.03,104.26) --
	(207.69,103.93);

\path[draw=drawColor,line width= 1.1pt,line join=round] (163.69, 37.94) --
	(162.83, 38.46) --
	(160.90, 39.89) --
	(160.87, 39.90) --
	(158.92, 41.26) --
	(158.22, 41.84) --
	(156.97, 42.70) --
	(155.59, 43.79) --
	(155.02, 44.18) --
	(153.07, 45.69) --
	(153.01, 45.74) --
	(151.12, 47.17) --
	(150.48, 47.69) --
	(149.17, 48.69) --
	(147.99, 49.64) --
	(147.22, 50.23) --
	(145.52, 51.59) --
	(145.27, 51.79) --
	(143.32, 53.35) --
	(143.07, 53.54) --
	(141.37, 54.90) --
	(140.62, 55.49) --
	(139.42, 56.47) --
	(138.19, 57.44) --
	(137.47, 58.05) --
	(135.78, 59.39) --
	(135.52, 59.62) --
	(133.57, 61.19) --
	(133.35, 61.34) --
	(131.62, 62.75) --
	(130.87, 63.29) --
	(129.67, 64.30);

\path[draw=drawColor,line width= 1.1pt,line join=round] (129.67, 95.59) --
	(130.83, 96.45) --
	(131.62, 97.00) --
	(133.47, 98.40) --
	(133.57, 98.47) --
	(135.52,100.04) --
	(135.90,100.35) --
	(137.47,101.66) --
	(138.23,102.31) --
	(139.42,103.33) --
	(140.49,104.26) --
	(141.37,105.05) --
	(142.67,106.21) --
	(143.32,106.82) --
	(144.77,108.16) --
	(145.27,108.67) --
	(146.77,110.11) --
	(147.22,110.60) --
	(148.66,112.06) --
	(149.17,112.68) --
	(150.39,114.01) --
	(151.12,115.00) --
	(151.93,115.96);

\path[draw=drawColor,line width= 1.1pt,line join=round] (195.97, 37.94) --
	(195.99, 37.95) --
	(197.93, 39.89) --
	(197.94, 39.89) --
	(199.89, 41.57) --
	(200.17, 41.84) --
	(201.84, 43.11) --
	(202.69, 43.79) --
	(203.79, 44.54) --
	(205.53, 45.74) --
	(205.74, 45.86) --
	(207.69, 46.88);

\path[draw=drawColor,line width= 1.1pt,line join=round] (129.67, 55.31) --
	(131.62, 54.55) --
	(133.57, 53.64) --
	(133.75, 53.54) --
	(135.52, 52.53) --
	(137.04, 51.59) --
	(137.47, 51.31) --
	(139.42, 49.96) --
	(139.86, 49.64) --
	(141.37, 48.47) --
	(142.37, 47.69) --
	(143.32, 46.88) --
	(144.66, 45.74) --
	(145.27, 45.16) --
	(146.74, 43.79) --
	(147.22, 43.26) --
	(148.58, 41.84) --
	(149.17, 41.09) --
	(150.16, 39.89) --
	(151.12, 38.40) --
	(151.44, 37.94);
\definecolor[named]{drawColor}{rgb}{0.00,0.00,0.00}

\path[draw=drawColor,draw opacity=0.33,line width= 1.1pt,line join=round] (141.37, 75.18) --
	(139.60, 76.95) --
	(141.37, 78.71) --
	(141.56, 78.90) --
	(143.32, 80.66) --
	(143.51, 80.85) --
	(145.27, 82.61) --
	(145.46, 82.80) --
	(147.22, 84.56) --
	(147.41, 84.75) --
	(149.17, 86.51) --
	(149.36, 86.70) --
	(151.12, 88.46) --
	(151.31, 88.65) --
	(153.07, 90.41) --
	(153.26, 90.60) --
	(155.02, 92.37) --
	(155.21, 92.55) --
	(156.97, 94.32) --
	(157.16, 94.50) --
	(158.92, 96.27) --
	(159.11, 96.45) --
	(160.87, 98.22) --
	(161.06, 98.40) --
	(162.83,100.17) --
	(163.01,100.35) --
	(164.78,102.12) --
	(164.96,102.31) --
	(166.73,104.07) --
	(166.91,104.26) --
	(168.68,106.02) --
	(170.44,104.26) --
	(170.63,104.07) --
	(172.39,102.31) --
	(172.58,102.12) --
	(174.34,100.35) --
	(174.53,100.17) --
	(176.29, 98.40) --
	(176.48, 98.22) --
	(178.24, 96.45) --
	(178.43, 96.27) --
	(180.19, 94.50) --
	(180.38, 94.32) --
	(182.14, 92.55) --
	(182.33, 92.37) --
	(184.09, 90.60) --
	(184.28, 90.41) --
	(186.05, 88.65) --
	(186.23, 88.46) --
	(188.00, 86.70) --
	(188.18, 86.51) --
	(189.95, 84.75) --
	(190.13, 84.56) --
	(191.90, 82.80) --
	(192.08, 82.61) --
	(193.85, 80.85) --
	(194.03, 80.66) --
	(195.80, 78.90) --
	(195.99, 78.71) --
	(197.75, 76.95) --
	(195.99, 75.18) --
	(195.80, 75.00) --
	(194.03, 73.23) --
	(193.85, 73.05) --
	(192.08, 71.28) --
	(191.90, 71.10) --
	(190.13, 69.33) --
	(189.95, 69.15) --
	(188.18, 67.38) --
	(188.00, 67.19) --
	(186.23, 65.43) --
	(186.05, 65.24) --
	(184.28, 63.48) --
	(184.09, 63.29) --
	(182.33, 61.53) --
	(182.14, 61.34) --
	(180.38, 59.58) --
	(180.19, 59.39) --
	(178.43, 57.63) --
	(178.24, 57.44) --
	(176.48, 55.68) --
	(176.29, 55.49) --
	(174.53, 53.73) --
	(174.34, 53.54) --
	(172.58, 51.78) --
	(172.39, 51.59) --
	(170.63, 49.83) --
	(170.44, 49.64) --
	(168.68, 47.88) --
	(166.91, 49.64) --
	(166.73, 49.83) --
	(164.96, 51.59) --
	(164.78, 51.78) --
	(163.01, 53.54) --
	(162.83, 53.73) --
	(161.06, 55.49) --
	(160.87, 55.68) --
	(159.11, 57.44) --
	(158.92, 57.63) --
	(157.16, 59.39) --
	(156.97, 59.58) --
	(155.21, 61.34) --
	(155.02, 61.53) --
	(153.26, 63.29) --
	(153.07, 63.48) --
	(151.31, 65.24) --
	(151.12, 65.43) --
	(149.36, 67.19) --
	(149.17, 67.38) --
	(147.41, 69.15) --
	(147.22, 69.33) --
	(145.46, 71.10) --
	(145.27, 71.28) --
	(143.51, 73.05) --
	(143.32, 73.23) --
	(141.56, 75.00) --
	(141.37, 75.18);

\path[draw=drawColor,draw opacity=0.33,line width= 1.1pt,line join=round] (166.57, 37.94) --
	(164.78, 39.73) --
	(164.62, 39.89) --
	(162.83, 41.68) --
	(162.67, 41.84) --
	(160.87, 43.63) --
	(160.72, 43.79) --
	(158.92, 45.58) --
	(158.77, 45.74) --
	(156.97, 47.53) --
	(156.82, 47.69) --
	(155.02, 49.48) --
	(154.87, 49.64) --
	(153.07, 51.43) --
	(152.92, 51.59) --
	(151.12, 53.38) --
	(150.97, 53.54) --
	(149.17, 55.33) --
	(149.01, 55.49) --
	(147.22, 57.29) --
	(147.06, 57.44) --
	(145.27, 59.24) --
	(145.11, 59.39) --
	(143.32, 61.19) --
	(143.16, 61.34) --
	(141.37, 63.14) --
	(141.21, 63.29) --
	(139.42, 65.09) --
	(139.26, 65.24) --
	(137.47, 67.04) --
	(137.31, 67.19) --
	(135.52, 68.99) --
	(135.36, 69.15) --
	(133.57, 70.94) --
	(133.41, 71.10) --
	(131.62, 72.89) --
	(131.46, 73.05) --
	(129.67, 74.84);

\path[draw=drawColor,draw opacity=0.33,line width= 1.1pt,line join=round] (129.67, 79.05) --
	(131.46, 80.85) --
	(131.62, 81.01) --
	(133.41, 82.80) --
	(133.57, 82.96) --
	(135.36, 84.75) --
	(135.52, 84.91) --
	(137.31, 86.70) --
	(137.47, 86.86) --
	(139.26, 88.65) --
	(139.42, 88.81) --
	(141.21, 90.60) --
	(141.37, 90.76) --
	(143.16, 92.55) --
	(143.32, 92.71) --
	(145.11, 94.50) --
	(145.27, 94.66) --
	(147.06, 96.45) --
	(147.22, 96.61) --
	(149.01, 98.40) --
	(149.17, 98.56) --
	(150.97,100.35) --
	(151.12,100.51) --
	(152.92,102.31) --
	(153.07,102.46) --
	(154.87,104.26) --
	(155.02,104.41) --
	(156.82,106.21) --
	(156.97,106.36) --
	(158.77,108.16) --
	(158.92,108.31) --
	(160.72,110.11) --
	(160.87,110.26) --
	(162.67,112.06) --
	(162.83,112.21) --
	(164.62,114.01) --
	(164.78,114.17) --
	(166.57,115.96);

\path[draw=drawColor,draw opacity=0.33,line width= 1.1pt,line join=round] (170.78, 37.94) --
	(172.58, 39.73) --
	(172.73, 39.89) --
	(174.53, 41.68) --
	(174.69, 41.84) --
	(176.48, 43.63) --
	(176.64, 43.79) --
	(178.43, 45.58) --
	(178.59, 45.74) --
	(180.38, 47.53) --
	(180.54, 47.69) --
	(182.33, 49.48) --
	(182.49, 49.64) --
	(184.28, 51.43) --
	(184.44, 51.59) --
	(186.23, 53.38) --
	(186.39, 53.54) --
	(188.18, 55.33) --
	(188.34, 55.49) --
	(190.13, 57.29) --
	(190.29, 57.44) --
	(192.08, 59.24) --
	(192.24, 59.39) --
	(194.03, 61.19) --
	(194.19, 61.34) --
	(195.99, 63.14) --
	(196.14, 63.29) --
	(197.94, 65.09) --
	(198.09, 65.24) --
	(199.89, 67.04) --
	(200.04, 67.19) --
	(201.84, 68.99) --
	(201.99, 69.15) --
	(203.79, 70.94) --
	(203.94, 71.10) --
	(205.74, 72.89) --
	(205.89, 73.05) --
	(207.69, 74.84);

\path[draw=drawColor,draw opacity=0.33,line width= 1.1pt,line join=round] (207.69, 79.05) --
	(205.89, 80.85) --
	(205.74, 81.01) --
	(203.94, 82.80) --
	(203.79, 82.96) --
	(201.99, 84.75) --
	(201.84, 84.91) --
	(200.04, 86.70) --
	(199.89, 86.86) --
	(198.09, 88.65) --
	(197.94, 88.81) --
	(196.14, 90.60) --
	(195.99, 90.76) --
	(194.19, 92.55) --
	(194.03, 92.71) --
	(192.24, 94.50) --
	(192.08, 94.66) --
	(190.29, 96.45) --
	(190.13, 96.61) --
	(188.34, 98.40) --
	(188.18, 98.56) --
	(186.39,100.35) --
	(186.23,100.51) --
	(184.44,102.31) --
	(184.28,102.46) --
	(182.49,104.26) --
	(182.33,104.41) --
	(180.54,106.21) --
	(180.38,106.36) --
	(178.59,108.16) --
	(178.43,108.31) --
	(176.64,110.11) --
	(176.48,110.26) --
	(174.69,112.06) --
	(174.53,112.21) --
	(172.73,114.01) --
	(172.58,114.17) --
	(170.78,115.96);

\path[draw=drawColor,draw opacity=0.33,line width= 1.1pt,line join=round] (157.33, 37.94) --
	(156.97, 38.29) --
	(155.38, 39.89) --
	(155.02, 40.24) --
	(153.43, 41.84) --
	(153.07, 42.19) --
	(151.48, 43.79) --
	(151.12, 44.14) --
	(149.53, 45.74) --
	(149.17, 46.10) --
	(147.58, 47.69) --
	(147.22, 48.05) --
	(145.63, 49.64) --
	(145.27, 50.00) --
	(143.68, 51.59) --
	(143.32, 51.95) --
	(141.73, 53.54) --
	(141.37, 53.90) --
	(139.78, 55.49) --
	(139.42, 55.85) --
	(137.82, 57.44) --
	(137.47, 57.80) --
	(135.87, 59.39) --
	(135.52, 59.75) --
	(133.92, 61.34) --
	(133.57, 61.70) --
	(131.97, 63.29) --
	(131.62, 63.65) --
	(130.02, 65.24) --
	(129.67, 65.60);

\path[draw=drawColor,draw opacity=0.33,line width= 1.1pt,line join=round] (129.67, 88.29) --
	(130.02, 88.65) --
	(131.62, 90.24) --
	(131.97, 90.60) --
	(133.57, 92.20) --
	(133.92, 92.55) --
	(135.52, 94.15) --
	(135.87, 94.50) --
	(137.47, 96.10) --
	(137.82, 96.45) --
	(139.42, 98.05) --
	(139.78, 98.40) --
	(141.37,100.00) --
	(141.73,100.35) --
	(143.32,101.95) --
	(143.68,102.31) --
	(145.27,103.90) --
	(145.63,104.26) --
	(147.22,105.85) --
	(147.58,106.21) --
	(149.17,107.80) --
	(149.53,108.16) --
	(151.12,109.75) --
	(151.48,110.11) --
	(153.07,111.70) --
	(153.43,112.06) --
	(155.02,113.65) --
	(155.38,114.01) --
	(156.97,115.60) --
	(157.33,115.96);

\path[draw=drawColor,draw opacity=0.33,line width= 1.1pt,line join=round] (180.02, 37.94) --
	(180.38, 38.29) --
	(181.97, 39.89) --
	(182.33, 40.24) --
	(183.92, 41.84) --
	(184.28, 42.19) --
	(185.88, 43.79) --
	(186.23, 44.14) --
	(187.83, 45.74) --
	(188.18, 46.10) --
	(189.78, 47.69) --
	(190.13, 48.05) --
	(191.73, 49.64) --
	(192.08, 50.00) --
	(193.68, 51.59) --
	(194.03, 51.95) --
	(195.63, 53.54) --
	(195.99, 53.90) --
	(197.58, 55.49) --
	(197.94, 55.85) --
	(199.53, 57.44) --
	(199.89, 57.80) --
	(201.48, 59.39) --
	(201.84, 59.75) --
	(203.43, 61.34) --
	(203.79, 61.70) --
	(205.38, 63.29) --
	(205.74, 63.65) --
	(207.33, 65.24) --
	(207.69, 65.60);

\path[draw=drawColor,draw opacity=0.33,line width= 1.1pt,line join=round] (207.69, 88.29) --
	(207.33, 88.65) --
	(205.74, 90.24) --
	(205.38, 90.60) --
	(203.79, 92.20) --
	(203.43, 92.55) --
	(201.84, 94.15) --
	(201.48, 94.50) --
	(199.89, 96.10) --
	(199.53, 96.45) --
	(197.94, 98.05) --
	(197.58, 98.40) --
	(195.99,100.00) --
	(195.63,100.35) --
	(194.03,101.95) --
	(193.68,102.31) --
	(192.08,103.90) --
	(191.73,104.26) --
	(190.13,105.85) --
	(189.78,106.21) --
	(188.18,107.80) --
	(187.83,108.16) --
	(186.23,109.75) --
	(185.88,110.11) --
	(184.28,111.70) --
	(183.92,112.06) --
	(182.33,113.65) --
	(181.97,114.01) --
	(180.38,115.60) --
	(180.02,115.96);

\path[draw=drawColor,draw opacity=0.33,line width= 1.1pt,line join=round] (149.54, 37.94) --
	(149.17, 38.30) --
	(147.59, 39.89) --
	(147.22, 40.25) --
	(145.64, 41.84) --
	(145.27, 42.20) --
	(143.69, 43.79) --
	(143.32, 44.15) --
	(141.74, 45.74) --
	(141.37, 46.11) --
	(139.79, 47.69) --
	(139.42, 48.06) --
	(137.83, 49.64) --
	(137.47, 50.01) --
	(135.88, 51.59) --
	(135.52, 51.96) --
	(133.93, 53.54) --
	(133.57, 53.91) --
	(131.98, 55.49) --
	(131.62, 55.86) --
	(130.03, 57.44) --
	(129.67, 57.81);

\path[draw=drawColor,draw opacity=0.33,line width= 1.1pt,line join=round] (129.67, 96.09) --
	(130.03, 96.45) --
	(131.62, 98.04) --
	(131.98, 98.40) --
	(133.57, 99.99) --
	(133.93,100.35) --
	(135.52,101.94) --
	(135.88,102.31) --
	(137.47,103.89) --
	(137.83,104.26) --
	(139.42,105.84) --
	(139.79,106.21) --
	(141.37,107.79) --
	(141.74,108.16) --
	(143.32,109.74) --
	(143.69,110.11) --
	(145.27,111.69) --
	(145.64,112.06) --
	(147.22,113.64) --
	(147.59,114.01) --
	(149.17,115.59) --
	(149.54,115.96);

\path[draw=drawColor,draw opacity=0.33,line width= 1.1pt,line join=round] (187.82, 37.94) --
	(188.18, 38.30) --
	(189.77, 39.89) --
	(190.13, 40.25) --
	(191.72, 41.84) --
	(192.08, 42.20) --
	(193.67, 43.79) --
	(194.03, 44.15) --
	(195.62, 45.74) --
	(195.99, 46.11) --
	(197.57, 47.69) --
	(197.94, 48.06) --
	(199.52, 49.64) --
	(199.89, 50.01) --
	(201.47, 51.59) --
	(201.84, 51.96) --
	(203.42, 53.54) --
	(203.79, 53.91) --
	(205.37, 55.49) --
	(205.74, 55.86) --
	(207.32, 57.44) --
	(207.69, 57.81);

\path[draw=drawColor,draw opacity=0.33,line width= 1.1pt,line join=round] (207.69, 96.09) --
	(207.32, 96.45) --
	(205.74, 98.04) --
	(205.37, 98.40) --
	(203.79, 99.99) --
	(203.42,100.35) --
	(201.84,101.94) --
	(201.47,102.31) --
	(199.89,103.89) --
	(199.52,104.26) --
	(197.94,105.84) --
	(197.57,106.21) --
	(195.99,107.79) --
	(195.62,108.16) --
	(194.03,109.74) --
	(193.67,110.11) --
	(192.08,111.69) --
	(191.72,112.06) --
	(190.13,113.64) --
	(189.77,114.01) --
	(188.18,115.59) --
	(187.82,115.96);

\path[draw=drawColor,draw opacity=0.33,line width= 1.1pt,line join=round] (142.68, 37.94) --
	(141.37, 39.24) --
	(140.72, 39.89) --
	(139.42, 41.19) --
	(138.77, 41.84) --
	(137.47, 43.14) --
	(136.82, 43.79) --
	(135.52, 45.09) --
	(134.87, 45.74) --
	(133.57, 47.05) --
	(132.92, 47.69) --
	(131.62, 49.00) --
	(130.97, 49.64) --
	(129.67, 50.95);

\path[draw=drawColor,draw opacity=0.33,line width= 1.1pt,line join=round] (129.67,102.95) --
	(130.97,104.26) --
	(131.62,104.90) --
	(132.92,106.21) --
	(133.57,106.85) --
	(134.87,108.16) --
	(135.52,108.80) --
	(136.82,110.11) --
	(137.47,110.75) --
	(138.77,112.06) --
	(139.42,112.70) --
	(140.72,114.01) --
	(141.37,114.65) --
	(142.68,115.96);

\path[draw=drawColor,draw opacity=0.33,line width= 1.1pt,line join=round] (194.68, 37.94) --
	(195.99, 39.24) --
	(196.63, 39.89) --
	(197.94, 41.19) --
	(198.58, 41.84) --
	(199.89, 43.14) --
	(200.53, 43.79) --
	(201.84, 45.09) --
	(202.48, 45.74) --
	(203.79, 47.05) --
	(204.43, 47.69) --
	(205.74, 49.00) --
	(206.38, 49.64) --
	(207.69, 50.95);

\path[draw=drawColor,draw opacity=0.33,line width= 1.1pt,line join=round] (207.69,102.95) --
	(206.38,104.26) --
	(205.74,104.90) --
	(204.43,106.21) --
	(203.79,106.85) --
	(202.48,108.16) --
	(201.84,108.80) --
	(200.53,110.11) --
	(199.89,110.75) --
	(198.58,112.06) --
	(197.94,112.70) --
	(196.63,114.01) --
	(195.99,114.65) --
	(194.68,115.96);

\path[draw=drawColor,draw opacity=0.33,line width= 1.1pt,line join=round] (136.47, 37.94) --
	(135.52, 38.89) --
	(134.52, 39.89) --
	(133.57, 40.84) --
	(132.57, 41.84) --
	(131.62, 42.79) --
	(130.62, 43.79) --
	(129.67, 44.74);

\path[draw=drawColor,draw opacity=0.33,line width= 1.1pt,line join=round] (129.67,109.15) --
	(130.62,110.11) --
	(131.62,111.11) --
	(132.57,112.06) --
	(133.57,113.06) --
	(134.52,114.01) --
	(135.52,115.01) --
	(136.47,115.96);

\path[draw=drawColor,draw opacity=0.33,line width= 1.1pt,line join=round] (200.88, 37.94) --
	(201.84, 38.89) --
	(202.83, 39.89) --
	(203.79, 40.84) --
	(204.79, 41.84) --
	(205.74, 42.79) --
	(206.74, 43.79) --
	(207.69, 44.74);

\path[draw=drawColor,draw opacity=0.33,line width= 1.1pt,line join=round] (207.69,109.15) --
	(206.74,110.11) --
	(205.74,111.11) --
	(204.79,112.06) --
	(203.79,113.06) --
	(202.83,114.01) --
	(201.84,115.01) --
	(200.88,115.96);

\path[draw=drawColor,draw opacity=0.33,line width= 1.1pt,line join=round] (130.76, 37.94) --
	(129.67, 39.03);

\path[draw=drawColor,draw opacity=0.33,line width= 1.1pt,line join=round] (129.67,114.86) --
	(130.76,115.96);

\path[draw=drawColor,draw opacity=0.33,line width= 1.1pt,line join=round] (206.59, 37.94) --
	(207.69, 39.03);

\path[draw=drawColor,draw opacity=0.33,line width= 1.1pt,line join=round] (207.69,114.86) --
	(206.59,115.96);
\definecolor[named]{drawColor}{rgb}{0.50,0.50,0.50}

\path[draw=drawColor,line width= 0.6pt,line join=round,line cap=round] (125.76, 34.03) rectangle (211.59,119.86);
\end{scope}
\begin{scope}
\path[clip] (211.59, 34.03) rectangle (297.42,119.86);
\definecolor[named]{fillColor}{rgb}{1.00,1.00,1.00}

\path[fill=fillColor] (211.59, 34.03) rectangle (297.42,119.86);
\definecolor[named]{drawColor}{rgb}{0.00,0.00,0.00}

\path[draw=drawColor,line width= 1.1pt,line join=round] (242.80, 72.98) --
	(242.76, 73.05) --
	(242.01, 75.00) --
	(241.58, 76.95) --
	(241.50, 78.90) --
	(241.76, 80.85) --
	(242.37, 82.80) --
	(242.80, 83.69) --
	(243.35, 84.75) --
	(244.67, 86.70) --
	(244.75, 86.80) --
	(246.43, 88.65) --
	(246.70, 88.91) --
	(248.65, 90.53) --
	(248.76, 90.60) --
	(250.60, 91.73) --
	(252.41, 92.55) --
	(252.55, 92.61) --
	(254.50, 93.13) --
	(256.45, 93.33) --
	(258.40, 93.22) --
	(260.35, 92.81) --
	(261.10, 92.55) --
	(262.31, 92.11) --
	(264.26, 91.14) --
	(265.14, 90.60) --
	(266.21, 89.90) --
	(267.85, 88.65) --
	(268.16, 88.39) --
	(269.93, 86.70) --
	(270.11, 86.50) --
	(271.51, 84.75) --
	(272.06, 83.86) --
	(272.66, 82.80) --
	(273.38, 80.85) --
	(273.66, 78.90) --
	(273.51, 76.95) --
	(272.96, 75.00) --
	(272.07, 73.05) --
	(272.06, 73.04) --
	(270.77, 71.10) --
	(270.11, 70.26) --
	(269.14, 69.15) --
	(268.16, 68.13) --
	(267.15, 67.19) --
	(266.21, 66.39) --
	(264.67, 65.24) --
	(264.26, 64.95) --
	(262.31, 63.82) --
	(261.06, 63.29) --
	(260.35, 63.00) --
	(258.40, 62.51) --
	(256.45, 62.38) --
	(254.50, 62.63) --
	(252.55, 63.27) --
	(252.50, 63.29) --
	(250.60, 64.31) --
	(249.27, 65.24) --
	(248.65, 65.71) --
	(247.02, 67.19) --
	(246.70, 67.51) --
	(245.27, 69.15) --
	(244.75, 69.83) --
	(243.87, 71.10) --
	(242.80, 72.98);

\path[draw=drawColor,line width= 1.1pt,line join=round] (233.05, 76.09) --
	(232.87, 76.95) --
	(232.86, 78.90) --
	(233.05, 79.96) --
	(233.20, 80.85) --
	(233.89, 82.80) --
	(234.87, 84.75) --
	(235.00, 84.94) --
	(236.12, 86.70) --
	(236.95, 87.81) --
	(237.58, 88.65) --
	(238.90, 90.21) --
	(239.23, 90.60) --
	(240.85, 92.33) --
	(241.07, 92.55) --
	(242.80, 94.23) --
	(243.10, 94.50) --
	(244.75, 95.94) --
	(245.40, 96.45) --
	(246.70, 97.46) --
	(248.11, 98.40) --
	(248.65, 98.77) --
	(250.60, 99.83) --
	(251.91,100.35) --
	(252.55,100.61) --
	(254.50,101.09) --
	(256.45,101.24) --
	(258.40,101.06) --
	(260.35,100.57) --
	(260.90,100.35) --
	(262.31, 99.81) --
	(264.26, 98.81) --
	(264.90, 98.40) --
	(266.21, 97.61) --
	(267.89, 96.45) --
	(268.16, 96.27) --
	(270.11, 94.80) --
	(270.48, 94.50) --
	(272.06, 93.21) --
	(272.83, 92.55) --
	(274.01, 91.50) --
	(274.98, 90.60) --
	(275.96, 89.63) --
	(276.92, 88.65) --
	(277.91, 87.49) --
	(278.58, 86.70) --
	(279.86, 84.84) --
	(279.92, 84.75) --
	(280.88, 82.80) --
	(281.43, 80.85) --
	(281.56, 78.90) --
	(281.28, 76.95) --
	(280.63, 75.00) --
	(279.86, 73.46) --
	(279.66, 73.05) --
	(278.42, 71.10) --
	(277.91, 70.42) --
	(276.95, 69.15) --
	(275.96, 67.96) --
	(275.32, 67.19) --
	(274.01, 65.75) --
	(273.54, 65.24) --
	(272.06, 63.70) --
	(271.65, 63.29) --
	(270.11, 61.76) --
	(269.66, 61.34) --
	(268.16, 59.92) --
	(267.55, 59.39) --
	(266.21, 58.18) --
	(265.28, 57.44) --
	(264.26, 56.57) --
	(262.76, 55.49) --
	(262.31, 55.13) --
	(260.35, 53.94) --
	(259.41, 53.54) --
	(258.40, 53.07) --
	(256.45, 52.63) --
	(254.50, 52.69) --
	(252.55, 53.26) --
	(252.05, 53.54) --
	(250.60, 54.28) --
	(248.95, 55.49) --
	(248.65, 55.70) --
	(246.70, 57.39) --
	(246.65, 57.44) --
	(244.75, 59.27) --
	(244.63, 59.39) --
	(242.80, 61.30) --
	(242.76, 61.34) --
	(240.98, 63.29) --
	(240.85, 63.45) --
	(239.30, 65.24) --
	(238.90, 65.74) --
	(237.72, 67.19) --
	(236.95, 68.24) --
	(236.27, 69.15) --
	(235.01, 71.10) --
	(235.00, 71.12) --
	(233.99, 73.05) --
	(233.26, 75.00) --
	(233.05, 76.09);

\path[draw=drawColor,line width= 1.1pt,line join=round] (249.38, 37.94) --
	(248.81, 39.89) --
	(248.65, 40.25) --
	(248.14, 41.84) --
	(247.30, 43.79) --
	(246.70, 44.88) --
	(246.31, 45.74) --
	(245.22, 47.69) --
	(244.75, 48.39) --
	(244.02, 49.64) --
	(242.80, 51.43) --
	(242.70, 51.59) --
	(241.31, 53.54) --
	(240.85, 54.13) --
	(239.83, 55.49) --
	(238.90, 56.63) --
	(238.24, 57.44) --
	(236.95, 58.96) --
	(236.58, 59.39) --
	(235.00, 61.17) --
	(234.83, 61.34) --
	(233.05, 63.27) --
	(233.03, 63.29) --
	(231.17, 65.24) --
	(231.10, 65.33) --
	(229.31, 67.19) --
	(229.15, 67.38) --
	(227.49, 69.15) --
	(227.19, 69.52) --
	(225.82, 71.10) --
	(225.24, 71.95) --
	(224.43, 73.05) --
	(223.49, 75.00) --
	(223.29, 75.96) --
	(223.08, 76.95) --
	(223.27, 78.90) --
	(223.29, 78.95) --
	(223.98, 80.85) --
	(225.15, 82.80) --
	(225.24, 82.92) --
	(226.59, 84.75) --
	(227.19, 85.42) --
	(228.25, 86.70) --
	(229.15, 87.64) --
	(230.04, 88.65) --
	(231.10, 89.73) --
	(231.90, 90.60) --
	(233.05, 91.75) --
	(233.81, 92.55) --
	(235.00, 93.73) --
	(235.75, 94.50) --
	(236.95, 95.69) --
	(237.71, 96.45) --
	(238.90, 97.63) --
	(239.70, 98.40) --
	(240.85, 99.54) --
	(241.71,100.35) --
	(242.80,101.41) --
	(243.79,102.31) --
	(244.75,103.21) --
	(245.97,104.26) --
	(246.70,104.92) --
	(248.34,106.21) --
	(248.65,106.47) --
	(250.60,107.82) --
	(251.27,108.16) --
	(252.55,108.88) --
	(254.50,109.57) --
	(256.45,109.81) --
	(258.40,109.59) --
	(260.35,108.91) --
	(261.68,108.16) --
	(262.31,107.83) --
	(264.26,106.48) --
	(264.58,106.21) --
	(266.21,104.95) --
	(266.99,104.26) --
	(268.16,103.31) --
	(269.28,102.31) --
	(270.11,101.62) --
	(271.53,100.35) --
	(272.06, 99.91) --
	(273.79, 98.40) --
	(274.01, 98.22) --
	(275.96, 96.53) --
	(276.05, 96.45) --
	(277.91, 94.85) --
	(278.32, 94.50) --
	(279.86, 93.15) --
	(280.56, 92.55) --
	(281.81, 91.41) --
	(282.72, 90.60) --
	(283.76, 89.57) --
	(284.72, 88.65) --
	(285.71, 87.52) --
	(286.46, 86.70) --
	(287.66, 84.98) --
	(287.83, 84.75) --
	(288.76, 82.80) --
	(289.18, 80.85) --
	(289.12, 78.90) --
	(288.60, 76.95) --
	(287.68, 75.00) --
	(287.66, 74.97) --
	(286.47, 73.05) --
	(285.71, 72.07) --
	(285.00, 71.10) --
	(283.76, 69.63) --
	(283.37, 69.15) --
	(281.81, 67.41) --
	(281.63, 67.19) --
	(279.86, 65.29) --
	(279.82, 65.24) --
	(277.98, 63.29) --
	(277.91, 63.22) --
	(276.15, 61.34) --
	(275.96, 61.14) --
	(274.34, 59.39) --
	(274.01, 59.02) --
	(272.57, 57.44) --
	(272.06, 56.84) --
	(270.86, 55.49) --
	(270.11, 54.57) --
	(269.22, 53.54) --
	(268.16, 52.16) --
	(267.67, 51.59) --
	(266.24, 49.64) --
	(266.21, 49.58) --
	(264.85, 47.69) --
	(264.26, 46.67) --
	(263.61, 45.74) --
	(262.53, 43.79) --
	(262.31, 43.27) --
	(261.53, 41.84) --
	(260.76, 39.89) --
	(260.35, 38.38) --
	(260.19, 37.94);

\path[draw=drawColor,line width= 1.1pt,line join=round] (240.78, 37.94) --
	(240.50, 39.89) --
	(240.09, 41.84) --
	(239.52, 43.79) --
	(238.90, 45.46) --
	(238.80, 45.74) --
	(237.94, 47.69) --
	(236.95, 49.54) --
	(236.90, 49.64) --
	(235.70, 51.59) --
	(235.00, 52.60) --
	(234.32, 53.54) --
	(233.05, 55.14) --
	(232.76, 55.49) --
	(231.10, 57.33) --
	(230.99, 57.44) --
	(229.15, 59.26) --
	(228.99, 59.39) --
	(227.19, 60.97) --
	(226.72, 61.34) --
	(225.24, 62.50) --
	(224.10, 63.29) --
	(223.29, 63.87) --
	(221.34, 65.08) --
	(221.03, 65.24) --
	(219.39, 66.17) --
	(217.44, 67.07) --
	(217.11, 67.19) --
	(215.49, 67.88);

\path[draw=drawColor,line width= 1.1pt,line join=round] (215.49, 85.18) --
	(217.44, 86.17) --
	(218.29, 86.70) --
	(219.39, 87.30) --
	(221.34, 88.57) --
	(221.45, 88.65) --
	(223.29, 89.90) --
	(224.20, 90.60) --
	(225.24, 91.36) --
	(226.72, 92.55) --
	(227.19, 92.92) --
	(229.06, 94.50) --
	(229.15, 94.57) --
	(231.10, 96.31) --
	(231.25, 96.45) --
	(233.05, 98.14) --
	(233.32, 98.40) --
	(235.00,100.06) --
	(235.29,100.35) --
	(236.95,102.08) --
	(237.17,102.31) --
	(238.90,104.19) --
	(238.96,104.26) --
	(240.68,106.21) --
	(240.85,106.41) --
	(242.34,108.16) --
	(242.80,108.76) --
	(243.92,110.11) --
	(244.75,111.23) --
	(245.43,112.06) --
	(246.70,113.84) --
	(246.84,114.01) --
	(248.23,115.96);

\path[draw=drawColor,line width= 1.1pt,line join=round] (293.52, 88.71) --
	(291.56, 90.22) --
	(291.10, 90.60) --
	(289.61, 91.67) --
	(288.47, 92.55) --
	(287.66, 93.12) --
	(285.82, 94.50) --
	(285.71, 94.58) --
	(283.76, 96.06) --
	(283.27, 96.45) --
	(281.81, 97.61) --
	(280.86, 98.40) --
	(279.86, 99.25) --
	(278.60,100.35) --
	(277.91,100.98) --
	(276.49,102.31) --
	(275.96,102.84) --
	(274.55,104.26) --
	(274.01,104.84) --
	(272.75,106.21) --
	(272.06,107.04) --
	(271.09,108.16) --
	(270.11,109.45) --
	(269.57,110.11) --
	(268.20,112.06) --
	(268.16,112.13) --
	(266.90,114.01) --
	(266.21,115.27) --
	(265.76,115.96);

\path[draw=drawColor,line width= 1.1pt,line join=round] (268.02, 37.94) --
	(268.16, 38.69) --
	(268.35, 39.89) --
	(268.80, 41.84) --
	(269.43, 43.79) --
	(270.11, 45.44) --
	(270.22, 45.74) --
	(271.14, 47.69) --
	(272.06, 49.30) --
	(272.24, 49.64) --
	(273.47, 51.59) --
	(274.01, 52.34) --
	(274.85, 53.54) --
	(275.96, 54.97) --
	(276.36, 55.49) --
	(277.91, 57.33) --
	(278.01, 57.44) --
	(279.76, 59.39) --
	(279.86, 59.50) --
	(281.62, 61.34) --
	(281.81, 61.55) --
	(283.56, 63.29) --
	(283.76, 63.50) --
	(285.58, 65.24) --
	(285.71, 65.38) --
	(287.64, 67.19) --
	(287.66, 67.22) --
	(289.61, 69.06) --
	(289.71, 69.15) --
	(291.56, 70.93) --
	(291.75, 71.10) --
	(293.52, 72.89);

\path[draw=drawColor,line width= 1.1pt,line join=round] (215.49, 60.60) --
	(217.44, 59.96) --
	(218.79, 59.39) --
	(219.39, 59.14) --
	(221.34, 58.13) --
	(222.46, 57.44) --
	(223.29, 56.90) --
	(225.17, 55.49) --
	(225.24, 55.43) --
	(227.19, 53.65) --
	(227.31, 53.54) --
	(229.06, 51.59) --
	(229.15, 51.48) --
	(230.51, 49.64) --
	(231.10, 48.71) --
	(231.71, 47.69) --
	(232.69, 45.74) --
	(233.05, 44.86) --
	(233.47, 43.79) --
	(234.09, 41.84) --
	(234.54, 39.89) --
	(234.85, 37.94);

\path[draw=drawColor,line width= 1.1pt,line join=round] (215.49, 91.56) --
	(217.44, 92.37) --
	(217.82, 92.55) --
	(219.39, 93.29) --
	(221.34, 94.38) --
	(221.54, 94.50) --
	(223.29, 95.59) --
	(224.52, 96.45) --
	(225.24, 96.96) --
	(227.11, 98.40) --
	(227.19, 98.47) --
	(229.15,100.14) --
	(229.39,100.35) --
	(231.10,101.97) --
	(231.44,102.31) --
	(233.05,103.98) --
	(233.31,104.26) --
	(235.00,106.20) --
	(235.01,106.21) --
	(236.56,108.16) --
	(236.95,108.69) --
	(237.98,110.11) --
	(238.90,111.52) --
	(239.26,112.06) --
	(240.40,114.01) --
	(240.85,114.89) --
	(241.42,115.96);

\path[draw=drawColor,line width= 1.1pt,line join=round] (273.01, 37.94) --
	(273.30, 39.89) --
	(273.74, 41.84) --
	(274.01, 42.73) --
	(274.32, 43.79) --
	(275.05, 45.74) --
	(275.96, 47.69) --
	(275.96, 47.70) --
	(277.01, 49.64) --
	(277.91, 51.07) --
	(278.24, 51.59) --
	(279.64, 53.54) --
	(279.86, 53.82) --
	(281.20, 55.49) --
	(281.81, 56.19) --
	(282.93, 57.44) --
	(283.76, 58.31) --
	(284.83, 59.39) --
	(285.71, 60.25) --
	(286.90, 61.34) --
	(287.66, 62.04) --
	(289.14, 63.29) --
	(289.61, 63.70) --
	(291.54, 65.24) --
	(291.56, 65.27) --
	(293.52, 66.77);

\path[draw=drawColor,line width= 1.1pt,line join=round] (293.52, 95.37) --
	(291.56, 96.44) --
	(291.55, 96.45) --
	(289.61, 97.60) --
	(288.38, 98.40) --
	(287.66, 98.87) --
	(285.71,100.26) --
	(285.59,100.35) --
	(283.76,101.80) --
	(283.17,102.31) --
	(281.81,103.52) --
	(281.04,104.26) --
	(279.86,105.47) --
	(279.17,106.21) --
	(277.91,107.72) --
	(277.55,108.16) --
	(276.15,110.11) --
	(275.96,110.42) --
	(274.95,112.06) --
	(274.01,113.91) --
	(273.96,114.01) --
	(273.14,115.96);

\path[draw=drawColor,line width= 1.1pt,line join=round] (215.49, 54.45) --
	(217.44, 53.68) --
	(217.73, 53.54) --
	(219.39, 52.67) --
	(221.04, 51.59) --
	(221.34, 51.37) --
	(223.29, 49.70) --
	(223.36, 49.64) --
	(225.09, 47.69) --
	(225.24, 47.49) --
	(226.43, 45.74) --
	(227.19, 44.35) --
	(227.48, 43.79) --
	(228.27, 41.84) --
	(228.87, 39.89) --
	(229.15, 38.60) --
	(229.28, 37.94);

\path[draw=drawColor,line width= 1.1pt,line join=round] (215.49, 96.41) --
	(215.60, 96.45) --
	(217.44, 97.20) --
	(219.39, 98.16) --
	(219.83, 98.40) --
	(221.34, 99.27) --
	(223.00,100.35) --
	(223.29,100.55) --
	(225.24,102.02) --
	(225.60,102.31) --
	(227.19,103.69) --
	(227.80,104.26) --
	(229.15,105.60) --
	(229.72,106.21) --
	(231.10,107.79) --
	(231.40,108.16) --
	(232.88,110.11) --
	(233.05,110.36) --
	(234.18,112.06) --
	(235.00,113.47) --
	(235.31,114.01) --
	(236.28,115.96);

\path[draw=drawColor,line width= 1.1pt,line join=round] (277.27, 37.94) --
	(277.55, 39.89) --
	(277.91, 41.49) --
	(277.99, 41.84) --
	(278.57, 43.79) --
	(279.33, 45.74) --
	(279.86, 46.86) --
	(280.26, 47.69) --
	(281.37, 49.64) --
	(281.81, 50.31) --
	(282.68, 51.59) --
	(283.76, 53.01) --
	(284.19, 53.54) --
	(285.71, 55.28) --
	(285.90, 55.49) --
	(287.66, 57.27) --
	(287.85, 57.44) --
	(289.61, 59.03) --
	(290.05, 59.39) --
	(291.56, 60.63) --
	(292.53, 61.34) --
	(293.52, 62.08);

\path[draw=drawColor,line width= 1.1pt,line join=round] (293.52,100.86) --
	(291.56,101.91) --
	(290.91,102.31) --
	(289.61,103.13) --
	(288.04,104.26) --
	(287.66,104.54) --
	(285.71,106.19) --
	(285.70,106.21) --
	(283.79,108.16) --
	(283.76,108.19) --
	(282.22,110.11) --
	(281.81,110.70) --
	(280.93,112.06) --
	(279.89,114.01) --
	(279.86,114.08) --
	(279.09,115.96);

\path[draw=drawColor,line width= 1.1pt,line join=round] (223.23, 37.94) --
	(222.57, 39.89) --
	(221.64, 41.84) --
	(221.34, 42.33) --
	(220.32, 43.79) --
	(219.39, 44.86) --
	(218.47, 45.74) --
	(217.44, 46.57) --
	(215.69, 47.69) --
	(215.49, 47.80);

\path[draw=drawColor,line width= 1.1pt,line join=round] (215.49,100.83) --
	(217.44,101.71) --
	(218.57,102.31) --
	(219.39,102.76) --
	(221.34,104.01) --
	(221.69,104.26) --
	(223.29,105.49) --
	(224.14,106.21) --
	(225.24,107.23) --
	(226.16,108.16) --
	(227.19,109.30) --
	(227.87,110.11) --
	(229.15,111.81) --
	(229.32,112.06) --
	(230.55,114.01) --
	(231.10,115.03) --
	(231.58,115.96);

\path[draw=drawColor,line width= 1.1pt,line join=round] (281.24, 37.94) --
	(281.54, 39.89) --
	(281.81, 41.07) --
	(281.99, 41.84) --
	(282.62, 43.79) --
	(283.43, 45.74) --
	(283.76, 46.39) --
	(284.44, 47.69) --
	(285.66, 49.64) --
	(285.71, 49.71) --
	(287.12, 51.59) --
	(287.66, 52.23) --
	(288.84, 53.54) --
	(289.61, 54.33) --
	(290.84, 55.49) --
	(291.56, 56.13) --
	(293.18, 57.44) --
	(293.52, 57.70);

\path[draw=drawColor,line width= 1.1pt,line join=round] (293.52,106.55) --
	(291.56,107.85) --
	(291.15,108.16) --
	(289.61,109.48) --
	(288.97,110.11) --
	(287.66,111.58) --
	(287.28,112.06) --
	(286.00,114.01) --
	(285.71,114.57) --
	(285.05,115.96);

\path[draw=drawColor,line width= 1.1pt,line join=round] (215.49,105.30) --
	(217.25,106.21) --
	(217.44,106.31) --
	(219.39,107.58) --
	(220.16,108.16) --
	(221.34,109.13) --
	(222.40,110.11) --
	(223.29,111.04) --
	(224.19,112.06) --
	(225.24,113.43) --
	(225.66,114.01) --
	(226.85,115.96);

\path[draw=drawColor,line width= 1.1pt,line join=round] (285.19, 37.94) --
	(285.50, 39.89) --
	(285.71, 40.74) --
	(286.00, 41.84) --
	(286.69, 43.79) --
	(287.60, 45.74) --
	(287.66, 45.85) --
	(288.75, 47.69) --
	(289.61, 48.90) --
	(290.17, 49.64) --
	(291.56, 51.25) --
	(291.89, 51.59) --
	(293.52, 53.14);

\path[draw=drawColor,line width= 1.1pt,line join=round] (293.52,114.24) --
	(292.21,115.96);

\path[draw=drawColor,line width= 1.1pt,line join=round] (215.49,110.23) --
	(217.44,111.58) --
	(218.03,112.06) --
	(219.39,113.32) --
	(220.04,114.01) --
	(221.34,115.61) --
	(221.60,115.96);

\path[draw=drawColor,line width= 1.1pt,line join=round] (289.31, 37.94) --
	(289.61, 39.66) --
	(289.66, 39.89) --
	(290.24, 41.84) --
	(291.05, 43.79) --
	(291.56, 44.76) --
	(292.13, 45.74) --
	(293.50, 47.69) --
	(293.52, 47.70);
\definecolor[named]{drawColor}{rgb}{0.00,0.00,0.00}

\path[draw=drawColor,draw opacity=0.33,line width= 1.1pt,line join=round] (227.19, 75.18) --
	(225.43, 76.95) --
	(227.19, 78.71) --
	(227.38, 78.90) --
	(229.15, 80.66) --
	(229.33, 80.85) --
	(231.10, 82.61) --
	(231.28, 82.80) --
	(233.05, 84.56) --
	(233.23, 84.75) --
	(235.00, 86.51) --
	(235.18, 86.70) --
	(236.95, 88.46) --
	(237.13, 88.65) --
	(238.90, 90.41) --
	(239.09, 90.60) --
	(240.85, 92.37) --
	(241.04, 92.55) --
	(242.80, 94.32) --
	(242.99, 94.50) --
	(244.75, 96.27) --
	(244.94, 96.45) --
	(246.70, 98.22) --
	(246.89, 98.40) --
	(248.65,100.17) --
	(248.84,100.35) --
	(250.60,102.12) --
	(250.79,102.31) --
	(252.55,104.07) --
	(252.74,104.26) --
	(254.50,106.02) --
	(256.27,104.26) --
	(256.45,104.07) --
	(258.22,102.31) --
	(258.40,102.12) --
	(260.17,100.35) --
	(260.35,100.17) --
	(262.12, 98.40) --
	(262.31, 98.22) --
	(264.07, 96.45) --
	(264.26, 96.27) --
	(266.02, 94.50) --
	(266.21, 94.32) --
	(267.97, 92.55) --
	(268.16, 92.37) --
	(269.92, 90.60) --
	(270.11, 90.41) --
	(271.87, 88.65) --
	(272.06, 88.46) --
	(273.82, 86.70) --
	(274.01, 86.51) --
	(275.77, 84.75) --
	(275.96, 84.56) --
	(277.72, 82.80) --
	(277.91, 82.61) --
	(279.67, 80.85) --
	(279.86, 80.66) --
	(281.62, 78.90) --
	(281.81, 78.71) --
	(283.58, 76.95) --
	(281.81, 75.18) --
	(281.62, 75.00) --
	(279.86, 73.23) --
	(279.67, 73.05) --
	(277.91, 71.28) --
	(277.72, 71.10) --
	(275.96, 69.33) --
	(275.77, 69.15) --
	(274.01, 67.38) --
	(273.82, 67.19) --
	(272.06, 65.43) --
	(271.87, 65.24) --
	(270.11, 63.48) --
	(269.92, 63.29) --
	(268.16, 61.53) --
	(267.97, 61.34) --
	(266.21, 59.58) --
	(266.02, 59.39) --
	(264.26, 57.63) --
	(264.07, 57.44) --
	(262.31, 55.68) --
	(262.12, 55.49) --
	(260.35, 53.73) --
	(260.17, 53.54) --
	(258.40, 51.78) --
	(258.22, 51.59) --
	(256.45, 49.83) --
	(256.27, 49.64) --
	(254.50, 47.88) --
	(252.74, 49.64) --
	(252.55, 49.83) --
	(250.79, 51.59) --
	(250.60, 51.78) --
	(248.84, 53.54) --
	(248.65, 53.73) --
	(246.89, 55.49) --
	(246.70, 55.68) --
	(244.94, 57.44) --
	(244.75, 57.63) --
	(242.99, 59.39) --
	(242.80, 59.58) --
	(241.04, 61.34) --
	(240.85, 61.53) --
	(239.09, 63.29) --
	(238.90, 63.48) --
	(237.13, 65.24) --
	(236.95, 65.43) --
	(235.18, 67.19) --
	(235.00, 67.38) --
	(233.23, 69.15) --
	(233.05, 69.33) --
	(231.28, 71.10) --
	(231.10, 71.28) --
	(229.33, 73.05) --
	(229.15, 73.23) --
	(227.38, 75.00) --
	(227.19, 75.18);

\path[draw=drawColor,draw opacity=0.33,line width= 1.1pt,line join=round] (252.40, 37.94) --
	(250.60, 39.73) --
	(250.45, 39.89) --
	(248.65, 41.68) --
	(248.50, 41.84) --
	(246.70, 43.63) --
	(246.54, 43.79) --
	(244.75, 45.58) --
	(244.59, 45.74) --
	(242.80, 47.53) --
	(242.64, 47.69) --
	(240.85, 49.48) --
	(240.69, 49.64) --
	(238.90, 51.43) --
	(238.74, 51.59) --
	(236.95, 53.38) --
	(236.79, 53.54) --
	(235.00, 55.33) --
	(234.84, 55.49) --
	(233.05, 57.29) --
	(232.89, 57.44) --
	(231.10, 59.24) --
	(230.94, 59.39) --
	(229.15, 61.19) --
	(228.99, 61.34) --
	(227.19, 63.14) --
	(227.04, 63.29) --
	(225.24, 65.09) --
	(225.09, 65.24) --
	(223.29, 67.04) --
	(223.14, 67.19) --
	(221.34, 68.99) --
	(221.19, 69.15) --
	(219.39, 70.94) --
	(219.24, 71.10) --
	(217.44, 72.89) --
	(217.29, 73.05) --
	(215.49, 74.84);

\path[draw=drawColor,draw opacity=0.33,line width= 1.1pt,line join=round] (215.49, 79.05) --
	(217.29, 80.85) --
	(217.44, 81.01) --
	(219.24, 82.80) --
	(219.39, 82.96) --
	(221.19, 84.75) --
	(221.34, 84.91) --
	(223.14, 86.70) --
	(223.29, 86.86) --
	(225.09, 88.65) --
	(225.24, 88.81) --
	(227.04, 90.60) --
	(227.19, 90.76) --
	(228.99, 92.55) --
	(229.15, 92.71) --
	(230.94, 94.50) --
	(231.10, 94.66) --
	(232.89, 96.45) --
	(233.05, 96.61) --
	(234.84, 98.40) --
	(235.00, 98.56) --
	(236.79,100.35) --
	(236.95,100.51) --
	(238.74,102.31) --
	(238.90,102.46) --
	(240.69,104.26) --
	(240.85,104.41) --
	(242.64,106.21) --
	(242.80,106.36) --
	(244.59,108.16) --
	(244.75,108.31) --
	(246.54,110.11) --
	(246.70,110.26) --
	(248.50,112.06) --
	(248.65,112.21) --
	(250.45,114.01) --
	(250.60,114.17) --
	(252.40,115.96);

\path[draw=drawColor,draw opacity=0.33,line width= 1.1pt,line join=round] (256.61, 37.94) --
	(258.40, 39.73) --
	(258.56, 39.89) --
	(260.35, 41.68) --
	(260.51, 41.84) --
	(262.31, 43.63) --
	(262.46, 43.79) --
	(264.26, 45.58) --
	(264.41, 45.74) --
	(266.21, 47.53) --
	(266.36, 47.69) --
	(268.16, 49.48) --
	(268.31, 49.64) --
	(270.11, 51.43) --
	(270.26, 51.59) --
	(272.06, 53.38) --
	(272.21, 53.54) --
	(274.01, 55.33) --
	(274.17, 55.49) --
	(275.96, 57.29) --
	(276.12, 57.44) --
	(277.91, 59.24) --
	(278.07, 59.39) --
	(279.86, 61.19) --
	(280.02, 61.34) --
	(281.81, 63.14) --
	(281.97, 63.29) --
	(283.76, 65.09) --
	(283.92, 65.24) --
	(285.71, 67.04) --
	(285.87, 67.19) --
	(287.66, 68.99) --
	(287.82, 69.15) --
	(289.61, 70.94) --
	(289.77, 71.10) --
	(291.56, 72.89) --
	(291.72, 73.05) --
	(293.52, 74.84);

\path[draw=drawColor,draw opacity=0.33,line width= 1.1pt,line join=round] (293.52, 79.05) --
	(291.72, 80.85) --
	(291.56, 81.01) --
	(289.77, 82.80) --
	(289.61, 82.96) --
	(287.82, 84.75) --
	(287.66, 84.91) --
	(285.87, 86.70) --
	(285.71, 86.86) --
	(283.92, 88.65) --
	(283.76, 88.81) --
	(281.97, 90.60) --
	(281.81, 90.76) --
	(280.02, 92.55) --
	(279.86, 92.71) --
	(278.07, 94.50) --
	(277.91, 94.66) --
	(276.12, 96.45) --
	(275.96, 96.61) --
	(274.17, 98.40) --
	(274.01, 98.56) --
	(272.21,100.35) --
	(272.06,100.51) --
	(270.26,102.31) --
	(270.11,102.46) --
	(268.31,104.26) --
	(268.16,104.41) --
	(266.36,106.21) --
	(266.21,106.36) --
	(264.41,108.16) --
	(264.26,108.31) --
	(262.46,110.11) --
	(262.31,110.26) --
	(260.51,112.06) --
	(260.35,112.21) --
	(258.56,114.01) --
	(258.40,114.17) --
	(256.61,115.96);

\path[draw=drawColor,draw opacity=0.33,line width= 1.1pt,line join=round] (243.16, 37.94) --
	(242.80, 38.29) --
	(241.21, 39.89) --
	(240.85, 40.24) --
	(239.26, 41.84) --
	(238.90, 42.19) --
	(237.30, 43.79) --
	(236.95, 44.14) --
	(235.35, 45.74) --
	(235.00, 46.10) --
	(233.40, 47.69) --
	(233.05, 48.05) --
	(231.45, 49.64) --
	(231.10, 50.00) --
	(229.50, 51.59) --
	(229.15, 51.95) --
	(227.55, 53.54) --
	(227.19, 53.90) --
	(225.60, 55.49) --
	(225.24, 55.85) --
	(223.65, 57.44) --
	(223.29, 57.80) --
	(221.70, 59.39) --
	(221.34, 59.75) --
	(219.75, 61.34) --
	(219.39, 61.70) --
	(217.80, 63.29) --
	(217.44, 63.65) --
	(215.85, 65.24) --
	(215.49, 65.60);

\path[draw=drawColor,draw opacity=0.33,line width= 1.1pt,line join=round] (215.49, 88.29) --
	(215.85, 88.65) --
	(217.44, 90.24) --
	(217.80, 90.60) --
	(219.39, 92.20) --
	(219.75, 92.55) --
	(221.34, 94.15) --
	(221.70, 94.50) --
	(223.29, 96.10) --
	(223.65, 96.45) --
	(225.24, 98.05) --
	(225.60, 98.40) --
	(227.19,100.00) --
	(227.55,100.35) --
	(229.15,101.95) --
	(229.50,102.31) --
	(231.10,103.90) --
	(231.45,104.26) --
	(233.05,105.85) --
	(233.40,106.21) --
	(235.00,107.80) --
	(235.35,108.16) --
	(236.95,109.75) --
	(237.30,110.11) --
	(238.90,111.70) --
	(239.26,112.06) --
	(240.85,113.65) --
	(241.21,114.01) --
	(242.80,115.60) --
	(243.16,115.96);

\path[draw=drawColor,draw opacity=0.33,line width= 1.1pt,line join=round] (265.85, 37.94) --
	(266.21, 38.29) --
	(267.80, 39.89) --
	(268.16, 40.24) --
	(269.75, 41.84) --
	(270.11, 42.19) --
	(271.70, 43.79) --
	(272.06, 44.14) --
	(273.65, 45.74) --
	(274.01, 46.10) --
	(275.60, 47.69) --
	(275.96, 48.05) --
	(277.55, 49.64) --
	(277.91, 50.00) --
	(279.50, 51.59) --
	(279.86, 51.95) --
	(281.45, 53.54) --
	(281.81, 53.90) --
	(283.41, 55.49) --
	(283.76, 55.85) --
	(285.36, 57.44) --
	(285.71, 57.80) --
	(287.31, 59.39) --
	(287.66, 59.75) --
	(289.26, 61.34) --
	(289.61, 61.70) --
	(291.21, 63.29) --
	(291.56, 63.65) --
	(293.16, 65.24) --
	(293.52, 65.60);

\path[draw=drawColor,draw opacity=0.33,line width= 1.1pt,line join=round] (293.52, 88.29) --
	(293.16, 88.65) --
	(291.56, 90.24) --
	(291.21, 90.60) --
	(289.61, 92.20) --
	(289.26, 92.55) --
	(287.66, 94.15) --
	(287.31, 94.50) --
	(285.71, 96.10) --
	(285.36, 96.45) --
	(283.76, 98.05) --
	(283.41, 98.40) --
	(281.81,100.00) --
	(281.45,100.35) --
	(279.86,101.95) --
	(279.50,102.31) --
	(277.91,103.90) --
	(277.55,104.26) --
	(275.96,105.85) --
	(275.60,106.21) --
	(274.01,107.80) --
	(273.65,108.16) --
	(272.06,109.75) --
	(271.70,110.11) --
	(270.11,111.70) --
	(269.75,112.06) --
	(268.16,113.65) --
	(267.80,114.01) --
	(266.21,115.60) --
	(265.85,115.96);

\path[draw=drawColor,draw opacity=0.33,line width= 1.1pt,line join=round] (235.36, 37.94) --
	(235.00, 38.30) --
	(233.41, 39.89) --
	(233.05, 40.25) --
	(231.46, 41.84) --
	(231.10, 42.20) --
	(229.51, 43.79) --
	(229.15, 44.15) --
	(227.56, 45.74) --
	(227.19, 46.11) --
	(225.61, 47.69) --
	(225.24, 48.06) --
	(223.66, 49.64) --
	(223.29, 50.01) --
	(221.71, 51.59) --
	(221.34, 51.96) --
	(219.76, 53.54) --
	(219.39, 53.91) --
	(217.81, 55.49) --
	(217.44, 55.86) --
	(215.86, 57.44) --
	(215.49, 57.81);

\path[draw=drawColor,draw opacity=0.33,line width= 1.1pt,line join=round] (215.49, 96.09) --
	(215.86, 96.45) --
	(217.44, 98.04) --
	(217.81, 98.40) --
	(219.39, 99.99) --
	(219.76,100.35) --
	(221.34,101.94) --
	(221.71,102.31) --
	(223.29,103.89) --
	(223.66,104.26) --
	(225.24,105.84) --
	(225.61,106.21) --
	(227.19,107.79) --
	(227.56,108.16) --
	(229.15,109.74) --
	(229.51,110.11) --
	(231.10,111.69) --
	(231.46,112.06) --
	(233.05,113.64) --
	(233.41,114.01) --
	(235.00,115.59) --
	(235.36,115.96);

\path[draw=drawColor,draw opacity=0.33,line width= 1.1pt,line join=round] (273.64, 37.94) --
	(274.01, 38.30) --
	(275.59, 39.89) --
	(275.96, 40.25) --
	(277.54, 41.84) --
	(277.91, 42.20) --
	(279.49, 43.79) --
	(279.86, 44.15) --
	(281.44, 45.74) --
	(281.81, 46.11) --
	(283.39, 47.69) --
	(283.76, 48.06) --
	(285.35, 49.64) --
	(285.71, 50.01) --
	(287.30, 51.59) --
	(287.66, 51.96) --
	(289.25, 53.54) --
	(289.61, 53.91) --
	(291.20, 55.49) --
	(291.56, 55.86) --
	(293.15, 57.44) --
	(293.52, 57.81);

\path[draw=drawColor,draw opacity=0.33,line width= 1.1pt,line join=round] (293.52, 96.09) --
	(293.15, 96.45) --
	(291.56, 98.04) --
	(291.20, 98.40) --
	(289.61, 99.99) --
	(289.25,100.35) --
	(287.66,101.94) --
	(287.30,102.31) --
	(285.71,103.89) --
	(285.35,104.26) --
	(283.76,105.84) --
	(283.39,106.21) --
	(281.81,107.79) --
	(281.44,108.16) --
	(279.86,109.74) --
	(279.49,110.11) --
	(277.91,111.69) --
	(277.54,112.06) --
	(275.96,113.64) --
	(275.59,114.01) --
	(274.01,115.59) --
	(273.64,115.96);

\path[draw=drawColor,draw opacity=0.33,line width= 1.1pt,line join=round] (228.50, 37.94) --
	(227.19, 39.24) --
	(226.55, 39.89) --
	(225.24, 41.19) --
	(224.60, 41.84) --
	(223.29, 43.14) --
	(222.65, 43.79) --
	(221.34, 45.09) --
	(220.70, 45.74) --
	(219.39, 47.05) --
	(218.75, 47.69) --
	(217.44, 49.00) --
	(216.80, 49.64) --
	(215.49, 50.95);

\path[draw=drawColor,draw opacity=0.33,line width= 1.1pt,line join=round] (215.49,102.95) --
	(216.80,104.26) --
	(217.44,104.90) --
	(218.75,106.21) --
	(219.39,106.85) --
	(220.70,108.16) --
	(221.34,108.80) --
	(222.65,110.11) --
	(223.29,110.75) --
	(224.60,112.06) --
	(225.24,112.70) --
	(226.55,114.01) --
	(227.19,114.65) --
	(228.50,115.96);

\path[draw=drawColor,draw opacity=0.33,line width= 1.1pt,line join=round] (280.50, 37.94) --
	(281.81, 39.24) --
	(282.46, 39.89) --
	(283.76, 41.19) --
	(284.41, 41.84) --
	(285.71, 43.14) --
	(286.36, 43.79) --
	(287.66, 45.09) --
	(288.31, 45.74) --
	(289.61, 47.05) --
	(290.26, 47.69) --
	(291.56, 49.00) --
	(292.21, 49.64) --
	(293.52, 50.95);

\path[draw=drawColor,draw opacity=0.33,line width= 1.1pt,line join=round] (293.52,102.95) --
	(292.21,104.26) --
	(291.56,104.90) --
	(290.26,106.21) --
	(289.61,106.85) --
	(288.31,108.16) --
	(287.66,108.80) --
	(286.36,110.11) --
	(285.71,110.75) --
	(284.41,112.06) --
	(283.76,112.70) --
	(282.46,114.01) --
	(281.81,114.65) --
	(280.50,115.96);

\path[draw=drawColor,draw opacity=0.33,line width= 1.1pt,line join=round] (222.30, 37.94) --
	(221.34, 38.89) --
	(220.35, 39.89) --
	(219.39, 40.84) --
	(218.39, 41.84) --
	(217.44, 42.79) --
	(216.44, 43.79) --
	(215.49, 44.74);

\path[draw=drawColor,draw opacity=0.33,line width= 1.1pt,line join=round] (215.49,109.15) --
	(216.44,110.11) --
	(217.44,111.11) --
	(218.39,112.06) --
	(219.39,113.06) --
	(220.35,114.01) --
	(221.34,115.01) --
	(222.30,115.96);

\path[draw=drawColor,draw opacity=0.33,line width= 1.1pt,line join=round] (286.71, 37.94) --
	(287.66, 38.89) --
	(288.66, 39.89) --
	(289.61, 40.84) --
	(290.61, 41.84) --
	(291.56, 42.79) --
	(292.56, 43.79) --
	(293.52, 44.74);

\path[draw=drawColor,draw opacity=0.33,line width= 1.1pt,line join=round] (293.52,109.15) --
	(292.56,110.11) --
	(291.56,111.11) --
	(290.61,112.06) --
	(289.61,113.06) --
	(288.66,114.01) --
	(287.66,115.01) --
	(286.71,115.96);

\path[draw=drawColor,draw opacity=0.33,line width= 1.1pt,line join=round] (216.59, 37.94) --
	(215.49, 39.03);

\path[draw=drawColor,draw opacity=0.33,line width= 1.1pt,line join=round] (215.49,114.86) --
	(216.59,115.96);

\path[draw=drawColor,draw opacity=0.33,line width= 1.1pt,line join=round] (292.42, 37.94) --
	(293.52, 39.03);

\path[draw=drawColor,draw opacity=0.33,line width= 1.1pt,line join=round] (293.52,114.86) --
	(292.42,115.96);
\definecolor[named]{drawColor}{rgb}{0.50,0.50,0.50}

\path[draw=drawColor,line width= 0.6pt,line join=round,line cap=round] (211.59, 34.03) rectangle (297.42,119.86);
\end{scope}
\begin{scope}
\path[clip] (297.42, 34.03) rectangle (383.24,119.86);
\definecolor[named]{fillColor}{rgb}{1.00,1.00,1.00}

\path[fill=fillColor] (297.42, 34.03) rectangle (383.24,119.86);
\definecolor[named]{drawColor}{rgb}{0.00,0.00,0.00}

\path[draw=drawColor,line width= 1.1pt,line join=round] (328.63, 73.63) --
	(328.14, 75.00) --
	(327.79, 76.95) --
	(327.82, 78.90) --
	(328.22, 80.85) --
	(328.63, 81.95) --
	(328.95, 82.80) --
	(329.95, 84.75) --
	(330.58, 85.79) --
	(331.16, 86.70) --
	(332.53, 88.64) --
	(332.54, 88.65) --
	(334.20, 90.60) --
	(334.48, 90.91) --
	(336.26, 92.55) --
	(336.43, 92.70) --
	(338.38, 93.92) --
	(340.09, 94.50) --
	(340.33, 94.58) --
	(342.28, 94.62) --
	(342.74, 94.50) --
	(344.23, 94.10) --
	(346.18, 93.12) --
	(347.00, 92.55) --
	(348.13, 91.76) --
	(349.53, 90.60) --
	(350.08, 90.13) --
	(351.68, 88.65) --
	(352.03, 88.30) --
	(353.59, 86.70) --
	(353.98, 86.25) --
	(355.27, 84.75) --
	(355.93, 83.81) --
	(356.63, 82.80) --
	(357.58, 80.85) --
	(357.88, 79.55) --
	(358.03, 78.90) --
	(357.97, 76.95) --
	(357.88, 76.63) --
	(357.43, 75.00) --
	(356.53, 73.05) --
	(355.93, 72.07) --
	(355.33, 71.10) --
	(353.98, 69.20) --
	(353.94, 69.15) --
	(352.35, 67.19) --
	(352.03, 66.82) --
	(350.58, 65.24) --
	(350.08, 64.72) --
	(348.53, 63.29) --
	(348.13, 62.93) --
	(346.18, 61.55) --
	(345.74, 61.34) --
	(344.23, 60.67) --
	(342.28, 60.33) --
	(340.33, 60.56) --
	(338.42, 61.34) --
	(338.38, 61.36) --
	(336.43, 62.77) --
	(335.87, 63.29) --
	(334.48, 64.69) --
	(334.00, 65.24) --
	(332.53, 67.04) --
	(332.41, 67.19) --
	(331.04, 69.15) --
	(330.58, 69.88) --
	(329.83, 71.10) --
	(328.84, 73.05) --
	(328.63, 73.63);

\path[draw=drawColor,line width= 1.1pt,line join=round] (320.82, 72.26) --
	(320.28, 73.05) --
	(319.38, 75.00) --
	(319.02, 76.95) --
	(319.22, 78.90) --
	(319.94, 80.85) --
	(320.82, 82.36) --
	(321.06, 82.80) --
	(322.42, 84.75) --
	(322.77, 85.19) --
	(323.91, 86.70) --
	(324.72, 87.70) --
	(325.46, 88.65) --
	(326.68, 90.17) --
	(327.01, 90.60) --
	(328.55, 92.55) --
	(328.63, 92.64) --
	(330.14, 94.50) --
	(330.58, 95.05) --
	(331.78, 96.45) --
	(332.53, 97.33) --
	(333.57, 98.40) --
	(334.48, 99.33) --
	(335.73,100.35) --
	(336.43,100.91) --
	(338.38,101.98) --
	(339.68,102.31) --
	(340.33,102.47) --
	(342.28,102.38) --
	(342.50,102.31) --
	(344.23,101.74) --
	(346.18,100.62) --
	(346.52,100.35) --
	(348.13, 99.12) --
	(348.92, 98.40) --
	(350.08, 97.37) --
	(351.02, 96.45) --
	(352.03, 95.50) --
	(353.05, 94.50) --
	(353.98, 93.60) --
	(355.08, 92.55) --
	(355.93, 91.73) --
	(357.16, 90.60) --
	(357.88, 89.91) --
	(359.28, 88.65) --
	(359.84, 88.10) --
	(361.35, 86.70) --
	(361.79, 86.23) --
	(363.22, 84.75) --
	(363.74, 84.06) --
	(364.70, 82.80) --
	(365.63, 80.85) --
	(365.69, 80.49) --
	(365.94, 78.90) --
	(365.69, 77.18) --
	(365.65, 76.95) --
	(364.87, 75.00) --
	(363.74, 73.11) --
	(363.70, 73.05) --
	(362.28, 71.10) --
	(361.79, 70.50) --
	(360.71, 69.15) --
	(359.84, 68.11) --
	(359.08, 67.19) --
	(357.88, 65.78) --
	(357.44, 65.24) --
	(355.93, 63.45) --
	(355.80, 63.29) --
	(354.16, 61.34) --
	(353.98, 61.14) --
	(352.45, 59.39) --
	(352.03, 58.91) --
	(350.62, 57.44) --
	(350.08, 56.89) --
	(348.50, 55.49) --
	(348.13, 55.17) --
	(346.18, 53.88) --
	(345.42, 53.54) --
	(344.23, 53.03) --
	(342.28, 52.67) --
	(340.33, 52.79) --
	(338.38, 53.39) --
	(338.09, 53.54) --
	(336.43, 54.49) --
	(335.14, 55.49) --
	(334.48, 56.03) --
	(333.05, 57.44) --
	(332.53, 57.98) --
	(331.28, 59.39) --
	(330.58, 60.19) --
	(329.62, 61.34) --
	(328.63, 62.54) --
	(328.00, 63.29) --
	(326.68, 64.89) --
	(326.37, 65.24) --
	(324.73, 67.19) --
	(324.72, 67.21) --
	(323.11, 69.15) --
	(322.77, 69.59) --
	(321.57, 71.10) --
	(320.82, 72.26);

\path[draw=drawColor,line width= 1.1pt,line join=round] (309.12, 74.66) --
	(308.89, 75.00) --
	(308.56, 76.95) --
	(309.12, 78.59) --
	(309.21, 78.90) --
	(310.64, 80.85) --
	(311.07, 81.24) --
	(312.59, 82.80) --
	(313.02, 83.14) --
	(314.80, 84.75) --
	(314.97, 84.88) --
	(316.92, 86.59) --
	(317.03, 86.70) --
	(318.87, 88.36) --
	(319.16, 88.65) --
	(320.82, 90.27) --
	(321.14, 90.60) --
	(322.77, 92.34) --
	(322.96, 92.55) --
	(324.64, 94.50) --
	(324.72, 94.60) --
	(326.22, 96.45) --
	(326.68, 97.04) --
	(327.73, 98.40) --
	(328.63, 99.61) --
	(329.20,100.35) --
	(330.58,102.20) --
	(330.66,102.31) --
	(332.23,104.26) --
	(332.53,104.63) --
	(334.00,106.21) --
	(334.48,106.72) --
	(336.24,108.16) --
	(336.43,108.31) --
	(338.38,109.35) --
	(340.33,109.77) --
	(342.28,109.60) --
	(344.23,108.84) --
	(345.26,108.16) --
	(346.18,107.56) --
	(347.70,106.21) --
	(348.13,105.84) --
	(349.64,104.26) --
	(350.08,103.81) --
	(351.41,102.31) --
	(352.03,101.64) --
	(353.15,100.35) --
	(353.98, 99.46) --
	(354.95, 98.40) --
	(355.93, 97.39) --
	(356.87, 96.45) --
	(357.88, 95.48) --
	(358.95, 94.50) --
	(359.84, 93.72) --
	(361.25, 92.55) --
	(361.79, 92.11) --
	(363.74, 90.63) --
	(363.78, 90.60) --
	(365.69, 89.19) --
	(366.50, 88.65) --
	(367.64, 87.78) --
	(369.18, 86.70) --
	(369.59, 86.34) --
	(371.50, 84.75) --
	(371.54, 84.71) --
	(373.18, 82.80) --
	(373.49, 82.06) --
	(374.01, 80.85) --
	(374.03, 78.90) --
	(373.49, 77.39) --
	(373.35, 76.95) --
	(372.13, 75.00) --
	(371.54, 74.32) --
	(370.52, 73.05) --
	(369.59, 72.10) --
	(368.67, 71.10) --
	(367.64, 70.09) --
	(366.73, 69.15) --
	(365.69, 68.13) --
	(364.79, 67.19) --
	(363.74, 66.13) --
	(362.91, 65.24) --
	(361.79, 64.04) --
	(361.12, 63.29) --
	(359.84, 61.82) --
	(359.43, 61.34) --
	(357.88, 59.46) --
	(357.83, 59.39) --
	(356.27, 57.44) --
	(355.93, 57.01) --
	(354.71, 55.49) --
	(353.98, 54.56) --
	(353.11, 53.54) --
	(352.03, 52.26) --
	(351.38, 51.59) --
	(350.08, 50.26) --
	(349.34, 49.64) --
	(348.13, 48.64) --
	(346.59, 47.69) --
	(346.18, 47.44) --
	(344.23, 46.68) --
	(342.28, 46.34) --
	(340.33, 46.42) --
	(338.38, 46.93) --
	(336.77, 47.69) --
	(336.43, 47.86) --
	(334.48, 49.22) --
	(334.00, 49.64) --
	(332.53, 50.98) --
	(331.96, 51.59) --
	(330.58, 53.09) --
	(330.20, 53.54) --
	(328.63, 55.41) --
	(328.56, 55.49) --
	(326.96, 57.44) --
	(326.68, 57.78) --
	(325.30, 59.39) --
	(324.72, 60.06) --
	(323.54, 61.34) --
	(322.77, 62.16) --
	(321.62, 63.29) --
	(320.82, 64.08) --
	(319.50, 65.24) --
	(318.87, 65.82) --
	(317.17, 67.19) --
	(316.92, 67.41) --
	(314.97, 68.94) --
	(314.67, 69.15) --
	(313.02, 70.50) --
	(312.21, 71.10) --
	(311.07, 72.22) --
	(310.16, 73.05) --
	(309.12, 74.66);

\path[draw=drawColor,line width= 1.1pt,line join=round] (301.32, 68.86) --
	(303.27, 68.18) --
	(305.22, 67.46) --
	(305.86, 67.19) --
	(307.17, 66.73) --
	(309.12, 65.94) --
	(310.60, 65.24) --
	(311.07, 65.04) --
	(313.02, 64.04) --
	(314.26, 63.29) --
	(314.97, 62.88) --
	(316.92, 61.53) --
	(317.16, 61.34) --
	(318.87, 59.97) --
	(319.51, 59.39) --
	(320.82, 58.15) --
	(321.51, 57.44) --
	(322.77, 56.07) --
	(323.28, 55.49) --
	(324.72, 53.74) --
	(324.88, 53.54) --
	(326.41, 51.59) --
	(326.68, 51.24) --
	(327.92, 49.64) --
	(328.63, 48.73) --
	(329.49, 47.69) --
	(330.58, 46.40) --
	(331.21, 45.74) --
	(332.53, 44.38) --
	(333.23, 43.79) --
	(334.48, 42.75) --
	(335.93, 41.84) --
	(336.43, 41.52) --
	(338.38, 40.69) --
	(340.33, 40.25) --
	(342.28, 40.18) --
	(344.23, 40.48) --
	(346.18, 41.16) --
	(347.40, 41.84) --
	(348.13, 42.24) --
	(350.08, 43.74) --
	(350.13, 43.79) --
	(352.03, 45.66) --
	(352.10, 45.74) --
	(353.74, 47.69) --
	(353.98, 47.98) --
	(355.22, 49.64) --
	(355.93, 50.58) --
	(356.64, 51.59) --
	(357.88, 53.32) --
	(358.04, 53.54) --
	(359.46, 55.49) --
	(359.84, 55.98) --
	(360.95, 57.44) --
	(361.79, 58.47) --
	(362.56, 59.39) --
	(363.74, 60.73) --
	(364.31, 61.34) --
	(365.69, 62.77) --
	(366.23, 63.29) --
	(367.64, 64.62) --
	(368.36, 65.24) --
	(369.59, 66.32) --
	(370.69, 67.19) --
	(371.54, 67.90) --
	(373.18, 69.15) --
	(373.49, 69.40) --
	(375.44, 70.86) --
	(375.78, 71.10) --
	(377.39, 72.35) --
	(378.36, 73.05) --
	(379.34, 73.90);

\path[draw=drawColor,line width= 1.1pt,line join=round] (301.32, 83.39) --
	(303.27, 84.30) --
	(304.16, 84.75) --
	(305.22, 85.19) --
	(307.17, 86.11) --
	(308.26, 86.70) --
	(309.12, 87.11) --
	(311.07, 88.18) --
	(311.82, 88.65) --
	(313.02, 89.36) --
	(314.81, 90.60) --
	(314.97, 90.71) --
	(316.92, 92.21) --
	(317.31, 92.55) --
	(318.87, 93.94) --
	(319.44, 94.50) --
	(320.82, 95.92) --
	(321.30, 96.45) --
	(322.77, 98.20) --
	(322.94, 98.40) --
	(324.41,100.35) --
	(324.72,100.80) --
	(325.78,102.31) --
	(326.68,103.69) --
	(327.05,104.26) --
	(328.28,106.21) --
	(328.63,106.79) --
	(329.51,108.16) --
	(330.58,109.88) --
	(330.74,110.11) --
	(332.09,112.06) --
	(332.53,112.71) --
	(333.63,114.01) --
	(334.48,115.03) --
	(335.57,115.96);

\path[draw=drawColor,line width= 1.1pt,line join=round] (379.34, 87.13) --
	(377.39, 88.06) --
	(376.22, 88.65) --
	(375.44, 88.96) --
	(373.49, 89.83) --
	(371.91, 90.60) --
	(371.54, 90.76) --
	(369.59, 91.72) --
	(368.11, 92.55) --
	(367.64, 92.81) --
	(365.69, 93.99) --
	(364.95, 94.50) --
	(363.74, 95.34) --
	(362.31, 96.45) --
	(361.79, 96.88) --
	(360.08, 98.40) --
	(359.84, 98.63) --
	(358.15,100.35) --
	(357.88,100.65) --
	(356.44,102.31) --
	(355.93,102.94) --
	(354.88,104.26) --
	(353.98,105.48) --
	(353.42,106.21) --
	(352.04,108.16) --
	(352.03,108.17) --
	(350.62,110.11) --
	(350.08,110.90) --
	(349.14,112.06) --
	(348.13,113.38) --
	(347.52,114.01) --
	(346.18,115.44) --
	(345.49,115.96);

\path[draw=drawColor,line width= 1.1pt,line join=round] (301.32, 62.63) --
	(303.27, 62.31) --
	(305.22, 61.90) --
	(307.17, 61.37) --
	(307.26, 61.34) --
	(309.12, 60.72) --
	(311.07, 59.90) --
	(312.07, 59.39) --
	(313.02, 58.89) --
	(314.97, 57.66) --
	(315.27, 57.44) --
	(316.92, 56.16) --
	(317.67, 55.49) --
	(318.87, 54.34) --
	(319.62, 53.54) --
	(320.82, 52.17) --
	(321.30, 51.59) --
	(322.77, 49.68) --
	(322.81, 49.64) --
	(324.19, 47.69) --
	(324.72, 46.91) --
	(325.54, 45.74) --
	(326.68, 44.07) --
	(326.89, 43.79) --
	(328.29, 41.84) --
	(328.63, 41.37) --
	(329.84, 39.89) --
	(330.58, 38.98) --
	(331.61, 37.94);

\path[draw=drawColor,line width= 1.1pt,line join=round] (301.32, 89.40) --
	(303.27, 89.91) --
	(305.22, 90.52) --
	(305.42, 90.60) --
	(307.17, 91.23) --
	(309.12, 92.06) --
	(310.08, 92.55) --
	(311.07, 93.06) --
	(313.02, 94.23) --
	(313.42, 94.50) --
	(314.97, 95.63) --
	(315.97, 96.45) --
	(316.92, 97.30) --
	(318.04, 98.40) --
	(318.87, 99.30) --
	(319.77,100.35) --
	(320.82,101.72) --
	(321.25,102.31) --
	(322.55,104.26) --
	(322.77,104.64) --
	(323.69,106.21) --
	(324.72,108.16) --
	(324.73,108.16) --
	(325.68,110.11) --
	(326.56,112.06) --
	(326.68,112.35) --
	(327.41,114.01) --
	(328.22,115.96);

\path[draw=drawColor,line width= 1.1pt,line join=round] (351.87, 37.94) --
	(352.03, 38.09) --
	(353.58, 39.89) --
	(353.98, 40.37) --
	(355.04, 41.84) --
	(355.93, 43.11) --
	(356.36, 43.79) --
	(357.59, 45.74) --
	(357.88, 46.20) --
	(358.78, 47.69) --
	(359.84, 49.39) --
	(359.99, 49.64) --
	(361.23, 51.59) --
	(361.79, 52.41) --
	(362.56, 53.54) --
	(363.74, 55.15) --
	(364.00, 55.49) --
	(365.60, 57.44) --
	(365.69, 57.54) --
	(367.42, 59.39) --
	(367.64, 59.61) --
	(369.50, 61.34) --
	(369.59, 61.42) --
	(371.54, 63.01) --
	(371.92, 63.29) --
	(373.49, 64.43) --
	(374.73, 65.24) --
	(375.44, 65.71) --
	(377.39, 66.87) --
	(377.99, 67.19) --
	(379.34, 67.96);

\path[draw=drawColor,line width= 1.1pt,line join=round] (379.34, 92.89) --
	(377.39, 93.31) --
	(375.44, 93.82) --
	(373.49, 94.45) --
	(373.35, 94.50) --
	(371.54, 95.18) --
	(369.59, 96.06) --
	(368.86, 96.45) --
	(367.64, 97.12) --
	(365.69, 98.37) --
	(365.64, 98.40) --
	(363.74, 99.87) --
	(363.18,100.35) --
	(361.79,101.67) --
	(361.17,102.31) --
	(359.84,103.84) --
	(359.49,104.26) --
	(358.04,106.21) --
	(357.88,106.44) --
	(356.76,108.16) --
	(355.93,109.57) --
	(355.61,110.11) --
	(354.55,112.06) --
	(353.98,113.22) --
	(353.56,114.01) --
	(352.62,115.96);

\path[draw=drawColor,line width= 1.1pt,line join=round] (301.32, 57.26) --
	(303.27, 57.10) --
	(305.22, 56.79) --
	(307.17, 56.31) --
	(309.12, 55.64) --
	(309.45, 55.49) --
	(311.07, 54.71) --
	(313.02, 53.54) --
	(313.02, 53.54) --
	(314.97, 51.99) --
	(315.41, 51.59) --
	(316.92, 50.04) --
	(317.28, 49.64) --
	(318.82, 47.69) --
	(318.87, 47.62) --
	(320.13, 45.74) --
	(320.82, 44.60) --
	(321.31, 43.79) --
	(322.40, 41.84) --
	(322.77, 41.11) --
	(323.42, 39.89) --
	(324.42, 37.94);

\path[draw=drawColor,line width= 1.1pt,line join=round] (301.32, 94.39) --
	(301.92, 94.50) --
	(303.27, 94.75) --
	(305.22, 95.26) --
	(307.17, 95.92) --
	(308.43, 96.45) --
	(309.12, 96.76) --
	(311.07, 97.84) --
	(311.93, 98.40) --
	(313.02, 99.19) --
	(314.40,100.35) --
	(314.97,100.89) --
	(316.30,102.31) --
	(316.92,103.06) --
	(317.82,104.26) --
	(318.87,105.88) --
	(319.07,106.21) --
	(320.10,108.16) --
	(320.82,109.78) --
	(320.96,110.11) --
	(321.67,112.06) --
	(322.26,114.01) --
	(322.73,115.96);

\path[draw=drawColor,line width= 1.1pt,line join=round] (379.34, 97.54) --
	(377.39, 97.78) --
	(375.44, 98.16) --
	(374.56, 98.40) --
	(373.49, 98.71) --
	(371.54, 99.44) --
	(369.60,100.35) --
	(369.59,100.36) --
	(367.64,101.57) --
	(366.64,102.31) --
	(365.69,103.09) --
	(364.44,104.26) --
	(363.74,105.01) --
	(362.73,106.21) --
	(361.79,107.49) --
	(361.33,108.16) --
	(360.19,110.11) --
	(359.84,110.82) --
	(359.23,112.06) --
	(358.43,114.01) --
	(357.88,115.63) --
	(357.77,115.96);

\path[draw=drawColor,line width= 1.1pt,line join=round] (359.18, 37.94) --
	(359.84, 39.43) --
	(360.02, 39.89) --
	(360.87, 41.84) --
	(361.74, 43.79) --
	(361.79, 43.89) --
	(362.66, 45.74) --
	(363.64, 47.69) --
	(363.74, 47.85) --
	(364.75, 49.64) --
	(365.69, 51.16) --
	(365.96, 51.59) --
	(367.36, 53.54) --
	(367.64, 53.89) --
	(369.00, 55.49) --
	(369.59, 56.13) --
	(370.93, 57.44) --
	(371.54, 57.99) --
	(373.26, 59.39) --
	(373.49, 59.57) --
	(375.44, 60.89) --
	(376.19, 61.34) --
	(377.39, 62.03) --
	(379.34, 63.02);

\path[draw=drawColor,line width= 1.1pt,line join=round] (301.32, 50.95) --
	(303.27, 50.97) --
	(305.22, 50.72) --
	(307.17, 50.19) --
	(308.47, 49.64) --
	(309.12, 49.32) --
	(311.07, 48.01) --
	(311.45, 47.69) --
	(313.02, 46.12) --
	(313.34, 45.74) --
	(314.72, 43.79) --
	(314.97, 43.35) --
	(315.75, 41.84) --
	(316.56, 39.89) --
	(316.92, 38.70) --
	(317.15, 37.94);

\path[draw=drawColor,line width= 1.1pt,line join=round] (301.32, 99.86) --
	(303.27,100.07) --
	(304.53,100.35) --
	(305.22,100.52) --
	(307.17,101.26) --
	(309.12,102.25) --
	(309.20,102.31) --
	(311.07,103.71) --
	(311.67,104.26) --
	(313.02,105.71) --
	(313.41,106.21) --
	(314.67,108.16) --
	(314.97,108.75) --
	(315.58,110.11) --
	(316.22,112.06) --
	(316.64,114.01) --
	(316.83,115.96);

\path[draw=drawColor,line width= 1.1pt,line join=round] (379.34,102.38) --
	(377.39,102.50) --
	(375.44,102.84) --
	(373.49,103.41) --
	(371.54,104.23) --
	(371.50,104.26) --
	(369.59,105.46) --
	(368.64,106.21) --
	(367.64,107.14) --
	(366.71,108.16) --
	(365.69,109.52) --
	(365.30,110.11) --
	(364.26,112.06) --
	(363.74,113.37) --
	(363.50,114.01) --
	(362.99,115.96);

\path[draw=drawColor,line width= 1.1pt,line join=round] (366.68, 37.94) --
	(366.85, 39.89) --
	(367.19, 41.84) --
	(367.64, 43.58) --
	(367.69, 43.79) --
	(368.41, 45.74) --
	(369.27, 47.69) --
	(369.59, 48.27) --
	(370.39, 49.64) --
	(371.54, 51.30) --
	(371.76, 51.59) --
	(373.49, 53.53) --
	(373.50, 53.54) --
	(375.44, 55.20) --
	(375.83, 55.49) --
	(377.39, 56.52) --
	(379.05, 57.44) --
	(379.34, 57.59);

\path[draw=drawColor,line width= 1.1pt,line join=round] (301.32,111.92) --
	(303.27,110.31) --
	(305.22,111.43) --
	(305.62,112.06) --
	(305.38,114.01) --
	(305.22,114.17) --
	(303.27,114.46) --
	(302.46,114.01) --
	(301.32,112.23);

\path[draw=drawColor,line width= 1.1pt,line join=round] (379.34,109.29) --
	(377.39,109.19) --
	(375.44,109.55) --
	(374.15,110.11) --
	(373.49,110.50) --
	(371.77,112.06) --
	(371.54,112.38) --
	(370.67,114.01) --
	(370.15,115.96);
\definecolor[named]{drawColor}{rgb}{0.00,0.00,0.00}

\path[draw=drawColor,draw opacity=0.33,line width= 1.1pt,line join=round] (313.02, 75.18) --
	(311.26, 76.95) --
	(313.02, 78.71) --
	(313.21, 78.90) --
	(314.97, 80.66) --
	(315.16, 80.85) --
	(316.92, 82.61) --
	(317.11, 82.80) --
	(318.87, 84.56) --
	(319.06, 84.75) --
	(320.82, 86.51) --
	(321.01, 86.70) --
	(322.77, 88.46) --
	(322.96, 88.65) --
	(324.72, 90.41) --
	(324.91, 90.60) --
	(326.68, 92.37) --
	(326.86, 92.55) --
	(328.63, 94.32) --
	(328.81, 94.50) --
	(330.58, 96.27) --
	(330.76, 96.45) --
	(332.53, 98.22) --
	(332.71, 98.40) --
	(334.48,100.17) --
	(334.66,100.35) --
	(336.43,102.12) --
	(336.61,102.31) --
	(338.38,104.07) --
	(338.57,104.26) --
	(340.33,106.02) --
	(342.09,104.26) --
	(342.28,104.07) --
	(344.04,102.31) --
	(344.23,102.12) --
	(345.99,100.35) --
	(346.18,100.17) --
	(347.94, 98.40) --
	(348.13, 98.22) --
	(349.90, 96.45) --
	(350.08, 96.27) --
	(351.85, 94.50) --
	(352.03, 94.32) --
	(353.80, 92.55) --
	(353.98, 92.37) --
	(355.75, 90.60) --
	(355.93, 90.41) --
	(357.70, 88.65) --
	(357.88, 88.46) --
	(359.65, 86.70) --
	(359.84, 86.51) --
	(361.60, 84.75) --
	(361.79, 84.56) --
	(363.55, 82.80) --
	(363.74, 82.61) --
	(365.50, 80.85) --
	(365.69, 80.66) --
	(367.45, 78.90) --
	(367.64, 78.71) --
	(369.40, 76.95) --
	(367.64, 75.18) --
	(367.45, 75.00) --
	(365.69, 73.23) --
	(365.50, 73.05) --
	(363.74, 71.28) --
	(363.55, 71.10) --
	(361.79, 69.33) --
	(361.60, 69.15) --
	(359.84, 67.38) --
	(359.65, 67.19) --
	(357.88, 65.43) --
	(357.70, 65.24) --
	(355.93, 63.48) --
	(355.75, 63.29) --
	(353.98, 61.53) --
	(353.80, 61.34) --
	(352.03, 59.58) --
	(351.85, 59.39) --
	(350.08, 57.63) --
	(349.90, 57.44) --
	(348.13, 55.68) --
	(347.94, 55.49) --
	(346.18, 53.73) --
	(345.99, 53.54) --
	(344.23, 51.78) --
	(344.04, 51.59) --
	(342.28, 49.83) --
	(342.09, 49.64) --
	(340.33, 47.88) --
	(338.57, 49.64) --
	(338.38, 49.83) --
	(336.61, 51.59) --
	(336.43, 51.78) --
	(334.66, 53.54) --
	(334.48, 53.73) --
	(332.71, 55.49) --
	(332.53, 55.68) --
	(330.76, 57.44) --
	(330.58, 57.63) --
	(328.81, 59.39) --
	(328.63, 59.58) --
	(326.86, 61.34) --
	(326.68, 61.53) --
	(324.91, 63.29) --
	(324.72, 63.48) --
	(322.96, 65.24) --
	(322.77, 65.43) --
	(321.01, 67.19) --
	(320.82, 67.38) --
	(319.06, 69.15) --
	(318.87, 69.33) --
	(317.11, 71.10) --
	(316.92, 71.28) --
	(315.16, 73.05) --
	(314.97, 73.23) --
	(313.21, 75.00) --
	(313.02, 75.18);

\path[draw=drawColor,draw opacity=0.33,line width= 1.1pt,line join=round] (338.22, 37.94) --
	(336.43, 39.73) --
	(336.27, 39.89) --
	(334.48, 41.68) --
	(334.32, 41.84) --
	(332.53, 43.63) --
	(332.37, 43.79) --
	(330.58, 45.58) --
	(330.42, 45.74) --
	(328.63, 47.53) --
	(328.47, 47.69) --
	(326.68, 49.48) --
	(326.52, 49.64) --
	(324.72, 51.43) --
	(324.57, 51.59) --
	(322.77, 53.38) --
	(322.62, 53.54) --
	(320.82, 55.33) --
	(320.67, 55.49) --
	(318.87, 57.29) --
	(318.72, 57.44) --
	(316.92, 59.24) --
	(316.77, 59.39) --
	(314.97, 61.19) --
	(314.82, 61.34) --
	(313.02, 63.14) --
	(312.86, 63.29) --
	(311.07, 65.09) --
	(310.91, 65.24) --
	(309.12, 67.04) --
	(308.96, 67.19) --
	(307.17, 68.99) --
	(307.01, 69.15) --
	(305.22, 70.94) --
	(305.06, 71.10) --
	(303.27, 72.89) --
	(303.11, 73.05) --
	(301.32, 74.84);

\path[draw=drawColor,draw opacity=0.33,line width= 1.1pt,line join=round] (301.32, 79.05) --
	(303.11, 80.85) --
	(303.27, 81.01) --
	(305.06, 82.80) --
	(305.22, 82.96) --
	(307.01, 84.75) --
	(307.17, 84.91) --
	(308.96, 86.70) --
	(309.12, 86.86) --
	(310.91, 88.65) --
	(311.07, 88.81) --
	(312.86, 90.60) --
	(313.02, 90.76) --
	(314.82, 92.55) --
	(314.97, 92.71) --
	(316.77, 94.50) --
	(316.92, 94.66) --
	(318.72, 96.45) --
	(318.87, 96.61) --
	(320.67, 98.40) --
	(320.82, 98.56) --
	(322.62,100.35) --
	(322.77,100.51) --
	(324.57,102.31) --
	(324.72,102.46) --
	(326.52,104.26) --
	(326.68,104.41) --
	(328.47,106.21) --
	(328.63,106.36) --
	(330.42,108.16) --
	(330.58,108.31) --
	(332.37,110.11) --
	(332.53,110.26) --
	(334.32,112.06) --
	(334.48,112.21) --
	(336.27,114.01) --
	(336.43,114.17) --
	(338.22,115.96);

\path[draw=drawColor,draw opacity=0.33,line width= 1.1pt,line join=round] (342.44, 37.94) --
	(344.23, 39.73) --
	(344.39, 39.89) --
	(346.18, 41.68) --
	(346.34, 41.84) --
	(348.13, 43.63) --
	(348.29, 43.79) --
	(350.08, 45.58) --
	(350.24, 45.74) --
	(352.03, 47.53) --
	(352.19, 47.69) --
	(353.98, 49.48) --
	(354.14, 49.64) --
	(355.93, 51.43) --
	(356.09, 51.59) --
	(357.88, 53.38) --
	(358.04, 53.54) --
	(359.84, 55.33) --
	(359.99, 55.49) --
	(361.79, 57.29) --
	(361.94, 57.44) --
	(363.74, 59.24) --
	(363.89, 59.39) --
	(365.69, 61.19) --
	(365.84, 61.34) --
	(367.64, 63.14) --
	(367.79, 63.29) --
	(369.59, 65.09) --
	(369.74, 65.24) --
	(371.54, 67.04) --
	(371.70, 67.19) --
	(373.49, 68.99) --
	(373.65, 69.15) --
	(375.44, 70.94) --
	(375.60, 71.10) --
	(377.39, 72.89) --
	(377.55, 73.05) --
	(379.34, 74.84);

\path[draw=drawColor,draw opacity=0.33,line width= 1.1pt,line join=round] (379.34, 79.05) --
	(377.55, 80.85) --
	(377.39, 81.01) --
	(375.60, 82.80) --
	(375.44, 82.96) --
	(373.65, 84.75) --
	(373.49, 84.91) --
	(371.70, 86.70) --
	(371.54, 86.86) --
	(369.74, 88.65) --
	(369.59, 88.81) --
	(367.79, 90.60) --
	(367.64, 90.76) --
	(365.84, 92.55) --
	(365.69, 92.71) --
	(363.89, 94.50) --
	(363.74, 94.66) --
	(361.94, 96.45) --
	(361.79, 96.61) --
	(359.99, 98.40) --
	(359.84, 98.56) --
	(358.04,100.35) --
	(357.88,100.51) --
	(356.09,102.31) --
	(355.93,102.46) --
	(354.14,104.26) --
	(353.98,104.41) --
	(352.19,106.21) --
	(352.03,106.36) --
	(350.24,108.16) --
	(350.08,108.31) --
	(348.29,110.11) --
	(348.13,110.26) --
	(346.34,112.06) --
	(346.18,112.21) --
	(344.39,114.01) --
	(344.23,114.17) --
	(342.44,115.96);

\path[draw=drawColor,draw opacity=0.33,line width= 1.1pt,line join=round] (328.98, 37.94) --
	(328.63, 38.29) --
	(327.03, 39.89) --
	(326.68, 40.24) --
	(325.08, 41.84) --
	(324.72, 42.19) --
	(323.13, 43.79) --
	(322.77, 44.14) --
	(321.18, 45.74) --
	(320.82, 46.10) --
	(319.23, 47.69) --
	(318.87, 48.05) --
	(317.28, 49.64) --
	(316.92, 50.00) --
	(315.33, 51.59) --
	(314.97, 51.95) --
	(313.38, 53.54) --
	(313.02, 53.90) --
	(311.43, 55.49) --
	(311.07, 55.85) --
	(309.48, 57.44) --
	(309.12, 57.80) --
	(307.53, 59.39) --
	(307.17, 59.75) --
	(305.58, 61.34) --
	(305.22, 61.70) --
	(303.63, 63.29) --
	(303.27, 63.65) --
	(301.67, 65.24) --
	(301.32, 65.60);

\path[draw=drawColor,draw opacity=0.33,line width= 1.1pt,line join=round] (301.32, 88.29) --
	(301.67, 88.65) --
	(303.27, 90.24) --
	(303.63, 90.60) --
	(305.22, 92.20) --
	(305.58, 92.55) --
	(307.17, 94.15) --
	(307.53, 94.50) --
	(309.12, 96.10) --
	(309.48, 96.45) --
	(311.07, 98.05) --
	(311.43, 98.40) --
	(313.02,100.00) --
	(313.38,100.35) --
	(314.97,101.95) --
	(315.33,102.31) --
	(316.92,103.90) --
	(317.28,104.26) --
	(318.87,105.85) --
	(319.23,106.21) --
	(320.82,107.80) --
	(321.18,108.16) --
	(322.77,109.75) --
	(323.13,110.11) --
	(324.72,111.70) --
	(325.08,112.06) --
	(326.68,113.65) --
	(327.03,114.01) --
	(328.63,115.60) --
	(328.98,115.96);

\path[draw=drawColor,draw opacity=0.33,line width= 1.1pt,line join=round] (351.68, 37.94) --
	(352.03, 38.29) --
	(353.63, 39.89) --
	(353.98, 40.24) --
	(355.58, 41.84) --
	(355.93, 42.19) --
	(357.53, 43.79) --
	(357.88, 44.14) --
	(359.48, 45.74) --
	(359.84, 46.10) --
	(361.43, 47.69) --
	(361.79, 48.05) --
	(363.38, 49.64) --
	(363.74, 50.00) --
	(365.33, 51.59) --
	(365.69, 51.95) --
	(367.28, 53.54) --
	(367.64, 53.90) --
	(369.23, 55.49) --
	(369.59, 55.85) --
	(371.18, 57.44) --
	(371.54, 57.80) --
	(373.13, 59.39) --
	(373.49, 59.75) --
	(375.08, 61.34) --
	(375.44, 61.70) --
	(377.03, 63.29) --
	(377.39, 63.65) --
	(378.98, 65.24) --
	(379.34, 65.60);

\path[draw=drawColor,draw opacity=0.33,line width= 1.1pt,line join=round] (379.34, 88.29) --
	(378.98, 88.65) --
	(377.39, 90.24) --
	(377.03, 90.60) --
	(375.44, 92.20) --
	(375.08, 92.55) --
	(373.49, 94.15) --
	(373.13, 94.50) --
	(371.54, 96.10) --
	(371.18, 96.45) --
	(369.59, 98.05) --
	(369.23, 98.40) --
	(367.64,100.00) --
	(367.28,100.35) --
	(365.69,101.95) --
	(365.33,102.31) --
	(363.74,103.90) --
	(363.38,104.26) --
	(361.79,105.85) --
	(361.43,106.21) --
	(359.84,107.80) --
	(359.48,108.16) --
	(357.88,109.75) --
	(357.53,110.11) --
	(355.93,111.70) --
	(355.58,112.06) --
	(353.98,113.65) --
	(353.63,114.01) --
	(352.03,115.60) --
	(351.68,115.96);

\path[draw=drawColor,draw opacity=0.33,line width= 1.1pt,line join=round] (321.19, 37.94) --
	(320.82, 38.30) --
	(319.24, 39.89) --
	(318.87, 40.25) --
	(317.29, 41.84) --
	(316.92, 42.20) --
	(315.34, 43.79) --
	(314.97, 44.15) --
	(313.39, 45.74) --
	(313.02, 46.11) --
	(311.44, 47.69) --
	(311.07, 48.06) --
	(309.49, 49.64) --
	(309.12, 50.01) --
	(307.54, 51.59) --
	(307.17, 51.96) --
	(305.59, 53.54) --
	(305.22, 53.91) --
	(303.64, 55.49) --
	(303.27, 55.86) --
	(301.68, 57.44) --
	(301.32, 57.81);

\path[draw=drawColor,draw opacity=0.33,line width= 1.1pt,line join=round] (301.32, 96.09) --
	(301.68, 96.45) --
	(303.27, 98.04) --
	(303.64, 98.40) --
	(305.22, 99.99) --
	(305.59,100.35) --
	(307.17,101.94) --
	(307.54,102.31) --
	(309.12,103.89) --
	(309.49,104.26) --
	(311.07,105.84) --
	(311.44,106.21) --
	(313.02,107.79) --
	(313.39,108.16) --
	(314.97,109.74) --
	(315.34,110.11) --
	(316.92,111.69) --
	(317.29,112.06) --
	(318.87,113.64) --
	(319.24,114.01) --
	(320.82,115.59) --
	(321.19,115.96);

\path[draw=drawColor,draw opacity=0.33,line width= 1.1pt,line join=round] (359.47, 37.94) --
	(359.84, 38.30) --
	(361.42, 39.89) --
	(361.79, 40.25) --
	(363.37, 41.84) --
	(363.74, 42.20) --
	(365.32, 43.79) --
	(365.69, 44.15) --
	(367.27, 45.74) --
	(367.64, 46.11) --
	(369.22, 47.69) --
	(369.59, 48.06) --
	(371.17, 49.64) --
	(371.54, 50.01) --
	(373.12, 51.59) --
	(373.49, 51.96) --
	(375.07, 53.54) --
	(375.44, 53.91) --
	(377.02, 55.49) --
	(377.39, 55.86) --
	(378.97, 57.44) --
	(379.34, 57.81);

\path[draw=drawColor,draw opacity=0.33,line width= 1.1pt,line join=round] (379.34, 96.09) --
	(378.97, 96.45) --
	(377.39, 98.04) --
	(377.02, 98.40) --
	(375.44, 99.99) --
	(375.07,100.35) --
	(373.49,101.94) --
	(373.12,102.31) --
	(371.54,103.89) --
	(371.17,104.26) --
	(369.59,105.84) --
	(369.22,106.21) --
	(367.64,107.79) --
	(367.27,108.16) --
	(365.69,109.74) --
	(365.32,110.11) --
	(363.74,111.69) --
	(363.37,112.06) --
	(361.79,113.64) --
	(361.42,114.01) --
	(359.84,115.59) --
	(359.47,115.96);

\path[draw=drawColor,draw opacity=0.33,line width= 1.1pt,line join=round] (314.33, 37.94) --
	(313.02, 39.24) --
	(312.38, 39.89) --
	(311.07, 41.19) --
	(310.43, 41.84) --
	(309.12, 43.14) --
	(308.48, 43.79) --
	(307.17, 45.09) --
	(306.53, 45.74) --
	(305.22, 47.05) --
	(304.57, 47.69) --
	(303.27, 49.00) --
	(302.62, 49.64) --
	(301.32, 50.95);

\path[draw=drawColor,draw opacity=0.33,line width= 1.1pt,line join=round] (301.32,102.95) --
	(302.62,104.26) --
	(303.27,104.90) --
	(304.57,106.21) --
	(305.22,106.85) --
	(306.53,108.16) --
	(307.17,108.80) --
	(308.48,110.11) --
	(309.12,110.75) --
	(310.43,112.06) --
	(311.07,112.70) --
	(312.38,114.01) --
	(313.02,114.65) --
	(314.33,115.96);

\path[draw=drawColor,draw opacity=0.33,line width= 1.1pt,line join=round] (366.33, 37.94) --
	(367.64, 39.24) --
	(368.28, 39.89) --
	(369.59, 41.19) --
	(370.23, 41.84) --
	(371.54, 43.14) --
	(372.18, 43.79) --
	(373.49, 45.09) --
	(374.13, 45.74) --
	(375.44, 47.05) --
	(376.08, 47.69) --
	(377.39, 49.00) --
	(378.03, 49.64) --
	(379.34, 50.95);

\path[draw=drawColor,draw opacity=0.33,line width= 1.1pt,line join=round] (379.34,102.95) --
	(378.03,104.26) --
	(377.39,104.90) --
	(376.08,106.21) --
	(375.44,106.85) --
	(374.13,108.16) --
	(373.49,108.80) --
	(372.18,110.11) --
	(371.54,110.75) --
	(370.23,112.06) --
	(369.59,112.70) --
	(368.28,114.01) --
	(367.64,114.65) --
	(366.33,115.96);

\path[draw=drawColor,draw opacity=0.33,line width= 1.1pt,line join=round] (308.12, 37.94) --
	(307.17, 38.89) --
	(306.17, 39.89) --
	(305.22, 40.84) --
	(304.22, 41.84) --
	(303.27, 42.79) --
	(302.27, 43.79) --
	(301.32, 44.74);

\path[draw=drawColor,draw opacity=0.33,line width= 1.1pt,line join=round] (301.32,109.15) --
	(302.27,110.11) --
	(303.27,111.11) --
	(304.22,112.06) --
	(305.22,113.06) --
	(306.17,114.01) --
	(307.17,115.01) --
	(308.12,115.96);

\path[draw=drawColor,draw opacity=0.33,line width= 1.1pt,line join=round] (372.54, 37.94) --
	(373.49, 38.89) --
	(374.49, 39.89) --
	(375.44, 40.84) --
	(376.44, 41.84) --
	(377.39, 42.79) --
	(378.39, 43.79) --
	(379.34, 44.74);

\path[draw=drawColor,draw opacity=0.33,line width= 1.1pt,line join=round] (379.34,109.15) --
	(378.39,110.11) --
	(377.39,111.11) --
	(376.44,112.06) --
	(375.44,113.06) --
	(374.49,114.01) --
	(373.49,115.01) --
	(372.54,115.96);

\path[draw=drawColor,draw opacity=0.33,line width= 1.1pt,line join=round] (302.41, 37.94) --
	(301.32, 39.03);

\path[draw=drawColor,draw opacity=0.33,line width= 1.1pt,line join=round] (301.32,114.86) --
	(302.41,115.96);

\path[draw=drawColor,draw opacity=0.33,line width= 1.1pt,line join=round] (378.24, 37.94) --
	(379.34, 39.03);

\path[draw=drawColor,draw opacity=0.33,line width= 1.1pt,line join=round] (379.34,114.86) --
	(378.24,115.96);
\definecolor[named]{drawColor}{rgb}{0.50,0.50,0.50}

\path[draw=drawColor,line width= 0.6pt,line join=round,line cap=round] (297.42, 34.03) rectangle (383.24,119.86);
\end{scope}
\begin{scope}
\path[clip] (  0.00,  0.00) rectangle (505.89,144.54);
\definecolor[named]{drawColor}{rgb}{0.00,0.00,0.00}

\node[text=drawColor,anchor=base east,inner sep=0pt, outer sep=0pt, scale=  0.96] at ( 32.82, 47.63) {-2};

\node[text=drawColor,anchor=base east,inner sep=0pt, outer sep=0pt, scale=  0.96] at ( 32.82, 60.64) {-1};

\node[text=drawColor,anchor=base east,inner sep=0pt, outer sep=0pt, scale=  0.96] at ( 32.82, 73.64) {0};

\node[text=drawColor,anchor=base east,inner sep=0pt, outer sep=0pt, scale=  0.96] at ( 32.82, 86.65) {1};

\node[text=drawColor,anchor=base east,inner sep=0pt, outer sep=0pt, scale=  0.96] at ( 32.82, 99.65) {2};
\end{scope}
\begin{scope}
\path[clip] (  0.00,  0.00) rectangle (505.89,144.54);
\definecolor[named]{drawColor}{rgb}{0.00,0.00,0.00}

\path[draw=drawColor,line width= 0.6pt,line join=round] ( 35.67, 50.94) --
	( 39.94, 50.94);

\path[draw=drawColor,line width= 0.6pt,line join=round] ( 35.67, 63.94) --
	( 39.94, 63.94);

\path[draw=drawColor,line width= 0.6pt,line join=round] ( 35.67, 76.95) --
	( 39.94, 76.95);

\path[draw=drawColor,line width= 0.6pt,line join=round] ( 35.67, 89.95) --
	( 39.94, 89.95);

\path[draw=drawColor,line width= 0.6pt,line join=round] ( 35.67,102.96) --
	( 39.94,102.96);
\end{scope}
\begin{scope}
\path[clip] (  0.00,  0.00) rectangle (505.89,144.54);
\definecolor[named]{drawColor}{rgb}{0.00,0.00,0.00}

\path[draw=drawColor,line width= 0.6pt,line join=round] ( 56.84, 29.77) --
	( 56.84, 34.03);

\path[draw=drawColor,line width= 0.6pt,line join=round] ( 69.85, 29.77) --
	( 69.85, 34.03);

\path[draw=drawColor,line width= 0.6pt,line join=round] ( 82.85, 29.77) --
	( 82.85, 34.03);

\path[draw=drawColor,line width= 0.6pt,line join=round] ( 95.85, 29.77) --
	( 95.85, 34.03);

\path[draw=drawColor,line width= 0.6pt,line join=round] (108.86, 29.77) --
	(108.86, 34.03);
\end{scope}
\begin{scope}
\path[clip] (  0.00,  0.00) rectangle (505.89,144.54);
\definecolor[named]{drawColor}{rgb}{0.00,0.00,0.00}

\node[text=drawColor,anchor=base,inner sep=0pt, outer sep=0pt, scale=  0.96] at ( 56.84, 20.31) {-2};

\node[text=drawColor,anchor=base,inner sep=0pt, outer sep=0pt, scale=  0.96] at ( 69.85, 20.31) {-1};

\node[text=drawColor,anchor=base,inner sep=0pt, outer sep=0pt, scale=  0.96] at ( 82.85, 20.31) {0};

\node[text=drawColor,anchor=base,inner sep=0pt, outer sep=0pt, scale=  0.96] at ( 95.85, 20.31) {1};

\node[text=drawColor,anchor=base,inner sep=0pt, outer sep=0pt, scale=  0.96] at (108.86, 20.31) {2};
\end{scope}
\begin{scope}
\path[clip] (  0.00,  0.00) rectangle (505.89,144.54);
\definecolor[named]{drawColor}{rgb}{0.00,0.00,0.00}

\path[draw=drawColor,line width= 0.6pt,line join=round] (142.67, 29.77) --
	(142.67, 34.03);

\path[draw=drawColor,line width= 0.6pt,line join=round] (155.67, 29.77) --
	(155.67, 34.03);

\path[draw=drawColor,line width= 0.6pt,line join=round] (168.68, 29.77) --
	(168.68, 34.03);

\path[draw=drawColor,line width= 0.6pt,line join=round] (181.68, 29.77) --
	(181.68, 34.03);

\path[draw=drawColor,line width= 0.6pt,line join=round] (194.68, 29.77) --
	(194.68, 34.03);
\end{scope}
\begin{scope}
\path[clip] (  0.00,  0.00) rectangle (505.89,144.54);
\definecolor[named]{drawColor}{rgb}{0.00,0.00,0.00}

\node[text=drawColor,anchor=base,inner sep=0pt, outer sep=0pt, scale=  0.96] at (142.67, 20.31) {-2};

\node[text=drawColor,anchor=base,inner sep=0pt, outer sep=0pt, scale=  0.96] at (155.67, 20.31) {-1};

\node[text=drawColor,anchor=base,inner sep=0pt, outer sep=0pt, scale=  0.96] at (168.68, 20.31) {0};

\node[text=drawColor,anchor=base,inner sep=0pt, outer sep=0pt, scale=  0.96] at (181.68, 20.31) {1};

\node[text=drawColor,anchor=base,inner sep=0pt, outer sep=0pt, scale=  0.96] at (194.68, 20.31) {2};
\end{scope}
\begin{scope}
\path[clip] (  0.00,  0.00) rectangle (505.89,144.54);
\definecolor[named]{drawColor}{rgb}{0.00,0.00,0.00}

\path[draw=drawColor,line width= 0.6pt,line join=round] (228.50, 29.77) --
	(228.50, 34.03);

\path[draw=drawColor,line width= 0.6pt,line join=round] (241.50, 29.77) --
	(241.50, 34.03);

\path[draw=drawColor,line width= 0.6pt,line join=round] (254.50, 29.77) --
	(254.50, 34.03);

\path[draw=drawColor,line width= 0.6pt,line join=round] (267.51, 29.77) --
	(267.51, 34.03);

\path[draw=drawColor,line width= 0.6pt,line join=round] (280.51, 29.77) --
	(280.51, 34.03);
\end{scope}
\begin{scope}
\path[clip] (  0.00,  0.00) rectangle (505.89,144.54);
\definecolor[named]{drawColor}{rgb}{0.00,0.00,0.00}

\node[text=drawColor,anchor=base,inner sep=0pt, outer sep=0pt, scale=  0.96] at (228.50, 20.31) {-2};

\node[text=drawColor,anchor=base,inner sep=0pt, outer sep=0pt, scale=  0.96] at (241.50, 20.31) {-1};

\node[text=drawColor,anchor=base,inner sep=0pt, outer sep=0pt, scale=  0.96] at (254.50, 20.31) {0};

\node[text=drawColor,anchor=base,inner sep=0pt, outer sep=0pt, scale=  0.96] at (267.51, 20.31) {1};

\node[text=drawColor,anchor=base,inner sep=0pt, outer sep=0pt, scale=  0.96] at (280.51, 20.31) {2};
\end{scope}
\begin{scope}
\path[clip] (  0.00,  0.00) rectangle (505.89,144.54);
\definecolor[named]{drawColor}{rgb}{0.00,0.00,0.00}

\path[draw=drawColor,line width= 0.6pt,line join=round] (314.32, 29.77) --
	(314.32, 34.03);

\path[draw=drawColor,line width= 0.6pt,line join=round] (327.33, 29.77) --
	(327.33, 34.03);

\path[draw=drawColor,line width= 0.6pt,line join=round] (340.33, 29.77) --
	(340.33, 34.03);

\path[draw=drawColor,line width= 0.6pt,line join=round] (353.33, 29.77) --
	(353.33, 34.03);

\path[draw=drawColor,line width= 0.6pt,line join=round] (366.34, 29.77) --
	(366.34, 34.03);
\end{scope}
\begin{scope}
\path[clip] (  0.00,  0.00) rectangle (505.89,144.54);
\definecolor[named]{drawColor}{rgb}{0.00,0.00,0.00}

\node[text=drawColor,anchor=base,inner sep=0pt, outer sep=0pt, scale=  0.96] at (314.32, 20.31) {-2};

\node[text=drawColor,anchor=base,inner sep=0pt, outer sep=0pt, scale=  0.96] at (327.33, 20.31) {-1};

\node[text=drawColor,anchor=base,inner sep=0pt, outer sep=0pt, scale=  0.96] at (340.33, 20.31) {0};

\node[text=drawColor,anchor=base,inner sep=0pt, outer sep=0pt, scale=  0.96] at (353.33, 20.31) {1};

\node[text=drawColor,anchor=base,inner sep=0pt, outer sep=0pt, scale=  0.96] at (366.34, 20.31) {2};
\end{scope}
\begin{scope}
\path[clip] (  0.00,  0.00) rectangle (505.89,144.54);
\definecolor[named]{drawColor}{rgb}{0.00,0.00,0.00}

\node[text=drawColor,anchor=base,inner sep=0pt, outer sep=0pt, scale=  1.20] at (211.59,  9.03) {feature 1};
\end{scope}
\begin{scope}
\path[clip] (  0.00,  0.00) rectangle (505.89,144.54);
\definecolor[named]{drawColor}{rgb}{0.00,0.00,0.00}

\node[text=drawColor,rotate= 90.00,anchor=base,inner sep=0pt, outer sep=0pt, scale=  1.20] at ( 21.82, 76.95) {feature 2};
\end{scope}
\begin{scope}
\path[clip] (  0.00,  0.00) rectangle (505.89,144.54);
\definecolor[named]{fillColor}{rgb}{1.00,1.00,1.00}

\path[fill=fillColor] (392.11, 70.32) rectangle (477.99,137.28);
\end{scope}
\begin{scope}
\path[clip] (  0.00,  0.00) rectangle (505.89,144.54);
\definecolor[named]{drawColor}{rgb}{0.00,0.00,0.00}

\node[text=drawColor,anchor=base west,inner sep=0pt, outer sep=0pt, scale=  0.96] at (396.38,126.38) {\bfseries label};
\end{scope}
\begin{scope}
\path[clip] (  0.00,  0.00) rectangle (505.89,144.54);
\definecolor[named]{drawColor}{rgb}{0.80,0.80,0.80}
\definecolor[named]{fillColor}{rgb}{1.00,1.00,1.00}

\path[draw=drawColor,line width= 0.6pt,line join=round,line cap=round,fill=fillColor] (396.38, 98.68) rectangle (410.83,122.77);
\end{scope}
\begin{scope}
\path[clip] (  0.00,  0.00) rectangle (505.89,144.54);
\definecolor[named]{drawColor}{rgb}{0.60,0.31,0.64}

\path[draw=drawColor,line width= 1.1pt,line join=round] (397.82,110.73) -- (409.39,110.73);
\end{scope}
\begin{scope}
\path[clip] (  0.00,  0.00) rectangle (505.89,144.54);
\definecolor[named]{drawColor}{rgb}{0.60,0.31,0.64}

\path[draw=drawColor,line width= 1.1pt,line join=round] (397.82,110.73) -- (409.39,110.73);
\end{scope}
\begin{scope}
\path[clip] (  0.00,  0.00) rectangle (505.89,144.54);
\definecolor[named]{drawColor}{rgb}{0.80,0.80,0.80}
\definecolor[named]{fillColor}{rgb}{1.00,1.00,1.00}

\path[draw=drawColor,line width= 0.6pt,line join=round,line cap=round,fill=fillColor] (396.38, 74.59) rectangle (410.83, 98.68);
\end{scope}
\begin{scope}
\path[clip] (  0.00,  0.00) rectangle (505.89,144.54);
\definecolor[named]{drawColor}{rgb}{1.00,0.50,0.00}

\path[draw=drawColor,line width= 1.1pt,line join=round] (397.82, 86.64) -- (409.39, 86.64);
\end{scope}
\begin{scope}
\path[clip] (  0.00,  0.00) rectangle (505.89,144.54);
\definecolor[named]{drawColor}{rgb}{1.00,0.50,0.00}

\path[draw=drawColor,line width= 1.1pt,line join=round] (397.82, 86.64) -- (409.39, 86.64);
\end{scope}
\begin{scope}
\path[clip] (  0.00,  0.00) rectangle (505.89,144.54);
\definecolor[named]{drawColor}{rgb}{0.00,0.00,0.00}

\node[text=drawColor,anchor=base west,inner sep=0pt, outer sep=0pt, scale=  0.96] at (412.64,112.60) {equality pair};

\node[text=drawColor,anchor=base west,inner sep=0pt, outer sep=0pt, scale=  0.96] at (412.64,102.24) {$y_i=0$};
\end{scope}
\begin{scope}
\path[clip] (  0.00,  0.00) rectangle (505.89,144.54);
\definecolor[named]{drawColor}{rgb}{0.00,0.00,0.00}

\node[text=drawColor,anchor=base west,inner sep=0pt, outer sep=0pt, scale=  0.96] at (412.64, 88.51) {inequality pair};

\node[text=drawColor,anchor=base west,inner sep=0pt, outer sep=0pt, scale=  0.96] at (412.64, 78.15) {$y_i\in\{-1,1\}$};
\end{scope}
\begin{scope}
\path[clip] (  0.00,  0.00) rectangle (505.89,144.54);
\definecolor[named]{fillColor}{rgb}{1.00,1.00,1.00}

\path[fill=fillColor] (392.11, 16.62) rectangle (480.46, 64.30);
\end{scope}
\begin{scope}
\path[clip] (  0.00,  0.00) rectangle (505.89,144.54);
\definecolor[named]{drawColor}{rgb}{0.00,0.00,0.00}

\node[text=drawColor,anchor=base west,inner sep=0pt, outer sep=0pt, scale=  0.96] at (396.38, 53.41) {\bfseries ranking function};
\end{scope}
\begin{scope}
\path[clip] (  0.00,  0.00) rectangle (505.89,144.54);
\definecolor[named]{drawColor}{rgb}{0.80,0.80,0.80}
\definecolor[named]{fillColor}{rgb}{1.00,1.00,1.00}

\path[draw=drawColor,line width= 0.6pt,line join=round,line cap=round,fill=fillColor] (396.38, 35.34) rectangle (410.83, 49.79);
\end{scope}
\begin{scope}
\path[clip] (  0.00,  0.00) rectangle (505.89,144.54);
\definecolor[named]{drawColor}{rgb}{0.00,0.00,0.00}

\path[draw=drawColor,line width= 1.1pt,line join=round] (397.82, 42.57) -- (409.39, 42.57);
\end{scope}
\begin{scope}
\path[clip] (  0.00,  0.00) rectangle (505.89,144.54);
\definecolor[named]{drawColor}{rgb}{0.80,0.80,0.80}
\definecolor[named]{fillColor}{rgb}{1.00,1.00,1.00}

\path[draw=drawColor,line width= 0.6pt,line join=round,line cap=round,fill=fillColor] (396.38, 20.89) rectangle (410.83, 35.34);
\end{scope}
\begin{scope}
\path[clip] (  0.00,  0.00) rectangle (505.89,144.54);
\definecolor[named]{drawColor}{rgb}{0.00,0.00,0.00}

\path[draw=drawColor,draw opacity=0.33,line width= 1.1pt,line join=round] (397.82, 28.11) -- (409.39, 28.11);
\end{scope}
\begin{scope}
\path[clip] (  0.00,  0.00) rectangle (505.89,144.54);
\definecolor[named]{drawColor}{rgb}{0.00,0.00,0.00}

\node[text=drawColor,anchor=base west,inner sep=0pt, outer sep=0pt, scale=  0.96] at (412.64, 39.26) {learned};
\end{scope}
\begin{scope}
\path[clip] (  0.00,  0.00) rectangle (505.89,144.54);
\definecolor[named]{drawColor}{rgb}{0.00,0.00,0.00}

\node[text=drawColor,anchor=base west,inner sep=0pt, outer sep=0pt, scale=  0.96] at (412.64, 24.81) {truth};
\end{scope}
\end{tikzpicture}

  \vskip -1cm
  \caption{Application to a simulated pattern $r(\mathbf x)=||\mathbf
    x||_1^2$ where $\mathbf x\in\RR^2$. \textbf{Left}: the training
    data are $n=100$ pairs, half equality (segments indicate two
    points of equal rank), and half inequality (arrows point to the
    higher rank). \textbf{Others}: level curves of the learned ranking
    functions. The rank model does not directly model the equality
    pairs, so the rank2 and compare models recover the true pattern
    better.}
  \label{fig:norm-level-curves}
\end{figure*}

\begin{figure*}[b!]
  % Created by tikzDevice version 0.10.1 on 2018-01-31 09:59:17
% !TEX encoding = UTF-8 Unicode
\begin{tikzpicture}[x=1pt,y=1pt]
\definecolor{fillColor}{RGB}{255,255,255}
\path[use as bounding box,fill=fillColor,fill opacity=0.00] (0,0) rectangle (505.89,144.54);
\begin{scope}
\path[clip] (  0.00,  0.00) rectangle (505.89,144.54);
\definecolor{drawColor}{RGB}{255,255,255}
\definecolor{fillColor}{RGB}{255,255,255}

\path[draw=drawColor,line width= 0.6pt,line join=round,line cap=round,fill=fillColor] ( -0.00, -0.00) rectangle (505.89,144.54);
\end{scope}
\begin{scope}
\path[clip] ( 49.29,119.96) rectangle (142.36,138.54);
\definecolor{drawColor}{gray}{0.50}
\definecolor{fillColor}{gray}{0.80}

\path[draw=drawColor,line width= 0.2pt,line join=round,line cap=round,fill=fillColor] ( 49.29,119.96) rectangle (142.36,138.54);
\definecolor{drawColor}{gray}{0.10}

\node[text=drawColor,anchor=base,inner sep=0pt, outer sep=0pt, scale=  0.87] at ( 95.82,125.96) {$r(\mathbf x) = ||\mathbf x||_\infty^2$};
\end{scope}
\begin{scope}
\path[clip] (142.36,119.96) rectangle (235.42,138.54);
\definecolor{drawColor}{gray}{0.50}
\definecolor{fillColor}{gray}{0.80}

\path[draw=drawColor,line width= 0.2pt,line join=round,line cap=round,fill=fillColor] (142.36,119.96) rectangle (235.42,138.54);
\definecolor{drawColor}{gray}{0.10}

\node[text=drawColor,anchor=base,inner sep=0pt, outer sep=0pt, scale=  0.87] at (188.89,125.96) {$r(\mathbf x) = ||\mathbf x||_1^2$};
\end{scope}
\begin{scope}
\path[clip] (235.42,119.96) rectangle (328.49,138.54);
\definecolor{drawColor}{gray}{0.50}
\definecolor{fillColor}{gray}{0.80}

\path[draw=drawColor,line width= 0.2pt,line join=round,line cap=round,fill=fillColor] (235.42,119.96) rectangle (328.49,138.54);
\definecolor{drawColor}{gray}{0.10}

\node[text=drawColor,anchor=base,inner sep=0pt, outer sep=0pt, scale=  0.87] at (281.96,125.96) {$r(\mathbf x) = ||\mathbf x||_2^2$};
\end{scope}
\begin{scope}
\path[clip] (328.49,119.96) rectangle (421.56,138.54);
\definecolor{drawColor}{gray}{0.50}
\definecolor{fillColor}{gray}{0.80}

\path[draw=drawColor,line width= 0.2pt,line join=round,line cap=round,fill=fillColor] (328.49,119.96) rectangle (421.56,138.54);
\definecolor{drawColor}{gray}{0.10}

\node[text=drawColor,anchor=base,inner sep=0pt, outer sep=0pt, scale=  0.87] at (375.02,125.96) {sushi};
\end{scope}
\begin{scope}
\path[clip] ( 49.29, 33.41) rectangle (142.36,119.96);
\definecolor{fillColor}{RGB}{255,255,255}

\path[fill=fillColor] ( 49.29, 33.41) rectangle (142.36,119.96);
\definecolor{fillColor}{RGB}{135,206,235}

\path[fill=fillColor,fill opacity=0.50] ( 53.52, 78.24) --
	( 59.16, 72.37) --
	( 70.44, 65.55) --
	( 93.00, 58.37) --
	(138.13, 60.17) --
	(138.13, 57.22) --
	( 93.00, 55.04) --
	( 70.44, 58.40) --
	( 59.16, 66.28) --
	( 53.52, 57.23) --
	cycle;
\definecolor{fillColor}{RGB}{0,0,255}

\path[fill=fillColor,fill opacity=0.50] ( 53.52, 87.73) --
	( 59.16, 64.93) --
	( 70.44, 55.15) --
	( 93.00, 56.57) --
	(138.13, 56.43) --
	(138.13, 52.12) --
	( 93.00, 51.28) --
	( 70.44, 51.71) --
	( 59.16, 55.44) --
	( 53.52, 62.04) --
	cycle;
\definecolor{fillColor}{RGB}{0,0,0}

\path[fill=fillColor,fill opacity=0.50] ( 53.52, 76.49) --
	( 59.16, 67.62) --
	( 70.44, 58.69) --
	( 93.00, 51.94) --
	(138.13, 50.13) --
	(138.13, 47.19) --
	( 93.00, 46.38) --
	( 70.44, 54.13) --
	( 59.16, 49.57) --
	( 53.52, 58.98) --
	cycle;
\definecolor{fillColor}{RGB}{204,204,204}

\path[fill=fillColor,fill opacity=0.50] ( 53.52, 50.45) --
	( 59.16, 49.71) --
	( 70.44, 44.27) --
	( 93.00, 44.96) --
	(138.13, 45.57) --
	(138.13, 43.22) --
	( 93.00, 41.24) --
	( 70.44, 40.74) --
	( 59.16, 42.85) --
	( 53.52, 37.34) --
	cycle;
\definecolor{drawColor}{RGB}{135,206,235}

\path[draw=drawColor,line width= 1.7pt,line join=round] ( 53.52, 67.73) --
	( 59.16, 69.32) --
	( 70.44, 61.97) --
	( 93.00, 56.71) --
	(138.13, 58.70);
\definecolor{drawColor}{RGB}{0,0,255}

\path[draw=drawColor,line width= 1.7pt,line join=round] ( 53.52, 74.89) --
	( 59.16, 60.19) --
	( 70.44, 53.43) --
	( 93.00, 53.93) --
	(138.13, 54.28);
\definecolor{drawColor}{RGB}{0,0,0}

\path[draw=drawColor,line width= 1.7pt,line join=round] ( 53.52, 67.73) --
	( 59.16, 58.60) --
	( 70.44, 56.41) --
	( 93.00, 49.16) --
	(138.13, 48.66);
\definecolor{drawColor}{gray}{0.80}

\path[draw=drawColor,line width= 1.7pt,line join=round] ( 53.52, 43.90) --
	( 59.16, 46.28) --
	( 70.44, 42.51) --
	( 93.00, 43.10) --
	(138.13, 44.39);
\definecolor{drawColor}{gray}{0.50}

\path[draw=drawColor,line width= 0.6pt,line join=round,line cap=round] ( 49.29, 33.41) rectangle (142.36,119.96);
\end{scope}
\begin{scope}
\path[clip] (142.36, 33.41) rectangle (235.42,119.96);
\definecolor{fillColor}{RGB}{255,255,255}

\path[fill=fillColor] (142.36, 33.41) rectangle (235.42,119.96);
\definecolor{drawColor}{RGB}{0,0,0}

\path[draw=drawColor,line width= 2.3pt,line join=round] (152.23, 33.41) -- (152.23,119.96);
\definecolor{fillColor}{RGB}{135,206,235}

\path[fill=fillColor,fill opacity=0.50] (146.59, 78.72) --
	(152.23, 71.07) --
	(163.51, 67.62) --
	(186.07, 65.89) --
	(231.19, 60.98) --
	(231.19, 53.43) --
	(186.07, 60.64) --
	(163.51, 57.12) --
	(152.23, 57.25) --
	(146.59, 61.51) --
	cycle;
\definecolor{fillColor}{RGB}{0,0,255}

\path[fill=fillColor,fill opacity=0.50] (146.59, 88.70) --
	(152.23, 60.71) --
	(163.51, 59.98) --
	(186.07, 56.96) --
	(231.19, 54.63) --
	(231.19, 51.74) --
	(186.07, 50.70) --
	(163.51, 55.23) --
	(152.23, 58.87) --
	(146.59, 69.02) --
	cycle;
\definecolor{fillColor}{RGB}{0,0,0}

\path[fill=fillColor,fill opacity=0.50] (146.59, 72.50) --
	(152.23, 65.06) --
	(163.51, 58.62) --
	(186.07, 55.14) --
	(231.19, 50.75) --
	(231.19, 47.37) --
	(186.07, 45.37) --
	(163.51, 52.62) --
	(152.23, 54.52) --
	(146.59, 62.97) --
	cycle;
\definecolor{fillColor}{RGB}{204,204,204}

\path[fill=fillColor,fill opacity=0.50] (146.59, 49.19) --
	(152.23, 48.53) --
	(163.51, 45.76) --
	(186.07, 43.78) --
	(231.19, 43.84) --
	(231.19, 41.77) --
	(186.07, 43.02) --
	(163.51, 40.05) --
	(152.23, 42.44) --
	(146.59, 40.20) --
	cycle;
\definecolor{drawColor}{RGB}{135,206,235}

\path[draw=drawColor,line width= 1.7pt,line join=round] (146.59, 70.12) --
	(152.23, 64.16) --
	(163.51, 62.37) --
	(186.07, 63.26) --
	(231.19, 57.21);
\definecolor{drawColor}{RGB}{0,0,255}

\path[draw=drawColor,line width= 1.7pt,line join=round] (146.59, 78.86) --
	(152.23, 59.79) --
	(163.51, 57.60) --
	(186.07, 53.83) --
	(231.19, 53.18);
\definecolor{drawColor}{RGB}{0,0,0}

\path[draw=drawColor,line width= 1.7pt,line join=round] (146.59, 67.73) --
	(152.23, 59.79) --
	(163.51, 55.62) --
	(186.07, 50.25) --
	(231.19, 49.06);
\definecolor{drawColor}{gray}{0.80}

\path[draw=drawColor,line width= 1.7pt,line join=round] (146.59, 44.69) --
	(152.23, 45.48) --
	(163.51, 42.90) --
	(186.07, 43.40) --
	(231.19, 42.80);
\definecolor{drawColor}{gray}{0.50}

\path[draw=drawColor,line width= 0.6pt,line join=round,line cap=round] (142.36, 33.41) rectangle (235.42,119.96);
\end{scope}
\begin{scope}
\path[clip] (235.42, 33.41) rectangle (328.49,119.96);
\definecolor{fillColor}{RGB}{255,255,255}

\path[fill=fillColor] (235.42, 33.41) rectangle (328.49,119.96);
\definecolor{fillColor}{RGB}{135,206,235}

\path[fill=fillColor,fill opacity=0.50] (239.65, 76.86) --
	(245.29, 69.34) --
	(256.57, 52.84) --
	(279.14, 51.81) --
	(324.26, 52.40) --
	(324.26, 51.28) --
	(279.14, 47.30) --
	(256.57, 48.86) --
	(245.29, 43.08) --
	(239.65, 58.60) --
	cycle;
\definecolor{fillColor}{RGB}{0,0,255}

\path[fill=fillColor,fill opacity=0.50] (239.65, 74.21) --
	(245.29, 61.56) --
	(256.57, 50.71) --
	(279.14, 53.97) --
	(324.26, 54.28) --
	(324.26, 49.90) --
	(279.14, 47.73) --
	(256.57, 47.41) --
	(245.29, 49.28) --
	(239.65, 50.14) --
	cycle;
\definecolor{fillColor}{RGB}{0,0,0}

\path[fill=fillColor,fill opacity=0.50] (239.65, 62.01) --
	(245.29, 53.88) --
	(256.57, 53.56) --
	(279.14, 46.21) --
	(324.26, 47.16) --
	(324.26, 44.90) --
	(279.14, 41.98) --
	(256.57, 44.17) --
	(245.29, 41.07) --
	(239.65, 46.44) --
	cycle;
\definecolor{fillColor}{RGB}{204,204,204}

\path[fill=fillColor,fill opacity=0.50] (239.65, 54.09) --
	(245.29, 51.88) --
	(256.57, 46.48) --
	(279.14, 45.78) --
	(324.26, 46.13) --
	(324.26, 43.85) --
	(279.14, 39.63) --
	(256.57, 42.51) --
	(245.29, 39.88) --
	(239.65, 43.24) --
	cycle;
\definecolor{drawColor}{RGB}{135,206,235}

\path[draw=drawColor,line width= 1.7pt,line join=round] (239.65, 67.73) --
	(245.29, 56.21) --
	(256.57, 50.85) --
	(279.14, 49.56) --
	(324.26, 51.84);
\definecolor{drawColor}{RGB}{0,0,255}

\path[draw=drawColor,line width= 1.7pt,line join=round] (239.65, 62.17) --
	(245.29, 55.42) --
	(256.57, 49.06) --
	(279.14, 50.85) --
	(324.26, 52.09);
\definecolor{drawColor}{RGB}{0,0,0}

\path[draw=drawColor,line width= 1.7pt,line join=round] (239.65, 54.23) --
	(245.29, 47.47) --
	(256.57, 48.86) --
	(279.14, 44.09) --
	(324.26, 46.03);
\definecolor{drawColor}{gray}{0.80}

\path[draw=drawColor,line width= 1.7pt,line join=round] (239.65, 48.66) --
	(245.29, 45.88) --
	(256.57, 44.49) --
	(279.14, 42.70) --
	(324.26, 44.99);
\definecolor{drawColor}{gray}{0.50}

\path[draw=drawColor,line width= 0.6pt,line join=round,line cap=round] (235.42, 33.41) rectangle (328.49,119.96);
\end{scope}
\begin{scope}
\path[clip] (328.49, 33.41) rectangle (421.56,119.96);
\definecolor{fillColor}{RGB}{255,255,255}

\path[fill=fillColor] (328.49, 33.41) rectangle (421.56,119.96);
\definecolor{fillColor}{RGB}{135,206,235}

\path[fill=fillColor,fill opacity=0.50] (332.72,111.26) --
	(338.36,101.28) --
	(349.64,107.14) --
	(372.20,104.22) --
	(417.33,105.24) --
	(417.33, 97.37) --
	(372.20, 96.00) --
	(349.64, 91.90) --
	(338.36, 91.40) --
	(332.72, 92.54) --
	cycle;
\definecolor{fillColor}{RGB}{0,0,255}

\path[fill=fillColor,fill opacity=0.50] (332.72,116.02) --
	(338.36,102.35) --
	(349.64,104.15) --
	(372.20,100.43) --
	(417.33, 97.54) --
	(417.33, 96.13) --
	(372.20, 96.62) --
	(349.64, 92.51) --
	(338.36, 89.54) --
	(332.72, 89.37) --
	cycle;
\definecolor{fillColor}{RGB}{0,0,0}

\path[fill=fillColor,fill opacity=0.50] (332.72,113.04) --
	(338.36, 96.04) --
	(349.64, 97.28) --
	(372.20, 95.07) --
	(417.33, 93.17) --
	(417.33, 91.76) --
	(372.20, 87.08) --
	(349.64, 82.29) --
	(338.36, 89.49) --
	(332.72, 95.53) --
	cycle;
\definecolor{drawColor}{RGB}{135,206,235}

\path[draw=drawColor,line width= 1.7pt,line join=round] (332.72,101.90) --
	(338.36, 96.34) --
	(349.64, 99.52) --
	(372.20,100.11) --
	(417.33,101.31);
\definecolor{drawColor}{RGB}{0,0,255}

\path[draw=drawColor,line width= 1.7pt,line join=round] (332.72,102.70) --
	(338.36, 95.94) --
	(349.64, 98.33) --
	(372.20, 98.52) --
	(417.33, 96.84);
\definecolor{drawColor}{RGB}{0,0,0}

\path[draw=drawColor,line width= 1.7pt,line join=round] (332.72,104.29) --
	(338.36, 92.76) --
	(349.64, 89.78) --
	(372.20, 91.08) --
	(417.33, 92.47);
\definecolor{drawColor}{gray}{0.50}

\path[draw=drawColor,line width= 0.6pt,line join=round,line cap=round] (328.49, 33.41) rectangle (421.56,119.96);
\end{scope}
\begin{scope}
\path[clip] (  0.00,  0.00) rectangle (505.89,144.54);
\definecolor{drawColor}{RGB}{0,0,0}

\node[text=drawColor,anchor=base east,inner sep=0pt, outer sep=0pt, scale=  0.87] at ( 43.89, 44.58) {10};

\node[text=drawColor,anchor=base east,inner sep=0pt, outer sep=0pt, scale=  0.87] at ( 43.89, 60.47) {20};

\node[text=drawColor,anchor=base east,inner sep=0pt, outer sep=0pt, scale=  0.87] at ( 43.89, 76.36) {30};

\node[text=drawColor,anchor=base east,inner sep=0pt, outer sep=0pt, scale=  0.87] at ( 43.89, 92.25) {40};

\node[text=drawColor,anchor=base east,inner sep=0pt, outer sep=0pt, scale=  0.87] at ( 43.89,108.15) {50};
\end{scope}
\begin{scope}
\path[clip] (  0.00,  0.00) rectangle (505.89,144.54);
\definecolor{drawColor}{RGB}{0,0,0}

\path[draw=drawColor,line width= 0.6pt,line join=round] ( 46.29, 47.87) --
	( 49.29, 47.87);

\path[draw=drawColor,line width= 0.6pt,line join=round] ( 46.29, 63.76) --
	( 49.29, 63.76);

\path[draw=drawColor,line width= 0.6pt,line join=round] ( 46.29, 79.65) --
	( 49.29, 79.65);

\path[draw=drawColor,line width= 0.6pt,line join=round] ( 46.29, 95.55) --
	( 49.29, 95.55);

\path[draw=drawColor,line width= 0.6pt,line join=round] ( 46.29,111.44) --
	( 49.29,111.44);
\end{scope}
\begin{scope}
\path[clip] (  0.00,  0.00) rectangle (505.89,144.54);
\definecolor{drawColor}{RGB}{0,0,0}

\path[draw=drawColor,line width= 0.6pt,line join=round] ( 70.44, 30.41) --
	( 70.44, 33.41);

\path[draw=drawColor,line width= 0.6pt,line join=round] ( 93.00, 30.41) --
	( 93.00, 33.41);

\path[draw=drawColor,line width= 0.6pt,line join=round] (115.56, 30.41) --
	(115.56, 33.41);

\path[draw=drawColor,line width= 0.6pt,line join=round] (138.13, 30.41) --
	(138.13, 33.41);
\end{scope}
\begin{scope}
\path[clip] (  0.00,  0.00) rectangle (505.89,144.54);
\definecolor{drawColor}{RGB}{0,0,0}

\node[text=drawColor,anchor=base,inner sep=0pt, outer sep=0pt, scale=  0.87] at ( 70.44, 21.43) {200};

\node[text=drawColor,anchor=base,inner sep=0pt, outer sep=0pt, scale=  0.87] at ( 93.00, 21.43) {400};

\node[text=drawColor,anchor=base,inner sep=0pt, outer sep=0pt, scale=  0.87] at (115.56, 21.43) {600};

\node[text=drawColor,anchor=base,inner sep=0pt, outer sep=0pt, scale=  0.87] at (138.13, 21.43) {800};
\end{scope}
\begin{scope}
\path[clip] (  0.00,  0.00) rectangle (505.89,144.54);
\definecolor{drawColor}{RGB}{0,0,0}

\path[draw=drawColor,line width= 0.6pt,line join=round] (163.51, 30.41) --
	(163.51, 33.41);

\path[draw=drawColor,line width= 0.6pt,line join=round] (186.07, 30.41) --
	(186.07, 33.41);

\path[draw=drawColor,line width= 0.6pt,line join=round] (208.63, 30.41) --
	(208.63, 33.41);

\path[draw=drawColor,line width= 0.6pt,line join=round] (231.19, 30.41) --
	(231.19, 33.41);
\end{scope}
\begin{scope}
\path[clip] (  0.00,  0.00) rectangle (505.89,144.54);
\definecolor{drawColor}{RGB}{0,0,0}

\node[text=drawColor,anchor=base,inner sep=0pt, outer sep=0pt, scale=  0.87] at (163.51, 21.43) {200};

\node[text=drawColor,anchor=base,inner sep=0pt, outer sep=0pt, scale=  0.87] at (186.07, 21.43) {400};

\node[text=drawColor,anchor=base,inner sep=0pt, outer sep=0pt, scale=  0.87] at (208.63, 21.43) {600};

\node[text=drawColor,anchor=base,inner sep=0pt, outer sep=0pt, scale=  0.87] at (231.19, 21.43) {800};
\end{scope}
\begin{scope}
\path[clip] (  0.00,  0.00) rectangle (505.89,144.54);
\definecolor{drawColor}{RGB}{0,0,0}

\path[draw=drawColor,line width= 0.6pt,line join=round] (256.57, 30.41) --
	(256.57, 33.41);

\path[draw=drawColor,line width= 0.6pt,line join=round] (279.14, 30.41) --
	(279.14, 33.41);

\path[draw=drawColor,line width= 0.6pt,line join=round] (301.70, 30.41) --
	(301.70, 33.41);

\path[draw=drawColor,line width= 0.6pt,line join=round] (324.26, 30.41) --
	(324.26, 33.41);
\end{scope}
\begin{scope}
\path[clip] (  0.00,  0.00) rectangle (505.89,144.54);
\definecolor{drawColor}{RGB}{0,0,0}

\node[text=drawColor,anchor=base,inner sep=0pt, outer sep=0pt, scale=  0.87] at (256.57, 21.43) {200};

\node[text=drawColor,anchor=base,inner sep=0pt, outer sep=0pt, scale=  0.87] at (279.14, 21.43) {400};

\node[text=drawColor,anchor=base,inner sep=0pt, outer sep=0pt, scale=  0.87] at (301.70, 21.43) {600};

\node[text=drawColor,anchor=base,inner sep=0pt, outer sep=0pt, scale=  0.87] at (324.26, 21.43) {800};
\end{scope}
\begin{scope}
\path[clip] (  0.00,  0.00) rectangle (505.89,144.54);
\definecolor{drawColor}{RGB}{0,0,0}

\path[draw=drawColor,line width= 0.6pt,line join=round] (349.64, 30.41) --
	(349.64, 33.41);

\path[draw=drawColor,line width= 0.6pt,line join=round] (372.20, 30.41) --
	(372.20, 33.41);

\path[draw=drawColor,line width= 0.6pt,line join=round] (394.76, 30.41) --
	(394.76, 33.41);

\path[draw=drawColor,line width= 0.6pt,line join=round] (417.33, 30.41) --
	(417.33, 33.41);
\end{scope}
\begin{scope}
\path[clip] (  0.00,  0.00) rectangle (505.89,144.54);
\definecolor{drawColor}{RGB}{0,0,0}

\node[text=drawColor,anchor=base,inner sep=0pt, outer sep=0pt, scale=  0.87] at (349.64, 21.43) {200};

\node[text=drawColor,anchor=base,inner sep=0pt, outer sep=0pt, scale=  0.87] at (372.20, 21.43) {400};

\node[text=drawColor,anchor=base,inner sep=0pt, outer sep=0pt, scale=  0.87] at (394.76, 21.43) {600};

\node[text=drawColor,anchor=base,inner sep=0pt, outer sep=0pt, scale=  0.87] at (417.33, 21.43) {800};
\end{scope}
\begin{scope}
\path[clip] (  0.00,  0.00) rectangle (505.89,144.54);
\definecolor{drawColor}{RGB}{0,0,0}

\node[text=drawColor,anchor=base,inner sep=0pt, outer sep=0pt, scale=  1.09] at (235.42,  8.40) {$n=$ number of labeled pairs, half equality and half inequality};
\end{scope}
\begin{scope}
\path[clip] (  0.00,  0.00) rectangle (505.89,144.54);
\definecolor{drawColor}{RGB}{0,0,0}

\node[text=drawColor,rotate= 90.00,anchor=base,inner sep=0pt, outer sep=0pt, scale=  1.09] at ( 16.63, 76.68) {percent incorrectly};

\node[text=drawColor,rotate= 90.00,anchor=base,inner sep=0pt, outer sep=0pt, scale=  1.09] at ( 29.59, 76.68) {predicted test pairs};
\end{scope}
\begin{scope}
\path[clip] (  0.00,  0.00) rectangle (505.89,144.54);
\definecolor{fillColor}{RGB}{255,255,255}

\path[fill=fillColor] (430.09, 34.52) rectangle (491.35,118.85);
\end{scope}
\begin{scope}
\path[clip] (  0.00,  0.00) rectangle (505.89,144.54);
\definecolor{drawColor}{RGB}{0,0,0}

\node[text=drawColor,anchor=base west,inner sep=0pt, outer sep=0pt, scale=  1.09] at (434.36,106.36) {function};
\end{scope}
\begin{scope}
\path[clip] (  0.00,  0.00) rectangle (505.89,144.54);
\definecolor{drawColor}{gray}{0.80}
\definecolor{fillColor}{RGB}{255,255,255}

\path[draw=drawColor,line width= 0.6pt,line join=round,line cap=round,fill=fillColor] (434.36, 86.48) rectangle (450.26,102.38);
\end{scope}
\begin{scope}
\path[clip] (  0.00,  0.00) rectangle (505.89,144.54);
\definecolor{fillColor}{RGB}{135,206,235}

\path[fill=fillColor,fill opacity=0.50] (435.07, 87.19) rectangle (449.55,101.67);
\end{scope}
\begin{scope}
\path[clip] (  0.00,  0.00) rectangle (505.89,144.54);
\definecolor{drawColor}{RGB}{135,206,235}

\path[draw=drawColor,line width= 1.7pt,line join=round] (435.95, 94.43) -- (448.67, 94.43);
\end{scope}
\begin{scope}
\path[clip] (  0.00,  0.00) rectangle (505.89,144.54);
\definecolor{drawColor}{gray}{0.80}
\definecolor{fillColor}{RGB}{255,255,255}

\path[draw=drawColor,line width= 0.6pt,line join=round,line cap=round,fill=fillColor] (434.36, 70.58) rectangle (450.26, 86.48);
\end{scope}
\begin{scope}
\path[clip] (  0.00,  0.00) rectangle (505.89,144.54);
\definecolor{fillColor}{RGB}{0,0,255}

\path[fill=fillColor,fill opacity=0.50] (435.07, 71.29) rectangle (449.55, 85.77);
\end{scope}
\begin{scope}
\path[clip] (  0.00,  0.00) rectangle (505.89,144.54);
\definecolor{drawColor}{RGB}{0,0,255}

\path[draw=drawColor,line width= 1.7pt,line join=round] (435.95, 78.53) -- (448.67, 78.53);
\end{scope}
\begin{scope}
\path[clip] (  0.00,  0.00) rectangle (505.89,144.54);
\definecolor{drawColor}{gray}{0.80}
\definecolor{fillColor}{RGB}{255,255,255}

\path[draw=drawColor,line width= 0.6pt,line join=round,line cap=round,fill=fillColor] (434.36, 54.68) rectangle (450.26, 70.58);
\end{scope}
\begin{scope}
\path[clip] (  0.00,  0.00) rectangle (505.89,144.54);
\definecolor{fillColor}{RGB}{0,0,0}

\path[fill=fillColor,fill opacity=0.50] (435.07, 55.39) rectangle (449.55, 69.87);
\end{scope}
\begin{scope}
\path[clip] (  0.00,  0.00) rectangle (505.89,144.54);
\definecolor{drawColor}{RGB}{0,0,0}

\path[draw=drawColor,line width= 1.7pt,line join=round] (435.95, 62.63) -- (448.67, 62.63);
\end{scope}
\begin{scope}
\path[clip] (  0.00,  0.00) rectangle (505.89,144.54);
\definecolor{drawColor}{gray}{0.80}
\definecolor{fillColor}{RGB}{255,255,255}

\path[draw=drawColor,line width= 0.6pt,line join=round,line cap=round,fill=fillColor] (434.36, 38.78) rectangle (450.26, 54.68);
\end{scope}
\begin{scope}
\path[clip] (  0.00,  0.00) rectangle (505.89,144.54);
\definecolor{fillColor}{RGB}{204,204,204}

\path[fill=fillColor,fill opacity=0.50] (435.07, 39.50) rectangle (449.55, 53.97);
\end{scope}
\begin{scope}
\path[clip] (  0.00,  0.00) rectangle (505.89,144.54);
\definecolor{drawColor}{gray}{0.80}

\path[draw=drawColor,line width= 1.7pt,line join=round] (435.95, 46.73) -- (448.67, 46.73);
\end{scope}
\begin{scope}
\path[clip] (  0.00,  0.00) rectangle (505.89,144.54);
\definecolor{drawColor}{RGB}{0,0,0}

\node[text=drawColor,anchor=base west,inner sep=0pt, outer sep=0pt, scale=  0.87] at (452.25, 91.14) {rank};
\end{scope}
\begin{scope}
\path[clip] (  0.00,  0.00) rectangle (505.89,144.54);
\definecolor{drawColor}{RGB}{0,0,0}

\node[text=drawColor,anchor=base west,inner sep=0pt, outer sep=0pt, scale=  0.87] at (452.25, 75.24) {rank2};
\end{scope}
\begin{scope}
\path[clip] (  0.00,  0.00) rectangle (505.89,144.54);
\definecolor{drawColor}{RGB}{0,0,0}

\node[text=drawColor,anchor=base west,inner sep=0pt, outer sep=0pt, scale=  0.87] at (452.25, 59.34) {compare};
\end{scope}
\begin{scope}
\path[clip] (  0.00,  0.00) rectangle (505.89,144.54);
\definecolor{drawColor}{RGB}{0,0,0}

\node[text=drawColor,anchor=base west,inner sep=0pt, outer sep=0pt, scale=  0.87] at (452.25, 43.44) {truth};
\end{scope}
\end{tikzpicture}

  \vskip -1cm
  \caption{Test error for 3 different simulated patterns $r(\mathbf
    x)$ where $\mathbf x\in\RR^2$ and one real
    sushi data set where $\mathbf x\in\RR^{14}$. We randomly generated data sets
    with $\rho=1/2$ equality and 1/2 inequality pairs, then plotted test
    error as a function of data set size $n$ (a vertical line
    shows the data set which was used in
    Figure~\ref{fig:norm-level-curves}). Lines show mean and shaded
    bands show standard deviation over 4 test sets.}
  \label{fig:simulation-samples}
\end{figure*}

\begin{figure*}[b!]
  % Created by tikzDevice version 0.7.0 on 2014-01-30 13:33:20
% !TEX encoding = UTF-8 Unicode
\begin{tikzpicture}[x=1pt,y=1pt]
\definecolor[named]{fillColor}{rgb}{1.00,1.00,1.00}
\path[use as bounding box,fill=fillColor,fill opacity=0.00] (0,0) rectangle (505.89,144.54);
\begin{scope}
\path[clip] (  0.00,  0.00) rectangle (505.89,144.54);
\definecolor[named]{drawColor}{rgb}{1.00,1.00,1.00}
\definecolor[named]{fillColor}{rgb}{1.00,1.00,1.00}

\path[draw=drawColor,line width= 0.6pt,line join=round,line cap=round,fill=fillColor] (  0.00,  0.00) rectangle (505.89,144.54);
\end{scope}
\begin{scope}
\path[clip] ( 39.69,119.86) rectangle (133.79,132.50);
\definecolor[named]{drawColor}{rgb}{0.50,0.50,0.50}
\definecolor[named]{fillColor}{rgb}{0.80,0.80,0.80}

\path[draw=drawColor,line width= 0.6pt,line join=round,line cap=round,fill=fillColor] ( 39.69,119.86) rectangle (133.79,132.50);
\definecolor[named]{drawColor}{rgb}{0.00,0.00,0.00}

\node[text=drawColor,anchor=base,inner sep=0pt, outer sep=0pt, scale=  0.96] at ( 86.74,122.87) {$r(\mathbf x) = ||\mathbf x||_1^2$};
\end{scope}
\begin{scope}
\path[clip] (133.79,119.86) rectangle (227.89,132.50);
\definecolor[named]{drawColor}{rgb}{0.50,0.50,0.50}
\definecolor[named]{fillColor}{rgb}{0.80,0.80,0.80}

\path[draw=drawColor,line width= 0.6pt,line join=round,line cap=round,fill=fillColor] (133.79,119.86) rectangle (227.89,132.50);
\definecolor[named]{drawColor}{rgb}{0.00,0.00,0.00}

\node[text=drawColor,anchor=base,inner sep=0pt, outer sep=0pt, scale=  0.96] at (180.84,122.87) {$r(\mathbf x) = ||\mathbf x||_2^2$};
\end{scope}
\begin{scope}
\path[clip] (227.89,119.86) rectangle (321.99,132.50);
\definecolor[named]{drawColor}{rgb}{0.50,0.50,0.50}
\definecolor[named]{fillColor}{rgb}{0.80,0.80,0.80}

\path[draw=drawColor,line width= 0.6pt,line join=round,line cap=round,fill=fillColor] (227.89,119.86) rectangle (321.99,132.50);
\definecolor[named]{drawColor}{rgb}{0.00,0.00,0.00}

\node[text=drawColor,anchor=base,inner sep=0pt, outer sep=0pt, scale=  0.96] at (274.94,122.87) {$r(\mathbf x) = ||\mathbf x||_\infty^2$};
\end{scope}
\begin{scope}
\path[clip] (321.99,119.86) rectangle (416.09,132.50);
\definecolor[named]{drawColor}{rgb}{0.50,0.50,0.50}
\definecolor[named]{fillColor}{rgb}{0.80,0.80,0.80}

\path[draw=drawColor,line width= 0.6pt,line join=round,line cap=round,fill=fillColor] (321.99,119.86) rectangle (416.09,132.50);
\definecolor[named]{drawColor}{rgb}{0.00,0.00,0.00}

\node[text=drawColor,anchor=base,inner sep=0pt, outer sep=0pt, scale=  0.96] at (369.04,122.87) {sushi};
\end{scope}
\begin{scope}
\path[clip] ( 39.69, 34.03) rectangle (133.79,119.86);
\definecolor[named]{fillColor}{rgb}{1.00,1.00,1.00}

\path[fill=fillColor] ( 39.69, 34.03) rectangle (133.79,119.86);
\definecolor[named]{fillColor}{rgb}{0.00,0.00,0.00}

\path[fill=fillColor,fill opacity=0.25] ( 43.96,111.35) --
	( 54.66,110.79) --
	( 65.35,110.96) --
	( 76.04,109.63) --
	( 86.74,111.72) --
	( 97.43,109.98) --
	(108.12,111.14) --
	(118.82,108.36) --
	(129.51,110.71) --
	(129.51,102.49) --
	(118.82,102.85) --
	(108.12,108.27) --
	( 97.43,104.92) --
	( 86.74,106.72) --
	( 76.04,106.80) --
	( 65.35,106.67) --
	( 54.66,101.91) --
	( 43.96,103.85) --
	cycle;
\definecolor[named]{fillColor}{rgb}{0.53,0.81,0.92}

\path[fill=fillColor,fill opacity=0.25] ( 43.96,107.63) --
	( 54.66,108.54) --
	( 65.35,105.92) --
	( 76.04,103.81) --
	( 86.74, 99.90) --
	( 97.43,102.64) --
	(108.12,103.00) --
	(118.82, 96.93) --
	(129.51, 93.25) --
	(129.51, 89.04) --
	(118.82, 87.25) --
	(108.12, 95.36) --
	( 97.43, 94.07) --
	( 86.74, 95.57) --
	( 76.04, 98.78) --
	( 65.35,101.30) --
	( 54.66, 98.26) --
	( 43.96, 99.15) --
	cycle;
\definecolor[named]{fillColor}{rgb}{0.00,0.00,1.00}

\path[fill=fillColor,fill opacity=0.25] ( 43.96,110.79) --
	( 54.66,112.26) --
	( 65.35,110.99) --
	( 76.04,109.46) --
	( 86.74,110.68) --
	( 97.43,110.50) --
	(108.12,111.50) --
	(118.82,108.24) --
	(129.51,107.42) --
	(129.51,103.44) --
	(118.82,102.79) --
	(108.12,109.33) --
	( 97.43,102.68) --
	( 86.74,106.30) --
	( 76.04,107.88) --
	( 65.35,104.95) --
	( 54.66,101.97) --
	( 43.96,106.05) --
	cycle;
\definecolor[named]{drawColor}{rgb}{0.00,0.00,0.00}

\path[draw=drawColor,line width= 2.3pt,line join=round] ( 43.96,107.60) --
	( 54.66,106.35) --
	( 65.35,108.82) --
	( 76.04,108.21) --
	( 86.74,109.22) --
	( 97.43,107.45) --
	(108.12,109.70) --
	(118.82,105.61) --
	(129.51,106.60);
\definecolor[named]{drawColor}{rgb}{0.53,0.81,0.92}

\path[draw=drawColor,line width= 2.3pt,line join=round] ( 43.96,103.39) --
	( 54.66,103.40) --
	( 65.35,103.61) --
	( 76.04,101.29) --
	( 86.74, 97.74) --
	( 97.43, 98.36) --
	(108.12, 99.18) --
	(118.82, 92.09) --
	(129.51, 91.15);
\definecolor[named]{drawColor}{rgb}{0.00,0.00,1.00}

\path[draw=drawColor,line width= 2.3pt,line join=round] ( 43.96,108.42) --
	( 54.66,107.11) --
	( 65.35,107.97) --
	( 76.04,108.67) --
	( 86.74,108.49) --
	( 97.43,106.59) --
	(108.12,110.42) --
	(118.82,105.51) --
	(129.51,105.43);
\definecolor[named]{drawColor}{rgb}{0.50,0.50,0.50}

\path[draw=drawColor,line width= 0.6pt,line join=round,line cap=round] ( 39.69, 34.03) rectangle (133.79,119.86);
\end{scope}
\begin{scope}
\path[clip] (133.79, 34.03) rectangle (227.89,119.86);
\definecolor[named]{fillColor}{rgb}{1.00,1.00,1.00}

\path[fill=fillColor] (133.79, 34.03) rectangle (227.89,119.86);
\definecolor[named]{fillColor}{rgb}{0.00,0.00,0.00}

\path[fill=fillColor,fill opacity=0.25] (138.07,115.91) --
	(148.76,114.22) --
	(159.45,115.53) --
	(170.15,114.84) --
	(180.84,115.49) --
	(191.53,114.63) --
	(202.23,114.82) --
	(212.92,113.99) --
	(223.61,115.77) --
	(223.61,113.07) --
	(212.92,113.45) --
	(202.23,113.15) --
	(191.53,114.15) --
	(180.84,110.86) --
	(170.15,112.08) --
	(159.45,112.51) --
	(148.76,113.63) --
	(138.07,112.75) --
	cycle;
\definecolor[named]{fillColor}{rgb}{0.53,0.81,0.92}

\path[fill=fillColor,fill opacity=0.25] (138.07,115.90) --
	(148.76,114.39) --
	(159.45,115.17) --
	(170.15,114.45) --
	(180.84,115.96) --
	(191.53,114.47) --
	(202.23,114.83) --
	(212.92,114.00) --
	(223.61,113.87) --
	(223.61,106.65) --
	(212.92,112.34) --
	(202.23,113.21) --
	(191.53,113.27) --
	(180.84,110.74) --
	(170.15,112.66) --
	(159.45,112.93) --
	(148.76,113.50) --
	(138.07,113.35) --
	cycle;
\definecolor[named]{fillColor}{rgb}{0.00,0.00,1.00}

\path[fill=fillColor,fill opacity=0.25] (138.07,115.84) --
	(148.76,114.43) --
	(159.45,115.60) --
	(170.15,114.63) --
	(180.84,115.70) --
	(191.53,114.37) --
	(202.23,114.86) --
	(212.92,113.74) --
	(223.61,115.53) --
	(223.61,110.37) --
	(212.92,112.87) --
	(202.23,113.02) --
	(191.53,113.74) --
	(180.84,111.39) --
	(170.15,112.38) --
	(159.45,112.59) --
	(148.76,113.54) --
	(138.07,112.54) --
	cycle;
\definecolor[named]{drawColor}{rgb}{0.00,0.00,0.00}

\path[draw=drawColor,line width= 2.3pt,line join=round] (138.07,114.33) --
	(148.76,113.93) --
	(159.45,114.02) --
	(170.15,113.46) --
	(180.84,113.17) --
	(191.53,114.39) --
	(202.23,113.98) --
	(212.92,113.72) --
	(223.61,114.42);
\definecolor[named]{drawColor}{rgb}{0.53,0.81,0.92}

\path[draw=drawColor,line width= 2.3pt,line join=round] (138.07,114.62) --
	(148.76,113.95) --
	(159.45,114.05) --
	(170.15,113.55) --
	(180.84,113.35) --
	(191.53,113.87) --
	(202.23,114.02) --
	(212.92,113.17) --
	(223.61,110.26);
\definecolor[named]{drawColor}{rgb}{0.00,0.00,1.00}

\path[draw=drawColor,line width= 2.3pt,line join=round] (138.07,114.19) --
	(148.76,113.98) --
	(159.45,114.09) --
	(170.15,113.51) --
	(180.84,113.54) --
	(191.53,114.05) --
	(202.23,113.94) --
	(212.92,113.31) --
	(223.61,112.95);
\definecolor[named]{drawColor}{rgb}{0.50,0.50,0.50}

\path[draw=drawColor,line width= 0.6pt,line join=round,line cap=round] (133.79, 34.03) rectangle (227.89,119.86);
\end{scope}
\begin{scope}
\path[clip] (227.89, 34.03) rectangle (321.99,119.86);
\definecolor[named]{fillColor}{rgb}{1.00,1.00,1.00}

\path[fill=fillColor] (227.89, 34.03) rectangle (321.99,119.86);
\definecolor[named]{fillColor}{rgb}{0.00,0.00,0.00}

\path[fill=fillColor,fill opacity=0.25] (232.17,112.78) --
	(242.86,111.69) --
	(253.55,111.64) --
	(264.25,111.82) --
	(274.94,112.76) --
	(285.63,114.17) --
	(296.33,111.95) --
	(307.02,112.78) --
	(317.72,110.93) --
	(317.72,107.02) --
	(307.02,107.96) --
	(296.33,110.29) --
	(285.63,108.59) --
	(274.94,112.05) --
	(264.25,109.39) --
	(253.55,108.09) --
	(242.86,107.64) --
	(232.17,110.59) --
	cycle;
\definecolor[named]{fillColor}{rgb}{0.53,0.81,0.92}

\path[fill=fillColor,fill opacity=0.25] (232.17,110.90) --
	(242.86,108.85) --
	(253.55,109.99) --
	(264.25,110.12) --
	(274.94,109.33) --
	(285.63,109.22) --
	(296.33,107.47) --
	(307.02,101.78) --
	(317.72, 99.27) --
	(317.72, 87.67) --
	(307.02, 87.59) --
	(296.33,104.01) --
	(285.63,105.07) --
	(274.94,107.23) --
	(264.25,104.50) --
	(253.55,105.80) --
	(242.86,104.37) --
	(232.17,106.84) --
	cycle;
\definecolor[named]{fillColor}{rgb}{0.00,0.00,1.00}

\path[fill=fillColor,fill opacity=0.25] (232.17,112.07) --
	(242.86,109.31) --
	(253.55,111.66) --
	(264.25,111.82) --
	(274.94,112.95) --
	(285.63,112.70) --
	(296.33,112.28) --
	(307.02,111.96) --
	(317.72,109.30) --
	(317.72,105.66) --
	(307.02,104.93) --
	(296.33,108.28) --
	(285.63,109.45) --
	(274.94,112.42) --
	(264.25,109.37) --
	(253.55,107.63) --
	(242.86,108.09) --
	(232.17,108.29) --
	cycle;
\definecolor[named]{drawColor}{rgb}{0.00,0.00,0.00}

\path[draw=drawColor,line width= 2.3pt,line join=round] (232.17,111.68) --
	(242.86,109.67) --
	(253.55,109.87) --
	(264.25,110.60) --
	(274.94,112.41) --
	(285.63,111.38) --
	(296.33,111.12) --
	(307.02,110.37) --
	(317.72,108.97);
\definecolor[named]{drawColor}{rgb}{0.53,0.81,0.92}

\path[draw=drawColor,line width= 2.3pt,line join=round] (232.17,108.87) --
	(242.86,106.61) --
	(253.55,107.90) --
	(264.25,107.31) --
	(274.94,108.28) --
	(285.63,107.15) --
	(296.33,105.74) --
	(307.02, 94.69) --
	(317.72, 93.47);
\definecolor[named]{drawColor}{rgb}{0.00,0.00,1.00}

\path[draw=drawColor,line width= 2.3pt,line join=round] (232.17,110.18) --
	(242.86,108.70) --
	(253.55,109.65) --
	(264.25,110.59) --
	(274.94,112.69) --
	(285.63,111.08) --
	(296.33,110.28) --
	(307.02,108.44) --
	(317.72,107.48);
\definecolor[named]{drawColor}{rgb}{0.50,0.50,0.50}

\path[draw=drawColor,line width= 0.6pt,line join=round,line cap=round] (227.89, 34.03) rectangle (321.99,119.86);
\end{scope}
\begin{scope}
\path[clip] (321.99, 34.03) rectangle (416.09,119.86);
\definecolor[named]{fillColor}{rgb}{1.00,1.00,1.00}

\path[fill=fillColor] (321.99, 34.03) rectangle (416.09,119.86);
\definecolor[named]{fillColor}{rgb}{0.00,0.00,0.00}

\path[fill=fillColor,fill opacity=0.25] (326.27, 66.23) --
	(336.96, 64.43) --
	(347.66, 67.47) --
	(358.35, 66.46) --
	(369.04, 67.22) --
	(379.74, 64.92) --
	(390.43, 65.64) --
	(401.12, 65.21) --
	(411.82, 59.15) --
	(411.82, 37.94) --
	(401.12, 48.78) --
	(390.43, 49.60) --
	(379.74, 56.28) --
	(369.04, 55.96) --
	(358.35, 56.05) --
	(347.66, 55.72) --
	(336.96, 47.17) --
	(326.27, 44.98) --
	cycle;
\definecolor[named]{fillColor}{rgb}{0.53,0.81,0.92}

\path[fill=fillColor,fill opacity=0.25] (326.27, 72.19) --
	(336.96, 68.69) --
	(347.66, 66.98) --
	(358.35, 67.14) --
	(369.04, 63.73) --
	(379.74, 62.63) --
	(390.43, 66.81) --
	(401.12, 67.12) --
	(411.82, 65.69) --
	(411.82, 45.84) --
	(401.12, 50.63) --
	(390.43, 50.98) --
	(379.74, 54.12) --
	(369.04, 53.21) --
	(358.35, 54.23) --
	(347.66, 54.29) --
	(336.96, 55.29) --
	(326.27, 52.98) --
	cycle;
\definecolor[named]{fillColor}{rgb}{0.00,0.00,1.00}

\path[fill=fillColor,fill opacity=0.25] (326.27, 71.91) --
	(336.96, 69.29) --
	(347.66, 67.24) --
	(358.35, 67.31) --
	(369.04, 63.15) --
	(379.74, 64.47) --
	(390.43, 66.46) --
	(401.12, 69.28) --
	(411.82, 66.58) --
	(411.82, 45.67) --
	(401.12, 47.92) --
	(390.43, 53.67) --
	(379.74, 53.76) --
	(369.04, 53.34) --
	(358.35, 55.23) --
	(347.66, 53.72) --
	(336.96, 55.40) --
	(326.27, 52.24) --
	cycle;
\definecolor[named]{drawColor}{rgb}{0.00,0.00,0.00}

\path[draw=drawColor,line width= 2.3pt,line join=round] (326.27, 55.61) --
	(336.96, 55.80) --
	(347.66, 61.60) --
	(358.35, 61.26) --
	(369.04, 61.59) --
	(379.74, 60.60) --
	(390.43, 57.62) --
	(401.12, 57.00) --
	(411.82, 48.54);
\definecolor[named]{drawColor}{rgb}{0.53,0.81,0.92}

\path[draw=drawColor,line width= 2.3pt,line join=round] (326.27, 62.59) --
	(336.96, 61.99) --
	(347.66, 60.63) --
	(358.35, 60.69) --
	(369.04, 58.47) --
	(379.74, 58.37) --
	(390.43, 58.89) --
	(401.12, 58.88) --
	(411.82, 55.76);
\definecolor[named]{drawColor}{rgb}{0.00,0.00,1.00}

\path[draw=drawColor,line width= 2.3pt,line join=round] (326.27, 62.08) --
	(336.96, 62.34) --
	(347.66, 60.48) --
	(358.35, 61.27) --
	(369.04, 58.24) --
	(379.74, 59.11) --
	(390.43, 60.06) --
	(401.12, 58.60) --
	(411.82, 56.12);
\definecolor[named]{drawColor}{rgb}{0.50,0.50,0.50}

\path[draw=drawColor,line width= 0.6pt,line join=round,line cap=round] (321.99, 34.03) rectangle (416.09,119.86);
\end{scope}
\begin{scope}
\path[clip] (  0.00,  0.00) rectangle (505.89,144.54);
\definecolor[named]{drawColor}{rgb}{0.00,0.00,0.00}

\node[text=drawColor,anchor=base east,inner sep=0pt, outer sep=0pt, scale=  0.96] at ( 32.57, 38.70) {0.6};

\node[text=drawColor,anchor=base east,inner sep=0pt, outer sep=0pt, scale=  0.96] at ( 32.57, 57.56) {0.7};

\node[text=drawColor,anchor=base east,inner sep=0pt, outer sep=0pt, scale=  0.96] at ( 32.57, 76.43) {0.8};

\node[text=drawColor,anchor=base east,inner sep=0pt, outer sep=0pt, scale=  0.96] at ( 32.57, 95.30) {0.9};

\node[text=drawColor,anchor=base east,inner sep=0pt, outer sep=0pt, scale=  0.96] at ( 32.57,114.16) {1.0};
\end{scope}
\begin{scope}
\path[clip] (  0.00,  0.00) rectangle (505.89,144.54);
\definecolor[named]{drawColor}{rgb}{0.00,0.00,0.00}

\path[draw=drawColor,line width= 0.6pt,line join=round] ( 35.42, 42.00) --
	( 39.69, 42.00);

\path[draw=drawColor,line width= 0.6pt,line join=round] ( 35.42, 60.87) --
	( 39.69, 60.87);

\path[draw=drawColor,line width= 0.6pt,line join=round] ( 35.42, 79.74) --
	( 39.69, 79.74);

\path[draw=drawColor,line width= 0.6pt,line join=round] ( 35.42, 98.60) --
	( 39.69, 98.60);

\path[draw=drawColor,line width= 0.6pt,line join=round] ( 35.42,117.47) --
	( 39.69,117.47);
\end{scope}
\begin{scope}
\path[clip] (  0.00,  0.00) rectangle (505.89,144.54);
\definecolor[named]{drawColor}{rgb}{0.00,0.00,0.00}

\path[draw=drawColor,line width= 0.6pt,line join=round] ( 54.66, 29.77) --
	( 54.66, 34.03);

\path[draw=drawColor,line width= 0.6pt,line join=round] ( 76.04, 29.77) --
	( 76.04, 34.03);

\path[draw=drawColor,line width= 0.6pt,line join=round] ( 97.43, 29.77) --
	( 97.43, 34.03);

\path[draw=drawColor,line width= 0.6pt,line join=round] (118.82, 29.77) --
	(118.82, 34.03);
\end{scope}
\begin{scope}
\path[clip] (  0.00,  0.00) rectangle (505.89,144.54);
\definecolor[named]{drawColor}{rgb}{0.00,0.00,0.00}

\node[text=drawColor,anchor=base,inner sep=0pt, outer sep=0pt, scale=  0.96] at ( 54.66, 20.31) {0.2};

\node[text=drawColor,anchor=base,inner sep=0pt, outer sep=0pt, scale=  0.96] at ( 76.04, 20.31) {0.4};

\node[text=drawColor,anchor=base,inner sep=0pt, outer sep=0pt, scale=  0.96] at ( 97.43, 20.31) {0.6};

\node[text=drawColor,anchor=base,inner sep=0pt, outer sep=0pt, scale=  0.96] at (118.82, 20.31) {0.8};
\end{scope}
\begin{scope}
\path[clip] (  0.00,  0.00) rectangle (505.89,144.54);
\definecolor[named]{drawColor}{rgb}{0.00,0.00,0.00}

\path[draw=drawColor,line width= 0.6pt,line join=round] (148.76, 29.77) --
	(148.76, 34.03);

\path[draw=drawColor,line width= 0.6pt,line join=round] (170.15, 29.77) --
	(170.15, 34.03);

\path[draw=drawColor,line width= 0.6pt,line join=round] (191.53, 29.77) --
	(191.53, 34.03);

\path[draw=drawColor,line width= 0.6pt,line join=round] (212.92, 29.77) --
	(212.92, 34.03);
\end{scope}
\begin{scope}
\path[clip] (  0.00,  0.00) rectangle (505.89,144.54);
\definecolor[named]{drawColor}{rgb}{0.00,0.00,0.00}

\node[text=drawColor,anchor=base,inner sep=0pt, outer sep=0pt, scale=  0.96] at (148.76, 20.31) {0.2};

\node[text=drawColor,anchor=base,inner sep=0pt, outer sep=0pt, scale=  0.96] at (170.15, 20.31) {0.4};

\node[text=drawColor,anchor=base,inner sep=0pt, outer sep=0pt, scale=  0.96] at (191.53, 20.31) {0.6};

\node[text=drawColor,anchor=base,inner sep=0pt, outer sep=0pt, scale=  0.96] at (212.92, 20.31) {0.8};
\end{scope}
\begin{scope}
\path[clip] (  0.00,  0.00) rectangle (505.89,144.54);
\definecolor[named]{drawColor}{rgb}{0.00,0.00,0.00}

\path[draw=drawColor,line width= 0.6pt,line join=round] (242.86, 29.77) --
	(242.86, 34.03);

\path[draw=drawColor,line width= 0.6pt,line join=round] (264.25, 29.77) --
	(264.25, 34.03);

\path[draw=drawColor,line width= 0.6pt,line join=round] (285.63, 29.77) --
	(285.63, 34.03);

\path[draw=drawColor,line width= 0.6pt,line join=round] (307.02, 29.77) --
	(307.02, 34.03);
\end{scope}
\begin{scope}
\path[clip] (  0.00,  0.00) rectangle (505.89,144.54);
\definecolor[named]{drawColor}{rgb}{0.00,0.00,0.00}

\node[text=drawColor,anchor=base,inner sep=0pt, outer sep=0pt, scale=  0.96] at (242.86, 20.31) {0.2};

\node[text=drawColor,anchor=base,inner sep=0pt, outer sep=0pt, scale=  0.96] at (264.25, 20.31) {0.4};

\node[text=drawColor,anchor=base,inner sep=0pt, outer sep=0pt, scale=  0.96] at (285.63, 20.31) {0.6};

\node[text=drawColor,anchor=base,inner sep=0pt, outer sep=0pt, scale=  0.96] at (307.02, 20.31) {0.8};
\end{scope}
\begin{scope}
\path[clip] (  0.00,  0.00) rectangle (505.89,144.54);
\definecolor[named]{drawColor}{rgb}{0.00,0.00,0.00}

\path[draw=drawColor,line width= 0.6pt,line join=round] (336.96, 29.77) --
	(336.96, 34.03);

\path[draw=drawColor,line width= 0.6pt,line join=round] (358.35, 29.77) --
	(358.35, 34.03);

\path[draw=drawColor,line width= 0.6pt,line join=round] (379.74, 29.77) --
	(379.74, 34.03);

\path[draw=drawColor,line width= 0.6pt,line join=round] (401.12, 29.77) --
	(401.12, 34.03);
\end{scope}
\begin{scope}
\path[clip] (  0.00,  0.00) rectangle (505.89,144.54);
\definecolor[named]{drawColor}{rgb}{0.00,0.00,0.00}

\node[text=drawColor,anchor=base,inner sep=0pt, outer sep=0pt, scale=  0.96] at (336.96, 20.31) {0.2};

\node[text=drawColor,anchor=base,inner sep=0pt, outer sep=0pt, scale=  0.96] at (358.35, 20.31) {0.4};

\node[text=drawColor,anchor=base,inner sep=0pt, outer sep=0pt, scale=  0.96] at (379.74, 20.31) {0.6};

\node[text=drawColor,anchor=base,inner sep=0pt, outer sep=0pt, scale=  0.96] at (401.12, 20.31) {0.8};
\end{scope}
\begin{scope}
\path[clip] (  0.00,  0.00) rectangle (505.89,144.54);
\definecolor[named]{drawColor}{rgb}{0.00,0.00,0.00}

\node[text=drawColor,anchor=base,inner sep=0pt, outer sep=0pt, scale=  1.20] at (227.89,  9.03) {$\rho =$ proportion of equality $y_i=0$ pairs};
\end{scope}
\begin{scope}
\path[clip] (  0.00,  0.00) rectangle (505.89,144.54);
\definecolor[named]{drawColor}{rgb}{0.00,0.00,0.00}

\node[text=drawColor,rotate= 90.00,anchor=base,inner sep=0pt, outer sep=0pt, scale=  1.20] at ( 17.30, 76.95) {Area under ROC curve};
\end{scope}
\begin{scope}
\path[clip] (  0.00,  0.00) rectangle (505.89,144.54);
\definecolor[named]{fillColor}{rgb}{1.00,1.00,1.00}

\path[fill=fillColor] (424.96, 45.88) rectangle (484.98,108.02);
\end{scope}
\begin{scope}
\path[clip] (  0.00,  0.00) rectangle (505.89,144.54);
\definecolor[named]{drawColor}{rgb}{0.00,0.00,0.00}

\node[text=drawColor,anchor=base west,inner sep=0pt, outer sep=0pt, scale=  0.96] at (429.23, 97.12) {\bfseries function};
\end{scope}
\begin{scope}
\path[clip] (  0.00,  0.00) rectangle (505.89,144.54);
\definecolor[named]{drawColor}{rgb}{0.80,0.80,0.80}
\definecolor[named]{fillColor}{rgb}{1.00,1.00,1.00}

\path[draw=drawColor,line width= 0.6pt,line join=round,line cap=round,fill=fillColor] (429.23, 79.06) rectangle (443.68, 93.51);
\end{scope}
\begin{scope}
\path[clip] (  0.00,  0.00) rectangle (505.89,144.54);
\definecolor[named]{fillColor}{rgb}{0.00,0.00,0.00}

\path[fill=fillColor,fill opacity=0.25] (429.23, 79.06) rectangle (443.68, 93.51);

\path[] (429.23, 79.06) --
	(443.68, 93.51);
\end{scope}
\begin{scope}
\path[clip] (  0.00,  0.00) rectangle (505.89,144.54);
\definecolor[named]{drawColor}{rgb}{0.00,0.00,0.00}

\path[draw=drawColor,line width= 2.3pt,line join=round] (430.68, 86.28) -- (442.24, 86.28);
\end{scope}
\begin{scope}
\path[clip] (  0.00,  0.00) rectangle (505.89,144.54);
\definecolor[named]{drawColor}{rgb}{0.80,0.80,0.80}
\definecolor[named]{fillColor}{rgb}{1.00,1.00,1.00}

\path[draw=drawColor,line width= 0.6pt,line join=round,line cap=round,fill=fillColor] (429.23, 64.60) rectangle (443.68, 79.06);
\end{scope}
\begin{scope}
\path[clip] (  0.00,  0.00) rectangle (505.89,144.54);
\definecolor[named]{fillColor}{rgb}{0.00,0.00,1.00}

\path[fill=fillColor,fill opacity=0.25] (429.23, 64.60) rectangle (443.68, 79.06);

\path[] (429.23, 64.60) --
	(443.68, 79.06);
\end{scope}
\begin{scope}
\path[clip] (  0.00,  0.00) rectangle (505.89,144.54);
\definecolor[named]{drawColor}{rgb}{0.00,0.00,1.00}

\path[draw=drawColor,line width= 2.3pt,line join=round] (430.68, 71.83) -- (442.24, 71.83);
\end{scope}
\begin{scope}
\path[clip] (  0.00,  0.00) rectangle (505.89,144.54);
\definecolor[named]{drawColor}{rgb}{0.80,0.80,0.80}
\definecolor[named]{fillColor}{rgb}{1.00,1.00,1.00}

\path[draw=drawColor,line width= 0.6pt,line join=round,line cap=round,fill=fillColor] (429.23, 50.15) rectangle (443.68, 64.60);
\end{scope}
\begin{scope}
\path[clip] (  0.00,  0.00) rectangle (505.89,144.54);
\definecolor[named]{fillColor}{rgb}{0.53,0.81,0.92}

\path[fill=fillColor,fill opacity=0.25] (429.23, 50.15) rectangle (443.68, 64.60);

\path[] (429.23, 50.15) --
	(443.68, 64.60);
\end{scope}
\begin{scope}
\path[clip] (  0.00,  0.00) rectangle (505.89,144.54);
\definecolor[named]{drawColor}{rgb}{0.53,0.81,0.92}

\path[draw=drawColor,line width= 2.3pt,line join=round] (430.68, 57.37) -- (442.24, 57.37);
\end{scope}
\begin{scope}
\path[clip] (  0.00,  0.00) rectangle (505.89,144.54);
\definecolor[named]{drawColor}{rgb}{0.00,0.00,0.00}

\node[text=drawColor,anchor=base west,inner sep=0pt, outer sep=0pt, scale=  0.96] at (445.49, 82.98) {compare};
\end{scope}
\begin{scope}
\path[clip] (  0.00,  0.00) rectangle (505.89,144.54);
\definecolor[named]{drawColor}{rgb}{0.00,0.00,0.00}

\node[text=drawColor,anchor=base west,inner sep=0pt, outer sep=0pt, scale=  0.96] at (445.49, 68.52) {rank2};
\end{scope}
\begin{scope}
\path[clip] (  0.00,  0.00) rectangle (505.89,144.54);
\definecolor[named]{drawColor}{rgb}{0.00,0.00,0.00}

\node[text=drawColor,anchor=base west,inner sep=0pt, outer sep=0pt, scale=  0.96] at (445.49, 54.07) {rank};
\end{scope}
\end{tikzpicture}

  \vskip -1cm
  \caption{Area under the ROC curve (AUC) for 3 different simulated
    patterns $r(\mathbf x)$ where $\mathbf x\in\RR^2$ and one real
    sushi data set where $\mathbf x\in\RR^{14}$. For each data set we
    picked $n=400$ pairs, varying the proportion $\rho$ of equality
    pairs. We plot mean and standard deviation of AUC over 4 test
    sets.}
  \label{fig:auc}
\end{figure*}

In Figure~\ref{fig:auc} we fixed the number of training pairs $n=400$
and varied the proportion $\rho$ of equality pairs for the three
simulated squared norm ranking functions $r$. We use area under the
ROC curve to evaluate the learned models, and all methods perform
close to the optimal true ranking function when $r(\mathbf
x)=||\mathbf x||^2_2$. For the other patterns, it is clear that all
the methods perform similarly when there are mostly inequality pairs
($\rho=0.1$), since SVMrank was designed for this type of training
data. In contrast, when there are mostly equality pairs ($\rho=0.9$),
the compare and rank2 methods clearly outperform the rank method,
which ignores the equality pairs. Although there is no difference
between the rank2 and compare methods when $r(\mathbf x)=||\mathbf
x||_1^2$, the compare method is slightly better when $r(\mathbf
x)=||\mathbf x||_\infty^2$. Overall, it is clear that when the data
contain equality pairs, it is advantageous to use a model such as the
proposed SVMcompare method which learns from them directly as a part
of the optimization problem.

\subsection{Learning to rank sushi data}

We downloaded the sushi data set of \citet{object-ranking-methods}
from \url{http://www.kamishima.net/sushi/}. We used the
\texttt{sushi3b.5000.10.score} data, which consist of 100 different
sushis rated by 5000 different people. Each person rated 10 sushis on
a 5 point scale, which we convert to 5 preference pairs, for a total
of 17,832 equality $y_i=0$ and 7,168 inequality $y_i\in\{-1,1\}$
pairs. For each pair $i$ we have features $\mathbf x_i,\mathbf
x_i'\in\RR^{14}$ consisting of 7 features of the sushi and 7 features
of the person. Sushi features are style, major, minor, oily, eating
frequency, price, and selling frequency. Person features are gender, age,
time, birthplace and current home (we converted Japanese prefecture
codes to latitude/longitude coordinates).

As in the simulations of Section~\ref{sec:simulations}, we picked
train, validation, and test sets, each with the same number of pairs
$n$ and the same proportion $\rho$ of equality pairs. We fit a
$10\times 7$ grid of models to the training set (cost parameter
$C=10^{-1},\dots,10^4$, Gaussian kernel width $10^{-5},\dots,10^1$),
select the model with minimal zero-one loss on the validation set, and
then use the test set to estimate the generalization ability of the
selected model.

In Figure~\ref{fig:simulation-samples} we fixed the proportion of
equality pairs $\rho=1/2$, varied the number of training pairs
$n\in\{50,\dots, 800\}$, and calculated test error. The relative
performance of the algorithms is the same as in the simulations: rank
has the highest test error, rank2 does better, and the proposed
SVMcompare algorithm has the lowest test error.

In Figure~\ref{fig:auc} we fixed the number of training pairs $n=400$,
varied the proportion $\rho$ of equality pairs, and calculated test
AUC. The methods perform about the same for extreme values of
$\rho\in\{0.1,0.9\}$, but the proposed SVMcompare algorithm has a
higher AUC when the classes are balanced $\rho=1/2$.



\section{Conclusions and future work}
\label{sec:conclusions}

We discussed the learning to compare problem, which has not yet been
extensively studied in the machine learning literature. In
Section~\ref{sec:lp-qp}, we proposed two different formulations for
max-margin comparison, and proved their relationship in
Lemma~\ref{lemma:feasible}. It justifies our proposed SVMcompare
algorithm, which uses a binary SVM dual QP solver to learn a nonlinear
comparison function.

Our results on simulated data clearly showed the importance of
directly modeling the equality pairs, when they are present. We showed
that when there are few equality pairs, as is the usual setup in
learning to rank problems, the baseline SVMrank algorithm performs as
well as our proposed SVMcompare algorithm. However, when there are
many equality pairs, it is clearly advantageous to use a model such
as SVMcompare which directly learns from the equality pairs.

For future work, it will be interesting to see if the same results are
observed in learning to rank data from search engines. For scaling to
these very large data sets, we would like to try algorithms based on
smooth discriminative loss functions, such as stochastic gradient
descent with a logistic loss.

\textbf{Acknowledgements}: 
%TDH was funded by KAKENHI 23120004, SS by a
%MEXT scholarship, and MS by KAKENHI 26700022. Thanks to Simon
%Lacoste-Julien and Hang Li for helpful discussions.



\bibliographystyle{abbrvnat}
\bibliography{refs}



\end{document}

