\documentclass{article}
\usepackage{listings}
\usepackage[table]{xcolor}
\usepackage{array}
\usepackage{slashbox}
\usepackage{multirow}
\usepackage{fullpage}
\usepackage{tikz}
\usepackage{stmaryrd}
\usepackage{graphicx}
\usepackage{amsmath,amssymb,amsthm}
\usepackage{stfloats}
\usepackage{float}

\renewcommand{\thefootnote}{\fnsymbol{footnote}}

\newtheorem{proposition}{Proposition}

\newtheorem{definition}{Definition}
\newtheorem{theo}{Theorem}    % numérotés par section
\newtheorem{lem}{Lemma}

\newcommand{\RR}{\mathbb R}
\newcommand{\NN}{\mathbb N}
\newcommand{\pkg}[1]{\texttt{#1}}
\newcommand{\plausibleK}{\textit{plausibleK}}

\DeclareMathOperator*{\argmin}{arg\,min}
\DeclareMathOperator*{\argmax}{arg\,max}
\DeclareMathOperator*{\maximize}{maximize}
\DeclareMathOperator*{\minimize}{minimize}


\newfloat{Algorithm}{thp}{lop}
\floatname{Algorithm}{Algorithm}

% For citations
\usepackage{natbib}

% For algorithms
\usepackage{algorithm}
\usepackage{algorithmic}
\usepackage{hyperref}
\newcommand{\theHalgorithm}{\arabic{algorithm}}

\usepackage{icml2013} 

\icmltitlerunning{Support vector comparison machines}

\begin{document}

\renewcommand{\arraystretch}{1.5}

\definecolor{lightgray}{rgb}{0.9,0.9,0.9}
\definecolor{pastelblue}{RGB}{213,229,255}
\newcolumntype{a}{>{\columncolor{lightgray}}c}

\twocolumn[
\icmltitle{Support vector comparison machines}

% It is OKAY to include author information, even for blind
% submissions: the style file will automatically remove it for you
% unless you've provided the [accepted] option to the icml2012
% package.


%\icmlauthor{Guillem Rigaill\footnotemark[1]}{rigaill@evry.inra.fr}
%\icmladdress{Unit\'e de Recherche en G\'enomique V\'eg\'etale INRA-CNRS-Universit\'e d'Evry Val d'Essonne, Evry, France}
%\icmlauthor{Toby Dylan Hocking\footnotemark[1], 
% Francis Bach}{toby.hocking@inria.fr, francis.bach@inria.fr}
\icmlauthor{Toby Dylan Hocking}{toby@sg.cs.titech.ac.jp}
\icmlauthor{Supaporn Spanurattana}{supaporn@sg.cs.titech.ac.jp}
\icmlauthor{Masashi Sugiyama}{sugi@cs.titech.ac.jp}
\icmladdress{Department of Computer Science, Tokyo Institute of
  Technology, Tokyo 152-8552, Japan}

% You may provide any keywords that you 
% find helpful for describing your paper; these are used to populate 
% the "keywords" metadata in the PDF but will not be shown in the document
\icmlkeywords{support vector machine, ranking, comparing, convex,
  optimization, relaxation, libsvm}

\vskip 0.3in
]

\begin{abstract}
  In ranking problems, the goal is to learn a ranking function
  $f(x)\in\RR$ from labeled pairs $x,x'$ of input points. In this
  paper, we consider the related comparison problem, where the label
  $y\in\{-1,0,1\}$ indicates which element of the pair is better, or
  if there is no significant difference. We cast the learning problem
  as a margin maximization, and show that it can be solved by
  converting it to a standard SVM. We compare our algorithm to SVMrank
  using a benchmark ranking data set.
\end{abstract}

\section{Introduction}

In this paper we consider the supervised comparison problem. Assume
that we have $n$ labeled training pairs and for each pair
$i\in\{1,\dots,n\}$ we have input features $x_i,x_i'\in\RR^p$ and a
label $y_i\in\{-1,0,1\}$ that indicates which element is better:
\begin{equation}
  \label{eq:z}
  y_i =
  \begin{cases}
    -1 & \text{ if $x_i$ is better than $x'_i$}\\
    0 & \text{ if $x_i$ is as good as $x'_i$}\\
    1 & \text{ if $x'_i$ is better than $x_i$}.
  \end{cases}
\end{equation}
The goal of learning is to find a comparison function $c:\RR^p \times
\RR^p \rightarrow \{-1,0,1\}$ which generalizes to a test set of data:
\begin{equation}
  \minimize_{c} 
  \sum_{j=1}^n
  e\left[ c(x_i, x_i'), y_i \right].
\end{equation}
For evaluation, we propose the zero-one loss function
$e:\{-1,0,1\}\times\{-1,0,1\}\rightarrow\{0,1\}$, described in
Table~\ref{tab:evaluation}.

This is similar to a ranking problem, for which a number of machine
learning algorithms exist \citep{learning-to-rank}, such as RankSVM
\citep{ranksvm}. There are two key differences between ranking and
comparing, which is what we want to do:
\begin{itemize}
\item Ranking algorithms ignore the $y_i=0$ equality constraints.
\item The goal of ranking is to give an absolute order to a set of
  items, typically documents in a search engine. Comparison is simpler
  since we only need to make a decision about exactly two items.
\end{itemize}

\section{Related work}
\label{sec:related}

First we discuss connections with several existing methods in the
machine learning literature, and then we discuss how ranking
algorithms can be applied to the comparison problem.

\begin{table}[b!]
  \centering
  \begin{tabular}{|a|c|c|}\hline
    \rowcolor{lightgray}
    \backslashbox{Outputs}{Inputs}
    &single items $x$&pairs $x,x'$\\ \hline
    $y\in\{-1,1\}$ &SVM  & SVMrank   	\\ \hline 
    $y\in\{-1,0,1\}$ &Reject option& this work\\ \hline
  \end{tabular}
  \caption{\label{tab:related} Comparison is similar to ranking 
    and classification with reject option.}
\end{table}

\subsection{Rank, reject, and rate}

Some related work appears in Table~\ref{tab:related}:

\begin{itemize}
\item \citet{reject-option} studied the statistical properties of the
  hinge loss for the classification with reject option.
\item \citet{ranksvm} proposed SVM for ranking.
\item \citet{rank-with-ties} proposed a boosting algorithm for ranking
  with ties, and observed that ties are more effective when there are
  more output values.
\item \citet{trueskill} proposed TrueSkill: a Bayesian skill rating
  system, a generalization of the Elo chess rating system.
\end{itemize}

The rest of this article is organized as follows. In
Section~\ref{sec:related} we explain how existing algorithms for
ranking can be applied to the comparison problem, then in
Section~\ref{sec:svm-compare} we propose a new algorithm:
SVMcompare. We show a comparison with SVMrank on a benchmark dataset
in Section~\ref{sec:results} and discuss future work in
Section~\ref{sec:conclusions}.

\subsection{SVMrank for comparing}

In this section we explain how to apply the existing SVMrank algorithm
to a comparison data set.

TODO


\begin{table}[b!]
  \centering
  \begin{tabular}{|a|c|c|c|}\hline
    \rowcolor{lightgray}
    \backslashbox{$\hat{y}$}{ $y$}
    &\textbf{-1}&\textbf{0}&\textbf{1}\\ \hline
    \textbf{-1}&0  & FP & Inversion   	\\ \hline 
    \textbf{0} &FN& 0& FP\\ \hline
    \textbf{1} & Inversion & FP &0	\\ \hline
  \end{tabular}
  % \cellcolor{pastelblue}
  \caption{Evaluation for comparison problems.}
  \label{tab:evaluation}
\end{table}

\subsection{Applying SVMrank}

First we learn a ranking function $r:\RR^p \rightarrow \RR$ and then
threshold it using 

\section{Support vector comparison machines}
\label{sec:svm-compare}

Given a kernel function $\kappa:\RR^p\times \RR^p\rightarrow\RR$,
define the $K\in\RR^{2n\times 2n}$ kernel matrix based on the $n$
pairs of training inputs $x_i,x_i'$.
% \begin{equation}
%   \label{eq:kernel}
%   K=  \left[
%     \begin{array}{cccccc}
%     \kappa(x_1, x_1) & \cdots & \kappa(x_n, x_1) &
%     \kappa(x_1', x_1) & \cdots & \kappa(x_n', x_1) \\
%     \vdots & & \vdots & \vdots && \vdots \\
%     \kappa(x_1, x_n) & \cdots & \kappa(x_n, x_n) &
%     \kappa(x_1', x_n) & \cdots & \kappa(x_n', x_n) \\
%     \kappa(x_1, x_1') & \cdots & \kappa(x_n, x_1') &
%     \kappa(x_1', x_1') & \cdots & \kappa(x_n', x_1') \\
%     \vdots & & \vdots & \vdots && \vdots 
%     \end{array}
%     \right].
% \end{equation}

%The three rows shown are $K_1,K_n,K_1'\in\RR^{2n}$. 

The primal problem for some $C\in\RR^+$ is
\begin{equation}
  \begin{aligned}
      \minimize_{a\in\RR^{2n},\xi\in\RR^n,\beta\in\RR}\ \ & 
      \frac 1 2 a^\intercal K a + C\sum_{i=1}^n \xi_i \\
      \text{subject to}\ \ & 
      \text{for all $i\in\{1,\dots,n\}$, }
      \xi_i \geq 0,\\
      &\text{and }
      \xi_i \geq 1-y_i(\beta + a^\intercal (K_i'-K_i)).
  \end{aligned}
\end{equation}
TODO: write the Lagrangian and dual feasibility conditions.

If $v\in\RR^n$ are the dual variables corresponding to the second
constraint, then the dual problem is
\begin{equation}
  \begin{aligned}
    \minimize_{v\in\RR^n}\ \ &
    \frac 1 2 v^\intercal Y M^\intercal K M Y v - v^\intercal 1\\
    \text{subject to}\ \ &
    \text{for all $i\in\{1,\dots,n\}$, } 0\leq v_i\leq C\\
    & \sum_{i=1}^n v_i y_i = 0,
  \end{aligned}
\end{equation}
where $M=[-I_n \, I_n]^\intercal\in\{-1,0,1\}^{2n\times n}$. This quadratic problem
is equivalent to a standard SVM with kernel $M^\intercal K M$.

So we can use any efficient SVM solver, such as libsvm
\citep{libsvm}.

After solving learned ranking function is $r(x) = \sum_{i\in\text{sv}}
a_i \kappa(x, x_i)/\beta$.

\section{Results}
\label{sec:results}

\section{Conclusions and future work}
\label{sec:conclusions}

\bibliographystyle{abbrvnat}
\bibliography{refs}

\end{document}

