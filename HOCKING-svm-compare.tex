\documentclass{article}
\usepackage{aistats2014} 
\usepackage{listings}
\usepackage{clrscode}
\usepackage[table]{xcolor}
\usepackage{array}
\usepackage{slashbox}
\usepackage{multirow}
%\usepackage{fullpage}
\usepackage{tikz}
\usepackage{stmaryrd}
\usepackage{graphicx}
\usepackage{amsmath,amssymb,amsthm}
\usepackage{stfloats}
\usepackage{float}
\newcommand{\SVMcompare}{SVMcompare}

\renewcommand{\thefootnote}{\fnsymbol{footnote}}

\newtheorem{proposition}{Proposition}

\newtheorem{definition}{Definition}
\newtheorem{theo}{Theorem}    % numérotés par section
\newtheorem{lemma}{Lemma}

\newcommand{\RR}{\mathbb R}
\newcommand{\NN}{\mathbb N}
\newcommand{\pkg}[1]{\texttt{#1}}
\newcommand{\plausibleK}{\textit{plausibleK}}

\DeclareMathOperator*{\argmin}{arg\,min}
\DeclareMathOperator*{\Diag}{Diag}
\DeclareMathOperator*{\argmax}{arg\,max}
\DeclareMathOperator*{\maximize}{maximize}
\DeclareMathOperator*{\minimize}{minimize}


\newfloat{Algorithm}{thp}{lop}
\floatname{Algorithm}{Algorithm}

% For citations
\usepackage{natbib}

% For algorithms
\usepackage{algorithm}
\usepackage{algorithmic}
\usepackage{hyperref}
\newcommand{\theHalgorithm}{\arabic{algorithm}}


%\icmltitlerunning{Support vector comparison machines}

\begin{document}

\renewcommand{\arraystretch}{1.5}

\definecolor{lightgray}{rgb}{0.9,0.9,0.9}
\definecolor{pastelblue}{RGB}{213,229,255}
\newcolumntype{a}{>{\columncolor{lightgray}}c}

\twocolumn[
\aistatstitle{Support vector comparison machines}

% It is OKAY to include author information, even for blind
% submissions: the style file will automatically remove it for you
% unless you've provided the [accepted] option to the icml2012
% package.


%\icmlauthor{Guillem Rigaill\footnotemark[1]}{rigaill@evry.inra.fr}
%\icmladdress{Unit\'e de Recherche en G\'enomique V\'eg\'etale INRA-CNRS-Universit\'e d'Evry Val d'Essonne, Evry, France}
%\icmlauthor{Toby Dylan Hocking\footnotemark[1], 
% Francis Bach}{toby.hocking@inria.fr, francis.bach@inria.fr}
\aistatsauthor{Toby Dylan Hocking \And Supaporn Spanurattana \And Masashi Sugiyama}
\aistatsaddress{Department of Computer Science, Tokyo Institute of
  Technology, Tokyo 152-8552, Japan}

% You may provide any keywords that you 
% find helpful for describing your paper; these are used to populate 
% the "keywords" metadata in the PDF but will not be shown in the document

\vskip 0.3in
]

\begin{abstract}
  In ranking problems, the goal is to learn a ranking function
  $r(x)\in\RR$ from labeled pairs $x,x'$ of input points. In this
  paper, we consider the related comparison problem, where the label
  $y\in\{-1,0,1\}$ indicates which element of the pair is better, or
  if there is no significant difference. We cast the learning problem
  as a margin maximization, and show that it can be solved by
  converting it to a standard SVM. We compare our algorithm to SVMrank
  using simulations and benchmark ranking data.
\end{abstract}

\section{Introduction}

In this paper we consider the supervised comparison problem. Assume
that we have $n$ labeled training pairs and for each pair
$i\in\{1,\dots,n\}$ we have input features $x_i,x_i'\in\RR^p$ and a
label $y_i\in\{-1,0,1\}$ that indicates which element is better:
\begin{equation}
  \label{eq:z}
  y_i =
  \begin{cases}
    -1 & \text{ if $x_i$ is better than $x'_i$},\\
    0 & \text{ if $x_i$ is as good as $x'_i$},\\
    1 & \text{ if $x'_i$ is better than $x_i$}.
  \end{cases}
\end{equation}
These data are geometrically represented by segments and arrows in the
top panel of Figure~\ref{fig:norm-data}. 

Comparison data naturally
arise when considering subjective human evaluations of pairs of
items. For example, if I were to compare some
pairs of movies I have watched, I would say \textit{Les Mis\'erables}
is better than \textit{Star Wars}, and \textit{The Empire Strikes
  Back} is as good as \textit{Star Wars}. Features $x_i,x_i'$ of the
movies can be length in minutes, year of theatrical release,
indicators for genre, actors/actresses, directors, etc.

The goal of learning is to find a comparison function $c:\RR^p \times
\RR^p \rightarrow \{-1,0,1\}$ which generalizes to a test set of data:
\begin{equation}
  \label{eq:min_c}
  \minimize_{c} 
  \sum_{i\in\text{test}}
  e\left[ c(x_i, x_i'), y_i \right],
\end{equation}
where $e(\hat y, y) = I(\hat y \neq y)$ is the zero-one loss.

% This is similar to a ranking problem, for which a number of machine
% learning algorithms exist \citep{learning-to-rank}, such as SVMrank
% \citep{ranksvm}. There are two key differences between ranking and
% comparing:
% \begin{itemize}
% \item Ranking algorithms ignore the $y_i=0$ equality pairs.
% \item The goal of ranking is to give an absolute order to a set of
%   items, typically documents in a search engine. Comparison is simpler
%   since we only need to make a decision about exactly two items.
% \end{itemize}

The rest of this article is organized as follows. In
Section~\ref{sec:related} we discuss links with related work on
classification and ranking, then in Section~\ref{sec:svm-compare} we
propose a new algorithm: \SVMcompare. We show results on a benchmark
dataset in Section~\ref{sec:results} and discuss future work in
Section~\ref{sec:conclusions}.

\begin{table}[b!]
  \centering
  \begin{tabular}{|a|c|c|}\hline
    \rowcolor{lightgray}
    \backslashbox{Outputs}{Inputs}
    &single items $x$&pairs $x,x'$\\ \hline
    $y\in\{-1,1\}$ &SVM  & SVMrank   	\\ \hline 
    $y\in\{-1,0,1\}$ &Reject option& this work\\ \hline
  \end{tabular}
  \caption{\label{tab:related} Comparison is similar to ranking 
    and classification with reject option.}
\end{table}

\section{Related work}
\label{sec:related}

First we discuss connections with several existing methods in the
machine learning literature, and then we discuss how ranking
algorithms can be applied to the comparison problem.

\subsection{Rank, reject, and rate}

Comparison is similar to ranking and classification with a reject
option (Table~\ref{tab:related}). 
\citet{reject-option} studied the
statistical properties of a hinge loss for an extension of SVM where
$y\in\{-1,0,1\}$, and 0 signifies ``rejection'' or ``no guess,''
calling this the ``classification with reject option'' problem.  There
are many algorithms for the supervised learning to rank problem
\citep{learning-to-rank}, which is similar to the supervised
comparison problem we consider in this paper. The main idea of
learning to rank is to train on labeled pairs of possible documents
$x_i,x_i'$ for the same search query. The labels are $y_i\in\{-1,1\}$,
where $y_i=1$ means document $x_i'$ is more relevant than $x_i$ and
$y_i=-1$ means the opposite. \citet{ranksvm} proposed the SVMrank
algorithm for this problem, and the algorithm we propose here is
similar. The difference is that we also consider the case where both
items/documents are judged to be equally good ($y_i=0$).
\citet{rank-with-ties} called this the ``ranking with ties'' problem,
proposed a boosting algorithm, and observed that modeling ties is more
effective when there are more output values. \citet{trueskill}
proposed TrueSkill, a generalization of the Elo chess rating system
which can be applied to comparison data. 
% Neither TrueSkill nor SVMrank
% can predict ties ($y_i=0$), and \citet{rank-with-ties} did not respond
% to a request for their code, which is what motivates us to propose a
% new algorithm.

When the inputs are discrete $x_i,x_i'\in\{1,\dots,k\}$, then the
problem is known as learning a relation \citep{relations}. In this
article we consider the case when inputs are continuous
$x_i,x_i'\in\RR^p$.


% Some related work appears in Table~\ref{tab:related}:

% \begin{itemize}
% \item \citet{reject-option} studied the statistical properties of the
%   hinge loss for the classification with reject option.
% \item \citet{ranksvm} proposed SVM for ranking.
% \item \citet{rank-with-ties} proposed a boosting algorithm for ranking
%   with ties, and observed that ties are more effective when there are
%   more output values.
% \item \citet{trueskill} proposed TrueSkill: a Bayesian skill rating
%   system, a generalization of the Elo chess rating system.
% \end{itemize}

\begin{figure}
  \centering
  % Created by tikzDevice version 0.10.1 on 2017-12-20 13:08:39
% !TEX encoding = UTF-8 Unicode
\begin{tikzpicture}[x=1pt,y=1pt]
\definecolor{fillColor}{RGB}{255,255,255}
\path[use as bounding box,fill=fillColor,fill opacity=0.00] (0,0) rectangle (433.62,180.67);
\begin{scope}
\path[clip] (  0.00,  0.00) rectangle (433.62,180.67);
\definecolor{drawColor}{RGB}{0,0,0}

\path[draw=drawColor,line width= 0.4pt,line join=round,line cap=round] (  0.00,  0.00) --
	(144.54,  0.00) --
	(144.54,164.83) --
	(  0.00,164.83) --
	(  0.00,  0.00);
\end{scope}
\begin{scope}
\path[clip] (  0.00,  0.00) rectangle (144.54,164.83);
\definecolor{drawColor}{RGB}{190,190,190}

\path[draw=drawColor,line width= 0.4pt,line join=round,line cap=round] (  0.00, 79.33) -- (144.54, 79.33);

\path[draw=drawColor,line width= 0.4pt,line join=round,line cap=round] ( 72.09,  0.00) -- ( 72.09,164.84);
\end{scope}
\begin{scope}
\path[clip] (  0.00,  0.00) rectangle (433.62,180.67);
\definecolor{drawColor}{RGB}{0,0,0}

\node[text=drawColor,anchor=base,inner sep=0pt, outer sep=0pt, scale=  1.00] at ( 72.27,167.47) {Original features $\mathbf x\in\mathbb R^2$};
\end{scope}
\begin{scope}
\path[clip] (  0.00,  0.00) rectangle (144.54,164.83);
\definecolor{drawColor}{gray}{0.50}

\path[draw=drawColor,line width= 0.4pt,line join=round,line cap=round] ( 46.90, 63.81) --
	( 43.63, 71.13) --
	( 42.37, 78.44) --
	( 43.14, 85.76) --
	( 45.93, 93.08) --
	( 49.26, 98.15) --
	( 51.30,100.40) --
	( 56.57,104.49) --
	( 63.30,107.71) --
	( 63.89,107.93) --
	( 71.21,108.97) --
	( 78.53,108.34) --
	( 80.50,107.71) --
	( 85.84,105.53) --
	( 93.14,100.40) --
	( 93.16,100.38) --
	( 98.30, 93.08) --
	(100.48, 87.74) --
	(101.10, 85.76) --
	(101.73, 78.44) --
	(100.70, 71.13) --
	(100.48, 70.53) --
	( 97.25, 63.81) --
	( 93.16, 58.53) --
	( 90.91, 56.49) --
	( 85.84, 53.16) --
	( 78.53, 50.37) --
	( 71.21, 49.61) --
	( 63.89, 50.86) --
	( 56.57, 54.14) --
	( 53.32, 56.49) --
	( 49.26, 60.55);

\path[draw=drawColor,line width= 0.4pt,line join=round,line cap=round] ( 49.26, 60.55) -- ( 48.82, 61.16);

\node[text=drawColor,rotate=-54.13,anchor=base west,inner sep=0pt, outer sep=0pt, scale=  0.60] at ( 45.07, 62.48) { 1 };

\path[draw=drawColor,line width= 0.4pt,line join=round,line cap=round] ( 27.31, 53.96) --
	( 25.92, 56.49);

\path[draw=drawColor,line width= 0.4pt,line join=round,line cap=round] ( 23.03, 63.81) --
	( 21.24, 71.13) --
	( 20.55, 78.44) --
	( 20.97, 85.76) --
	( 22.49, 93.08) --
	( 25.12,100.40) --
	( 27.31,104.67) --
	( 29.13,107.71) --
	( 34.62,114.76) --
	( 34.88,115.03) --
	( 41.94,121.09) --
	( 43.80,122.35) --
	( 49.26,125.44) --
	( 56.57,128.45) --
	( 61.36,129.67) --
	( 63.89,130.22) --
	( 71.21,130.84) --
	( 78.53,130.46) --
	( 82.78,129.67) --
	( 85.84,129.00) --
	( 93.16,126.27) --
	(100.48,122.40) --
	(100.55,122.35) --
	(107.80,116.45) --
	(109.22,115.03) --
	(115.11,107.79) --
	(115.16,107.71) --
	(119.04,100.40) --
	(121.77, 93.08) --
	(122.43, 90.02) --
	(123.23, 85.76) --
	(123.60, 78.44) --
	(122.99, 71.13) --
	(122.43, 68.59) --
	(121.21, 63.81) --
	(118.21, 56.49) --
	(115.11, 51.04) --
	(113.86, 49.18) --
	(107.80, 42.12) --
	(107.52, 41.86) --
	(100.48, 36.37) --
	( 97.44, 34.54) --
	( 93.16, 32.36) --
	( 85.84, 29.73) --
	( 78.53, 28.20) --
	( 71.21, 27.78) --
	( 63.89, 28.47) --
	( 56.57, 30.26) --
	( 49.26, 33.16) --
	( 46.73, 34.54) --
	( 41.94, 37.63) --
	( 36.79, 41.86) --
	( 34.62, 44.03) --
	( 30.39, 49.18) --
	( 27.31, 53.96);

\path[draw=drawColor,line width= 0.4pt,line join=round,line cap=round] ( 25.92, 56.49) -- ( 24.23, 60.77);

\node[text=drawColor,rotate=-68.44,anchor=base west,inner sep=0pt, outer sep=0pt, scale=  0.60] at ( 20.92, 62.98) { 3 };

\path[draw=drawColor,line width= 0.4pt,line join=round,line cap=round] ( 12.67, 49.26) --
	(  9.63, 56.49) --
	(  7.41, 63.81) --
	(  6.03, 71.13) --
	(  5.51, 78.44) --
	(  5.83, 85.76) --
	(  7.00, 93.08) --
	(  9.02,100.40) --
	( 11.89,107.71) --
	( 12.67,109.25);

\path[draw=drawColor,line width= 0.4pt,line join=round,line cap=round] ( 15.99,115.03) --
	( 19.99,120.69) --
	( 21.33,122.35) --
	( 27.31,128.55) --
	( 28.57,129.67) --
	( 34.62,134.27) --
	( 38.95,136.98) --
	( 41.94,138.63) --
	( 49.26,141.79) --
	( 56.57,144.08) --
	( 57.71,144.30) --
	( 63.89,145.37) --
	( 71.21,145.85) --
	( 78.53,145.56) --
	( 85.84,144.48) --
	( 86.56,144.30) --
	( 93.16,142.43) --
	(100.48,139.47) --
	(105.24,136.98) --
	(107.80,135.47) --
	(115.11,130.14) --
	(115.66,129.67) --
	(122.43,122.89) --
	(122.90,122.35) --
	(128.23,115.03) --
	(129.75,112.48) --
	(132.24,107.71) --
	(135.19,100.40) --
	(137.07, 93.79) --
	(137.25, 93.08) --
	(138.32, 85.76) --
	(138.62, 78.44) --
	(138.13, 71.13) --
	(137.07, 64.94) --
	(136.85, 63.81) --
	(134.56, 56.49) --
	(131.40, 49.18) --
	(129.75, 46.19) --
	(127.04, 41.86) --
	(122.43, 35.81) --
	(121.31, 34.54) --
	(115.11, 28.57) --
	(113.46, 27.22) --
	(107.80, 23.23) --
	(102.02, 19.91) --
	(100.48, 19.12) --
	( 93.16, 16.26) --
	( 85.84, 14.24) --
	( 78.53, 13.06) --
	( 71.21, 12.74) --
	( 63.89, 13.27) --
	( 56.57, 14.65) --
	( 49.26, 16.87) --
	( 42.03, 19.91) --
	( 41.94, 19.95) --
	( 34.62, 24.38) --
	( 30.77, 27.22) --
	( 27.31, 30.17) --
	( 22.93, 34.54) --
	( 19.99, 38.01) --
	( 17.15, 41.86) --
	( 12.71, 49.18) --
	( 12.67, 49.26);

\path[draw=drawColor,line width= 0.4pt,line join=round,line cap=round] ( 14.30,112.09) -- ( 15.99,115.03);

\node[text=drawColor,rotate= 60.15,anchor=base west,inner sep=0pt, outer sep=0pt, scale=  0.60] at ( 14.63,108.13) { 5 };

\path[draw=drawColor,line width= 0.4pt,line join=round,line cap=round] ( 27.31, 14.57) --
	( 20.36, 19.91) --
	( 19.99, 20.23) --
	( 13.00, 27.22) --
	( 12.67, 27.60) --
	(  7.34, 34.54) --
	(  5.35, 37.58);

\path[draw=drawColor,line width= 0.4pt,line join=round,line cap=round] ( 27.61, 14.37) -- ( 27.31, 14.57);

\node[text=drawColor,rotate=-33.12,anchor=base west,inner sep=0pt, outer sep=0pt, scale=  0.60] at ( 26.38, 12.48) { 7 };

\path[draw=drawColor,line width= 0.4pt,line join=round,line cap=round] (  5.35,121.14) --
	(  6.11,122.35) --
	( 11.52,129.67) --
	( 12.67,131.01);

\path[draw=drawColor,line width= 0.4pt,line join=round,line cap=round] ( 18.46,136.98) --
	( 19.99,138.37) --
	( 27.31,144.16) --
	( 27.52,144.30) --
	( 34.62,148.56) --
	( 40.84,151.62);

\path[draw=drawColor,line width= 0.4pt,line join=round,line cap=round] ( 14.94,133.35) -- ( 18.46,136.98);

\node[text=drawColor,rotate= 45.95,anchor=base west,inner sep=0pt, outer sep=0pt, scale=  0.60] at ( 14.30,129.43) { 7 };

\path[draw=drawColor,line width= 0.4pt,line join=round,line cap=round] (142.26,115.03) --
	(138.07,122.35) --
	(137.07,123.82) --
	(132.61,129.67) --
	(129.75,132.92) --
	(125.69,136.98) --
	(122.43,139.85) --
	(116.58,144.30) --
	(115.11,145.30) --
	(107.80,149.50) --
	(103.25,151.62);

\path[draw=drawColor,line width= 0.4pt,line join=round,line cap=round] (143.00,113.45) -- (142.26,115.03);

\node[text=drawColor,rotate=-65.00,anchor=base west,inner sep=0pt, outer sep=0pt, scale=  0.60] at (140.95,112.49) { 7 };

\path[draw=drawColor,line width= 0.4pt,line join=round,line cap=round] (113.90, 12.59) --
	(115.11, 13.34);

\path[draw=drawColor,line width= 0.4pt,line join=round,line cap=round] (122.43, 18.76) --
	(123.77, 19.91) --
	(129.75, 25.69) --
	(131.14, 27.22) --
	(136.92, 34.54) --
	(137.07, 34.76) --
	(141.32, 41.86) --
	(144.38, 48.07);

\path[draw=drawColor,line width= 0.4pt,line join=round,line cap=round] (117.74, 15.29) -- (122.43, 18.76);

\node[text=drawColor,rotate= 36.52,anchor=base west,inner sep=0pt, outer sep=0pt, scale=  0.60] at (116.46, 11.53) { 7 };

\path[draw=drawColor,line width= 0.4pt,line join=round,line cap=round] ( 12.68, 12.59) --
	( 12.67, 12.60) --
	(  5.36, 19.91) --
	(  5.35, 19.92);

\path[draw=drawColor,line width= 0.4pt,line join=round,line cap=round] (  5.35,138.66) --
	( 10.84,144.30) --
	( 12.67,145.99) --
	( 19.60,151.62);

\path[draw=drawColor,line width= 0.4pt,line join=round,line cap=round] (140.26,136.98) --
	(137.07,140.56) --
	(133.32,144.30) --
	(129.75,147.49) --
	(124.50,151.62);

\path[draw=drawColor,line width= 0.4pt,line join=round,line cap=round] (142.36,134.31) -- (140.26,136.98);

\node[text=drawColor,rotate=-51.83,anchor=base west,inner sep=0pt, outer sep=0pt, scale=  0.60] at (140.58,132.91) { 9 };

\path[draw=drawColor,line width= 0.4pt,line join=round,line cap=round] (137.07, 18.07) --
	(138.75, 19.91) --
	(144.38, 26.84);

\path[draw=drawColor,line width= 0.4pt,line join=round,line cap=round] (131.42, 12.59) -- (134.72, 15.80);

\node[text=drawColor,rotate= 44.21,anchor=base west,inner sep=0pt, outer sep=0pt, scale=  0.60] at (136.30, 14.17) { 9 };
\definecolor{drawColor}{RGB}{152,78,163}

\path[draw=drawColor,line width= 1.2pt,line join=round,line cap=round] ( 61.12, 83.95) -- ( 43.66, 65.31);

\path[draw=drawColor,line width= 1.2pt,line join=round,line cap=round] ( 84.09, 50.95) -- ( 92.61, 52.51);

\path[draw=drawColor,line width= 1.2pt,line join=round,line cap=round] (120.03, 30.80) -- (124.47, 35.29);

\path[draw=drawColor,line width= 1.2pt,line join=round,line cap=round] (102.12, 95.27) -- ( 87.09, 98.32);

\path[draw=drawColor,line width= 1.2pt,line join=round,line cap=round] ( 45.37, 85.82) -- ( 67.82, 63.12);

\path[draw=drawColor,line width= 1.2pt,line join=round,line cap=round] ( 77.66, 47.49) -- ( 83.75, 60.15);
\definecolor{drawColor}{RGB}{255,127,0}

\path[draw=drawColor,line width= 1.2pt,line join=round,line cap=round] ( 31.00,135.13) -- ( 29.11,151.62);

\path[draw=drawColor,line width= 1.2pt,line join=round,line cap=round] ( 33.41,145.81) --
	( 29.11,151.62) --
	( 26.23,144.99);

\path[draw=drawColor,line width= 1.2pt,line join=round,line cap=round] (105.45, 42.87) -- (101.53, 13.22);

\path[draw=drawColor,line width= 1.2pt,line join=round,line cap=round] ( 98.77, 19.90) --
	(101.53, 13.22) --
	(105.93, 18.95);

\path[draw=drawColor,line width= 1.2pt,line join=round,line cap=round] (111.97,118.65) -- (139.19,145.37);

\path[draw=drawColor,line width= 1.2pt,line join=round,line cap=round] (137.25,138.41) --
	(139.19,145.37) --
	(132.19,143.56);

\path[draw=drawColor,line width= 1.2pt,line join=round,line cap=round] (117.50, 92.13) -- (131.79,110.19);

\path[draw=drawColor,line width= 1.2pt,line join=round,line cap=round] (130.74,103.04) --
	(131.79,110.19) --
	(125.08,107.53);

\path[draw=drawColor,line width= 1.2pt,line join=round,line cap=round] (125.01, 45.85) -- (123.70, 18.73);

\path[draw=drawColor,line width= 1.2pt,line join=round,line cap=round] (120.40, 25.15) --
	(123.70, 18.73) --
	(127.62, 24.80);

\path[draw=drawColor,line width= 1.2pt,line join=round,line cap=round] (107.73, 63.64) -- (111.50, 48.56);

\path[draw=drawColor,line width= 1.2pt,line join=round,line cap=round] (106.47, 53.76) --
	(111.50, 48.56) --
	(113.49, 55.51);

\path[draw=drawColor,line width= 1.2pt,line join=round,line cap=round] (100.01,103.59) -- (114.12,121.57);

\path[draw=drawColor,line width= 1.2pt,line join=round,line cap=round] (113.10,114.42) --
	(114.12,121.57) --
	(107.41,118.88);

\path[draw=drawColor,line width= 1.2pt,line join=round,line cap=round] ( 89.83, 42.11) -- ( 74.25, 16.51);

\path[draw=drawColor,line width= 1.2pt,line join=round,line cap=round] ( 74.42, 23.74) --
	( 74.25, 16.51) --
	( 80.59, 19.98);

\path[draw=drawColor,line width= 1.2pt,line join=round,line cap=round] ( 26.76, 93.40) -- (  7.01, 72.31);

\path[draw=drawColor,line width= 1.2pt,line join=round,line cap=round] (  8.65, 79.35) --
	(  7.01, 72.31) --
	( 13.93, 74.41);

\path[draw=drawColor,line width= 1.2pt,line join=round,line cap=round] ( 35.13, 88.32) -- (  5.35, 78.10);

\path[draw=drawColor,line width= 1.2pt,line join=round,line cap=round] ( 10.10, 83.55) --
	(  5.35, 78.10) --
	( 12.45, 76.71);

\path[draw=drawColor,line width= 1.2pt,line join=round,line cap=round] ( 21.85, 91.74) -- ( 11.53,109.04);

\path[draw=drawColor,line width= 1.2pt,line join=round,line cap=round] ( 17.84,105.52) --
	( 11.53,109.04) --
	( 11.63,101.81);

\path[draw=drawColor,line width= 1.2pt,line join=round,line cap=round] ( 38.60,116.27) -- ( 20.89,116.63);

\path[draw=drawColor,line width= 1.2pt,line join=round,line cap=round] ( 27.23,120.11) --
	( 20.89,116.63) --
	( 27.08,112.89);
\definecolor{drawColor}{RGB}{77,175,74}

\path[draw=drawColor,line width= 1.2pt,line join=round,line cap=round] ( 67.47, 71.83) -- ( 40.49, 50.57);

\path[draw=drawColor,line width= 1.2pt,line join=round,line cap=round] ( 43.17, 57.28) --
	( 40.49, 50.57) --
	( 47.65, 51.61);

\path[draw=drawColor,line width= 1.2pt,line join=round,line cap=round] ( 60.78, 48.55) -- ( 74.55, 29.80);

\path[draw=drawColor,line width= 1.2pt,line join=round,line cap=round] ( 67.94, 32.70) --
	( 74.55, 29.80) --
	( 73.76, 36.98);
\definecolor{drawColor}{RGB}{0,0,0}
\definecolor{fillColor}{RGB}{255,255,255}

\path[draw=drawColor,line width= 0.4pt,line join=round,line cap=round,fill=fillColor] (  0.00, 66.00) rectangle ( 53.62,  0.00);
\definecolor{drawColor}{RGB}{255,127,0}

\path[draw=drawColor,line width= 1.2pt,line join=round,line cap=round] (  9.90, 39.60) -- ( 29.70, 39.60);
\definecolor{drawColor}{RGB}{152,78,163}

\path[draw=drawColor,line width= 1.2pt,line join=round,line cap=round] (  9.90, 26.40) -- ( 29.70, 26.40);
\definecolor{drawColor}{RGB}{77,175,74}

\path[draw=drawColor,line width= 1.2pt,line join=round,line cap=round] (  9.90, 13.20) -- ( 29.70, 13.20);
\definecolor{drawColor}{RGB}{0,0,0}

\node[text=drawColor,anchor=base,inner sep=0pt, outer sep=0pt, scale=  1.00] at ( 26.81, 52.80) {label $y_i$};

\node[text=drawColor,anchor=base west,inner sep=0pt, outer sep=0pt, scale=  1.00] at ( 39.60, 35.83) {1};

\node[text=drawColor,anchor=base west,inner sep=0pt, outer sep=0pt, scale=  1.00] at ( 39.60, 22.63) {0};

\node[text=drawColor,anchor=base west,inner sep=0pt, outer sep=0pt, scale=  1.00] at ( 39.60,  9.43) {-1};

\path[draw=drawColor,line width= 0.4pt,line join=round,line cap=round] ( 31.00,135.13) circle (  2.47);

\path[draw=drawColor,line width= 0.4pt,line join=round,line cap=round] ( 29.11,151.62) circle (  2.47);

\node[text=drawColor,anchor=base,inner sep=0pt, outer sep=0pt, scale=  1.00] at ( 45.89,132.39) {$\mathbf x_{1}$};

\node[text=drawColor,anchor=base,inner sep=0pt, outer sep=0pt, scale=  1.00] at ( 44.00,148.88) {$\mathbf x_{1}'$};
\end{scope}
\begin{scope}
\path[clip] (  0.00,  0.00) rectangle (433.62,180.67);
\definecolor{drawColor}{RGB}{0,0,0}

\path[draw=drawColor,line width= 0.4pt,line join=round,line cap=round] (144.54,  0.00) --
	(289.08,  0.00) --
	(289.08,164.83) --
	(144.54,164.83) --
	(144.54,  0.00);
\end{scope}
\begin{scope}
\path[clip] (  0.00,  0.00) rectangle (433.62,180.67);
\definecolor{drawColor}{RGB}{0,0,0}

\node[text=drawColor,anchor=base,inner sep=0pt, outer sep=0pt, scale=  1.00] at (216.81,167.47) {Enlarged features $\Phi(\mathbf x)$};
\end{scope}
\end{tikzpicture}
\begin{tikzpicture}[x=1pt,y=1pt]
\definecolor{fillColor}{RGB}{255,255,255}
\path[use as bounding box,fill=fillColor,fill opacity=0.00] (0,0) rectangle (433.62,180.67);
\begin{scope}
\path[clip] (  0.00,  0.00) rectangle (433.62,180.67);
\definecolor{drawColor}{RGB}{0,0,0}

\path[draw=drawColor,line width= 0.4pt,line join=round,line cap=round] (  0.00,  0.00) --
	(144.54,  0.00) --
	(144.54,164.83) --
	(  0.00,164.83) --
	(  0.00,  0.00);
\end{scope}
\begin{scope}
\path[clip] (  0.00,  0.00) rectangle (144.54,164.83);
\definecolor{drawColor}{RGB}{190,190,190}

\path[draw=drawColor,line width= 0.4pt,line join=round,line cap=round] (  0.00, 79.33) -- (144.54, 79.33);

\path[draw=drawColor,line width= 0.4pt,line join=round,line cap=round] ( 72.09,  0.00) -- ( 72.09,164.84);
\end{scope}
\begin{scope}
\path[clip] (  0.00,  0.00) rectangle (433.62,180.67);
\definecolor{drawColor}{RGB}{0,0,0}

\node[text=drawColor,anchor=base,inner sep=0pt, outer sep=0pt, scale=  1.00] at ( 72.27,167.47) {Original features $\mathbf x\in\mathbb R^2$};
\end{scope}
\begin{scope}
\path[clip] (  0.00,  0.00) rectangle (144.54,164.83);
\definecolor{drawColor}{gray}{0.50}

\path[draw=drawColor,line width= 0.4pt,line join=round,line cap=round] ( 46.90, 63.81) --
	( 43.63, 71.13) --
	( 42.37, 78.44) --
	( 43.14, 85.76) --
	( 45.93, 93.08) --
	( 49.26, 98.15) --
	( 51.30,100.40) --
	( 56.57,104.49) --
	( 63.30,107.71) --
	( 63.89,107.93) --
	( 71.21,108.97) --
	( 78.53,108.34) --
	( 80.50,107.71) --
	( 85.84,105.53) --
	( 93.14,100.40) --
	( 93.16,100.38) --
	( 98.30, 93.08) --
	(100.48, 87.74) --
	(101.10, 85.76) --
	(101.73, 78.44) --
	(100.70, 71.13) --
	(100.48, 70.53) --
	( 97.25, 63.81) --
	( 93.16, 58.53) --
	( 90.91, 56.49) --
	( 85.84, 53.16) --
	( 78.53, 50.37) --
	( 71.21, 49.61) --
	( 63.89, 50.86) --
	( 56.57, 54.14) --
	( 53.32, 56.49) --
	( 49.26, 60.55);

\path[draw=drawColor,line width= 0.4pt,line join=round,line cap=round] ( 49.26, 60.55) -- ( 48.82, 61.16);

\node[text=drawColor,rotate=-54.13,anchor=base west,inner sep=0pt, outer sep=0pt, scale=  0.60] at ( 45.07, 62.48) { 1 };

\path[draw=drawColor,line width= 0.4pt,line join=round,line cap=round] ( 27.31, 53.96) --
	( 25.92, 56.49);

\path[draw=drawColor,line width= 0.4pt,line join=round,line cap=round] ( 23.03, 63.81) --
	( 21.24, 71.13) --
	( 20.55, 78.44) --
	( 20.97, 85.76) --
	( 22.49, 93.08) --
	( 25.12,100.40) --
	( 27.31,104.67) --
	( 29.13,107.71) --
	( 34.62,114.76) --
	( 34.88,115.03) --
	( 41.94,121.09) --
	( 43.80,122.35) --
	( 49.26,125.44) --
	( 56.57,128.45) --
	( 61.36,129.67) --
	( 63.89,130.22) --
	( 71.21,130.84) --
	( 78.53,130.46) --
	( 82.78,129.67) --
	( 85.84,129.00) --
	( 93.16,126.27) --
	(100.48,122.40) --
	(100.55,122.35) --
	(107.80,116.45) --
	(109.22,115.03) --
	(115.11,107.79) --
	(115.16,107.71) --
	(119.04,100.40) --
	(121.77, 93.08) --
	(122.43, 90.02) --
	(123.23, 85.76) --
	(123.60, 78.44) --
	(122.99, 71.13) --
	(122.43, 68.59) --
	(121.21, 63.81) --
	(118.21, 56.49) --
	(115.11, 51.04) --
	(113.86, 49.18) --
	(107.80, 42.12) --
	(107.52, 41.86) --
	(100.48, 36.37) --
	( 97.44, 34.54) --
	( 93.16, 32.36) --
	( 85.84, 29.73) --
	( 78.53, 28.20) --
	( 71.21, 27.78) --
	( 63.89, 28.47) --
	( 56.57, 30.26) --
	( 49.26, 33.16) --
	( 46.73, 34.54) --
	( 41.94, 37.63) --
	( 36.79, 41.86) --
	( 34.62, 44.03) --
	( 30.39, 49.18) --
	( 27.31, 53.96);

\path[draw=drawColor,line width= 0.4pt,line join=round,line cap=round] ( 25.92, 56.49) -- ( 24.23, 60.77);

\node[text=drawColor,rotate=-68.44,anchor=base west,inner sep=0pt, outer sep=0pt, scale=  0.60] at ( 20.92, 62.98) { 3 };

\path[draw=drawColor,line width= 0.4pt,line join=round,line cap=round] ( 12.67, 49.26) --
	(  9.63, 56.49) --
	(  7.41, 63.81) --
	(  6.03, 71.13) --
	(  5.51, 78.44) --
	(  5.83, 85.76) --
	(  7.00, 93.08) --
	(  9.02,100.40) --
	( 11.89,107.71) --
	( 12.67,109.25);

\path[draw=drawColor,line width= 0.4pt,line join=round,line cap=round] ( 15.99,115.03) --
	( 19.99,120.69) --
	( 21.33,122.35) --
	( 27.31,128.55) --
	( 28.57,129.67) --
	( 34.62,134.27) --
	( 38.95,136.98) --
	( 41.94,138.63) --
	( 49.26,141.79) --
	( 56.57,144.08) --
	( 57.71,144.30) --
	( 63.89,145.37) --
	( 71.21,145.85) --
	( 78.53,145.56) --
	( 85.84,144.48) --
	( 86.56,144.30) --
	( 93.16,142.43) --
	(100.48,139.47) --
	(105.24,136.98) --
	(107.80,135.47) --
	(115.11,130.14) --
	(115.66,129.67) --
	(122.43,122.89) --
	(122.90,122.35) --
	(128.23,115.03) --
	(129.75,112.48) --
	(132.24,107.71) --
	(135.19,100.40) --
	(137.07, 93.79) --
	(137.25, 93.08) --
	(138.32, 85.76) --
	(138.62, 78.44) --
	(138.13, 71.13) --
	(137.07, 64.94) --
	(136.85, 63.81) --
	(134.56, 56.49) --
	(131.40, 49.18) --
	(129.75, 46.19) --
	(127.04, 41.86) --
	(122.43, 35.81) --
	(121.31, 34.54) --
	(115.11, 28.57) --
	(113.46, 27.22) --
	(107.80, 23.23) --
	(102.02, 19.91) --
	(100.48, 19.12) --
	( 93.16, 16.26) --
	( 85.84, 14.24) --
	( 78.53, 13.06) --
	( 71.21, 12.74) --
	( 63.89, 13.27) --
	( 56.57, 14.65) --
	( 49.26, 16.87) --
	( 42.03, 19.91) --
	( 41.94, 19.95) --
	( 34.62, 24.38) --
	( 30.77, 27.22) --
	( 27.31, 30.17) --
	( 22.93, 34.54) --
	( 19.99, 38.01) --
	( 17.15, 41.86) --
	( 12.71, 49.18) --
	( 12.67, 49.26);

\path[draw=drawColor,line width= 0.4pt,line join=round,line cap=round] ( 14.30,112.09) -- ( 15.99,115.03);

\node[text=drawColor,rotate= 60.15,anchor=base west,inner sep=0pt, outer sep=0pt, scale=  0.60] at ( 14.63,108.13) { 5 };

\path[draw=drawColor,line width= 0.4pt,line join=round,line cap=round] ( 27.31, 14.57) --
	( 20.36, 19.91) --
	( 19.99, 20.23) --
	( 13.00, 27.22) --
	( 12.67, 27.60) --
	(  7.34, 34.54) --
	(  5.35, 37.58);

\path[draw=drawColor,line width= 0.4pt,line join=round,line cap=round] ( 27.61, 14.37) -- ( 27.31, 14.57);

\node[text=drawColor,rotate=-33.12,anchor=base west,inner sep=0pt, outer sep=0pt, scale=  0.60] at ( 26.38, 12.48) { 7 };

\path[draw=drawColor,line width= 0.4pt,line join=round,line cap=round] (  5.35,121.14) --
	(  6.11,122.35) --
	( 11.52,129.67) --
	( 12.67,131.01);

\path[draw=drawColor,line width= 0.4pt,line join=round,line cap=round] ( 18.46,136.98) --
	( 19.99,138.37) --
	( 27.31,144.16) --
	( 27.52,144.30) --
	( 34.62,148.56) --
	( 40.84,151.62);

\path[draw=drawColor,line width= 0.4pt,line join=round,line cap=round] ( 14.94,133.35) -- ( 18.46,136.98);

\node[text=drawColor,rotate= 45.95,anchor=base west,inner sep=0pt, outer sep=0pt, scale=  0.60] at ( 14.30,129.43) { 7 };

\path[draw=drawColor,line width= 0.4pt,line join=round,line cap=round] (142.26,115.03) --
	(138.07,122.35) --
	(137.07,123.82) --
	(132.61,129.67) --
	(129.75,132.92) --
	(125.69,136.98) --
	(122.43,139.85) --
	(116.58,144.30) --
	(115.11,145.30) --
	(107.80,149.50) --
	(103.25,151.62);

\path[draw=drawColor,line width= 0.4pt,line join=round,line cap=round] (143.00,113.45) -- (142.26,115.03);

\node[text=drawColor,rotate=-65.00,anchor=base west,inner sep=0pt, outer sep=0pt, scale=  0.60] at (140.95,112.49) { 7 };

\path[draw=drawColor,line width= 0.4pt,line join=round,line cap=round] (113.90, 12.59) --
	(115.11, 13.34);

\path[draw=drawColor,line width= 0.4pt,line join=round,line cap=round] (122.43, 18.76) --
	(123.77, 19.91) --
	(129.75, 25.69) --
	(131.14, 27.22) --
	(136.92, 34.54) --
	(137.07, 34.76) --
	(141.32, 41.86) --
	(144.38, 48.07);

\path[draw=drawColor,line width= 0.4pt,line join=round,line cap=round] (117.74, 15.29) -- (122.43, 18.76);

\node[text=drawColor,rotate= 36.52,anchor=base west,inner sep=0pt, outer sep=0pt, scale=  0.60] at (116.46, 11.53) { 7 };

\path[draw=drawColor,line width= 0.4pt,line join=round,line cap=round] ( 12.68, 12.59) --
	( 12.67, 12.60) --
	(  5.36, 19.91) --
	(  5.35, 19.92);

\path[draw=drawColor,line width= 0.4pt,line join=round,line cap=round] (  5.35,138.66) --
	( 10.84,144.30) --
	( 12.67,145.99) --
	( 19.60,151.62);

\path[draw=drawColor,line width= 0.4pt,line join=round,line cap=round] (140.26,136.98) --
	(137.07,140.56) --
	(133.32,144.30) --
	(129.75,147.49) --
	(124.50,151.62);

\path[draw=drawColor,line width= 0.4pt,line join=round,line cap=round] (142.36,134.31) -- (140.26,136.98);

\node[text=drawColor,rotate=-51.83,anchor=base west,inner sep=0pt, outer sep=0pt, scale=  0.60] at (140.58,132.91) { 9 };

\path[draw=drawColor,line width= 0.4pt,line join=round,line cap=round] (137.07, 18.07) --
	(138.75, 19.91) --
	(144.38, 26.84);

\path[draw=drawColor,line width= 0.4pt,line join=round,line cap=round] (131.42, 12.59) -- (134.72, 15.80);

\node[text=drawColor,rotate= 44.21,anchor=base west,inner sep=0pt, outer sep=0pt, scale=  0.60] at (136.30, 14.17) { 9 };
\definecolor{drawColor}{RGB}{152,78,163}

\path[draw=drawColor,line width= 1.2pt,line join=round,line cap=round] ( 61.12, 83.95) -- ( 43.66, 65.31);

\path[draw=drawColor,line width= 1.2pt,line join=round,line cap=round] ( 84.09, 50.95) -- ( 92.61, 52.51);

\path[draw=drawColor,line width= 1.2pt,line join=round,line cap=round] (120.03, 30.80) -- (124.47, 35.29);

\path[draw=drawColor,line width= 1.2pt,line join=round,line cap=round] (102.12, 95.27) -- ( 87.09, 98.32);

\path[draw=drawColor,line width= 1.2pt,line join=round,line cap=round] ( 45.37, 85.82) -- ( 67.82, 63.12);

\path[draw=drawColor,line width= 1.2pt,line join=round,line cap=round] ( 77.66, 47.49) -- ( 83.75, 60.15);
\definecolor{drawColor}{RGB}{255,127,0}

\path[draw=drawColor,line width= 1.2pt,line join=round,line cap=round] ( 31.00,135.13) -- ( 29.11,151.62);

\path[draw=drawColor,line width= 1.2pt,line join=round,line cap=round] ( 33.41,145.81) --
	( 29.11,151.62) --
	( 26.23,144.99);

\path[draw=drawColor,line width= 1.2pt,line join=round,line cap=round] (105.45, 42.87) -- (101.53, 13.22);

\path[draw=drawColor,line width= 1.2pt,line join=round,line cap=round] ( 98.77, 19.90) --
	(101.53, 13.22) --
	(105.93, 18.95);

\path[draw=drawColor,line width= 1.2pt,line join=round,line cap=round] (111.97,118.65) -- (139.19,145.37);

\path[draw=drawColor,line width= 1.2pt,line join=round,line cap=round] (137.25,138.41) --
	(139.19,145.37) --
	(132.19,143.56);

\path[draw=drawColor,line width= 1.2pt,line join=round,line cap=round] (117.50, 92.13) -- (131.79,110.19);

\path[draw=drawColor,line width= 1.2pt,line join=round,line cap=round] (130.74,103.04) --
	(131.79,110.19) --
	(125.08,107.53);

\path[draw=drawColor,line width= 1.2pt,line join=round,line cap=round] (125.01, 45.85) -- (123.70, 18.73);

\path[draw=drawColor,line width= 1.2pt,line join=round,line cap=round] (120.40, 25.15) --
	(123.70, 18.73) --
	(127.62, 24.80);

\path[draw=drawColor,line width= 1.2pt,line join=round,line cap=round] (107.73, 63.64) -- (111.50, 48.56);

\path[draw=drawColor,line width= 1.2pt,line join=round,line cap=round] (106.47, 53.76) --
	(111.50, 48.56) --
	(113.49, 55.51);

\path[draw=drawColor,line width= 1.2pt,line join=round,line cap=round] (100.01,103.59) -- (114.12,121.57);

\path[draw=drawColor,line width= 1.2pt,line join=round,line cap=round] (113.10,114.42) --
	(114.12,121.57) --
	(107.41,118.88);

\path[draw=drawColor,line width= 1.2pt,line join=round,line cap=round] ( 89.83, 42.11) -- ( 74.25, 16.51);

\path[draw=drawColor,line width= 1.2pt,line join=round,line cap=round] ( 74.42, 23.74) --
	( 74.25, 16.51) --
	( 80.59, 19.98);

\path[draw=drawColor,line width= 1.2pt,line join=round,line cap=round] ( 26.76, 93.40) -- (  7.01, 72.31);

\path[draw=drawColor,line width= 1.2pt,line join=round,line cap=round] (  8.65, 79.35) --
	(  7.01, 72.31) --
	( 13.93, 74.41);

\path[draw=drawColor,line width= 1.2pt,line join=round,line cap=round] ( 35.13, 88.32) -- (  5.35, 78.10);

\path[draw=drawColor,line width= 1.2pt,line join=round,line cap=round] ( 10.10, 83.55) --
	(  5.35, 78.10) --
	( 12.45, 76.71);

\path[draw=drawColor,line width= 1.2pt,line join=round,line cap=round] ( 21.85, 91.74) -- ( 11.53,109.04);

\path[draw=drawColor,line width= 1.2pt,line join=round,line cap=round] ( 17.84,105.52) --
	( 11.53,109.04) --
	( 11.63,101.81);

\path[draw=drawColor,line width= 1.2pt,line join=round,line cap=round] ( 38.60,116.27) -- ( 20.89,116.63);

\path[draw=drawColor,line width= 1.2pt,line join=round,line cap=round] ( 27.23,120.11) --
	( 20.89,116.63) --
	( 27.08,112.89);
\definecolor{drawColor}{RGB}{77,175,74}

\path[draw=drawColor,line width= 1.2pt,line join=round,line cap=round] ( 67.47, 71.83) -- ( 40.49, 50.57);

\path[draw=drawColor,line width= 1.2pt,line join=round,line cap=round] ( 43.17, 57.28) --
	( 40.49, 50.57) --
	( 47.65, 51.61);

\path[draw=drawColor,line width= 1.2pt,line join=round,line cap=round] ( 60.78, 48.55) -- ( 74.55, 29.80);

\path[draw=drawColor,line width= 1.2pt,line join=round,line cap=round] ( 67.94, 32.70) --
	( 74.55, 29.80) --
	( 73.76, 36.98);
\definecolor{drawColor}{RGB}{0,0,0}
\definecolor{fillColor}{RGB}{255,255,255}

\path[draw=drawColor,line width= 0.4pt,line join=round,line cap=round,fill=fillColor] (  0.00, 66.00) rectangle ( 53.62,  0.00);
\definecolor{drawColor}{RGB}{255,127,0}

\path[draw=drawColor,line width= 1.2pt,line join=round,line cap=round] (  9.90, 39.60) -- ( 29.70, 39.60);
\definecolor{drawColor}{RGB}{152,78,163}

\path[draw=drawColor,line width= 1.2pt,line join=round,line cap=round] (  9.90, 26.40) -- ( 29.70, 26.40);
\definecolor{drawColor}{RGB}{77,175,74}

\path[draw=drawColor,line width= 1.2pt,line join=round,line cap=round] (  9.90, 13.20) -- ( 29.70, 13.20);
\definecolor{drawColor}{RGB}{0,0,0}

\node[text=drawColor,anchor=base,inner sep=0pt, outer sep=0pt, scale=  1.00] at ( 26.81, 52.80) {label $y_i$};

\node[text=drawColor,anchor=base west,inner sep=0pt, outer sep=0pt, scale=  1.00] at ( 39.60, 35.83) {1};

\node[text=drawColor,anchor=base west,inner sep=0pt, outer sep=0pt, scale=  1.00] at ( 39.60, 22.63) {0};

\node[text=drawColor,anchor=base west,inner sep=0pt, outer sep=0pt, scale=  1.00] at ( 39.60,  9.43) {-1};

\path[draw=drawColor,line width= 0.4pt,line join=round,line cap=round] ( 31.00,135.13) circle (  2.47);

\path[draw=drawColor,line width= 0.4pt,line join=round,line cap=round] ( 29.11,151.62) circle (  2.47);

\node[text=drawColor,anchor=base,inner sep=0pt, outer sep=0pt, scale=  1.00] at ( 45.89,132.39) {$\mathbf x_{1}$};

\node[text=drawColor,anchor=base,inner sep=0pt, outer sep=0pt, scale=  1.00] at ( 44.00,148.88) {$\mathbf x_{1}'$};
\end{scope}
\begin{scope}
\path[clip] (  0.00,  0.00) rectangle (433.62,180.67);
\definecolor{drawColor}{RGB}{0,0,0}

\path[draw=drawColor,line width= 0.4pt,line join=round,line cap=round] (144.54,  0.00) --
	(289.08,  0.00) --
	(289.08,164.83) --
	(144.54,164.83) --
	(144.54,  0.00);
\end{scope}
\begin{scope}
\path[clip] (  0.00,  0.00) rectangle (433.62,180.67);
\definecolor{drawColor}{RGB}{0,0,0}

\node[text=drawColor,anchor=base,inner sep=0pt, outer sep=0pt, scale=  1.00] at (216.81,167.47) {Enlarged features $\Phi(\mathbf x)$};
\end{scope}
\begin{scope}
\path[clip] (144.54,  0.00) rectangle (289.08,164.83);
\definecolor{drawColor}{gray}{0.50}

\node[text=drawColor,rotate=-45.00,anchor=base,inner sep=0pt, outer sep=0pt, scale=  0.80] at (259.58, 62.82) {$r(\mathbf x)=\mathbf w^\intercal \Phi(\mathbf x)$};

\path[draw=drawColor,line width= 0.4pt,line join=round,line cap=round] (176.86,  6.10) --
	(175.13,  7.83) --
	(168.82, 14.14) --
	(167.10, 15.87);

\path[draw=drawColor,line width= 0.4pt,line join=round,line cap=round] (160.79, 22.17) --
	(159.06, 23.90) --
	(152.76, 30.20) --
	(151.03, 31.93);

\path[draw=drawColor,line width= 0.4pt,line join=round,line cap=round] (164.79, 18.18) -- (160.79, 22.17);

\node[text=drawColor,rotate=-45.02,anchor=base west,inner sep=0pt, outer sep=0pt, scale=  0.60] at (163.19, 16.58) { 1 };

\path[draw=drawColor,line width= 0.4pt,line join=round,line cap=round] (199.23,  9.65) --
	(194.74, 14.14) --
	(191.19, 17.68) --
	(186.71, 22.17) --
	(183.16, 25.72) --
	(178.67, 30.20) --
	(175.13, 33.75) --
	(170.64, 38.24) --
	(167.10, 41.78) --
	(162.61, 46.27) --
	(159.06, 49.81) --
	(154.58, 54.30) --
	(151.03, 57.85);

\path[draw=drawColor,line width= 0.4pt,line join=round,line cap=round] (200.46,  8.41) -- (199.23,  9.65);

\node[text=drawColor,rotate=-45.02,anchor=base west,inner sep=0pt, outer sep=0pt, scale=  0.60] at (198.86,  6.82) { 2 };

\path[draw=drawColor,line width= 0.4pt,line join=round,line cap=round] (228.69,  6.10) --
	(223.33, 11.47);

\path[draw=drawColor,line width= 0.4pt,line join=round,line cap=round] (220.65, 14.14) --
	(215.29, 19.50) --
	(212.62, 22.17) --
	(207.26, 27.53) --
	(204.59, 30.20) --
	(199.23, 35.56) --
	(196.56, 38.24) --
	(191.19, 43.60) --
	(188.52, 46.27) --
	(183.16, 51.63) --
	(180.49, 54.30) --
	(175.13, 59.66) --
	(172.46, 62.34) --
	(167.10, 67.70) --
	(164.42, 70.37) --
	(159.06, 75.73) --
	(156.39, 78.40) --
	(151.03, 83.76);

\path[draw=drawColor,line width= 0.4pt,line join=round,line cap=round] (221.02, 13.78) -- (220.65, 14.14);

\node[text=drawColor,rotate=-45.02,anchor=base west,inner sep=0pt, outer sep=0pt, scale=  0.60] at (219.42, 12.18) { 3 };

\path[draw=drawColor,line width= 0.4pt,line join=round,line cap=round] (254.60,  6.10) --
	(247.43, 13.28) --
	(246.57, 14.14);

\path[draw=drawColor,line width= 0.4pt,line join=round,line cap=round] (239.39, 21.32) --
	(238.54, 22.17) --
	(231.36, 29.35) --
	(230.50, 30.20) --
	(223.33, 37.38) --
	(222.47, 38.24) --
	(215.29, 45.41) --
	(214.44, 46.27) --
	(207.26, 53.45) --
	(206.41, 54.30) --
	(199.23, 61.48) --
	(198.37, 62.34) --
	(191.19, 69.51) --
	(190.34, 70.37) --
	(183.16, 77.55) --
	(182.31, 78.40) --
	(175.13, 85.58) --
	(174.27, 86.43) --
	(167.10, 93.61) --
	(166.24, 94.47) --
	(159.06,101.64) --
	(158.21,102.50) --
	(151.03,109.68);

\path[draw=drawColor,line width= 0.4pt,line join=round,line cap=round] (244.26, 16.45) -- (239.39, 21.32);

\node[text=drawColor,rotate=-45.02,anchor=base west,inner sep=0pt, outer sep=0pt, scale=  0.60] at (242.66, 14.85) { 4 };

\path[draw=drawColor,line width= 0.4pt,line join=round,line cap=round] (280.52,  6.10) --
	(279.56,  7.07) --
	(272.49, 14.14) --
	(271.52, 15.10);

\path[draw=drawColor,line width= 0.4pt,line join=round,line cap=round] (264.45, 22.17) --
	(263.49, 23.13) --
	(256.42, 30.20) --
	(255.46, 31.17) --
	(248.39, 38.24) --
	(247.43, 39.20) --
	(240.35, 46.27) --
	(239.39, 47.23) --
	(232.32, 54.30) --
	(231.36, 55.26) --
	(224.29, 62.34) --
	(223.33, 63.30) --
	(216.26, 70.37) --
	(215.29, 71.33) --
	(208.22, 78.40) --
	(207.26, 79.36) --
	(200.19, 86.43) --
	(199.23, 87.40) --
	(192.16, 94.47) --
	(191.19, 95.43) --
	(184.12,102.50) --
	(183.16,103.46) --
	(176.09,110.53) --
	(175.13,111.49) --
	(168.06,118.57) --
	(167.10,119.53) --
	(160.03,126.60) --
	(159.06,127.56) --
	(151.99,134.63) --
	(151.03,135.59);

\path[draw=drawColor,line width= 0.4pt,line join=round,line cap=round] (269.21, 17.41) -- (264.45, 22.17);

\node[text=drawColor,rotate=-45.02,anchor=base west,inner sep=0pt, outer sep=0pt, scale=  0.60] at (267.61, 15.81) { 5 };

\path[draw=drawColor,line width= 0.4pt,line join=round,line cap=round] (298.40, 14.14) --
	(295.62, 16.92) --
	(290.37, 22.17) --
	(287.59, 24.95) --
	(282.34, 30.20) --
	(279.56, 32.98) --
	(274.30, 38.24) --
	(271.52, 41.01) --
	(266.27, 46.27) --
	(263.49, 49.05) --
	(258.24, 54.30) --
	(255.46, 57.08) --
	(250.20, 62.34) --
	(247.43, 65.11) --
	(242.17, 70.37) --
	(239.39, 73.15) --
	(234.14, 78.40) --
	(231.36, 81.18) --
	(226.10, 86.43) --
	(223.33, 89.21) --
	(218.07, 94.47) --
	(215.29, 97.25) --
	(210.04,102.50) --
	(207.26,105.28) --
	(202.01,110.53) --
	(199.23,113.31) --
	(193.97,118.57) --
	(191.19,121.34) --
	(185.94,126.60) --
	(183.16,129.38) --
	(177.91,134.63) --
	(175.13,137.41) --
	(169.87,142.66) --
	(167.10,145.44) --
	(161.84,150.70) --
	(159.06,153.48) --
	(153.81,158.73);

\path[draw=drawColor,line width= 0.4pt,line join=round,line cap=round] (301.35, 11.19) -- (298.40, 14.14);

\node[text=drawColor,rotate=-45.02,anchor=base west,inner sep=0pt, outer sep=0pt, scale=  0.60] at (299.75,  9.59) { 6 };

\path[draw=drawColor,line width= 0.4pt,line join=round,line cap=round] (300.22, 38.24) --
	(295.62, 42.83) --
	(292.18, 46.27) --
	(287.59, 50.86) --
	(284.15, 54.30) --
	(279.56, 58.90) --
	(276.12, 62.34) --
	(271.52, 66.93) --
	(268.09, 70.37) --
	(263.49, 74.96) --
	(260.05, 78.40) --
	(255.46, 83.00) --
	(252.02, 86.43) --
	(247.43, 91.03) --
	(243.99, 94.47) --
	(239.39, 99.06) --
	(235.95,102.50) --
	(231.36,107.09) --
	(227.92,110.53) --
	(223.33,115.13) --
	(219.89,118.57) --
	(215.29,123.16) --
	(211.86,126.60) --
	(207.26,131.19) --
	(203.82,134.63) --
	(199.23,139.23) --
	(195.79,142.66) --
	(191.19,147.26) --
	(187.76,150.70) --
	(183.16,155.29) --
	(179.72,158.73);

\path[draw=drawColor,line width= 0.4pt,line join=round,line cap=round] (301.35, 37.11) -- (300.22, 38.24);

\node[text=drawColor,rotate=-45.02,anchor=base west,inner sep=0pt, outer sep=0pt, scale=  0.60] at (299.75, 35.51) { 7 };

\path[draw=drawColor,line width= 0.4pt,line join=round,line cap=round] (303.66, 60.71) --
	(302.03, 62.34);

\path[draw=drawColor,line width= 0.4pt,line join=round,line cap=round] (295.62, 68.75) --
	(294.00, 70.37) --
	(287.59, 76.78) --
	(285.97, 78.40) --
	(279.56, 84.81) --
	(277.94, 86.43) --
	(271.52, 92.85) --
	(269.90, 94.47) --
	(263.49,100.88) --
	(261.87,102.50) --
	(255.46,108.91) --
	(253.84,110.53) --
	(247.43,116.94) --
	(245.80,118.57) --
	(239.39,124.98) --
	(237.77,126.60) --
	(231.36,133.01) --
	(229.74,134.63) --
	(223.33,141.04) --
	(221.71,142.66) --
	(215.29,149.08) --
	(213.67,150.70) --
	(207.26,157.11) --
	(205.64,158.73);

\path[draw=drawColor,line width= 0.4pt,line join=round,line cap=round] (299.72, 64.64) -- (295.62, 68.75);

\node[text=drawColor,rotate=-45.02,anchor=base west,inner sep=0pt, outer sep=0pt, scale=  0.60] at (298.12, 63.05) { 8 };

\path[draw=drawColor,line width= 0.4pt,line join=round,line cap=round] (295.82, 94.47) --
	(295.62, 94.66) --
	(287.79,102.50) --
	(287.59,102.70) --
	(279.75,110.53) --
	(279.56,110.73) --
	(271.72,118.57) --
	(271.52,118.76) --
	(263.69,126.60) --
	(263.49,126.79) --
	(255.65,134.63) --
	(255.46,134.83) --
	(247.62,142.66) --
	(247.43,142.86) --
	(239.59,150.70) --
	(239.39,150.89) --
	(231.55,158.73);

\path[draw=drawColor,line width= 0.4pt,line join=round,line cap=round] (301.35, 88.94) -- (295.82, 94.47);

\node[text=drawColor,rotate=-45.02,anchor=base west,inner sep=0pt, outer sep=0pt, scale=  0.60] at (299.75, 87.34) { 9 };

\path[draw=drawColor,line width= 0.4pt,line join=round,line cap=round] (297.63,118.57) --
	(295.62,120.58) --
	(289.60,126.60) --
	(287.59,128.61) --
	(281.57,134.63) --
	(279.56,136.64) --
	(273.54,142.66) --
	(271.52,144.68) --
	(265.50,150.70) --
	(263.49,152.71) --
	(257.47,158.73);

\path[draw=drawColor,line width= 0.4pt,line join=round,line cap=round] (299.04,117.16) -- (297.63,118.57);

\node[text=drawColor,rotate=-45.02,anchor=base west,inner sep=0pt, outer sep=0pt, scale=  0.60] at (297.44,115.56) { 10 };

\path[draw=drawColor,line width= 0.4pt,line join=round,line cap=round] (303.66,138.46) --
	(299.45,142.66);

\path[draw=drawColor,line width= 0.4pt,line join=round,line cap=round] (291.42,150.70) --
	(287.59,154.53) --
	(283.39,158.73);

\path[draw=drawColor,line width= 0.4pt,line join=round,line cap=round] (294.83,147.28) -- (291.42,150.70);

\node[text=drawColor,rotate=-45.02,anchor=base west,inner sep=0pt, outer sep=0pt, scale=  0.60] at (293.23,145.68) { 11 };
\definecolor{drawColor}{RGB}{152,78,163}

\path[draw=drawColor,line width= 1.2pt,line join=round,line cap=round] (154.50,  6.68) -- (174.61, 11.80);

\path[draw=drawColor,line width= 1.2pt,line join=round,line cap=round] (155.19, 29.59) -- (163.29, 27.07);

\path[draw=drawColor,line width= 1.2pt,line join=round,line cap=round] (218.13, 74.86) -- (231.14, 62.71);

\path[draw=drawColor,line width= 1.2pt,line join=round,line cap=round] (177.33, 13.48) -- (157.56, 16.60);

\path[draw=drawColor,line width= 1.2pt,line join=round,line cap=round] (171.85,  7.29) -- (151.52, 13.74);

\path[draw=drawColor,line width= 1.2pt,line join=round,line cap=round] (151.89, 35.67) -- (154.96, 16.81);
\definecolor{drawColor}{RGB}{255,127,0}

\path[draw=drawColor,line width= 1.2pt,line join=round,line cap=round] (200.32, 97.03) -- (204.97,158.73);

\path[draw=drawColor,line width= 1.2pt,line join=round,line cap=round] (208.10,152.22) --
	(204.97,158.73) --
	(200.90,152.76);

\path[draw=drawColor,line width= 1.2pt,line join=round,line cap=round] (183.49, 44.90) -- (176.30,133.74);

\path[draw=drawColor,line width= 1.2pt,line join=round,line cap=round] (180.41,127.80) --
	(176.30,133.74) --
	(173.20,127.21);

\path[draw=drawColor,line width= 1.2pt,line join=round,line cap=round] (197.43, 51.24) -- (282.50,133.47);

\path[draw=drawColor,line width= 1.2pt,line join=round,line cap=round] (280.51,126.53) --
	(282.50,133.47) --
	(275.49,131.72);

\path[draw=drawColor,line width= 1.2pt,line join=round,line cap=round] (211.23, 10.85) -- (255.12, 33.89);

\path[draw=drawColor,line width= 1.2pt,line join=round,line cap=round] (251.25, 27.78) --
	(255.12, 33.89) --
	(247.90, 34.18);

\path[draw=drawColor,line width= 1.2pt,line join=round,line cap=round] (232.81, 38.80) -- (228.81,113.35);

\path[draw=drawColor,line width= 1.2pt,line join=round,line cap=round] (232.75,107.30) --
	(228.81,113.35) --
	(225.53,106.91);

\path[draw=drawColor,line width= 1.2pt,line join=round,line cap=round] (188.08, 13.25) -- (196.35, 33.71);

\path[draw=drawColor,line width= 1.2pt,line join=round,line cap=round] (197.36, 26.55) --
	(196.35, 33.71) --
	(190.66, 29.26);

\path[draw=drawColor,line width= 1.2pt,line join=round,line cap=round] (173.75, 23.26) -- (202.58, 58.20);

\path[draw=drawColor,line width= 1.2pt,line join=round,line cap=round] (201.38, 51.07) --
	(202.58, 58.20) --
	(195.81, 55.67);

\path[draw=drawColor,line width= 1.2pt,line join=round,line cap=round] (160.18, 46.54) -- (151.12,121.33);

\path[draw=drawColor,line width= 1.2pt,line join=round,line cap=round] (155.46,115.55) --
	(151.12,121.33) --
	(148.29,114.68);

\path[draw=drawColor,line width= 1.2pt,line join=round,line cap=round] (211.02, 11.85) -- (274.72,  7.50);

\path[draw=drawColor,line width= 1.2pt,line join=round,line cap=round] (268.23,  4.32) --
	(274.72,  7.50) --
	(268.72, 11.53);

\path[draw=drawColor,line width= 1.2pt,line join=round,line cap=round] (190.91,  8.42) -- (281.11,  6.10);

\path[draw=drawColor,line width= 1.2pt,line join=round,line cap=round] (274.76,  2.65) --
	(281.11,  6.10) --
	(274.94,  9.88);

\path[draw=drawColor,line width= 1.2pt,line join=round,line cap=round] (224.73, 10.56) -- (258.13, 31.85);

\path[draw=drawColor,line width= 1.2pt,line join=round,line cap=round] (254.79, 25.44) --
	(258.13, 31.85) --
	(250.91, 31.53);

\path[draw=drawColor,line width= 1.2pt,line join=round,line cap=round] (183.75, 45.92) -- (227.56, 46.70);

\path[draw=drawColor,line width= 1.2pt,line join=round,line cap=round] (221.37, 42.98) --
	(227.56, 46.70) --
	(221.24, 50.20);
\definecolor{drawColor}{RGB}{77,175,74}

\path[draw=drawColor,line width= 1.2pt,line join=round,line cap=round] (151.61,  7.70) -- (180.16, 30.22);

\path[draw=drawColor,line width= 1.2pt,line join=round,line cap=round] (177.48, 23.51) --
	(180.16, 30.22) --
	(173.01, 29.18);

\path[draw=drawColor,line width= 1.2pt,line join=round,line cap=round] (154.72, 33.74) -- (151.16, 77.73);

\path[draw=drawColor,line width= 1.2pt,line join=round,line cap=round] (155.27, 71.78) --
	(151.16, 77.73) --
	(148.07, 71.20);
\definecolor{drawColor}{RGB}{0,0,0}

\path[draw=drawColor,line width= 0.4pt,line join=round,line cap=round] (200.32, 97.03) circle (  2.47);

\path[draw=drawColor,line width= 0.4pt,line join=round,line cap=round] (204.97,158.73) circle (  2.47);

\node[text=drawColor,anchor=base,inner sep=0pt, outer sep=0pt, scale=  0.80] at (215.87, 94.84) {$\Phi(\mathbf x_{1})$};

\node[text=drawColor,anchor=base,inner sep=0pt, outer sep=0pt, scale=  0.80] at (220.52,156.54) {$\Phi(\mathbf x_{1}')$};
\definecolor{drawColor}{RGB}{190,190,190}

\path[draw=drawColor,line width= 0.4pt,line join=round,line cap=round] (144.54,  6.06) -- (289.08,  6.06);

\path[draw=drawColor,line width= 0.4pt,line join=round,line cap=round] (150.99,  0.00) -- (150.99,164.84);
\end{scope}
\begin{scope}
\path[clip] (289.08,  0.00) rectangle (433.62,164.83);
\definecolor{drawColor}{RGB}{255,127,0}
\definecolor{fillColor}{RGB}{255,127,0}

\path[draw=drawColor,line width= 0.4pt,line join=round,line cap=round,fill=fillColor] (331.84,126.68) circle (  1.65);
\definecolor{drawColor}{RGB}{152,78,163}
\definecolor{fillColor}{RGB}{152,78,163}

\path[draw=drawColor,line width= 0.4pt,line join=round,line cap=round,fill=fillColor] (349.26, 62.90) circle (  1.65);
\definecolor{drawColor}{RGB}{255,127,0}
\definecolor{fillColor}{RGB}{255,127,0}

\path[draw=drawColor,line width= 0.4pt,line join=round,line cap=round,fill=fillColor] (318.51,157.27) circle (  1.65);

\path[draw=drawColor,line width= 0.4pt,line join=round,line cap=round,fill=fillColor] (422.48,149.82) circle (  1.65);
\definecolor{drawColor}{RGB}{152,78,163}
\definecolor{fillColor}{RGB}{152,78,163}

\path[draw=drawColor,line width= 0.4pt,line join=round,line cap=round,fill=fillColor] (335.73, 54.30) circle (  1.65);
\definecolor{drawColor}{RGB}{255,127,0}
\definecolor{fillColor}{RGB}{255,127,0}

\path[draw=drawColor,line width= 0.4pt,line join=round,line cap=round,fill=fillColor] (376.07, 83.10) circle (  1.65);
\definecolor{drawColor}{RGB}{152,78,163}
\definecolor{fillColor}{RGB}{152,78,163}

\path[draw=drawColor,line width= 0.4pt,line join=round,line cap=round,fill=fillColor] (341.27, 43.44) circle (  1.65);
\definecolor{drawColor}{RGB}{255,127,0}
\definecolor{fillColor}{RGB}{255,127,0}

\path[draw=drawColor,line width= 0.4pt,line join=round,line cap=round,fill=fillColor] (322.10,141.16) circle (  1.65);

\path[draw=drawColor,line width= 0.4pt,line join=round,line cap=round,fill=fillColor] (335.92, 80.19) circle (  1.65);

\path[draw=drawColor,line width= 0.4pt,line join=round,line cap=round,fill=fillColor] (359.09, 96.51) circle (  1.65);
\definecolor{drawColor}{RGB}{152,78,163}
\definecolor{fillColor}{RGB}{152,78,163}

\path[draw=drawColor,line width= 0.4pt
  \vskip -0.5cm
  \caption{Geometric interpretation. \textbf{Top}: input feature pairs
    $x_i,x_i'\in\RR^p$ are segments and arrows, colored using the
    labels $y_i\in\{-1,0,1\}$. The level curves of the ranking
    function $r(x)=||x||_2^2$ are grey, and differences
    $|r(x)-r(x')|\leq 1$ are considered insignificant
    ($y_i=0$). \textbf{Middle}: in the enlarged feature space, the
    ranking function is linear: $r(x)=w^\intercal
    \Phi(x)$. \textbf{Bottom}: two symmetric hyperplanes
    $w^\intercal[\Phi(x_i')-\Phi(x_i)]\in\{-1,1\}$ are used to
    classify the difference vectors.}
  \label{fig:norm-data}
\end{figure}

\subsection{SVMrank for comparing}
\label{sec:svmrank}
In this section we explain how to apply the existing SVMrank algorithm
of \citet{ranksvm} to a comparison data set.

The goal of the SVMrank algorithm is to learn a ranking function
$r:\RR^p \rightarrow \RR$. When $r(x)=w^\intercal x$ is linear, the
primal problem for some cost parameter $C\in\RR^+$ is the following
quadratic program (QP):
\begin{equation}
  \begin{aligned}
    \minimize_{w, \xi}\ \  & \frac 1 2 w^\intercal w 
    + C \sum_{i\in I_1\cup I_{-1}} \xi_i \\
    \text{subject to}\ \  & 
    \forall i\in I_1\cup I_{-1},\ \xi_i \geq 0,\\
&    \text{and }\xi_i \geq 1-w^\intercal(x_i'-x_i)y_i,
  \end{aligned}
  \label{eq:svmrank}
\end{equation}
where $I_y=\{i\mid y_i=y\}$ are the sets of indices for the different
labels. Note that the equality pairs $i\in I_0$ are not
used in the optimization problem.

\begin{figure*}[b!]
  \centering
  % Created by tikzDevice version 0.10.1 on 2017-12-13 15:58:48
% !TEX encoding = UTF-8 Unicode
\begin{tikzpicture}[x=1pt,y=1pt]
\definecolor{fillColor}{RGB}{255,255,255}
\path[use as bounding box,fill=fillColor,fill opacity=0.00] (0,0) rectangle (433.62,274.63);
\begin{scope}
\path[clip] (  0.00,  0.00) rectangle (433.62,274.63);
\definecolor{drawColor}{RGB}{255,255,255}
\definecolor{fillColor}{RGB}{255,255,255}

\path[draw=drawColor,line width= 0.6pt,line join=round,line cap=round,fill=fillColor] (  0.00,  0.00) rectangle (433.62,274.63);
\end{scope}
\begin{scope}
\path[clip] ( 34.75,250.04) rectangle (165.70,268.63);
\definecolor{drawColor}{gray}{0.50}
\definecolor{fillColor}{gray}{0.80}

\path[draw=drawColor,line width= 0.2pt,line join=round,line cap=round,fill=fillColor] ( 34.75,250.04) rectangle (165.70,268.63);
\definecolor{drawColor}{gray}{0.10}

\node[text=drawColor,anchor=base,inner sep=0pt, outer sep=0pt, scale=  0.87] at (100.23,256.04) {original $\mathbf x'-\mathbf x$};
\end{scope}
\begin{scope}
\path[clip] (165.70,250.04) rectangle (296.66,268.63);
\definecolor{drawColor}{gray}{0.50}
\definecolor{fillColor}{gray}{0.80}

\path[draw=drawColor,line width= 0.2pt,line join=round,line cap=round,fill=fillColor] (165.70,250.04) rectangle (296.66,268.63);
\definecolor{drawColor}{gray}{0.10}

\node[text=drawColor,anchor=base,inner sep=0pt, outer sep=0pt, scale=  0.87] at (231.18,256.04) {flipped $\mathbf{\tilde x'}-\mathbf{\tilde x}$};
\end{scope}
\begin{scope}
\path[clip] (296.66,250.04) rectangle (427.62,268.63);
\definecolor{drawColor}{gray}{0.50}
\definecolor{fillColor}{gray}{0.80}

\path[draw=drawColor,line width= 0.2pt,line join=round,line cap=round,fill=fillColor] (296.66,250.04) rectangle (427.62,268.63);
\definecolor{drawColor}{gray}{0.10}

\node[text=drawColor,anchor=base,inner sep=0pt, outer sep=0pt, scale=  0.87] at (362.14,256.04) {scaled and flipped $\mathbf{\tilde x'}-\mathbf{\tilde x}$};
\end{scope}
\begin{scope}
\path[clip] ( 34.75,105.04) rectangle (165.70,250.04);
\definecolor{fillColor}{RGB}{255,255,255}

\path[fill=fillColor] ( 34.75,105.04) rectangle (165.70,250.04);
\definecolor{drawColor}{gray}{0.60}

\path[draw=drawColor,line width= 1.1pt,line join=round] ( 44.25,274.63) -- (125.52,  0.00);

\path[draw=drawColor,line width= 1.1pt,line join=round] ( 98.39,274.63) -- (165.70, 47.12);

\path[draw=drawColor,line width= 1.1pt,dash pattern=on 1pt off 3pt ,line join=round] ( 86.29,274.63) -- (165.70,  6.24);

\path[draw=drawColor,line width= 1.1pt,dash pattern=on 1pt off 3pt ,line join=round] (110.48,274.63) -- (165.70, 88.01);

\path[draw=drawColor,line width= 1.1pt,dash pattern=on 1pt off 3pt ,line join=round] ( 34.75,265.87) -- (113.42,  0.00);

\path[draw=drawColor,line width= 1.1pt,dash pattern=on 1pt off 3pt ,line join=round] ( 56.35,274.63) -- (137.62,  0.00);
\definecolor{drawColor}{RGB}{152,78,163}

\path[draw=drawColor,line width= 0.4pt,line join=round,line cap=round] (105.90,149.19) circle (  2.28);

\path[draw=drawColor,line width= 0.4pt,line join=round,line cap=round] (118.10,134.98) circle (  2.28);

\path[draw=drawColor,line width= 0.4pt,line join=round,line cap=round] (120.55,111.63) circle (  2.28);

\path[draw=drawColor,line width= 0.4pt,line join=round,line cap=round] (128.40,123.74) circle (  2.28);

\path[draw=drawColor,line width= 0.4pt,line join=round,line cap=round] (103.16,148.80) circle (  2.28);

\path[draw=drawColor,line width= 0.4pt,line join=round,line cap=round] (127.24,113.29) circle (  2.28);

\path[draw=drawColor,line width= 0.4pt,line join=round,line cap=round] (113.73,159.23) circle (  2.28);

\path[draw=drawColor,line width= 0.4pt,line join=round,line cap=round] (103.54,152.23) circle (  2.28);

\path[draw=drawColor,line width= 0.4pt,line join=round,line cap=round] (107.06,121.71) circle (  2.28);
\definecolor{drawColor}{RGB}{255,127,0}

\path[draw=drawColor,line width= 0.4pt,line join=round,line cap=round] (139.89,214.37) circle (  2.28);

\path[draw=drawColor,line width= 0.4pt,line join=round,line cap=round] (141.88,221.51) circle (  2.28);
\definecolor{drawColor}{RGB}{77,175,74}

\path[draw=drawColor,line width= 0.4pt,line join=round,line cap=round] ( 62.62,135.64) circle (  2.28);

\path[draw=drawColor,line width= 0.4pt,line join=round,line cap=round] ( 63.38,160.55) circle (  2.28);

\path[draw=drawColor,line width= 0.4pt,line join=round,line cap=round] ( 61.92,133.12) circle (  2.28);
\definecolor{drawColor}{RGB}{255,127,0}

\path[draw=drawColor,line width= 0.4pt,line join=round,line cap=round] (138.91,222.04) circle (  2.28);

\path[draw=drawColor,line width= 0.4pt,line join=round,line cap=round] (141.64,185.30) circle (  2.28);
\definecolor{drawColor}{RGB}{77,175,74}

\path[draw=drawColor,line width= 0.4pt,line join=round,line cap=round] ( 64.43,155.18) circle (  2.28);

\path[draw=drawColor,line width= 0.4pt,line join=round,line cap=round] ( 55.18,124.74) circle (  2.28);
\definecolor{drawColor}{RGB}{255,127,0}

\path[draw=drawColor,line width= 0.4pt,line join=round,line cap=round] (140.38,196.34) circle (  2.28);
\definecolor{drawColor}{RGB}{152,78,163}

\path[draw=drawColor,line width= 0.4pt,line join=round,line cap=round] (112.42,162.72) circle (  2.28);

\path[draw=drawColor,line width= 0.4pt,line join=round,line cap=round] (114.87,139.38) circle (  2.28);

\path[draw=drawColor,line width= 0.4pt,line join=round,line cap=round] (120.44,149.20) -- (125.01,153.77);

\path[draw=drawColor,line width= 0.4pt,line join=round,line cap=round] (120.44,153.77) -- (125.01,149.20);

\path[draw=drawColor,line width= 0.4pt,line join=round,line cap=round] (119.50,151.49) -- (125.95,151.49);

\path[draw=drawColor,line width= 0.4pt,line join=round,line cap=round] (122.72,148.26) -- (122.72,154.72);

\path[draw=drawColor,line width= 0.4pt,line join=round,line cap=round] ( 97.48,176.55) circle (  2.28);

\path[draw=drawColor,line width= 0.4pt,line join=round,line cap=round] (121.57,141.04) circle (  2.28);

\path[draw=drawColor,line width= 0.4pt,line join=round,line cap=round] (108.05,186.98) circle (  2.28);

\path[draw=drawColor,line width= 0.4pt,line join=round,line cap=round] ( 97.86,179.97) circle (  2.28);

\path[draw=drawColor,line width= 0.4pt,line join=round,line cap=round] (101.38,149.45) circle (  2.28);
\definecolor{drawColor}{RGB}{77,175,74}

\path[draw=drawColor,line width= 0.4pt,line join=round,line cap=round] ( 54.88,167.25) circle (  2.28);

\path[draw=drawColor,line width= 0.4pt,line join=round,line cap=round] ( 52.90,160.11) circle (  2.28);
\definecolor{drawColor}{RGB}{255,127,0}

\path[draw=drawColor,line width= 0.4pt,line join=round,line cap=round] (143.51,190.49) circle (  2.28);
\definecolor{drawColor}{RGB}{77,175,74}

\path[draw=drawColor,line width= 0.4pt,line join=round,line cap=round] ( 55.42,186.01) -- ( 59.98,190.58);

\path[draw=drawColor,line width= 0.4pt,line join=round,line cap=round] ( 55.42,190.58) -- ( 59.98,186.01);

\path[draw=drawColor,line width= 0.4pt,line join=round,line cap=round] ( 54.47,188.30) -- ( 60.93,188.30);

\path[draw=drawColor,line width= 0.4pt,line join=round,line cap=round] ( 57.70,185.07) -- ( 57.70,191.52);

\path[draw=drawColor,line width= 0.4pt,line join=round,line cap=round] ( 56.25,160.87) circle (  2.28);
\definecolor{drawColor}{RGB}{255,127,0}

\path[draw=drawColor,line width= 0.4pt,line join=round,line cap=round] (144.59,194.30) circle (  2.28);

\path[draw=drawColor,line width= 0.4pt,line join=round,line cap=round] (147.32,157.55) circle (  2.28);
\definecolor{drawColor}{RGB}{77,175,74}

\path[draw=drawColor,line width= 0.4pt,line join=round,line cap=round] ( 58.75,182.93) circle (  2.28);
\definecolor{drawColor}{RGB}{255,127,0}

\path[draw=drawColor,line width= 0.4pt,line join=round,line cap=round] (150.95,201.39) circle (  2.28);

\path[draw=drawColor,line width= 0.4pt,line join=round,line cap=round] (146.06,168.59) circle (  2.28);
\definecolor{drawColor}{RGB}{152,78,163}

\path[draw=drawColor,line width= 0.4pt,line join=round,line cap=round] (102.68,153.59) circle (  2.28);

\path[draw=drawColor,line width= 0.4pt,line join=round,line cap=round] (110.53,165.70) circle (  2.28);

\path[draw=drawColor,line width= 0.4pt,line join=round,line cap=round] ( 85.28,190.76) circle (  2.28);

\path[draw=drawColor,line width= 0.4pt,line join=round,line cap=round] (109.37,155.25) circle (  2.28);

\path[draw=drawColor,line width= 0.4pt,line join=round,line cap=round] ( 95.85,201.19) circle (  2.28);

\path[draw=drawColor,line width= 0.4pt,line join=round,line cap=round] ( 85.67,194.19) circle (  2.28);

\path[draw=drawColor,line width= 0.4pt,line join=round,line cap=round] ( 86.90,161.38) -- ( 91.47,165.95);

\path[draw=drawColor,line width= 0.4pt,line join=round,line cap=round] ( 86.90,165.95) -- ( 91.47,161.38);

\path[draw=drawColor,line width= 0.4pt,line join=round,line cap=round] ( 85.96,163.67) -- ( 92.41,163.67);

\path[draw=drawColor,line width= 0.4pt,line join=round,line cap=round] ( 89.18,160.44) -- ( 89.18,166.90);
\definecolor{drawColor}{RGB}{255,127,0}

\path[draw=drawColor,line width= 0.4pt,line join=round,line cap=round] (157.76,172.41) circle (  2.28);
\definecolor{drawColor}{RGB}{77,175,74}

\path[draw=drawColor,line width= 0.4pt,line join=round,line cap=round] ( 40.70,174.33) circle (  2.28);

\path[draw=drawColor,line width= 0.4pt,line join=round,line cap=round] ( 44.75,177.60) circle (  2.28);

\path[draw=drawColor,line width= 0.4pt,line join=round,line cap=round] ( 45.50,202.51) circle (  2.28);
\definecolor{drawColor}{RGB}{255,127,0}

\path[draw=drawColor,line width= 0.4pt,line join=round,line cap=round] (156.40,178.79) circle (  2.28);

\path[draw=drawColor,line width= 0.4pt,line join=round,line cap=round] (156.78,180.08) circle (  2.28);
\definecolor{drawColor}{RGB}{77,175,74}

\node[text=drawColor,anchor=base,inner sep=0pt, outer sep=0pt, scale=  0.78] at ( 53.30,140.75) {$y_i=-1$};
\definecolor{drawColor}{RGB}{152,78,163}

\node[text=drawColor,anchor=base,inner sep=0pt, outer sep=0pt, scale=  0.78] at ( 87.88,216.14) {$y_i=0$};
\definecolor{drawColor}{RGB}{255,127,0}

\node[text=drawColor,anchor=base,inner sep=0pt, outer sep=0pt, scale=  0.78] at (144.68,227.22) {$y_i=1$};
\definecolor{drawColor}{gray}{0.50}

\path[draw=drawColor,line width= 0.6pt,line join=round,line cap=round] ( 34.75,105.04) rectangle (165.70,250.04);
\end{scope}
\begin{scope}
\path[clip] (165.70,105.04) rectangle (296.66,250.04);
\definecolor{fillColor}{RGB}{255,255,255}

\path[fill=fillColor] (165.70,105.04) rectangle (296.66,250.04);
\definecolor{drawColor}{gray}{0.60}

\path[draw=drawColor,line width= 1.1pt,line join=round] (175.21,274.63) -- (256.47,  0.00);

\path[draw=drawColor,line width= 1.1pt,line join=round] (229.34,274.63) -- (296.66, 47.12);

\path[draw=drawColor,line width= 1.1pt,dash pattern=on 1pt off 3pt ,line join=round] (217.24,274.63) -- (296.66,  6.24);

\path[draw=drawColor,line width= 1.1pt,dash pattern=on 1pt off 3pt ,line join=round] (241.44,274.63) -- (296.66, 88.01);

\path[draw=drawColor,line width= 1.1pt,dash pattern=on 1pt off 3pt ,line join=round] (165.70,265.87) -- (244.38,  0.00);

\path[draw=drawColor,line width= 1.1pt,dash pattern=on 1pt off 3pt ,line join=round] (187.31,274.63) -- (268.57,  0.00);
\definecolor{drawColor}{RGB}{0,0,0}

\path[draw=drawColor,line width= 0.9pt,dash pattern=on 1pt off 3pt ,line join=round] (259.36,111.63) -- (259.20,242.24);

\path[draw=drawColor,line width= 0.9pt,dash pattern=on 1pt off 3pt ,line join=round] (267.10,111.63) -- (266.94,242.24);

\path[draw=drawColor,line width= 0.9pt,line join=round] (263.23,111.63) -- (263.07,242.24);

\node[text=drawColor,anchor=base,inner sep=0pt, outer sep=0pt, scale=  0.78] at (280.58,240.53) {QP};
\definecolor{drawColor}{gray}{0.60}

\node[text=drawColor,anchor=base,inner sep=0pt, outer sep=0pt, scale=  0.78] at (181.79,174.01) {LP};
\definecolor{drawColor}{RGB}{152,78,163}

\path[draw=drawColor,line width= 0.4pt,line join=round,line cap=round] (236.86,149.19) circle (  2.28);

\path[draw=drawColor,line width= 0.4pt,line join=round,line cap=round] (249.06,134.98) circle (  2.28);

\path[draw=drawColor,line width= 0.4pt,line join=round,line cap=round] (251.51,111.63) circle (  2.28);

\path[draw=drawColor,line width= 0.4pt,line join=round,line cap=round] (259.36,123.74) circle (  2.28);

\path[draw=drawColor,line width= 0.4pt,line join=round,line cap=round] (234.12,148.80) circle (  2.28);

\path[draw=drawColor,line width= 0.4pt,line join=round,line cap=round] (258.20,113.29) circle (  2.28);

\path[draw=drawColor,line width= 0.4pt,line join=round,line cap=round] (244.68,159.23) circle (  2.28);

\path[draw=drawColor,line width= 0.4pt,line join=round,line cap=round] (234.50,152.23) circle (  2.28);

\path[draw=drawColor,line width= 0.4pt,line join=round,line cap=round] (238.02,121.71) circle (  2.28);
\definecolor{drawColor}{RGB}{255,127,0}

\path[draw=drawColor,line width= 0.4pt,line join=round,line cap=round] (270.85,214.37) circle (  2.28);

\path[draw=drawColor,line width= 0.4pt,line join=round,line cap=round] (272.84,221.51) circle (  2.28);

\path[draw=drawColor,line width= 0.4pt,line join=round,line cap=round] (268.79,218.24) circle (  2.28);

\path[draw=drawColor,line width= 0.4pt,line join=round,line cap=round] (268.03,193.33) circle (  2.28);

\path[draw=drawColor,line width= 0.4pt,line join=round,line cap=round] (269.48,220.75) circle (  2.28);

\path[draw=drawColor,line width= 0.4pt,line join=round,line cap=round] (269.87,222.04) circle (  2.28);

\path[draw=drawColor,line width= 0.4pt,line join=round,line cap=round] (272.60,185.30) circle (  2.28);

\path[draw=drawColor,line width= 0.4pt,line join=round,line cap=round] (266.98,198.69) circle (  2.28);

\path[draw=drawColor,line width= 0.4pt,line join=round,line cap=round] (276.23,229.13) circle (  2.28);

\path[draw=drawColor,line width= 0.4pt,line join=round,line cap=round] (271.34,196.34) circle (  2.28);
\definecolor{drawColor}{RGB}{152,78,163}

\path[draw=drawColor,line width= 0.4pt,line join=round,line cap=round] (243.38,162.72) circle (  2.28);

\path[draw=drawColor,line width= 0.4pt,line join=round,line cap=round] (245.83,139.38) circle (  2.28);

\path[draw=drawColor,line width= 0.4pt,line join=round,line cap=round] (251.40,149.20) -- (255.97,153.77);

\path[draw=drawColor,line width= 0.4pt,line join=round,line cap=round] (251.40,153.77) -- (255.97,149.20);

\path[draw=drawColor,line width= 0.4pt,line join=round,line cap=round] (250.45,151.49) -- (256.91,151.49);

\path[draw=drawColor,line width= 0.4pt,line join=round,line cap=round] (253.68,148.26) -- (253.68,154.72);

\path[draw=drawColor,line width= 0.4pt,line join=round,line cap=round] (228.44,176.55) circle (  2.28);

\path[draw=drawColor,line width= 0.4pt,line join=round,line cap=round] (252.52,141.04) circle (  2.28);

\path[draw=drawColor,line width= 0.4pt,line join=round,line cap=round] (239.01,186.98) circle (  2.28);

\path[draw=drawColor,line width= 0.4pt,line join=round,line cap=round] (228.82,179.97) circle (  2.28);

\path[draw=drawColor,line width= 0.4pt,line join=round,line cap=round] (232.34,149.45) circle (  2.28);
\definecolor{drawColor}{RGB}{255,127,0}

\path[draw=drawColor,line width= 0.4pt,line join=round,line cap=round] (276.53,186.63) circle (  2.28);

\path[draw=drawColor,line width= 0.4pt,line join=round,line cap=round] (278.51,193.76) circle (  2.28);

\path[draw=drawColor,line width= 0.4pt,line join=round,line cap=round] (274.46,190.49) circle (  2.28);

\path[draw=drawColor,line width= 0.4pt,line join=round,line cap=round] (271.42,163.30) -- (275.99,167.86);

\path[draw=drawColor,line width= 0.4pt,line join=round,line cap=round] (271.42,167.86) -- (275.99,163.30);

\path[draw=drawColor,line width= 0.4pt,line join=round,line cap=round] (270.48,165.58) -- (276.94,165.58);

\path[draw=drawColor,line width= 0.4pt,line join=round,line cap=round] (273.71,162.35) -- (273.71,168.81);

\path[draw=drawColor,line width= 0.4pt,line join=round,line cap=round] (275.16,193.01) circle (  2.28);

\path[draw=drawColor,line width= 0.4pt,line join=round,line cap=round] (275.54,194.30) circle (  2.28);

\path[draw=drawColor,line width= 0.4pt,line join=round,line cap=round] (278.28,157.55) circle (  2.28);

\path[draw=drawColor,line width= 0.4pt,line join=round,line cap=round] (272.66,170.95) circle (  2.28);

\path[draw=drawColor,line width= 0.4pt,line join=round,line cap=round] (281.91,201.39) circle (  2.28);

\path[draw=drawColor,line width= 0.4pt,line join=round,line cap=round] (277.01,168.59) circle (  2.28);
\definecolor{drawColor}{RGB}{152,78,163}

\path[draw=drawColor,line width= 0.4pt,line join=round,line cap=round] (233.63,153.59) circle (  2.28);

\path[draw=drawColor,line width= 0.4pt,line join=round,line cap=round] (241.49,165.70) circle (  2.28);

\path[draw=drawColor,line width= 0.4pt,line join=round,line cap=round] (216.24,190.76) circle (  2.28);

\path[draw=drawColor,line width= 0.4pt,line join=round,line cap=round] (240.33,155.25) circle (  2.28);

\path[draw=drawColor,line width= 0.4pt,line join=round,line cap=round] (226.81,201.19) circle (  2.28);

\path[draw=drawColor,line width= 0.4pt,line join=round,line cap=round] (216.62,194.19) circle (  2.28);

\path[draw=drawColor,line width= 0.4pt,line join=round,line cap=round] (217.86,161.38) -- (222.43,165.95);

\path[draw=drawColor,line width= 0.4pt,line join=round,line cap=round] (217.86,165.95) -- (222.43,161.38);

\path[draw=drawColor,line width= 0.4pt,line join=round,line cap=round] (216.91,163.67) -- (223.37,163.67);

\path[draw=drawColor,line width= 0.4pt,line join=round,line cap=round] (220.14,160.44) -- (220.14,166.90);
\definecolor{drawColor}{RGB}{255,127,0}

\path[draw=drawColor,line width= 0.4pt,line join=round,line cap=round] (288.72,172.41) circle (  2.28);

\path[draw=drawColor,line width= 0.4pt,line join=round,line cap=round] (290.71,179.55) circle (  2.28);

\path[draw=drawColor,line width= 0.4pt,line join=round,line cap=round] (286.66,176.28) circle (  2.28);

\path[draw=drawColor,line width= 0.4pt,line join=round,line cap=round] (285.90,151.37) circle (  2.28);

\path[draw=drawColor,line width= 0.4pt,line join=round,line cap=round] (287.36,178.79) circle (  2.28);

\path[draw=drawColor,line width= 0.4pt,line join=round,line cap=round] (287.74,180.08) circle (  2.28);
\definecolor{drawColor}{RGB}{152,78,163}

\path[draw=drawColor,line width= 0.4pt,line join=round,line cap=round] (225.51,204.68) circle (  2.28);

\path[draw=drawColor,line width= 0.4pt,line join=round,line cap=round] (213.31,218.90) circle (  2.28);

\path[draw=drawColor,line width= 0.4pt,line join=round,line cap=round] (210.86,242.24) circle (  2.28);

\path[draw=drawColor,line width= 0.4pt,line join=round,line cap=round] (203.01,230.14) circle (  2.28);

\path[draw=drawColor,line width= 0.4pt,line join=round,line cap=round] (228.25,205.07) circle (  2.28);

\path[draw=drawColor,line width= 0.4pt,line join=round,line cap=round] (204.17,240.59) circle (  2.28);

\path[draw=drawColor,line width= 0.4pt,line join=round,line cap=round] (217.68,194.64) circle (  2.28);

\path[draw=drawColor,line width= 0.4pt,line join=round,line cap=round] (227.87,201.65) circle (  2.28);

\path[draw=drawColor,line width= 0.4pt,line join=round,line cap=round] (224.35,232.17) circle (  2.28);

\path[draw=drawColor,line width= 0.4pt,line join=round,line cap=round] (218.99,191.15) circle (  2.28);

\path[draw=drawColor,line width= 0.4pt,line join=round,line cap=round] (216.54,214.50) circle (  2.28);

\path[draw=drawColor,line width= 0.4pt,line join=round,line cap=round] (206.40,200.11) -- (210.97,204.67);

\path[draw=drawColor,line width= 0.4pt,line join=round,line cap=round] (206.40,204.67) -- (210.97,200.11);

\path[draw=drawColor,line width= 0.4pt,line join=round,line cap=round] (205.46,202.39) -- (211.91,202.39);

\path[draw=drawColor,line width= 0.4pt,line join=round,line cap=round] (208.68,199.16) -- (208.68,205.62);

\path[draw=drawColor,line width= 0.4pt,line join=round,line cap=round] (233.93,177.33) circle (  2.28);

\path[draw=drawColor,line width= 0.4pt,line join=round,line cap=round] (209.84,212.84) circle (  2.28);

\path[draw=drawColor,line width= 0.4pt,line join=round,line cap=round] (223.36,166.90) circle (  2.28);

\path[draw=drawColor,line width= 0.4pt,line join=round,line cap=round] (233.55,173.90) circle (  2.28);

\path[draw=drawColor,line width= 0.4pt,line join=round,line cap=round] (230.03,204.42) circle (  2.28);

\path[draw=drawColor,line width= 0.4pt,line join=round,line cap=round] (228.73,200.28) circle (  2.28);

\path[draw=drawColor,line width= 0.4pt,line join=round,line cap=round] (220.88,188.17) circle (  2.28);

\path[draw=drawColor,line width= 0.4pt,line join=round,line cap=round] (246.12,163.11) circle (  2.28);

\path[draw=drawColor,line width= 0.4pt,line join=round,line cap=round] (222.04,198.62) circle (  2.28);

\path[draw=drawColor,line width= 0.4pt,line join=round,line cap=round] (235.56,152.68) circle (  2.28);

\path[draw=drawColor,line width= 0.4pt,line join=round,line cap=round] (245.74,159.69) circle (  2.28);

\path[draw=drawColor,line width= 0.4pt,line join=round,line cap=round] (239.94,187.93) -- (244.51,192.49);

\path[draw=drawColor,line width= 0.4pt,line join=round,line cap=round] (239.94,192.49) -- (244.51,187.93);

\path[draw=drawColor,line width= 0.4pt,line join=round,line cap=round] (239.00,190.21) -- (245.45,190.21);

\path[draw=drawColor,line width= 0.4pt,line join=round,line cap=round] (242.22,186.98) -- (242.22,193.44);
\definecolor{drawColor}{RGB}{0,0,0}
\definecolor{fillColor}{RGB}{0,0,0}

\path[draw=drawColor,line width= 0.4pt,line join=round,line cap=round,fill=fillColor] (259.36,123.74) circle (  1.52);

\path[draw=drawColor,line width= 0.4pt,line join=round,line cap=round,fill=fillColor] (266.98,198.69) circle (  1.52);
\definecolor{drawColor}{gray}{0.50}

\path[draw=drawColor,line width= 0.6pt,line join=round,line cap=round] (165.70,105.04) rectangle (296.66,250.04);
\end{scope}
\begin{scope}
\path[clip] (296.66,105.04) rectangle (427.62,250.04);
\definecolor{fillColor}{RGB}{255,255,255}

\path[fill=fillColor] (296.66,105.04) rectangle (427.62,250.04);
\definecolor{drawColor}{gray}{0.60}

\path[draw=drawColor,line width= 1.1pt,line join=round] (306.17,274.63) -- (387.43,  0.00);

\path[draw=drawColor,line width= 1.1pt,line join=round] (360.30,274.63) -- (427.62, 47.12);

\path[draw=drawColor,line width= 1.1pt,dash pattern=on 1pt off 3pt ,line join=round] (348.20,274.63) -- (427.62,  6.24);

\path[draw=drawColor,line width= 1.1pt,dash pattern=on 1pt off 3pt ,line join=round] (372.40,274.63) -- (427.62, 88.01);

\path[draw=drawColor,line width= 1.1pt,dash pattern=on 1pt off 3pt ,line join=round] (296.66,265.87) -- (375.33,  0.00);

\path[draw=drawColor,line width= 1.1pt,dash pattern=on 1pt off 3pt ,line join=round] (318.27,274.63) -- (399.53,  0.00);
\definecolor{drawColor}{RGB}{0,0,0}

\path[draw=drawColor,line width= 0.9pt,dash pattern=on 1pt off 3pt ,line join=round] (396.43,111.63) -- (357.78,242.24);

\path[draw=drawColor,line width= 0.9pt,dash pattern=on 1pt off 3pt ,line join=round] (420.63,111.63) -- (381.98,242.24);

\path[draw=drawColor,line width= 0.9pt,line join=round] (408.53,111.63) -- (369.88,242.24);
\definecolor{drawColor}{RGB}{152,78,163}

\path[draw=drawColor,line width= 0.4pt,line join=round,line cap=round] (367.82,149.19) circle (  2.28);

\path[draw=drawColor,line width= 0.4pt,line join=round,line cap=round] (380.01,134.98) circle (  2.28);

\path[draw=drawColor,line width= 0.4pt,line join=round,line cap=round] (382.46,111.63) circle (  2.28);

\path[draw=drawColor,line width= 0.4pt,line join=round,line cap=round] (390.32,123.74) circle (  2.28);

\path[draw=drawColor,line width= 0.4pt,line join=round,line cap=round] (365.07,148.80) circle (  2.28);

\path[draw=drawColor,line width= 0.4pt,line join=round,line cap=round] (389.16,113.29) circle (  2.28);

\path[draw=drawColor,line width= 0.4pt,line join=round,line cap=round] (375.64,159.23) circle (  2.28);

\path[draw=drawColor,line width= 0.4pt,line join=round,line cap=round] (365.45,152.23) circle (  2.28);

\path[draw=drawColor,line width= 0.4pt,line join=round,line cap=round] (368.97,121.71) circle (  2.28);
\definecolor{drawColor}{RGB}{255,127,0}

\path[draw=drawColor,line width= 0.4pt,line join=round,line cap=round] (401.81,214.37) circle (  2.28);

\path[draw=drawColor,line width= 0.4pt,line join=round,line cap=round] (403.79,221.51) circle (  2.28);

\path[draw=drawColor,line width= 0.4pt,line join=round,line cap=round] (399.75,218.24) circle (  2.28);

\path[draw=drawColor,line width= 0.4pt,line join=round,line cap=round] (398.99,193.33) circle (  2.28);

\path[draw=drawColor,line width= 0.4pt,line join=round,line cap=round] (400.44,220.75) circle (  2.28);

\path[draw=drawColor,line width= 0.4pt,line join=round,line cap=round] (400.83,222.04) circle (  2.28);

\path[draw=drawColor,line width= 0.4pt,line join=round,line cap=round] (403.56,185.30) circle (  2.28);

\path[draw=drawColor,line width= 0.4pt,line join=round,line cap=round] (397.94,198.69) circle (  2.28);

\path[draw=drawColor,line width= 0.4pt,line join=round,line cap=round] (407.19,229.13) circle (  2.28);

\path[draw=drawColor,line width= 0.4pt,line join=round,line cap=round] (402.29,196.34) circle (  2.28);
\definecolor{drawColor}{RGB}{152,78,163}

\path[draw=drawColor,line width= 0.4pt,line join=round,line cap=round] (374.34,162.72) circle (  2.28);

\path[draw=drawColor,line width= 0.4pt,line join=round,line cap=round] (376.79,139.38) circle (  2.28);

\path[draw=drawColor,line width= 0.4pt,line join=round,line cap=round] (382.36,149.20) -- (386.92,153.77);

\path[draw=drawColor,line width= 0.4pt,line join=round,line cap=round] (382.36,153.77) -- (386.92,149.20);

\path[draw=drawColor,line width= 0.4pt,line join=round,line cap=round] (381.41,151.49) -- (387.87,151.49);

\path[draw=drawColor,line width= 0.4pt,line join=round,line cap=round] (384.64,148.26) -- (384.64,154.72);

\path[draw=drawColor,line width= 0.4pt,line join=round,line cap=round] (359.40,176.55) circle (  2.28);

\path[draw=drawColor,line width= 0.4pt,line join=round,line cap=round] (383.48,141.04) circle (  2.28);

\path[draw=drawColor,line width= 0.4pt,line join=round,line cap=round] (369.97,186.98) circle (  2.28);

\path[draw=drawColor,line width= 0.4pt,line join=round,line cap=round] (359.78,179.97) circle (  2.28);

\path[draw=drawColor,line width= 0.4pt,line join=round,line cap=round] (363.30,149.45) circle (  2.28);
\definecolor{drawColor}{RGB}{255,127,0}

\path[draw=drawColor,line width= 0.4pt,line join=round,line cap=round] (407.48,186.63) circle (  2.28);

\path[draw=drawColor,line width= 0.4pt,line join=round,line cap=round] (409.47,193.76) circle (  2.28);

\path[draw=drawColor,line width= 0.4pt,line join=round,line cap=round] (405.42,190.49) circle (  2.28);

\path[draw=drawColor,line width= 0.4pt,line join=round,line cap=round] (402.38,163.30) -- (406.95,167.86);

\path[draw=drawColor,line width= 0.4pt,line join=round,line cap=round] (402.38,167.86) -- (406.95,163.30);

\path[draw=drawColor,line width= 0.4pt,line join=round,line cap=round] (401.44,165.58) -- (407.89,165.58);

\path[draw=drawColor,line width= 0.4pt,line join=round,line cap=round] (404.67,162.35) -- (404.67,168.81);

\path[draw=drawColor,line width= 0.4pt,line join=round,line cap=round] (406.12,193.01) circle (  2.28);

\path[draw=drawColor,line width= 0.4pt,line join=round,line cap=round] (406.50,194.30) circle (  2.28);

\path[draw=drawColor,line width= 0.4pt,line join=round,line cap=round] (409.23,157.55) circle (  2.28);

\path[draw=drawColor,line width= 0.4pt,line join=round,line cap=round] (403.61,170.95) circle (  2.28);

\path[draw=drawColor,line width= 0.4pt,line join=round,line cap=round] (412.87,201.39) circle (  2.28);

\path[draw=drawColor,line width= 0.4pt,line join=round,line cap=round] (407.97,168.59) circle (  2.28);
\definecolor{drawColor}{RGB}{152,78,163}

\path[draw=drawColor,line width= 0.4pt,line join=round,line cap=round] (364.59,153.59) circle (  2.28);

\path[draw=drawColor,line width= 0.4pt,line join=round,line cap=round] (372.44,165.70) circle (  2.28);

\path[draw=drawColor,line width= 0.4pt,line join=round,line cap=round] (347.20,190.76) circle (  2.28);

\path[draw=drawColor,line width= 0.4pt,line join=round,line cap=round] (371.29,155.25) circle (  2.28);

\path[draw=drawColor,line width= 0.4pt,line join=round,line cap=round] (357.77,201.19) circle (  2.28);

\path[draw=drawColor,line width= 0.4pt,line join=round,line cap=round] (347.58,194.19) circle (  2.28);

\path[draw=drawColor,line width= 0.4pt,line join=round,line cap=round] (348.82,161.38) -- (353.38,165.95);

\path[draw=drawColor,line width= 0.4pt,line join=round,line cap=round] (348.82,165.95) -- (353.38,161.38);

\path[draw=drawColor,line width= 0.4pt,line join=round,line cap=round] (347.87,163.67) -- (354.33,163.67);

\path[draw=drawColor,line width= 0.4pt,line join=round,line cap=round] (351.10,160.44) -- (351.10,166.90);
\definecolor{drawColor}{RGB}{255,127,0}

\path[draw=drawColor,line width= 0.4pt,line join=round,line cap=round] (419.68,172.41) circle (  2.28);

\path[draw=drawColor,line width= 0.4pt,line join=round,line cap=round] (421.67,179.55) circle (  2.28);

\path[draw=drawColor,line width= 0.4pt,line join=round,line cap=round] (417.62,176.28) circle (  2.28);

\path[draw=drawColor,line width= 0.4pt,line join=round,line cap=round] (416.86,151.37) circle (  2.28);

\path[draw=drawColor,line width= 0.4pt,line join=round,line cap=round] (418.31,178.79) circle (  2.28);

\path[draw=drawColor,line width= 0.4pt,line join=round,line cap=round] (418.70,180.08) circle (  2.28);
\definecolor{drawColor}{RGB}{152,78,163}

\path[draw=drawColor,line width= 0.4pt,line join=round,line cap=round] (356.47,204.68) circle (  2.28);

\path[draw=drawColor,line width= 0.4pt,line join=round,line cap=round] (344.27,218.90) circle (  2.28);

\path[draw=drawColor,line width= 0.4pt,line join=round,line cap=round] (341.82,242.24) circle (  2.28);

\path[draw=drawColor,line width= 0.4pt,line join=round,line cap=round] (333.97,230.14) circle (  2.28);

\path[draw=drawColor,line width= 0.4pt,line join=round,line cap=round] (359.21,205.07) circle (  2.28);

\path[draw=drawColor,line width= 0.4pt,line join=round,line cap=round] (335.12,240.59) circle (  2.28);

\path[draw=drawColor,line width= 0.4pt,line join=round,line cap=round] (348.64,194.64) circle (  2.28);

\path[draw=drawColor,line width= 0.4pt,line join=round,line cap=round] (358.83,201.65) circle (  2.28);

\path[draw=drawColor,line width= 0.4pt,line join=round,line cap=round] (355.31,232.17) circle (  2.28);

\path[draw=drawColor,line width= 0.4pt,line join=round,line cap=round] (349.94,191.15) circle (  2.28);

\path[draw=drawColor,line width= 0.4pt,line join=round,line cap=round] (347.50,214.50) circle (  2.28);

\path[draw=drawColor,line width= 0.4pt,line join=round,line cap=round] (337.36,200.11) -- (341.93,204.67);

\path[draw=drawColor,line width= 0.4pt,line join=round,line cap=round] (337.36,204.67) -- (341.93,200.11);

\path[draw=drawColor,line width= 0.4pt,line join=round,line cap=round] (336.41,202.39) -- (342.87,202.39);

\path[draw=drawColor,line width= 0.4pt,line join=round,line cap=round] (339.64,199.16) -- (339.64,205.62);

\path[draw=drawColor,line width= 0.4pt,line join=round,line cap=round] (364.89,177.33) circle (  2.28);

\path[draw=drawColor,line width= 0.4pt,line join=round,line cap=round] (340.80,212.84) circle (  2.28);

\path[draw=drawColor,line width= 0.4pt,line join=round,line cap=round] (354.32,166.90) circle (  2.28);

\path[draw=drawColor,line width= 0.4pt,line join=round,line cap=round] (364.50,173.90) circle (  2.28);

\path[draw=drawColor,line width= 0.4pt,line join=round,line cap=round] (360.99,204.42) circle (  2.28);

\path[draw=drawColor,line width= 0.4pt,line join=round,line cap=round] (359.69,200.28) circle (  2.28);

\path[draw=drawColor,line width= 0.4pt,line join=round,line cap=round] (351.84,188.17) circle (  2.28);

\path[draw=drawColor,line width= 0.4pt,line join=round,line cap=round] (377.08,163.11) circle (  2.28);

\path[draw=drawColor,line width= 0.4pt,line join=round,line cap=round] (353.00,198.62) circle (  2.28);

\path[draw=drawColor,line width= 0.4pt,line join=round,line cap=round] (366.51,152.68) circle (  2.28);

\path[draw=drawColor,line width= 0.4pt,line join=round,line cap=round] (376.70,159.69) circle (  2.28);

\path[draw=drawColor,line width= 0.4pt,line join=round,line cap=round] (370.90,187.93) -- (375.47,192.49);

\path[draw=drawColor,line width= 0.4pt,line join=round,line cap=round] (370.90,192.49) -- (375.47,187.93);

\path[draw=drawColor,line width= 0.4pt,line join=round,line cap=round] (369.95,190.21) -- (376.41,190.21);

\path[draw=drawColor,line width= 0.4pt,line join=round,line cap=round] (373.18,186.98) -- (373.18,193.44);
\definecolor{drawColor}{RGB}{0,0,0}
\definecolor{fillColor}{RGB}{0,0,0}

\path[draw=drawColor,line width= 0.4pt,line join=round,line cap=round,fill=fillColor] (384.64,151.49) circle (  1.52);

\path[draw=drawColor,line width= 0.4pt,line join=round,line cap=round,fill=fillColor] (404.67,165.58) circle (  1.52);

\path[draw=drawColor,line width= 0.4pt,line join=round,line cap=round,fill=fillColor] (373.18,190.21) circle (  1.52);
\definecolor{drawColor}{RGB}{152,78,163}

\node[text=drawColor,anchor=base,inner sep=0pt, outer sep=0pt, scale=  0.78] at (354.73,134.10) {$\tilde y_i=-1$};
\definecolor{drawColor}{RGB}{255,127,0}

\node[text=drawColor,anchor=base,inner sep=0pt, outer sep=0pt, scale=  0.78] at (406.59,240.53) {$\tilde y_i=1$};
\definecolor{drawColor}{gray}{0.50}

\path[draw=drawColor,line width= 0.6pt,line join=round,line cap=round] (296.66,105.04) rectangle (427.62,250.04);
\end{scope}
\begin{scope}
\path[clip] (  0.00,  0.00) rectangle (433.62,274.63);
\definecolor{drawColor}{RGB}{0,0,0}

\node[text=drawColor,anchor=base east,inner sep=0pt, outer sep=0pt, scale=  0.87] at ( 29.35,129.30) {-2};

\node[text=drawColor,anchor=base east,inner sep=0pt, outer sep=0pt, scale=  0.87] at ( 29.35,173.65) {0};

\node[text=drawColor,anchor=base east,inner sep=0pt, outer sep=0pt, scale=  0.87] at ( 29.35,217.99) {2};
\end{scope}
\begin{scope}
\path[clip] (  0.00,  0.00) rectangle (433.62,274.63);
\definecolor{drawColor}{RGB}{0,0,0}

\path[draw=drawColor,line width= 0.6pt,line join=round] ( 31.75,132.59) --
	( 34.75,132.59);

\path[draw=drawColor,line width= 0.6pt,line join=round] ( 31.75,176.94) --
	( 34.75,176.94);

\path[draw=drawColor,line width= 0.6pt,line join=round] ( 31.75,221.28) --
	( 34.75,221.28);
\end{scope}
\begin{scope}
\path[clip] (  0.00,  0.00) rectangle (433.62,274.63);
\definecolor{drawColor}{RGB}{0,0,0}

\path[draw=drawColor,line width= 0.6pt,line join=round] ( 50.83,102.04) --
	( 50.83,105.04);

\path[draw=drawColor,line width= 0.6pt,line join=round] ( 75.53,102.04) --
	( 75.53,105.04);

\path[draw=drawColor,line width= 0.6pt,line join=round] (100.23,102.04) --
	(100.23,105.04);

\path[draw=drawColor,line width= 0.6pt,line join=round] (124.92,102.04) --
	(124.92,105.04);

\path[draw=drawColor,line width= 0.6pt,line join=round] (149.62,102.04) --
	(149.62,105.04);
\end{scope}
\begin{scope}
\path[clip] (  0.00,  0.00) rectangle (433.62,274.63);
\definecolor{drawColor}{RGB}{0,0,0}

\node[text=drawColor,anchor=base,inner sep=0pt, outer sep=0pt, scale=  0.87] at ( 50.83, 93.06) {-200};

\node[text=drawColor,anchor=base,inner sep=0pt, outer sep=0pt, scale=  0.87] at ( 75.53, 93.06) {-100};

\node[text=drawColor,anchor=base,inner sep=0pt, outer sep=0pt, scale=  0.87] at (100.23, 93.06) {0};

\node[text=drawColor,anchor=base,inner sep=0pt, outer sep=0pt, scale=  0.87] at (124.92, 93.06) {100};

\node[text=drawColor,anchor=base,inner sep=0pt, outer sep=0pt, scale=  0.87] at (149.62, 93.06) {200};
\end{scope}
\begin{scope}
\path[clip] (  0.00,  0.00) rectangle (433.62,274.63);
\definecolor{drawColor}{RGB}{0,0,0}

\path[draw=drawColor,line width= 0.6pt,line join=round] (181.79,102.04) --
	(181.79,105.04);

\path[draw=drawColor,line width= 0.6pt,line join=round] (206.49,102.04) --
	(206.49,105.04);

\path[draw=drawColor,line width= 0.6pt,line join=round] (231.18,102.04) --
	(231.18,105.04);

\path[draw=drawColor,line width= 0.6pt,line join=round] (255.88,102.04) --
	(255.88,105.04);

\path[draw=drawColor,line width= 0.6pt,line join=round] (280.58,102.04) --
	(280.58,105.04);
\end{scope}
\begin{scope}
\path[clip] (  0.00,  0.00) rectangle (433.62,274.63);
\definecolor{drawColor}{RGB}{0,0,0}

\node[text=drawColor,anchor=base,inner sep=0pt, outer sep=0pt, scale=  0.87] at (181.79, 93.06) {-200};

\node[text=drawColor,anchor=base,inner sep=0pt, outer sep=0pt, scale=  0.87] at (206.49, 93.06) {-100};

\node[text=drawColor,anchor=base,inner sep=0pt, outer sep=0pt, scale=  0.87] at (231.18, 93.06) {0};

\node[text=drawColor,anchor=base,inner sep=0pt, outer sep=0pt, scale=  0.87] at (255.88, 93.06) {100};

\node[text=drawColor,anchor=base,inner sep=0pt, outer sep=0pt, scale=  0.87] at (280.58, 93.06) {200};
\end{scope}
\begin{scope}
\path[clip] (  0.00,  0.00) rectangle (433.62,274.63);
\definecolor{drawColor}{RGB}{0,0,0}

\path[draw=drawColor,line width= 0.6pt,line join=round] (312.75,102.04) --
	(312.75,105.04);

\path[draw=drawColor,line width= 0.6pt,line join=round] (337.45,102.04) --
	(337.45,105.04);

\path[draw=drawColor,line width= 0.6pt,line join=round] (362.14,102.04) --
	(362.14,105.04);

\path[draw=drawColor,line width= 0.6pt,line join=round] (386.84,102.04) --
	(386.84,105.04);

\path[draw=drawColor,line width= 0.6pt,line join=round] (411.53,102.04) --
	(411.53,105.04);
\end{scope}
\begin{scope}
\path[clip] (  0.00,  0.00) rectangle (433.62,274.63);
\definecolor{drawColor}{RGB}{0,0,0}

\node[text=drawColor,anchor=base,inner sep=0pt, outer sep=0pt, scale=  0.87] at (312.75, 93.06) {-2};

\node[text=drawColor,anchor=base,inner sep=0pt, outer sep=0pt, scale=  0.87] at (337.45, 93.06) {-1};

\node[text=drawColor,anchor=base,inner sep=0pt, outer sep=0pt, scale=  0.87] at (362.14, 93.06) {0};

\node[text=drawColor,anchor=base,inner sep=0pt, outer sep=0pt, scale=  0.87] at (386.84, 93.06) {1};

\node[text=drawColor,anchor=base,inner sep=0pt, outer sep=0pt, scale=  0.87] at (411.53, 93.06) {2};
\end{scope}
\begin{scope}
\path[clip] (  0.00,  0.00) rectangle (433.62,274.63);
\definecolor{drawColor}{RGB}{0,0,0}

\node[text=drawColor,anchor=base,inner sep=0pt, outer sep=0pt, scale=  1.09] at (231.18, 80.03) {difference feature 1};
\end{scope}
\begin{scope}
\path[clip] (  0.00,  0.00) rectangle (433.62,274.63);
\definecolor{drawColor}{RGB}{0,0,0}

\node[text=drawColor,rotate= 90.00,anchor=base,inner sep=0pt, outer sep=0pt, scale=  1.09] at ( 16.63,177.54) {difference feature 2};
\end{scope}
\begin{scope}
\path[clip] (  0.00,  0.00) rectangle (433.62,274.63);
\definecolor{fillColor}{RGB}{255,255,255}

\path[fill=fillColor] ( 95.74, 44.66) rectangle (366.62, 69.10);
\end{scope}
\begin{scope}
\path[clip] (  0.00,  0.00) rectangle (433.62,274.63);
\definecolor{drawColor}{RGB}{0,0,0}

\node[text=drawColor,anchor=base west,inner sep=0pt, outer sep=0pt, scale=  1.09] at (100.01, 52.77) {boundary};
\end{scope}
\begin{scope}
\path[clip] (  0.00,  0.00) rectangle (433.62,274.63);
\definecolor{drawColor}{gray}{0.80}
\definecolor{fillColor}{RGB}{255,255,255}

\path[draw=drawColor,line width= 0.6pt,line join=round,line cap=round,fill=fillColor] (153.48, 48.93) rectangle (169.38, 64.83);
\end{scope}
\begin{scope}
\path[clip] (  0.00,  0.00) rectangle (433.62,274.63);
\definecolor{drawColor}{gray}{0.60}

\path[draw=drawColor,line width= 1.1pt,line join=round] (153.48, 48.93) -- (169.38, 64.83);
\end{scope}
\begin{scope}
\path[clip] (  0.00,  0.00) rectangle (433.62,274.63);
\definecolor{drawColor}{RGB}{0,0,0}

\path[draw=drawColor,line width= 0.9pt,line join=round] (155.07, 56.88) -- (167.79, 56.88);
\end{scope}
\begin{scope}
\path[clip] (  0.00,  0.00) rectangle (433.62,274.63);
\definecolor{drawColor}{gray}{0.80}
\definecolor{fillColor}{RGB}{255,255,255}

\path[draw=drawColor,line width= 0.6pt,line join=round,line cap=round,fill=fillColor] (252.07, 48.93) rectangle (267.97, 64.83);
\end{scope}
\begin{scope}
\path[clip] (  0.00,  0.00) rectangle (433.62,274.63);
\definecolor{drawColor}{gray}{0.60}

\path[draw=drawColor,line width= 1.1pt,dash pattern=on 1pt off 3pt ,line join=round] (252.07, 48.93) -- (267.97, 64.83);
\end{scope}
\begin{scope}
\path[clip] (  0.00,  0.00) rectangle (433.62,274.63);
\definecolor{drawColor}{RGB}{0,0,0}

\path[draw=drawColor,line width= 0.9pt,dash pattern=on 1pt off 3pt ,line join=round] (253.66, 56.88) -- (266.38, 56.88);
\end{scope}
\begin{scope}
\path[clip] (  0.00,  0.00) rectangle (433.62,274.63);
\definecolor{drawColor}{RGB}{0,0,0}

\node[text=drawColor,anchor=base west,inner sep=0pt, outer sep=0pt, scale=  0.87] at (171.37, 53.59) {decision $r(\mathbf x)=\pm 1$};
\end{scope}
\begin{scope}
\path[clip] (  0.00,  0.00) rectangle (433.62,274.63);
\definecolor{drawColor}{RGB}{0,0,0}

\node[text=drawColor,anchor=base west,inner sep=0pt, outer sep=0pt, scale=  0.87] at (269.96, 53.59) {margin $r(\mathbf x)=\pm 1\pm\mu$};
\end{scope}
\begin{scope}
\path[clip] (  0.00,  0.00) rectangle (433.62,274.63);
\definecolor{fillColor}{RGB}{255,255,255}

\path[fill=fillColor] ( 55.34, 14.54) rectangle (407.02, 38.97);
\end{scope}
\begin{scope}
\path[clip] (  0.00,  0.00) rectangle (433.62,274.63);
\definecolor{drawColor}{RGB}{0,0,0}

\node[text=drawColor,anchor=base west,inner sep=0pt, outer sep=0pt, scale=  1.09] at ( 59.61, 22.64) {point};
\end{scope}
\begin{scope}
\path[clip] (  0.00,  0.00) rectangle (433.62,274.63);
\definecolor{drawColor}{gray}{0.80}
\definecolor{fillColor}{RGB}{255,255,255}

\path[draw=drawColor,line width= 0.6pt,line join=round,line cap=round,fill=fillColor] ( 90.65, 18.80) rectangle (106.54, 34.70);
\end{scope}
\begin{scope}
\path[clip] (  0.00,  0.00) rectangle (433.62,274.63);
\definecolor{drawColor}{RGB}{0,0,0}

\path[draw=drawColor,line width= 0.4pt,line join=round,line cap=round] ( 96.31, 24.47) -- (100.88, 29.04);

\path[draw=drawColor,line width= 0.4pt,line join=round,line cap=round] ( 96.31, 29.04) -- (100.88, 24.47);

\path[draw=drawColor,line width= 0.4pt,line join=round,line cap=round] ( 95.37, 26.75) -- (101.82, 26.75);

\path[draw=drawColor,line width= 0.4pt,line join=round,line cap=round] ( 98.60, 23.52) -- ( 98.60, 29.98);
\end{scope}
\begin{scope}
\path[clip] (  0.00,  0.00) rectangle (433.62,274.63);
\definecolor{drawColor}{gray}{0.80}
\definecolor{fillColor}{RGB}{255,255,255}

\path[draw=drawColor,line width= 0.6pt,line join=round,line cap=round,fill=fillColor] (195.04, 18.80) rectangle (210.94, 34.70);
\end{scope}
\begin{scope}
\path[clip] (  0.00,  0.00) rectangle (433.62,274.63);
\definecolor{drawColor}{RGB}{0,0,0}

\path[draw=drawColor,line width= 0.4pt,line join=round,line cap=round] (202.99, 26.75) circle (  2.28);
\end{scope}
\begin{scope}
\path[clip] (  0.00,  0.00) rectangle (433.62,274.63);
\definecolor{drawColor}{gray}{0.80}
\definecolor{fillColor}{RGB}{255,255,255}

\path[draw=drawColor,line width= 0.6pt,line join=round,line cap=round,fill=fillColor] (307.35, 18.80) rectangle (323.25, 34.70);
\end{scope}
\begin{scope}
\path[clip] (  0.00,  0.00) rectangle (433.62,274.63);
\definecolor{drawColor}{RGB}{0,0,0}
\definecolor{fillColor}{RGB}{0,0,0}

\path[draw=drawColor,line width= 0.4pt,line join=round,line cap=round,fill=fillColor] (315.30, 26.75) circle (  1.52);
\end{scope}
\begin{scope}
\path[clip] (  0.00,  0.00) rectangle (433.62,274.63);
\definecolor{drawColor}{RGB}{0,0,0}

\node[text=drawColor,anchor=base west,inner sep=0pt, outer sep=0pt, scale=  0.87] at (108.53, 23.46) {LP constraint active};
\end{scope}
\begin{scope}
\path[clip] (  0.00,  0.00) rectangle (433.62,274.63);
\definecolor{drawColor}{RGB}{0,0,0}

\node[text=drawColor,anchor=base west,inner sep=0pt, outer sep=0pt, scale=  0.87] at (212.92, 23.46) {LP constraint inactive};
\end{scope}
\begin{scope}
\path[clip] (  0.00,  0.00) rectangle (433.62,274.63);
\definecolor{drawColor}{RGB}{0,0,0}

\node[text=drawColor,anchor=base west,inner sep=0pt, outer sep=0pt, scale=  0.87] at (325.24, 23.46) {QP support vector};
\end{scope}
\end{tikzpicture}

  \vskip -1cm
  \caption{The separable LP and QP comparison problems. \textbf{Left}:
    the difference vectors $x'-x$ of the original data and the optimal
    solution to the LP (\ref{eq:max-margin-lp}). \textbf{Middle}: for
    the unscaled flipped data $\tilde x'-\tilde x$ (\ref{eq:tilde}),
    the LP is not the same as the QP
    (\ref{eq:max-margin-qp}). \textbf{Right}: in these scaled
    data, the QP is equivalent to the LP.}
  \label{fig:hard-margin}
\end{figure*}

After obtaining a weight vector $w\in\RR^p$ and ranking
function $r$ by solving (\ref{eq:svmrank}), we define a threshold
$\tau\in\RR^+$ and a thresholding function
$t_\tau:\RR\rightarrow\{-1,0,1\}$
\begin{equation}
  \label{eq:threshold}
  t_\tau(z) = 
  \begin{cases}
    -1 & \text{ if } z < -\tau, \\
    0 & \text{ if } |z| \leq \tau, \\
    1 & \text{ if } z > \tau. \\
  \end{cases}
\end{equation}
A comparison function $c_\tau:\RR^p\times \RR^p\rightarrow \{-1, 0,
1\}$ is defined as the thresholded difference of predicted ranks
\begin{equation}
  \label{eq:svmrank_c_t}
  c_\tau(x, x') = 
  t_\tau\big(
  r(x') - r(x)
  \big).
\end{equation}
We can then use grid search to estimate an optimal threshold $\hat
\tau$, by minimizing the zero-one loss $e$ with respect to all the
training pairs:
\begin{equation}
  \hat \tau = \argmin_{\tau}
  \sum_{i=1}^n
  e\left[ c_\tau(x_i, x_i'), y_i \right].
\end{equation}
However, there are two potential problems with the learned comparison
function $c_{\hat\tau}$. First, the equality pairs $i\in I_0$ are not
used to learn the weight vector $w$ in (\ref{eq:svmrank}). Second, the
threshold $\hat \tau$ is learned in a separate optimization step, which
may be suboptimal. In the next section, we propose a new algorithm
that fixes these potential problems.
%Supplementary File for Online Publication
\section{Support vector comparison machines}
\label{sec:svm-compare}

In this section we discuss new learning algorithms for comparison
problems. In all cases, we will first learn a ranking function
$r:\RR^p\rightarrow\RR$ and then a comparison function $c:\RR^p\times
\RR^p\rightarrow\{-1,0,1\}$:
\begin{equation}
  \label{eq:compare_general}
  c(x,x')=t_1\big( r(x') - r(x) \big)
\end{equation}
using the threshold
function $t_1:\RR\rightarrow\{-1,0,1\}$ (\ref{eq:threshold}). 
In other
words, any small difference in ranks $r(x')-r(x)\leq 1$ is considered
insignificant, and there are two decision boundaries
$r(x')-r(x)\in\{-1,1\}$.

\subsection{LP and QP for separable data}
\label{sec:lp-qp}

To illustrate the nature of the max-margin comparison problem, in this
section we assume that the training data are linearly separable. Later
in Section~\ref{sec:kernelized-qp}, we propose an algorithm for
learning a nonlinear function from non-separable data.

Using the following linear program (LP), we learn a linear ranking
function $r(x)=w^\intercal x$ that maximizes the geometric margin
$\mu$. As shown in the left panel of Figure~\ref{fig:hard-margin}, the
geometric margin $\mu$ is the smallest distance from any difference
vector $x_i'-x_i$ to a decision boundary $r(x)\in\{-1,1\}$. 
%Letting $I_y=\{i\mid y_i=y\}$ be the sets of indices for the different labels, the
The
 max margin LP is
%In the following linear program (LP), we consider learning a linear
%ranking function $r(x)=w^\intercal x$ that maximizes the margin $\mu$:
\begin{eqnarray}
  \label{eq:max-margin-lp}
  \maximize_{\mu\in\RR^+, w\in\RR^p}\ &&\hskip -0.5cm \mu \\
\nonumber
    \text{subject to}\ && \hskip -0.5cm \mu \leq 1-|w^\intercal (x_i' - x_i)|,\ 
    \forall\  i\in I_0,\\
\nonumber
    &&\hskip -0.5cm
\mu \leq -1 +  w^\intercal(x_i'-x_i)y_i,\ \forall\ i\in I_1\cup I_{-1}.
\end{eqnarray}
Note that finding a feasible point for this LP is a test of linear
separability. If there are no feasible points then the data are not
linearly separable. 
% The geometric interpretation of the LP margin is
% shown in the left panel of Figure~\ref{fig:hard-margin}. It is the
% distance from any difference vector $x_i'-x_i$ to its nearest decision
% boundary $r(x)\in\{-1,1\}$.

Another way to formulate the comparison problem is by first performing
a change of variables, and then solving a binary SVM QP. The idea is to
maximize the margin between significant differences $y_i\in\{-1,1\}$
and equality pairs $y_i=0$. Let $X_y,X_y'$ be the $|I_y|\times p$
matrices formed by all the pairs $i\in I_y$. We define a new
``flipped'' data set with $m=|I_1|+|I_{-1}|+2|I_0|$ pairs suitable for
training a binary SVM:
\begin{equation}
\label{eq:tilde}
  \tilde X = \left[
    \begin{array}{c}
      X_1 \\
      X_{-1}'\\
      X_0\\
      X_0'
    \end{array}
  \right],\ 
  \tilde X' = \left[
    \begin{array}{c}
      X_1' \\
      X_{-1}\\
      X_0'\\
      X_0
    \end{array}
  \right],\ 
  \tilde y = \left[
    \begin{array}{c}
      1_{|I_1|} \\
      1_{|I_{-1}|}\\
      -1_{|I_0|}\\
      -1_{|I_0|}
    \end{array}
  \right],
\end{equation}
where $1_n$ is an $n$-vector of ones, $\tilde X,\tilde
X'\in\RR^{m\times p}$ and $\tilde y\in\{-1,1\}^m$. Note that $\tilde
y_i=-1$ implies no significant difference between $\tilde x_i$ and
$\tilde x_i'$, and $\tilde y_i=1$ implies that $\tilde x_i'$ is better
than $\tilde x_i$. We then learn an affine function
$f(x)=\beta+u^\intercal x$ using a binary SVM QP (black lines in
middle and right panels of Figure~\ref{fig:hard-margin}):
\begin{eqnarray}
  \label{eq:max-margin-qp-tilde}
  \minimize_{u\in\RR^p, \beta\in\RR}\ &&\hskip -0.5cm
 u^\intercal u  \\
\nonumber    \text{subject to}\ &&\hskip -0.5cm 
    \tilde y_i (\beta + u^\intercal( \tilde x_i'-\tilde x_i) ) \geq 1,
    \ \forall i\in\{1,\dots,m\}.
\end{eqnarray}
Intuitively, the SVM QP learns a separator $f(x)=0$ between
significant difference pairs $\tilde y_i=1$ and insignificant
difference pairs $\tilde y_i=-1$. However, we want a comparison
function that predicts $c(x,x')\in\{-1,0,1\}$. So we use the next
Lemma to construct a ranking function $\hat r(x)= \hat w^\intercal x$
that is feasible for the original max margin comparison LP
(\ref{eq:max-margin-lp}).
\begin{lemma}
  Let $u\in\RR^p,\beta\in\RR$ be a solution of
  (\ref{eq:max-margin-qp-tilde}). Then $\hat \mu = -1/\beta$
  and $\hat w = -u/\beta$ are feasible for
  (\ref{eq:max-margin-lp}).
  \label{lemma:feasible}
\end{lemma}
\begin{proof}
  Begin by assuming that we want to find a ranking function $r(x)=\hat
  w^\intercal x = \gamma u^\intercal x$, where $\gamma\in\RR$ is a
  scaling constant.  
  Then consider that for all $x$ on the decision
  boundary, we have the following two equalities
  \begin{equation}
    \label{eq:dec-boundary-rank}
    r(x) = \hat w^\intercal x = 1\text{ and } f(x)=u^\intercal x + \beta = 0.
  \end{equation}
  Taken together, it is clear that $\gamma=-1/\beta$ and thus $\hat w
  = -u/\beta$. Consider for all $x$ on the margin we have the
  following two inequalities
  \begin{equation}
    \label{eq:margin-rank}
    r(x) = \hat w^\intercal x = 1+\hat\mu\text{ and } f(x)=u^\intercal x + \beta= 1.
  \end{equation}
  Taken together, these imply $\hat \mu=-1/\beta$. Now, by definition
  of the flipped data (\ref{eq:tilde}), we can re-write the max margin
  QP (\ref{eq:max-margin-qp-tilde}) as
\begin{eqnarray}
  \label{eq:max-margin-qp}
    \minimize_{u\in\RR^p, \beta\in\RR}\ &&\hskip -0.5cm u^\intercal u  \\
    \text{subject to}\ &&\hskip -0.5cm
    \nonumber \beta + |u^\intercal (x_i'-x_i)| \leq -1,\
    \forall\  i\in I_0,\\
    &&\hskip -0.5cm
\nonumber \beta + u^\intercal(x_i'-x_i)y_i \geq 1,\ \forall\ i\in I_1\cup I_{-1}.
\end{eqnarray}
By plugging in $\hat \mu,\hat w$ into the constraints of
(\ref{eq:max-margin-qp}), it is clear that $\hat \mu, \hat w$ are
feasible for the max margin comparison LP (\ref{eq:max-margin-lp}).
\end{proof}

One may also wonder: are $\hat \mu,\hat w$ optimal for the max margin
comparison LP? In general, the answer is no, and we give one
counterexample in the middle panel of
Figure~\ref{fig:hard-margin}. This is because the LP defines the
margin in terms of ranking function values, but the QP defines the
margin in terms of $||u||$, which depends on the scale of the inputs
$x,x'$. However, when the input variables are on the same scale, we
have observed that the solutions to the LP and QP are equivalent
(right panel of Figure~\ref{fig:hard-margin}).

Lemma~\ref{lemma:feasible} is very useful in practice. It means that a
ranking function $\hat r$ can be learned on comparison data by first
solving a standard binary SVM, and then transforming the solution. We
use this reult in the next section to build a general support vector
algorithm for comparison data.

\subsection{Kernelized QP for non-separable data}
\label{sec:kernelized-qp}
In this section, we assume the data are not separable, and want to
learn a nonlinear ranking function. We define a positive
definite kernel $\kappa:\RR^p\times \RR^p\rightarrow\RR$, which
implicitly defines an enlarged set of features $\Phi(x)$ (middle panel
of Figure~\ref{fig:norm-data}). As in (\ref{eq:max-margin-qp-tilde}),
we learn a function $f(x)=\beta + u^\intercal \Phi(x)$ which is affine
in the feature space. Let $\alpha,\alpha'\in\RR^m$ be coefficients
such that $u=\sum_{i=1}^m \alpha_i \Phi(\tilde x_i) + \alpha_i'
\Phi(\tilde x_i')$, and so we have
%the learned function is thus
 $f(x) =\beta + \sum_{i=1}^m \alpha_i \kappa(\tilde x_i, x) +
\alpha_i' \kappa(\tilde x_i', x)$. We then use Lemma~\ref{lemma:feasible} to
define the ranking function
\begin{equation}
  \label{eq:kernelized_r}
  r(x)= \frac{u^\intercal \Phi(x)}{-\beta} = %\frac{1}{-\beta}
  \sum_{i=1}^m
  \frac{\alpha_i \kappa(\tilde x_i, x) + \alpha_i'  \kappa(\tilde x_i', x)}
{-\beta}.
\end{equation}
% which implies the comparison function
% \begin{equation}
%   \label{eq:kernelized_c}
%   c(x, x') =
%   \begin{cases}
%     -1 & \text{if } r(x') - r(x) < -1, \\
%     0 & \text{if } |r(x') - r(x)| \leq 1, \\
%     1 & \text{if } r(x') - r(x) > 1. \\
%   \end{cases}
% \end{equation}

Let $K=[K_1\cdots K_m\ K_1'\cdots K_m']\in\RR^{2m\times 2m}$ be the
kernel matrix, where for all pairs $i\in\{1, \dots, m\}$, the kernel
vectors $K_i,K_i'\in\RR^{2m}$ are defined as
\begin{equation}
  K_i = \left[
    \begin{array}{c}
      \kappa(\tilde x_1, \tilde x_i)\\
      \vdots\\
      \kappa(\tilde x_m, \tilde x_i)\\
      \kappa(\tilde x_1', \tilde x_i)\\
      \vdots\\
      \kappa(\tilde x_m', \tilde x_i)
    \end{array}
  \right],\ 
  K_i' = \left[
    \begin{array}{c}
      \kappa(\tilde x_1, \tilde x_i')\\
      \vdots\\
      \kappa(\tilde x_m, \tilde x_i')\\
      \kappa(\tilde x_1', \tilde x_i')\\
      \vdots\\
      \kappa(\tilde x_m', \tilde x_i')
    \end{array}
  \right].
\end{equation}
Letting $a=[\alpha^\intercal\
\alpha'^\intercal]^\intercal\in\RR^{2m}$, the norm of the linear
function in the feature space is $w^\intercal w = a^\intercal K a$,
and we can write the primal soft-margin comparison QP for some
$C\in\RR^+$ as
\begin{equation}
  \begin{aligned}
      \minimize_{a\in\RR^{2m},\xi\in\RR^m,\beta\in\RR}\ \ & 
      \frac 1 2 a^\intercal K a + C\sum_{i=1}^m \xi_i \\
      \text{subject to}\ \ & 
      \text{for all $i\in\{1,\dots,m\}$, }
      \xi_i \geq 0,\\
      &\text{and }
      \xi_i \geq 1-\tilde y_i(\beta + a^\intercal (K_i'-K_i)).
  \end{aligned}
\end{equation}
Let $\lambda, v\in\RR^m$ be the dual variables, let $Y=\Diag(\tilde
y)$ be the diagonal matrix of $m$ labels. Then the Lagrangian can be
written as
\begin{equation}
  \label{eq:lagrangian}
  \mathcal L = \frac 1 2 a^\intercal K a + C\xi^\intercal 1_{m}\\
  -\lambda^\intercal \xi + v^\intercal(1_m - \tilde y\beta - Y M^\intercal K a - \xi),
\end{equation}
where $M=[-I_m \, I_m]^\intercal\in\{-1,0,1\}^{2m\times m}$. Solving
$\nabla_a \mathcal L=0$ results in the following stationary condition:
\begin{equation}
  \label{eq:stationarity}
  a = M Y v.
\end{equation}
The rest of the derivation of the dual comparison problem is the same
as for the standard binary SVM. The resulting dual QP is
\begin{equation}
  \begin{aligned}
    \label{eq:svm-dual}
    \minimize_{v\in\RR^m}\ \ &
    \frac 1 2 v^\intercal Y M^\intercal K M Y v - v^\intercal 1_m\\
    \text{subject to}\ \ &
    \sum_{i=1}^m v_i \tilde y_i = 0,\\
&    \text{for all $i\in\{1,\dots,m\}$, } 0\leq v_i\leq C,
  \end{aligned}
\end{equation}
which is equivalent to the dual problem of a standard binary SVM with
kernel $\tilde K = M^\intercal K M\in\RR^{m\times m}$ and labels
$\tilde y=\{-1,1\}^m$.


\begin{algorithm}[b!]
   \caption{\SVMcompare}
   \label{alg:SVMcompare}
\begin{algorithmic}
  \STATE {\bfseries Input:} cost $C\in\RR^+$, kernel
  $\kappa:\RR^p\times \RR^p \rightarrow \RR$, features $X,X'\in\RR^{n \times p}$,
  labels $y\in\{-1,0,1\}^n$.

  \STATE \makebox[0.5cm]{$\tilde X$} $\gets [$
  \makebox[1cm]{$X_1^\intercal$}
  \makebox[1cm]{$X_{-1}'^\intercal$}
  \makebox[1cm]{$X_0^\intercal$}
  \makebox[1cm]{$X_0'^\intercal$}
  $]^\intercal$.

  \STATE \makebox[0.5cm]{$\tilde X'$} $\gets [$
  \makebox[1cm]{$X_1'^\intercal$}
  \makebox[1cm]{$X_{-1}^\intercal$}
  \makebox[1cm]{$X_0'^\intercal$}
  \makebox[1cm]{$X_0^\intercal$}
  $]^\intercal$.

  \STATE \makebox[0.5cm]{$\tilde y$} $\gets [$
  \makebox[1cm]{$1_{|I_1|}^\intercal$}
  \makebox[1cm]{$1_{|I_{-1}|}^\intercal$}
  \makebox[1cm]{$-1_{|I_0|}^\intercal$}
  \makebox[1cm]{$-1_{|I_0|}^\intercal$}
  $]^\intercal$.

  \STATE $K \gets \proc{KernelMatrix}(\tilde X, \tilde X', \kappa)$.

  \STATE $M \gets [ -I_m\ I_m ]^\intercal$.

  \STATE $\tilde K \gets M^\intercal K M$.

  \STATE $v,\beta \gets \proc{SVMdual}(\tilde K, \tilde y, C)$.

  % \STATE $\alpha_i \gets $ 
  % \makebox[1cm][r]{$-\tilde y_i v_i$},
  % $\forall i\in\{1,\dots, m\}$.

  % \STATE $\alpha_i' \gets $
  % \makebox[1cm][r]{$\tilde y_i v_i$},
  % $\forall i\in\{1,\dots, m\}$.

  \STATE $\textbf{sv}\gets\{i: v_i>0\}$.
  
  \STATE {\bfseries Output:} Support vectors $\tilde
  X_{\textbf{sv}},\tilde X_{\textbf{sv}}'$, labels $\tilde y_{\textbf{sv}}$,
  bias~$\beta$, dual variables $v$.

   \end{algorithmic}
\end{algorithm}

So we can solve the dual comparison problem (\ref{eq:svm-dual}) using
any efficient SVM solver, such as libsvm \citep{libsvm}. We used the R
interface in the \texttt{kernlab} package of \citet{kernlab}, and our
code is available in the \texttt{tdhock/rankSVMcompare} package on Github.

After obtaining optimal dual variables $v\in\RR^m$ as the solution of
(\ref{eq:svm-dual}), the SVM solver also gives us the optimal bias
$\beta$ by analyzing the complementary slackness conditions.
The learned ranking function can be quickly evaluated since the
optimal $v$ is sparse. Let $\textbf{sv}=\{i: v_i > 0\}$ be the indices
of the support vectors. Since we need only $2|\textbf{sv}|$ kernel
evaluations, the ranking function (\ref{eq:kernelized_r}) becomes
\begin{equation}
  \label{eq:r_sv}
  r(x)= 
  \sum_{i\in \textbf{sv}}
  \tilde y_i v_i\left[ 
    \kappa(\tilde x_i, x)
    - \kappa(\tilde x_i', x)
  \right]/\beta.
\end{equation}
Note that for all $i\in\{1,\dots,m\}$, the optimal primal variables
$\alpha_i=-\tilde y_i v_i$ and $\alpha_i'=\tilde y_i v_i$ are
recovered using the stationary condition (\ref{eq:stationarity}). The
learned comparison function remains the same (\ref{eq:compare_general}).

The training procedure is summarized as
Algorithm~\ref{alg:SVMcompare}. There are two sub-routines:
\proc{KernelMatrix} computes the $2m\times 2m$ kernel matrix, and
\proc{SVMdual} solves the SVM dual QP. There are two hyper-parameters
to tune: the cost $C$ and the kernel $\kappa$. As with standard SVM
for binary classification, these parameters can be tuned by minimizing
the prediction error on a held-out validation set.

\begin{table}[b!]
  \centering
  \begin{tabular}{|a|c|c|c|}\hline
    \rowcolor{lightgray}
    \backslashbox{$\hat{y}$}{ $y$}
    &\textbf{-1}&\textbf{0}&\textbf{1}\\ \hline
    \textbf{-1}&0  & FP & Inversion   	\\ \hline 
    \textbf{0} &FN& 0& FN\\ \hline
    \textbf{1} & Inversion & FP &0	\\ \hline
  \end{tabular}
  % \cellcolor{pastelblue}
  \caption{We use the zero-one loss to evaluate a predicted label
    $\hat y$ given the true label $y$. False positives (FP) occur 
    when predicting a significant difference $\hat y\in\{-1,1\}$ 
    when there is none $y=0$, and False Negatives (FN) are the opposite.
  Inversions occur when predicting the opposite of the true label
  $\hat y = -y$.}
  \label{tab:evaluation}
\end{table}

\section{Results}
\label{sec:results}

The goal of our learning algorithm is to accurately predict a test set
of labeled pairs (\ref{eq:min_c}). We use the zero-one loss for
evaluating the learned comparison function on the test set
(Table~\ref{tab:evaluation}).

\subsection{Simulation: norms in 2d}

We used a simulation to test the proposed SVMcompare model and a
baseline ranking model that ignores the equality $y_i=0$ pairs,
SVMrank \citep{ranksvm}. The goal of our simulation is to demonstrate
that our model can perform better by learning from the equality
$y_i=0$ pairs, when there are few inequality $y_i\in\{-1,1\}$
pairs. We generated pairs $x_i,x_i'\in[-3,3]^2$ and noisy labels
$y_i=t_1[r(x'_i)-r(x_i)+\epsilon_i]$, where $t_1$ is the threshold
function (\ref{eq:threshold}), $r$ is the latent ranking function, and
$\epsilon_i\sim N(0, 1/4)$ is noise. We picked train, validation, and
test sets, each with $n/2$ equality pairs and $n/2$ inequality pairs,
for $n\in\{50,\dots,300\}$. We fit a $10\times 10$ grid of models to
the training set (cost parameter $C=10^{-3},\dots,10^3$, Gaussian
kernel width $2^{-7},\dots,2^4$), select the model with minimal
zero-one loss on the validation set, and then use the test set to
estimate the generalization ability of the selected model. For the
SVMrank model explained in Section~\ref{sec:svmrank}, equality pairs
are ignored when learning the ranking function, but used to learn a
threshold $\hat \tau$ via grid search for when to predict $c(x,x')=0$.

In Figure~\ref{fig:norm-level-curves} we show the training set for
$n=100$ pairs, and the level curves of the ranking functions learned
by the SVMrank and SVMcompare models. It is clear that SVMcompare
recovers a ranking function that is closer to the latent $r$,
especially in the case of the squared $\ell_1$ norm $r(x)=||x||_1^2$.

In Figure~\ref{fig:simulation-samples} the test error of the proposed
\SVMcompare\ model is clearly lower than that of the baseline SVMrank
model. The only exception is when $r(x)=||x||^2_2$, the methods
perform about the same.

%\subsection{Learning to rank data}

\begin{figure*}[b!]
  \centering
  % Created by tikzDevice version 0.10.1 on 2018-01-31 09:59:25
% !TEX encoding = UTF-8 Unicode
\begin{tikzpicture}[x=1pt,y=1pt]
\definecolor{fillColor}{RGB}{255,255,255}
\path[use as bounding box,fill=fillColor,fill opacity=0.00] (0,0) rectangle (505.89,144.54);
\begin{scope}
\path[clip] (  3.77,  0.00) rectangle (502.12,144.54);
\definecolor{drawColor}{RGB}{255,255,255}
\definecolor{fillColor}{RGB}{255,255,255}

\path[draw=drawColor,line width= 0.6pt,line join=round,line cap=round,fill=fillColor] (  3.77, -0.00) rectangle (502.12,144.54);
\end{scope}
\begin{scope}
\path[clip] ( 38.52,119.96) rectangle (125.07,138.54);
\definecolor{drawColor}{gray}{0.50}
\definecolor{fillColor}{gray}{0.80}

\path[draw=drawColor,line width= 0.2pt,line join=round,line cap=round,fill=fillColor] ( 38.52,119.96) rectangle (125.07,138.54);
\definecolor{drawColor}{gray}{0.10}

\node[text=drawColor,anchor=base,inner sep=0pt, outer sep=0pt, scale=  0.87] at ( 81.79,125.96) {training data};
\end{scope}
\begin{scope}
\path[clip] (125.07,119.96) rectangle (211.62,138.54);
\definecolor{drawColor}{gray}{0.50}
\definecolor{fillColor}{gray}{0.80}

\path[draw=drawColor,line width= 0.2pt,line join=round,line cap=round,fill=fillColor] (125.07,119.96) rectangle (211.62,138.54);
\definecolor{drawColor}{gray}{0.10}

\node[text=drawColor,anchor=base,inner sep=0pt, outer sep=0pt, scale=  0.87] at (168.34,125.96) {rank model};
\end{scope}
\begin{scope}
\path[clip] (211.62,119.96) rectangle (298.17,138.54);
\definecolor{drawColor}{gray}{0.50}
\definecolor{fillColor}{gray}{0.80}

\path[draw=drawColor,line width= 0.2pt,line join=round,line cap=round,fill=fillColor] (211.62,119.96) rectangle (298.17,138.54);
\definecolor{drawColor}{gray}{0.10}

\node[text=drawColor,anchor=base,inner sep=0pt, outer sep=0pt, scale=  0.87] at (254.89,125.96) {rank2 model};
\end{scope}
\begin{scope}
\path[clip] (298.17,119.96) rectangle (384.72,138.54);
\definecolor{drawColor}{gray}{0.50}
\definecolor{fillColor}{gray}{0.80}

\path[draw=drawColor,line width= 0.2pt,line join=round,line cap=round,fill=fillColor] (298.17,119.96) rectangle (384.72,138.54);
\definecolor{drawColor}{gray}{0.10}

\node[text=drawColor,anchor=base,inner sep=0pt, outer sep=0pt, scale=  0.87] at (341.44,125.96) {compare model};
\end{scope}
\begin{scope}
\path[clip] ( 38.52, 33.41) rectangle (125.07,119.96);
\definecolor{fillColor}{RGB}{255,255,255}

\path[fill=fillColor] ( 38.52, 33.41) rectangle (125.07,119.96);
\definecolor{drawColor}{RGB}{152,78,163}

\path[draw=drawColor,line width= 1.1pt,line join=round] ( 70.89, 73.20) -- ( 70.91, 77.90);

\path[draw=drawColor,line width= 1.1pt,line join=round] ( 83.74, 82.96) -- ( 87.10, 77.37);

\path[draw=drawColor,line width= 1.1pt,line join=round] ( 88.99, 73.40) -- ( 92.39, 76.25);

\path[draw=drawColor,line width= 1.1pt,line join=round] ( 74.98, 83.61) -- ( 74.77, 79.66);

\path[draw=drawColor,line width= 1.1pt,line join=round] ( 99.56, 75.54) -- ( 97.07, 79.82);

\path[draw=drawColor,line width= 1.1pt,line join=round] ( 86.42, 81.80) -- ( 80.39, 87.15);

\path[draw=drawColor,line width= 1.1pt,line join=round] ( 82.46, 60.37) -- ( 86.61, 60.95);

\path[draw=drawColor,line width= 1.1pt,line join=round] (103.91, 51.00) -- (107.46, 56.51);

\path[draw=drawColor,line width= 1.1pt,line join=round] ( 75.85, 84.46) -- ( 81.14, 86.11);

\path[draw=drawColor,line width= 1.1pt,line join=round] (101.05, 71.95) -- (102.62, 78.21);

\path[draw=drawColor,line width= 1.1pt,line join=round] ( 93.64, 80.84) -- ( 91.99, 87.33);

\path[draw=drawColor,line width= 1.1pt,line join=round] ( 66.73, 67.81) -- ( 60.43, 72.75);

\path[draw=drawColor,line width= 1.1pt,line join=round] ( 56.98, 90.03) -- ( 56.02, 88.25);

\path[draw=drawColor,line width= 1.1pt,line join=round] ( 61.61, 75.11) -- ( 58.68, 76.24);

\path[draw=drawColor,line width= 1.1pt,line join=round] ( 86.24, 59.95) -- ( 90.62, 65.27);

\path[draw=drawColor,line width= 1.1pt,line join=round] ( 74.65, 79.95) -- ( 78.74, 75.31);

\path[draw=drawColor,line width= 1.1pt,line join=round] ( 68.65, 80.76) -- ( 74.07, 84.79);

\path[draw=drawColor,line width= 1.1pt,line join=round] ( 84.95, 67.79) -- ( 89.15, 70.83);

\path[draw=drawColor,line width= 1.1pt,line join=round] ( 56.15,100.79) -- ( 59.63,103.03);

\path[draw=drawColor,line width= 1.1pt,line join=round] ( 80.19, 75.57) -- ( 81.50, 79.11);

\path[draw=drawColor,line width= 1.1pt,line join=round] ( 95.68, 61.44) -- ( 94.89, 64.11);

\path[draw=drawColor,line width= 1.1pt,line join=round] ( 76.34, 59.87) -- ( 82.19, 54.67);

\path[draw=drawColor,line width= 1.1pt,line join=round] ( 71.76, 69.25) -- ( 75.15, 69.67);

\path[draw=drawColor,line width= 1.1pt,line join=round] ( 92.91, 74.96) -- ( 89.12, 76.61);

\path[draw=drawColor,line width= 1.1pt,line join=round] ( 88.11, 68.99) -- ( 86.61, 74.37);

\path[draw=drawColor,line width= 1.1pt,line join=round] ( 72.42, 89.66) -- ( 74.61, 94.69);

\path[draw=drawColor,line width= 1.1pt,line join=round] ( 85.46, 58.67) -- ( 83.36, 55.46);

\path[draw=drawColor,line width= 1.1pt,line join=round] (107.22, 69.67) -- (108.54, 73.37);

\path[draw=drawColor,line width= 1.1pt,line join=round] ( 71.12, 81.79) -- ( 73.96, 84.44);

\path[draw=drawColor,line width= 1.1pt,line join=round] ( 72.44, 91.08) -- ( 72.16, 89.49);

\path[draw=drawColor,line width= 1.1pt,line join=round] ( 60.46, 97.08) -- ( 60.02, 98.07);

\path[draw=drawColor,line width= 1.1pt,line join=round] ( 81.26, 64.92) -- ( 79.14, 70.93);

\path[draw=drawColor,line width= 1.1pt,line join=round] ( 81.40, 69.30) -- ( 77.85, 74.30);

\path[draw=drawColor,line width= 1.1pt,line join=round] (107.68, 67.06) -- (109.88, 71.07);

\path[draw=drawColor,line width= 1.1pt,line join=round] ( 78.15, 52.10) -- ( 71.79, 56.04);

\path[draw=drawColor,line width= 1.1pt,line join=round] (106.35, 72.19) -- (109.25, 76.57);

\path[draw=drawColor,line width= 1.1pt,line join=round] ( 65.31, 91.14) -- ( 68.55, 96.78);

\path[draw=drawColor,line width= 1.1pt,line join=round] (106.50, 52.29) -- (101.25, 46.05);

\path[draw=drawColor,line width= 1.1pt,line join=round] ( 82.55, 88.69) -- ( 76.56, 84.54);

\path[draw=drawColor,line width= 1.1pt,line join=round] ( 83.11, 73.49) -- ( 82.23, 77.48);

\path[draw=drawColor,line width= 1.1pt,line join=round] ( 87.17, 83.94) -- ( 89.83, 83.76);

\path[draw=drawColor,line width= 1.1pt,line join=round] ( 76.18, 76.56) -- ( 78.36, 75.16);

\path[draw=drawColor,line width= 1.1pt,line join=round] ( 79.54, 99.79) -- ( 79.03, 98.47);

\path[draw=drawColor,line width= 1.1pt,line join=round] ( 88.07, 88.43) -- ( 85.18, 93.13);

\path[draw=drawColor,line width= 1.1pt,line join=round] ( 80.63, 81.15) -- ( 76.08, 75.32);

\path[draw=drawColor,line width= 1.1pt,line join=round] ( 74.85, 67.62) -- ( 80.28, 63.62);

\path[draw=drawColor,line width= 1.1pt,line join=round] ( 69.55, 88.68) -- ( 69.53, 89.08);

\path[draw=drawColor,line width= 1.1pt,line join=round] (105.92, 79.55) -- (101.42, 83.65);

\path[draw=drawColor,line width= 1.1pt,line join=round] ( 95.35, 82.37) -- ( 93.94, 82.09);

\path[draw=drawColor,line width= 1.1pt,line join=round] (107.88, 67.86) -- (110.70, 68.94);
\definecolor{drawColor}{RGB}{255,127,0}

\path[draw=drawColor,line width= 1.1pt,line join=round] ( 96.71, 76.21) -- (101.57, 73.50);

\path[draw=drawColor,line width= 1.1pt,line join=round] ( 97.96, 73.45) --
	(101.57, 73.50) --
	( 99.71, 76.60);

\path[draw=drawColor,line width= 1.1pt,line join=round] (105.90, 72.35) -- (104.10, 66.23);

\path[draw=drawColor,line width= 1.1pt,line join=round] (103.25, 69.74) --
	(104.10, 66.23) --
	(106.71, 68.72);

\path[draw=drawColor,line width= 1.1pt,line join=round] (100.55, 92.03) -- (106.90, 90.68);

\path[draw=drawColor,line width= 1.1pt,line join=round] (103.46, 89.56) --
	(106.90, 90.68) --
	(104.21, 93.10);

\path[draw=drawColor,line width= 1.1pt,line join=round] ( 91.90, 86.17) -- ( 93.22, 89.18);

\path[draw=drawColor,line width= 1.1pt,line join=round] ( 93.62, 85.59) --
	( 93.22, 89.18) --
	( 90.31, 87.03);

\path[draw=drawColor,line width= 1.1pt,line join=round] ( 76.97, 62.33) -- ( 70.89, 57.77);

\path[draw=drawColor,line width= 1.1pt,line join=round] ( 72.31, 61.10) --
	( 70.89, 57.77) --
	( 74.47, 58.20);

\path[draw=drawColor,line width= 1.1pt,line join=round] ( 67.79, 99.99) -- ( 62.13,101.94);

\path[draw=drawColor,line width= 1.1pt,line join=round] ( 65.67,102.63) --
	( 62.13,101.94) --
	( 64.50, 99.22);

\path[draw=drawColor,line width= 1.1pt,line join=round] (109.91, 93.26) -- (105.46, 99.36);

\path[draw=drawColor,line width= 1.1pt,line join=round] (108.76, 97.90) --
	(105.46, 99.36) --
	(105.84, 95.77);

\path[draw=drawColor,line width= 1.1pt,line join=round] ( 86.56, 90.67) -- ( 85.79, 96.10);

\path[draw=drawColor,line width= 1.1pt,line join=round] ( 88.02, 93.26) --
	( 85.79, 96.10) --
	( 84.44, 92.75);

\path[draw=drawColor,line width= 1.1pt,line join=round] ( 99.58, 97.44) -- ( 99.69,101.32);

\path[draw=drawColor,line width= 1.1pt,line join=round] (101.41, 98.14) --
	( 99.69,101.32) --
	( 97.80, 98.25);

\path[draw=drawColor,line width= 1.1pt,line join=round] ( 56.48, 64.81) -- ( 59.79, 58.38);

\path[draw=drawColor,line width= 1.1pt,line join=round] ( 56.76, 60.33) --
	( 59.79, 58.38) --
	( 59.97, 61.99);

\path[draw=drawColor,line width= 1.1pt,line join=round] ( 89.50, 65.22) -- ( 95.86, 59.15);

\path[draw=drawColor,line width= 1.1pt,line join=round] ( 92.34, 60.01) --
	( 95.86, 59.15) --
	( 94.84, 62.62);

\path[draw=drawColor,line width= 1.1pt,line join=round] ( 61.83, 95.33) -- ( 57.01, 99.91);

\path[draw=drawColor,line width= 1.1pt,line join=round] ( 60.52, 99.07) --
	( 57.01, 99.91) --
	( 58.03, 96.45);

\path[draw=drawColor,line width= 1.1pt,line join=round] ( 59.77, 61.33) -- ( 56.63, 59.89);

\path[draw=drawColor,line width= 1.1pt,line join=round] ( 58.72, 62.83) --
	( 56.63, 59.89) --
	( 60.22, 59.55);

\path[draw=drawColor,line width= 1.1pt,line join=round] ( 61.06, 60.60) -- ( 57.66, 55.71);

\path[draw=drawColor,line width= 1.1pt,line join=round] ( 57.96, 59.31) --
	( 57.66, 55.71) --
	( 60.93, 57.25);

\path[draw=drawColor,line width= 1.1pt,line join=round] (102.45, 90.19) -- (103.82, 93.28);

\path[draw=drawColor,line width= 1.1pt,line join=round] (104.20, 89.68) --
	(103.82, 93.28) --
	(100.90, 91.14);

\path[draw=drawColor,line width= 1.1pt,line join=round] ( 99.07, 70.55) -- ( 96.29, 65.22);

\path[draw=drawColor,line width= 1.1pt,line join=round] ( 96.13, 68.83) --
	( 96.29, 65.22) --
	( 99.34, 67.16);

\path[draw=drawColor,line width= 1.1pt,line join=round] (102.86, 76.98) -- (106.88, 74.33);

\path[draw=drawColor,line width= 1.1pt,line join=round] (103.27, 74.54) --
	(106.88, 74.33) --
	(105.26, 77.56);

\path[draw=drawColor,line width= 1.1pt,line join=round] ( 91.31, 68.42) -- ( 93.42, 62.53);

\path[draw=drawColor,line width= 1.1pt,line join=round] ( 90.66, 64.87) --
	( 93.42, 62.53) --
	( 94.07, 66.08);

\path[draw=drawColor,line width= 1.1pt,line join=round] ( 70.90, 67.22) -- ( 66.77, 65.78);

\path[draw=drawColor,line width= 1.1pt,line join=round] ( 69.13, 68.52) --
	( 66.77, 65.78) --
	( 70.32, 65.11);

\path[draw=drawColor,line width= 1.1pt,line join=round] ( 75.50, 58.02) -- ( 71.30, 56.77);

\path[draw=drawColor,line width= 1.1pt,line join=round] ( 73.78, 59.40) --
	( 71.30, 56.77) --
	( 74.81, 55.93);

\path[draw=drawColor,line width= 1.1pt,line join=round] ( 72.50, 88.61) -- ( 66.12, 89.66);

\path[draw=drawColor,line width= 1.1pt,line join=round] ( 69.50, 90.93) --
	( 66.12, 89.66) --
	( 68.91, 87.37);

\path[draw=drawColor,line width= 1.1pt,line join=round] ( 71.68, 49.32) -- ( 65.63, 52.28);

\path[draw=drawColor,line width= 1.1pt,line join=round] ( 69.23, 52.53) --
	( 65.63, 52.28) --
	( 67.65, 49.28);

\path[draw=drawColor,line width= 1.1pt,line join=round] ( 57.19, 92.04) -- ( 57.16, 95.52);

\path[draw=drawColor,line width= 1.1pt,line join=round] ( 58.99, 92.41) --
	( 57.16, 95.52) --
	( 55.38, 92.38);

\path[draw=drawColor,line width= 1.1pt,line join=round] ( 79.45,101.39) -- ( 73.20,100.62);

\path[draw=drawColor,line width= 1.1pt,line join=round] ( 76.09,102.79) --
	( 73.20,100.62) --
	( 76.53, 99.21);

\path[draw=drawColor,line width= 1.1pt,line join=round] ( 66.12, 90.28) -- ( 64.34, 91.88);

\path[draw=drawColor,line width= 1.1pt,line join=round] ( 67.88, 91.13) --
	( 64.34, 91.88) --
	( 65.46, 88.45);

\path[draw=drawColor,line width= 1.1pt,line join=round] (103.80, 96.54) -- (102.50,101.34);

\path[draw=drawColor,line width= 1.1pt,line join=round] (105.06, 98.80) --
	(102.50,101.34) --
	(101.57, 97.85);

\path[draw=drawColor,line width= 1.1pt,line join=round] (103.24, 88.29) -- (108.08, 94.72);

\path[draw=drawColor,line width= 1.1pt,line join=round] (107.64, 91.14) --
	(108.08, 94.72) --
	(104.76, 93.31);

\path[draw=drawColor,line width= 1.1pt,line join=round] ( 57.80, 88.25) -- ( 56.77, 91.29);

\path[draw=drawColor,line width= 1.1pt,line join=round] ( 59.48, 88.91) --
	( 56.77, 91.29) --
	( 56.06, 87.75);

\path[draw=drawColor,line width= 1.1pt,line join=round] ( 90.76, 60.62) -- ( 96.96, 56.81);

\path[draw=drawColor,line width= 1.1pt,line join=round] ( 93.35, 56.91) --
	( 96.96, 56.81) --
	( 95.24, 59.99);

\path[draw=drawColor,line width= 1.1pt,line join=round] ( 57.78, 58.17) -- ( 54.55, 53.13);

\path[draw=drawColor,line width= 1.1pt,line join=round] ( 54.72, 56.74) --
	( 54.55, 53.13) --
	( 57.76, 54.79);

\path[draw=drawColor,line width= 1.1pt,line join=round] ( 76.56, 59.54) -- ( 70.99, 60.73);

\path[draw=drawColor,line width= 1.1pt,line join=round] ( 74.43, 61.84) --
	( 70.99, 60.73) --
	( 73.68, 58.31);

\path[draw=drawColor,line width= 1.1pt,line join=round] ( 66.56, 60.60) -- ( 64.87, 54.40);

\path[draw=drawColor,line width= 1.1pt,line join=round] ( 63.95, 57.89) --
	( 64.87, 54.40) --
	( 67.44, 56.94);

\path[draw=drawColor,line width= 1.1pt,line join=round] ( 76.33, 53.18) -- ( 71.05, 47.80);

\path[draw=drawColor,line width= 1.1pt,line join=round] ( 71.95, 51.30) --
	( 71.05, 47.80) --
	( 74.53, 48.77);

\path[draw=drawColor,line width= 1.1pt,line join=round] ( 62.52, 73.44) -- ( 56.13, 75.77);

\path[draw=drawColor,line width= 1.1pt,line join=round] ( 59.68, 76.40) --
	( 56.13, 75.77) --
	( 58.45, 73.00);

\path[draw=drawColor,line width= 1.1pt,line join=round] ( 60.18, 63.26) -- ( 54.88, 56.90);

\path[draw=drawColor,line width= 1.1pt,line join=round] ( 55.50, 60.46) --
	( 54.88, 56.90) --
	( 58.27, 58.14);

\path[draw=drawColor,line width= 1.1pt,line join=round] ( 66.90, 83.88) -- ( 66.01, 89.70);

\path[draw=drawColor,line width= 1.1pt,line join=round] ( 68.27, 86.89) --
	( 66.01, 89.70) --
	( 64.70, 86.34);

\path[draw=drawColor,line width= 1.1pt,line join=round] ( 62.10, 56.97) -- ( 55.74, 52.32);

\path[draw=drawColor,line width= 1.1pt,line join=round] ( 57.20, 55.63) --
	( 55.74, 52.32) --
	( 59.33, 52.71);

\path[draw=drawColor,line width= 1.1pt,line join=round] ( 75.56, 61.02) -- ( 69.43, 62.69);

\path[draw=drawColor,line width= 1.1pt,line join=round] ( 72.92, 63.61) --
	( 69.43, 62.69) --
	( 71.97, 60.12);

\path[draw=drawColor,line width= 1.1pt,line join=round] ( 71.62, 68.70) -- ( 65.80, 65.27);

\path[draw=drawColor,line width= 1.1pt,line join=round] ( 67.58, 68.41) --
	( 65.80, 65.27) --
	( 69.41, 65.30);

\path[draw=drawColor,line width= 1.1pt,line join=round] ( 74.17, 76.28) -- ( 70.01, 73.47);

\path[draw=drawColor,line width= 1.1pt,line join=round] ( 71.59, 76.71) --
	( 70.01, 73.47) --
	( 73.61, 73.72);

\path[draw=drawColor,line width= 1.1pt,line join=round] (107.40, 75.41) -- (108.57, 72.25);

\path[draw=drawColor,line width= 1.1pt,line join=round] (105.79, 74.57) --
	(108.57, 72.25) --
	(109.18, 75.81);

\path[draw=drawColor,line width= 1.1pt,line join=round] ( 92.18, 76.13) -- ( 96.27, 82.35);

\path[draw=drawColor,line width= 1.1pt,line join=round] ( 96.06, 78.74) --
	( 96.27, 82.35) --
	( 93.04, 80.73);

\path[draw=drawColor,line width= 1.1pt,line join=round] (104.18, 60.88) -- (108.76, 55.88);

\path[draw=drawColor,line width= 1.1pt,line join=round] (105.31, 56.97) --
	(108.76, 55.88) --
	(107.98, 59.41);

\path[draw=drawColor,line width= 1.1pt,line join=round] ( 98.61, 71.61) -- (103.76, 70.82);

\path[draw=drawColor,line width= 1.1pt,line join=round] (100.39, 69.51) --
	(103.76, 70.82) --
	(100.94, 73.08);

\path[draw=drawColor,line width= 1.1pt,line join=round] ( 90.82, 98.44) -- ( 94.32,101.66);

\path[draw=drawColor,line width= 1.1pt,line join=round] ( 93.23, 98.21) --
	( 94.32,101.66) --
	( 90.79,100.87);

\path[draw=drawColor,line width= 1.1pt,line join=round] ( 68.57, 75.92) -- ( 63.58, 81.96);

\path[draw=drawColor,line width= 1.1pt,line join=round] ( 66.97, 80.70) --
	( 63.58, 81.96) --
	( 64.18, 78.40);

\path[draw=drawColor,line width= 1.1pt,line join=round] ( 64.27, 58.86) -- ( 63.97, 52.86);

\path[draw=drawColor,line width= 1.1pt,line join=round] ( 62.32, 56.07) --
	( 63.97, 52.86) --
	( 65.93, 55.89);

\path[draw=drawColor,line width= 1.1pt,line join=round] ( 89.09, 52.72) -- ( 90.05, 47.51);

\path[draw=drawColor,line width= 1.1pt,line join=round] ( 87.71, 50.26) --
	( 90.05, 47.51) --
	( 91.26, 50.92);

\path[draw=drawColor,line width= 1.1pt,line join=round] (104.09, 65.55) -- (106.70, 60.44);

\path[draw=drawColor,line width= 1.1pt,line join=round] (103.67, 62.41) --
	(106.70, 60.44) --
	(106.89, 64.05);

\path[draw=drawColor,line width= 1.1pt,line join=round] ( 68.28, 89.12) -- ( 62.06, 92.07);

\path[draw=drawColor,line width= 1.1pt,line join=round] ( 65.66, 92.36) --
	( 62.06, 92.07) --
	( 64.11, 89.09);
\definecolor{drawColor}{gray}{0.50}

\path[draw=drawColor,line width= 0.6pt,line join=round,line cap=round] ( 38.52, 33.41) rectangle (125.07,119.96);
\end{scope}
\begin{scope}
\path[clip] (125.07, 33.41) rectangle (211.62,119.96);
\definecolor{fillColor}{RGB}{255,255,255}

\path[fill=fillColor] (125.07, 33.41) rectangle (211.62,119.96);
\definecolor{drawColor}{RGB}{0,0,0}

\path[draw=drawColor,line width= 1.1pt,line join=round] (162.44, 71.43) --
	(161.61, 72.75) --
	(160.94, 74.72) --
	(160.83, 76.68) --
	(161.27, 78.65) --
	(162.25, 80.62) --
	(162.44, 80.86) --
	(164.04, 82.58) --
	(164.41, 82.88) --
	(166.37, 84.12) --
	(167.35, 84.55) --
	(168.34, 84.92) --
	(170.31, 85.34) --
	(172.28, 85.47) --
	(174.24, 85.29) --
	(176.21, 84.78) --
	(176.76, 84.55) --
	(178.18, 83.79) --
	(179.75, 82.58) --
	(180.14, 82.14) --
	(181.22, 80.62) --
	(181.96, 78.65) --
	(182.10, 76.68) --
	(181.71, 74.72) --
	(180.80, 72.75) --
	(180.14, 71.84) --
	(179.24, 70.78) --
	(178.18, 69.86) --
	(176.63, 68.82) --
	(176.21, 68.59) --
	(174.24, 67.85) --
	(172.28, 67.44) --
	(170.31, 67.38) --
	(168.34, 67.68) --
	(166.37, 68.33) --
	(165.42, 68.82) --
	(164.41, 69.48) --
	(162.94, 70.78) --
	(162.44, 71.43);

\path[draw=drawColor,line width= 1.1pt,line join=round] (156.54, 73.30) --
	(156.21, 74.72) --
	(156.13, 76.68) --
	(156.42, 78.65) --
	(156.54, 79.00) --
	(157.13, 80.62) --
	(158.25, 82.58) --
	(158.51, 82.93) --
	(159.91, 84.55) --
	(160.47, 85.08) --
	(162.28, 86.52) --
	(162.44, 86.63) --
	(164.41, 87.74) --
	(166.11, 88.49) --
	(166.37, 88.59) --
	(168.34, 89.16) --
	(170.31, 89.51) --
	(172.28, 89.64) --
	(174.24, 89.56) --
	(176.21, 89.27) --
	(178.18, 88.74) --
	(178.85, 88.49) --
	(180.14, 87.91) --
	(182.11, 86.76) --
	(182.45, 86.52) --
	(184.08, 85.06) --
	(184.55, 84.55) --
	(185.91, 82.58) --
	(186.04, 82.29) --
	(186.71, 80.62) --
	(187.09, 78.65) --
	(187.09, 76.68) --
	(186.74, 74.72) --
	(186.04, 72.79) --
	(186.03, 72.75) --
	(184.91, 70.78) --
	(184.08, 69.68) --
	(183.36, 68.82) --
	(182.11, 67.61) --
	(181.20, 66.85) --
	(180.14, 66.10) --
	(178.18, 64.97) --
	(177.99, 64.88) --
	(176.21, 64.18) --
	(174.24, 63.65) --
	(172.28, 63.37) --
	(170.31, 63.33) --
	(168.34, 63.55) --
	(166.37, 64.01) --
	(164.41, 64.73) --
	(164.10, 64.88) --
	(162.44, 65.81) --
	(160.97, 66.85) --
	(160.47, 67.27) --
	(158.95, 68.82) --
	(158.51, 69.40) --
	(157.58, 70.78) --
	(156.68, 72.75) --
	(156.54, 73.30);

\path[draw=drawColor,line width= 1.1pt,line join=round] (152.61, 73.09) --
	(152.28, 74.72) --
	(152.19, 76.68) --
	(152.42, 78.65) --
	(152.61, 79.34) --
	(152.97, 80.62) --
	(153.87, 82.58) --
	(154.57, 83.72) --
	(155.13, 84.55) --
	(156.54, 86.23) --
	(156.81, 86.52) --
	(158.51, 88.05) --
	(159.06, 88.49) --
	(160.47, 89.48) --
	(162.11, 90.45) --
	(162.44, 90.63) --
	(164.41, 91.52) --
	(166.37, 92.22) --
	(167.15, 92.42) --
	(168.34, 92.72) --
	(170.31, 93.03) --
	(172.28, 93.17) --
	(174.24, 93.15) --
	(176.21, 92.95) --
	(178.18, 92.57) --
	(178.72, 92.42) --
	(180.14, 91.99) --
	(182.11, 91.20) --
	(183.59, 90.45) --
	(184.08, 90.17) --
	(186.04, 88.81) --
	(186.44, 88.49) --
	(188.01, 86.96) --
	(188.41, 86.52) --
	(189.80, 84.55) --
	(189.98, 84.20) --
	(190.73, 82.58) --
	(191.29, 80.62) --
	(191.52, 78.65) --
	(191.44, 76.68) --
	(191.06, 74.72) --
	(190.40, 72.75) --
	(189.98, 71.88) --
	(189.43, 70.78) --
	(188.16, 68.82) --
	(188.01, 68.62) --
	(186.52, 66.85) --
	(186.04, 66.37) --
	(184.43, 64.88) --
	(184.08, 64.60) --
	(182.11, 63.20) --
	(181.64, 62.91) --
	(180.14, 62.09) --
	(178.18, 61.23) --
	(177.35, 60.95) --
	(176.21, 60.59) --
	(174.24, 60.17) --
	(172.28, 59.96) --
	(170.31, 59.95) --
	(168.34, 60.14) --
	(166.37, 60.53) --
	(165.01, 60.95) --
	(164.41, 61.14) --
	(162.44, 61.99) --
	(160.73, 62.91) --
	(160.47, 63.07) --
	(158.51, 64.47) --
	(158.01, 64.88) --
	(156.54, 66.29) --
	(156.01, 66.85) --
	(154.57, 68.72) --
	(154.50, 68.82) --
	(153.42, 70.78) --
	(152.68, 72.75) --
	(152.61, 73.09);

\path[draw=drawColor,line width= 1.1pt,line join=round] (148.67, 74.31) --
	(148.59, 74.72) --
	(148.49, 76.68) --
	(148.67, 78.62) --
	(148.67, 78.65) --
	(149.14, 80.62) --
	(149.90, 82.58) --
	(150.64, 83.98) --
	(150.95, 84.55) --
	(152.34, 86.52) --
	(152.61, 86.84) --
	(154.09, 88.49) --
	(154.57, 88.96) --
	(156.27, 90.45) --
	(156.54, 90.66) --
	(158.51, 92.07) --
	(159.07, 92.42) --
	(160.47, 93.24) --
	(162.44, 94.20) --
	(162.89, 94.39) --
	(164.41, 94.99) --
	(166.37, 95.60) --
	(168.34, 96.06) --
	(170.28, 96.35) --
	(170.31, 96.36) --
	(172.28, 96.51) --
	(174.24, 96.52) --
	(176.21, 96.37) --
	(176.33, 96.35) --
	(178.18, 96.08) --
	(180.14, 95.64) --
	(182.11, 95.03) --
	(183.77, 94.39) --
	(184.08, 94.26) --
	(186.04, 93.28) --
	(187.49, 92.42) --
	(188.01, 92.07) --
	(189.98, 90.57) --
	(190.11, 90.45) --
	(191.95, 88.61) --
	(192.06, 88.49) --
	(193.52, 86.52) --
	(193.91, 85.80) --
	(194.57, 84.55) --
	(195.28, 82.58) --
	(195.68, 80.62) --
	(195.80, 78.65) --
	(195.65, 76.68) --
	(195.24, 74.72) --
	(194.59, 72.75) --
	(193.91, 71.29) --
	(193.68, 70.78) --
	(192.52, 68.82) --
	(191.95, 68.01) --
	(191.09, 66.85) --
	(189.98, 65.56) --
	(189.35, 64.88) --
	(188.01, 63.59) --
	(187.25, 62.91) --
	(186.04, 61.96) --
	(184.63, 60.95) --
	(184.08, 60.58) --
	(182.11, 59.45) --
	(181.12, 58.98) --
	(180.14, 58.53) --
	(178.18, 57.81) --
	(176.21, 57.28) --
	(174.68, 57.01) --
	(174.24, 56.94) --
	(172.28, 56.78) --
	(170.31, 56.79) --
	(168.34, 56.98) --
	(168.18, 57.01) --
	(166.37, 57.36) --
	(164.41, 57.91) --
	(162.44, 58.65) --
	(161.72, 58.98) --
	(160.47, 59.59) --
	(158.51, 60.73) --
	(158.19, 60.95) --
	(156.54, 62.14) --
	(155.59, 62.91) --
	(154.57, 63.84) --
	(153.53, 64.88) --
	(152.61, 65.97) --
	(151.90, 66.85) --
	(150.64, 68.77) --
	(150.61, 68.82) --
	(149.65, 70.78) --
	(148.98, 72.75) --
	(148.67, 74.31);

\path[draw=drawColor,line width= 1.1pt,line join=round] (146.70, 69.11) --
	(145.93, 70.78) --
	(145.29, 72.75) --
	(144.91, 74.72) --
	(144.79, 76.68) --
	(144.92, 78.65) --
	(145.32, 80.62) --
	(145.98, 82.58) --
	(146.70, 84.11) --
	(146.92, 84.55) --
	(148.12, 86.52) --
	(148.67, 87.26) --
	(149.61, 88.49) --
	(150.64, 89.63) --
	(151.42, 90.45) --
	(152.61, 91.57) --
	(153.59, 92.42) --
	(154.57, 93.20) --
	(156.22, 94.39) --
	(156.54, 94.61) --
	(158.51, 95.81) --
	(159.53, 96.35) --
	(160.47, 96.85) --
	(162.44, 97.72) --
	(164.11, 98.32) --
	(164.41, 98.43) --
	(166.37, 99.01) --
	(168.34, 99.44) --
	(170.31, 99.73) --
	(172.28, 99.89) --
	(174.24, 99.91) --
	(176.21, 99.81) --
	(178.18, 99.58) --
	(180.14, 99.23) --
	(182.11, 98.73) --
	(183.41, 98.32) --
	(184.08, 98.11) --
	(186.04, 97.34) --
	(188.01, 96.42) --
	(188.13, 96.35) --
	(189.98, 95.31) --
	(191.41, 94.39) --
	(191.95, 94.00) --
	(193.91, 92.43) --
	(193.93, 92.42) --
	(195.88, 90.47) --
	(195.89, 90.45) --
	(197.43, 88.49) --
	(197.85, 87.80) --
	(198.60, 86.52) --
	(199.44, 84.55) --
	(199.81, 83.19) --
	(199.98, 82.58) --
	(200.26, 80.62) --
	(200.29, 78.65) --
	(200.06, 76.68) --
	(199.81, 75.64) --
	(199.61, 74.72) --
	(198.94, 72.75) --
	(198.03, 70.78) --
	(197.85, 70.46) --
	(196.93, 68.82) --
	(195.88, 67.28) --
	(195.58, 66.85) --
	(194.01, 64.88) --
	(193.91, 64.77) --
	(192.19, 62.91) --
	(191.95, 62.68) --
	(190.05, 60.95) --
	(189.98, 60.88) --
	(188.01, 59.33) --
	(187.52, 58.98) --
	(186.04, 57.98) --
	(184.40, 57.01) --
	(184.08, 56.83) --
	(182.11, 55.86) --
	(180.14, 55.07) --
	(180.06, 55.05) --
	(178.18, 54.45) --
	(176.21, 54.00) --
	(174.24, 53.71) --
	(172.28, 53.60) --
	(170.31, 53.65) --
	(168.34, 53.85) --
	(166.37, 54.23) --
	(164.41, 54.76) --
	(163.61, 55.05) --
	(162.44, 55.46) --
	(160.47, 56.32) --
	(159.14, 57.01) --
	(158.51, 57.35) --
	(156.54, 58.56) --
	(155.94, 58.98) --
	(154.57, 59.98) --
	(153.38, 60.95) --
	(152.61, 61.63) --
	(151.26, 62.91) --
	(150.64, 63.58) --
	(149.49, 64.88) --
	(148.67, 65.96) --
	(148.02, 66.85) --
	(146.84, 68.82) --
	(146.70, 69.11);

\path[draw=drawColor,line width= 1.1pt,line join=round] (142.77, 69.19) --
	(142.03, 70.78) --
	(141.38, 72.75) --
	(140.98, 74.72) --
	(140.83, 76.68) --
	(140.92, 78.65) --
	(141.26, 80.62) --
	(141.85, 82.58) --
	(142.71, 84.55) --
	(142.77, 84.66) --
	(143.79, 86.52) --
	(144.74, 87.91) --
	(145.13, 88.49) --
	(146.70, 90.43) --
	(146.73, 90.45) --
	(148.58, 92.42) --
	(148.67, 92.51) --
	(150.64, 94.31) --
	(150.73, 94.39) --
	(152.61, 95.90) --
	(153.22, 96.35) --
	(154.57, 97.31) --
	(156.17, 98.32) --
	(156.54, 98.56) --
	(158.51, 99.66) --
	(159.82,100.29) --
	(160.47,100.61) --
	(162.44,101.43) --
	(164.41,102.10) --
	(164.99,102.26) --
	(166.37,102.65) --
	(168.34,103.08) --
	(170.31,103.37) --
	(172.28,103.54) --
	(174.24,103.58) --
	(176.21,103.51) --
	(178.18,103.33) --
	(180.14,103.03) --
	(182.11,102.61) --
	(183.42,102.26) --
	(184.08,102.08) --
	(186.04,101.46) --
	(188.01,100.71) --
	(188.97,100.29) --
	(189.98, 99.84) --
	(191.95, 98.84) --
	(192.86, 98.32) --
	(193.91, 97.69) --
	(195.88, 96.38) --
	(195.91, 96.35) --
	(197.85, 94.84) --
	(198.38, 94.39) --
	(199.81, 93.01) --
	(200.39, 92.42) --
	(201.78, 90.73) --
	(202.01, 90.45) --
	(203.29, 88.49) --
	(203.75, 87.54) --
	(204.25, 86.52) --
	(204.93, 84.55) --
	(205.33, 82.58) --
	(205.48, 80.62) --
	(205.40, 78.65) --
	(205.10, 76.68) --
	(204.58, 74.72) --
	(203.83, 72.75) --
	(203.75, 72.59) --
	(202.91, 70.78) --
	(201.78, 68.86) --
	(201.76, 68.82) --
	(200.45, 66.85) --
	(199.81, 66.03) --
	(198.94, 64.88) --
	(197.85, 63.63) --
	(197.22, 62.91) --
	(195.88, 61.53) --
	(195.30, 60.95) --
	(193.91, 59.66) --
	(193.14, 58.98) --
	(191.95, 57.99) --
	(190.68, 57.01) --
	(189.98, 56.50) --
	(188.01, 55.17) --
	(187.80, 55.05) --
	(186.04, 54.00) --
	(184.24, 53.08) --
	(184.08, 53.00) --
	(182.11, 52.13) --
	(180.14, 51.45) --
	(178.90, 51.11) --
	(178.18, 50.91) --
	(176.21, 50.53) --
	(174.24, 50.30) --
	(172.28, 50.24) --
	(170.31, 50.33) --
	(168.34, 50.57) --
	(166.37, 50.97) --
	(165.88, 51.11) --
	(164.41, 51.51) --
	(162.44, 52.19) --
	(160.47, 53.04) --
	(160.39, 53.08) --
	(158.51, 54.01) --
	(156.69, 55.05) --
	(156.54, 55.14) --
	(154.57, 56.41) --
	(153.72, 57.01) --
	(152.61, 57.84) --
	(151.19, 58.98) --
	(150.64, 59.46) --
	(149.01, 60.95) --
	(148.67, 61.28) --
	(147.11, 62.91) --
	(146.70, 63.39) --
	(145.47, 64.88) --
	(144.74, 65.92) --
	(144.08, 66.85) --
	(142.94, 68.82) --
	(142.77, 69.19);

\path[draw=drawColor,line width= 1.1pt,line join=round] (207.68, 68.67) --
	(206.39, 66.85) --
	(205.72, 66.03) --
	(204.82, 64.88) --
	(203.75, 63.69) --
	(203.08, 62.91) --
	(201.78, 61.57) --
	(201.19, 60.95) --
	(199.81, 59.62) --
	(199.15, 58.98) --
	(197.85, 57.81) --
	(196.94, 57.01) --
	(195.88, 56.13) --
	(194.53, 55.05) --
	(193.91, 54.57) --
	(191.95, 53.15) --
	(191.85, 53.08) --
	(189.98, 51.83) --
	(188.78, 51.11) --
	(188.01, 50.65) --
	(186.04, 49.61) --
	(185.02, 49.15) --
	(184.08, 48.70) --
	(182.11, 47.94) --
	(180.14, 47.35) --
	(179.39, 47.18) --
	(178.18, 46.89) --
	(176.21, 46.57) --
	(174.24, 46.42) --
	(172.28, 46.43) --
	(170.31, 46.58) --
	(168.34, 46.89) --
	(167.15, 47.18) --
	(166.37, 47.35) --
	(164.41, 47.92) --
	(162.44, 48.64) --
	(161.31, 49.15) --
	(160.47, 49.50) --
	(158.51, 50.47) --
	(157.37, 51.11) --
	(156.54, 51.57) --
	(154.57, 52.79) --
	(154.14, 53.08) --
	(152.61, 54.12) --
	(151.35, 55.05) --
	(150.64, 55.58) --
	(148.85, 57.01) --
	(148.67, 57.17) --
	(146.70, 58.89) --
	(146.61, 58.98) --
	(144.74, 60.79) --
	(144.58, 60.95) --
	(142.77, 62.91) --
	(142.77, 62.91) --
	(141.15, 64.88) --
	(140.80, 65.38) --
	(139.75, 66.85) --
	(138.84, 68.40) --
	(138.58, 68.82) --
	(137.62, 70.78) --
	(136.94, 72.75) --
	(136.87, 73.05) --
	(136.46, 74.72) --
	(136.24, 76.68) --
	(136.29, 78.65) --
	(136.59, 80.62) --
	(136.87, 81.60) --
	(137.13, 82.58) --
	(137.91, 84.55) --
	(138.84, 86.30) --
	(138.95, 86.52) --
	(140.19, 88.49) --
	(140.80, 89.31) --
	(141.66, 90.45) --
	(142.77, 91.75) --
	(143.35, 92.42) --
	(144.74, 93.87) --
	(145.25, 94.39) --
	(146.70, 95.76) --
	(147.37, 96.35) --
	(148.67, 97.47) --
	(149.74, 98.32) --
	(150.64, 99.03) --
	(152.39,100.29) --
	(152.61,100.44) --
	(154.57,101.75) --
	(155.43,102.26) --
	(156.54,102.93) --
	(158.51,103.96) --
	(159.07,104.22) --
	(160.47,104.90) --
	(162.44,105.69) --
	(163.95,106.19) --
	(164.41,106.35) --
	(166.37,106.92) --
	(168.34,107.34) --
	(170.31,107.64) --
	(172.28,107.81) --
	(174.24,107.87) --
	(176.21,107.83) --
	(178.18,107.68) --
	(180.14,107.43) --
	(182.11,107.08) --
	(184.08,106.62) --
	(185.60,106.19) --
	(186.04,106.07) --
	(188.01,105.46) --
	(189.98,104.75) --
	(191.25,104.22) --
	(191.95,103.94) --
	(193.91,103.06) --
	(195.52,102.26) --
	(195.88,102.07) --
	(197.85,100.99) --
	(199.03,100.29) --
	(199.81, 99.79) --
	(201.78, 98.46) --
	(201.98, 98.32) --
	(203.75, 96.96) --
	(204.49, 96.35) --
	(205.72, 95.22) --
	(206.59, 94.39) --
	(207.68, 93.15);

\path[draw=drawColor,line width= 1.1pt,line join=round] (207.68, 59.34) --
	(207.28, 58.98) --
	(205.72, 57.67) --
	(204.94, 57.01) --
	(203.75, 56.04) --
	(202.54, 55.05) --
	(201.78, 54.44) --
	(200.06, 53.08) --
	(199.81, 52.89) --
	(197.85, 51.37) --
	(197.50, 51.11) --
	(195.88, 49.90) --
	(194.79, 49.15) --
	(193.91, 48.52) --
	(191.95, 47.25) --
	(191.83, 47.18) --
	(189.98, 46.02) --
	(188.45, 45.21) --
	(188.01, 44.96) --
	(186.04, 43.98) --
	(184.18, 43.24) --
	(184.08, 43.20) --
	(182.11, 42.50) --
	(180.14, 41.98) --
	(178.18, 41.64) --
	(176.21, 41.46) --
	(174.24, 41.43) --
	(172.28, 41.55) --
	(170.31, 41.82) --
	(168.34, 42.25) --
	(166.37, 42.84) --
	(165.32, 43.24) --
	(164.41, 43.55) --
	(162.44, 44.37) --
	(160.76, 45.21) --
	(160.47, 45.34) --
	(158.51, 46.37) --
	(157.18, 47.18) --
	(156.54, 47.54) --
	(154.57, 48.78) --
	(154.05, 49.15) --
	(152.61, 50.10) --
	(151.21, 51.11) --
	(150.64, 51.52) --
	(148.67, 53.01) --
	(148.58, 53.08) --
	(146.70, 54.56) --
	(146.11, 55.05) --
	(144.74, 56.20) --
	(143.79, 57.01) --
	(142.77, 57.93) --
	(141.61, 58.98) --
	(140.80, 59.77) --
	(139.57, 60.95) --
	(138.84, 61.73) --
	(137.68, 62.91) --
	(136.87, 63.87) --
	(135.96, 64.88) --
	(134.90, 66.26) --
	(134.42, 66.85) --
	(133.09, 68.82) --
	(132.93, 69.11) --
	(131.94, 70.78) --
	(131.09, 72.75) --
	(130.97, 73.19) --
	(130.46, 74.72) --
	(130.10, 76.68) --
	(130.04, 78.65) --
	(130.28, 80.62) --
	(130.81, 82.58) --
	(130.97, 82.95) --
	(131.57, 84.55) --
	(132.62, 86.52) --
	(132.93, 86.99) --
	(133.87, 88.49) --
	(134.90, 89.83) --
	(135.36, 90.45) --
	(136.87, 92.22) --
	(137.04, 92.42) --
	(138.84, 94.35) --
	(138.87, 94.39) --
	(140.80, 96.31) --
	(140.85, 96.35) --
	(142.77, 98.15) --
	(142.96, 98.32) --
	(144.74, 99.90) --
	(145.20,100.29) --
	(146.70,101.56) --
	(147.59,102.26) --
	(148.67,103.13) --
	(150.15,104.22) --
	(150.64,104.60) --
	(152.61,105.99) --
	(152.92,106.19) --
	(154.57,107.32) --
	(155.99,108.16) --
	(156.54,108.51) --
	(158.51,109.61) --
	(159.58,110.12) --
	(160.47,110.59) --
	(162.44,111.46) --
	(164.24,112.09) --
	(164.41,112.16) --
	(166.37,112.80) --
	(168.34,113.27) --
	(170.31,113.60) --
	(172.28,113.79) --
	(174.24,113.88) --
	(176.21,113.85) --
	(178.18,113.72) --
	(180.14,113.50) --
	(182.11,113.18) --
	(184.08,112.77) --
	(186.04,112.26) --
	(186.62,112.09) --
	(188.01,111.73) --
	(189.98,111.14) --
	(191.95,110.47) --
	(192.87,110.12) --
	(193.91,109.77) --
	(195.88,109.04) --
	(197.85,108.23) --
	(198.03,108.16) --
	(199.81,107.45) --
	(201.78,106.60) --
	(202.70,106.19) --
	(203.75,105.73) --
	(205.72,104.84) --
	(207.02,104.22) --
	(207.68,103.90);

\path[draw=drawColor,line width= 1.1pt,line join=round] (163.31, 37.34) --
	(162.44, 37.88) --
	(160.49, 39.31) --
	(160.47, 39.32) --
	(158.51, 40.70) --
	(157.79, 41.28) --
	(156.54, 42.14) --
	(155.14, 43.24) --
	(154.57, 43.64) --
	(152.61, 45.16) --
	(152.55, 45.21) --
	(150.64, 46.65) --
	(149.99, 47.18) --
	(148.67, 48.18) --
	(147.48, 49.15) --
	(146.70, 49.74) --
	(144.99, 51.11) --
	(144.74, 51.31) --
	(142.77, 52.88) --
	(142.52, 53.08) --
	(140.80, 54.45) --
	(140.05, 55.05) --
	(138.84, 56.03) --
	(137.60, 57.01) --
	(136.87, 57.62) --
	(135.17, 58.98) --
	(134.90, 59.21) --
	(132.93, 60.79) --
	(132.72, 60.95) --
	(130.97, 62.37) --
	(130.21, 62.91) --
	(129.00, 63.93);

\path[draw=drawColor,line width= 1.1pt,line join=round] (129.00, 95.48) --
	(130.18, 96.35) --
	(130.97, 96.90) --
	(132.84, 98.32) --
	(132.93, 98.39) --
	(134.90, 99.97) --
	(135.28,100.29) --
	(136.87,101.60) --
	(137.64,102.26) --
	(138.84,103.29) --
	(139.92,104.22) --
	(140.80,105.02) --
	(142.12,106.19) --
	(142.77,106.81) --
	(144.23,108.16) --
	(144.74,108.67) --
	(146.25,110.12) --
	(146.70,110.62) --
	(148.15,112.09) --
	(148.67,112.72) --
	(149.90,114.06) --
	(150.64,115.05) --
	(151.45,116.02);

\path[draw=drawColor,line width= 1.1pt,line join=round] (195.86, 37.34) --
	(195.88, 37.36) --
	(197.84, 39.31) --
	(197.85, 39.32) --
	(199.81, 41.01) --
	(200.10, 41.28) --
	(201.78, 42.56) --
	(202.64, 43.24) --
	(203.75, 44.00) --
	(205.51, 45.21) --
	(205.72, 45.34) --
	(207.68, 46.36);

\path[draw=drawColor,line width= 1.1pt,line join=round] (129.00, 54.86) --
	(130.97, 54.10) --
	(132.93, 53.18) --
	(133.12, 53.08) --
	(134.90, 52.06) --
	(136.43, 51.11) --
	(136.87, 50.84) --
	(138.84, 49.47) --
	(139.28, 49.15) --
	(140.80, 47.97) --
	(141.81, 47.18) --
	(142.77, 46.36) --
	(144.12, 45.21) --
	(144.74, 44.63) --
	(146.22, 43.24) --
	(146.70, 42.72) --
	(148.08, 41.28) --
	(148.67, 40.52) --
	(149.67, 39.31) --
	(150.64, 37.81) --
	(150.96, 37.34);
\definecolor{drawColor}{RGB}{0,0,0}

\path[draw=drawColor,draw opacity=0.33,line width= 1.1pt,line join=round] (140.80, 74.91) --
	(139.02, 76.68) --
	(140.80, 78.46) --
	(140.99, 78.65) --
	(142.77, 80.43) --
	(142.96, 80.62) --
	(144.74, 82.40) --
	(144.93, 82.58) --
	(146.70, 84.36) --
	(146.89, 84.55) --
	(148.67, 86.33) --
	(148.86, 86.52) --
	(150.64, 88.30) --
	(150.83, 88.49) --
	(152.61, 90.26) --
	(152.79, 90.45) --
	(154.57, 92.23) --
	(154.76, 92.42) --
	(156.54, 94.20) --
	(156.73, 94.39) --
	(158.51, 96.17) --
	(158.69, 96.35) --
	(160.47, 98.13) --
	(160.66, 98.32) --
	(162.44,100.10) --
	(162.63,100.29) --
	(164.41,102.07) --
	(164.60,102.26) --
	(166.37,104.03) --
	(166.56,104.22) --
	(168.34,106.00) --
	(170.12,104.22) --
	(170.31,104.03) --
	(172.09,102.26) --
	(172.28,102.07) --
	(174.05,100.29) --
	(174.24,100.10) --
	(176.02, 98.32) --
	(176.21, 98.13) --
	(177.99, 96.35) --
	(178.18, 96.17) --
	(179.96, 94.39) --
	(180.14, 94.20) --
	(181.92, 92.42) --
	(182.11, 92.23) --
	(183.89, 90.45) --
	(184.08, 90.26) --
	(185.86, 88.49) --
	(186.04, 88.30) --
	(187.82, 86.52) --
	(188.01, 86.33) --
	(189.79, 84.55) --
	(189.98, 84.36) --
	(191.76, 82.58) --
	(191.95, 82.40) --
	(193.72, 80.62) --
	(193.91, 80.43) --
	(195.69, 78.65) --
	(195.88, 78.46) --
	(197.66, 76.68) --
	(195.88, 74.91) --
	(195.69, 74.72) --
	(193.91, 72.94) --
	(193.72, 72.75) --
	(191.95, 70.97) --
	(191.76, 70.78) --
	(189.98, 69.00) --
	(189.79, 68.82) --
	(188.01, 67.04) --
	(187.82, 66.85) --
	(186.04, 65.07) --
	(185.86, 64.88) --
	(184.08, 63.10) --
	(183.89, 62.91) --
	(182.11, 61.14) --
	(181.92, 60.95) --
	(180.14, 59.17) --
	(179.96, 58.98) --
	(178.18, 57.20) --
	(177.99, 57.01) --
	(176.21, 55.23) --
	(176.02, 55.05) --
	(174.24, 53.27) --
	(174.05, 53.08) --
	(172.28, 51.30) --
	(172.09, 51.11) --
	(170.31, 49.33) --
	(170.12, 49.15) --
	(168.34, 47.37) --
	(166.56, 49.15) --
	(166.37, 49.33) --
	(164.60, 51.11) --
	(164.41, 51.30) --
	(162.63, 53.08) --
	(162.44, 53.27) --
	(160.66, 55.05) --
	(160.47, 55.23) --
	(158.69, 57.01) --
	(158.51, 57.20) --
	(156.73, 58.98) --
	(156.54, 59.17) --
	(154.76, 60.95) --
	(154.57, 61.14) --
	(152.79, 62.91) --
	(152.61, 63.10) --
	(150.83, 64.88) --
	(150.64, 65.07) --
	(148.86, 66.85) --
	(148.67, 67.04) --
	(146.89, 68.82) --
	(146.70, 69.00) --
	(144.93, 70.78) --
	(144.74, 70.97) --
	(142.96, 72.75) --
	(142.77, 72.94) --
	(140.99, 74.72) --
	(140.80, 74.91);

\path[draw=drawColor,draw opacity=0.33,line width= 1.1pt,line join=round] (166.22, 37.34) --
	(164.41, 39.15) --
	(164.25, 39.31) --
	(162.44, 41.12) --
	(162.28, 41.28) --
	(160.47, 43.09) --
	(160.32, 43.24) --
	(158.51, 45.05) --
	(158.35, 45.21) --
	(156.54, 47.02) --
	(156.38, 47.18) --
	(154.57, 48.99) --
	(154.41, 49.15) --
	(152.61, 50.95) --
	(152.45, 51.11) --
	(150.64, 52.92) --
	(150.48, 53.08) --
	(148.67, 54.89) --
	(148.51, 55.05) --
	(146.70, 56.86) --
	(146.55, 57.01) --
	(144.74, 58.82) --
	(144.58, 58.98) --
	(142.77, 60.79) --
	(142.61, 60.95) --
	(140.80, 62.76) --
	(140.65, 62.91) --
	(138.84, 64.72) --
	(138.68, 64.88) --
	(136.87, 66.69) --
	(136.71, 66.85) --
	(134.90, 68.66) --
	(134.74, 68.82) --
	(132.93, 70.63) --
	(132.78, 70.78) --
	(130.97, 72.59) --
	(130.81, 72.75) --
	(129.00, 74.56);

\path[draw=drawColor,draw opacity=0.33,line width= 1.1pt,line join=round] (129.00, 78.81) --
	(130.81, 80.62) --
	(130.97, 80.78) --
	(132.78, 82.58) --
	(132.93, 82.74) --
	(134.74, 84.55) --
	(134.90, 84.71) --
	(136.71, 86.52) --
	(136.87, 86.68) --
	(138.68, 88.49) --
	(138.84, 88.64) --
	(140.65, 90.45) --
	(140.80, 90.61) --
	(142.61, 92.42) --
	(142.77, 92.58) --
	(144.58, 94.39) --
	(144.74, 94.54) --
	(146.55, 96.35) --
	(146.70, 96.51) --
	(148.51, 98.32) --
	(148.67, 98.48) --
	(150.48,100.29) --
	(150.64,100.45) --
	(152.45,102.26) --
	(152.61,102.41) --
	(154.41,104.22) --
	(154.57,104.38) --
	(156.38,106.19) --
	(156.54,106.35) --
	(158.35,108.16) --
	(158.51,108.31) --
	(160.32,110.12) --
	(160.47,110.28) --
	(162.28,112.09) --
	(162.44,112.25) --
	(164.25,114.06) --
	(164.41,114.22) --
	(166.22,116.02);

\path[draw=drawColor,draw opacity=0.33,line width= 1.1pt,line join=round] (170.47, 37.34) --
	(172.28, 39.15) --
	(172.43, 39.31) --
	(174.24, 41.12) --
	(174.40, 41.28) --
	(176.21, 43.09) --
	(176.37, 43.24) --
	(178.18, 45.05) --
	(178.33, 45.21) --
	(180.14, 47.02) --
	(180.30, 47.18) --
	(182.11, 48.99) --
	(182.27, 49.15) --
	(184.08, 50.95) --
	(184.24, 51.11) --
	(186.04, 52.92) --
	(186.20, 53.08) --
	(188.01, 54.89) --
	(188.17, 55.05) --
	(189.98, 56.86) --
	(190.14, 57.01) --
	(191.95, 58.82) --
	(192.10, 58.98) --
	(193.91, 60.79) --
	(194.07, 60.95) --
	(195.88, 62.76) --
	(196.04, 62.91) --
	(197.85, 64.72) --
	(198.00, 64.88) --
	(199.81, 66.69) --
	(199.97, 66.85) --
	(201.78, 68.66) --
	(201.94, 68.82) --
	(203.75, 70.63) --
	(203.91, 70.78) --
	(205.72, 72.59) --
	(205.87, 72.75) --
	(207.68, 74.56);

\path[draw=drawColor,draw opacity=0.33,line width= 1.1pt,line join=round] (207.68, 78.81) --
	(205.87, 80.62) --
	(205.72, 80.78) --
	(203.91, 82.58) --
	(203.75, 82.74) --
	(201.94, 84.55) --
	(201.78, 84.71) --
	(199.97, 86.52) --
	(199.81, 86.68) --
	(198.00, 88.49) --
	(197.85, 88.64) --
	(196.04, 90.45) --
	(195.88, 90.61) --
	(194.07, 92.42) --
	(193.91, 92.58) --
	(192.10, 94.39) --
	(191.95, 94.54) --
	(190.14, 96.35) --
	(189.98, 96.51) --
	(188.17, 98.32) --
	(188.01, 98.48) --
	(186.20,100.29) --
	(186.04,100.45) --
	(184.24,102.26) --
	(184.08,102.41) --
	(182.27,104.22) --
	(182.11,104.38) --
	(180.30,106.19) --
	(180.14,106.35) --
	(178.33,108.16) --
	(178.18,108.31) --
	(176.37,110.12) --
	(176.21,110.28) --
	(174.40,112.09) --
	(174.24,112.25) --
	(172.43,114.06) --
	(172.28,114.22) --
	(170.47,116.02);

\path[draw=drawColor,draw opacity=0.33,line width= 1.1pt,line join=round] (156.90, 37.34) --
	(156.54, 37.70) --
	(154.93, 39.31) --
	(154.57, 39.67) --
	(152.97, 41.28) --
	(152.61, 41.64) --
	(151.00, 43.24) --
	(150.64, 43.60) --
	(149.03, 45.21) --
	(148.67, 45.57) --
	(147.06, 47.18) --
	(146.70, 47.54) --
	(145.10, 49.15) --
	(144.74, 49.51) --
	(143.13, 51.11) --
	(142.77, 51.47) --
	(141.16, 53.08) --
	(140.80, 53.44) --
	(139.20, 55.05) --
	(138.84, 55.41) --
	(137.23, 57.01) --
	(136.87, 57.37) --
	(135.26, 58.98) --
	(134.90, 59.34) --
	(133.29, 60.95) --
	(132.93, 61.31) --
	(131.33, 62.91) --
	(130.97, 63.27) --
	(129.36, 64.88) --
	(129.00, 65.24);

\path[draw=drawColor,draw opacity=0.33,line width= 1.1pt,line join=round] (129.00, 88.13) --
	(129.36, 88.49) --
	(130.97, 90.09) --
	(131.33, 90.45) --
	(132.93, 92.06) --
	(133.29, 92.42) --
	(134.90, 94.03) --
	(135.26, 94.39) --
	(136.87, 95.99) --
	(137.23, 96.35) --
	(138.84, 97.96) --
	(139.20, 98.32) --
	(140.80, 99.93) --
	(141.16,100.29) --
	(142.77,101.90) --
	(143.13,102.26) --
	(144.74,103.86) --
	(145.10,104.22) --
	(146.70,105.83) --
	(147.06,106.19) --
	(148.67,107.80) --
	(149.03,108.16) --
	(150.64,109.76) --
	(151.00,110.12) --
	(152.61,111.73) --
	(152.97,112.09) --
	(154.57,113.70) --
	(154.93,114.06) --
	(156.54,115.66) --
	(156.90,116.02);

\path[draw=drawColor,draw opacity=0.33,line width= 1.1pt,line join=round] (179.78, 37.34) --
	(180.14, 37.70) --
	(181.75, 39.31) --
	(182.11, 39.67) --
	(183.72, 41.28) --
	(184.08, 41.64) --
	(185.68, 43.24) --
	(186.04, 43.60) --
	(187.65, 45.21) --
	(188.01, 45.57) --
	(189.62, 47.18) --
	(189.98, 47.54) --
	(191.59, 49.15) --
	(191.95, 49.51) --
	(193.55, 51.11) --
	(193.91, 51.47) --
	(195.52, 53.08) --
	(195.88, 53.44) --
	(197.49, 55.05) --
	(197.85, 55.41) --
	(199.45, 57.01) --
	(199.81, 57.37) --
	(201.42, 58.98) --
	(201.78, 59.34) --
	(203.39, 60.95) --
	(203.75, 61.31) --
	(205.36, 62.91) --
	(205.72, 63.27) --
	(207.32, 64.88) --
	(207.68, 65.24);

\path[draw=drawColor,draw opacity=0.33,line width= 1.1pt,line join=round] (207.68, 88.13) --
	(207.32, 88.49) --
	(205.72, 90.09) --
	(205.36, 90.45) --
	(203.75, 92.06) --
	(203.39, 92.42) --
	(201.78, 94.03) --
	(201.42, 94.39) --
	(199.81, 95.99) --
	(199.45, 96.35) --
	(197.85, 97.96) --
	(197.49, 98.32) --
	(195.88, 99.93) --
	(195.52,100.29) --
	(193.91,101.90) --
	(193.55,102.26) --
	(191.95,103.86) --
	(191.59,104.22) --
	(189.98,105.83) --
	(189.62,106.19) --
	(188.01,107.80) --
	(187.65,108.16) --
	(186.04,109.76) --
	(185.68,110.12) --
	(184.08,111.73) --
	(183.72,112.09) --
	(182.11,113.70) --
	(181.75,114.06) --
	(180.14,115.66) --
	(179.78,116.02);

\path[draw=drawColor,draw opacity=0.33,line width= 1.1pt,line join=round] (149.04, 37.34) --
	(148.67, 37.71) --
	(147.07, 39.31) --
	(146.70, 39.68) --
	(145.11, 41.28) --
	(144.74, 41.65) --
	(143.14, 43.24) --
	(142.77, 43.61) --
	(141.17, 45.21) --
	(140.80, 45.58) --
	(139.21, 47.18) --
	(138.84, 47.55) --
	(137.24, 49.15) --
	(136.87, 49.52) --
	(135.27, 51.11) --
	(134.90, 51.48) --
	(133.31, 53.08) --
	(132.93, 53.45) --
	(131.34, 55.05) --
	(130.97, 55.42) --
	(129.37, 57.01) --
	(129.00, 57.38);

\path[draw=drawColor,draw opacity=0.33,line width= 1.1pt,line join=round] (129.00, 95.98) --
	(129.37, 96.35) --
	(130.97, 97.95) --
	(131.34, 98.32) --
	(132.93, 99.92) --
	(133.31,100.29) --
	(134.90,101.88) --
	(135.27,102.26) --
	(136.87,103.85) --
	(137.24,104.22) --
	(138.84,105.82) --
	(139.21,106.19) --
	(140.80,107.79) --
	(141.17,108.16) --
	(142.77,109.75) --
	(143.14,110.12) --
	(144.74,111.72) --
	(145.11,112.09) --
	(146.70,113.69) --
	(147.07,114.06) --
	(148.67,115.65) --
	(149.04,116.02);

\path[draw=drawColor,draw opacity=0.33,line width= 1.1pt,line join=round] (187.64, 37.34) --
	(188.01, 37.71) --
	(189.61, 39.31) --
	(189.98, 39.68) --
	(191.58, 41.28) --
	(191.95, 41.65) --
	(193.54, 43.24) --
	(193.91, 43.61) --
	(195.51, 45.21) --
	(195.88, 45.58) --
	(197.48, 47.18) --
	(197.85, 47.55) --
	(199.44, 49.15) --
	(199.81, 49.52) --
	(201.41, 51.11) --
	(201.78, 51.48) --
	(203.38, 53.08) --
	(203.75, 53.45) --
	(205.34, 55.05) --
	(205.72, 55.42) --
	(207.31, 57.01) --
	(207.68, 57.38);

\path[draw=drawColor,draw opacity=0.33,line width= 1.1pt,line join=round] (207.68, 95.98) --
	(207.31, 96.35) --
	(205.72, 97.95) --
	(205.34, 98.32) --
	(203.75, 99.92) --
	(203.38,100.29) --
	(201.78,101.88) --
	(201.41,102.26) --
	(199.81,103.85) --
	(199.44,104.22) --
	(197.85,105.82) --
	(197.48,106.19) --
	(195.88,107.79) --
	(195.51,108.16) --
	(193.91,109.75) --
	(193.54,110.12) --
	(191.95,111.72) --
	(191.58,112.09) --
	(189.98,113.69) --
	(189.61,114.06) --
	(188.01,115.65) --
	(187.64,116.02);

\path[draw=drawColor,draw opacity=0.33,line width= 1.1pt,line join=round] (142.12, 37.34) --
	(140.80, 38.66) --
	(140.15, 39.31) --
	(138.84, 40.63) --
	(138.19, 41.28) --
	(136.87, 42.60) --
	(136.22, 43.24) --
	(134.90, 44.56) --
	(134.25, 45.21) --
	(132.93, 46.53) --
	(132.29, 47.18) --
	(130.97, 48.50) --
	(130.32, 49.15) --
	(129.00, 50.46);

\path[draw=drawColor,draw opacity=0.33,line width= 1.1pt,line join=round] (129.00,102.90) --
	(130.32,104.22) --
	(130.97,104.87) --
	(132.29,106.19) --
	(132.93,106.84) --
	(134.25,108.16) --
	(134.90,108.81) --
	(136.22,110.12) --
	(136.87,110.77) --
	(138.19,112.09) --
	(138.84,112.74) --
	(140.15,114.06) --
	(140.80,114.71) --
	(142.12,116.02);

\path[draw=drawColor,draw opacity=0.33,line width= 1.1pt,line join=round] (194.56, 37.34) --
	(195.88, 38.66) --
	(196.53, 39.31) --
	(197.85, 40.63) --
	(198.50, 41.28) --
	(199.81, 42.60) --
	(200.46, 43.24) --
	(201.78, 44.56) --
	(202.43, 45.21) --
	(203.75, 46.53) --
	(204.40, 47.18) --
	(205.72, 48.50) --
	(206.36, 49.15) --
	(207.68, 50.46);

\path[draw=drawColor,draw opacity=0.33,line width= 1.1pt,line join=round] (207.68,102.90) --
	(206.36,104.22) --
	(205.72,104.87) --
	(204.40,106.19) --
	(203.75,106.84) --
	(202.43,108.16) --
	(201.78,108.81) --
	(200.46,110.12) --
	(199.81,110.77) --
	(198.50,112.09) --
	(197.85,112.74) --
	(196.53,114.06) --
	(195.88,114.71) --
	(194.56,116.02);

\path[draw=drawColor,draw opacity=0.33,line width= 1.1pt,line join=round] (135.86, 37.34) --
	(134.90, 38.30) --
	(133.90, 39.31) --
	(132.93, 40.27) --
	(131.93, 41.28) --
	(130.97, 42.24) --
	(129.96, 43.24) --
	(129.00, 44.21);

\path[draw=drawColor,draw opacity=0.33,line width= 1.1pt,line join=round] (129.00,109.16) --
	(129.96,110.12) --
	(130.97,111.13) --
	(131.93,112.09) --
	(132.93,113.10) --
	(133.90,114.06) --
	(134.90,115.06) --
	(135.86,116.02);

\path[draw=drawColor,draw opacity=0.33,line width= 1.1pt,line join=round] (200.82, 37.34) --
	(201.78, 38.30) --
	(202.79, 39.31) --
	(203.75, 40.27) --
	(204.75, 41.28) --
	(205.72, 42.24) --
	(206.72, 43.24) --
	(207.68, 44.21);

\path[draw=drawColor,draw opacity=0.33,line width= 1.1pt,line join=round] (207.68,109.16) --
	(206.72,110.12) --
	(205.72,111.13) --
	(204.75,112.09) --
	(203.75,113.10) --
	(202.79,114.06) --
	(201.78,115.06) --
	(200.82,116.02);

\path[draw=drawColor,draw opacity=0.33,line width= 1.1pt,line join=round] (130.11, 37.34) --
	(129.00, 38.45);

\path[draw=drawColor,draw opacity=0.33,line width= 1.1pt,line join=round] (129.00,114.92) --
	(130.11,116.02);

\path[draw=drawColor,draw opacity=0.33,line width= 1.1pt,line join=round] (206.58, 37.34) --
	(207.68, 38.45);

\path[draw=drawColor,draw opacity=0.33,line width= 1.1pt,line join=round] (207.68,114.92) --
	(206.58,116.02);
\definecolor{drawColor}{gray}{0.50}

\path[draw=drawColor,line width= 0.6pt,line join=round,line cap=round] (125.07, 33.41) rectangle (211.62,119.96);
\end{scope}
\begin{scope}
\path[clip] (211.62, 33.41) rectangle (298.17,119.96);
\definecolor{fillColor}{RGB}{255,255,255}

\path[fill=fillColor] (211.62, 33.41) rectangle (298.17,119.96);
\definecolor{drawColor}{RGB}{0,0,0}

\path[draw=drawColor,line width= 1.1pt,line join=round] (243.09, 72.68) --
	(243.05, 72.75) --
	(242.30, 74.72) --
	(241.86, 76.68) --
	(241.77, 78.65) --
	(242.04, 80.62) --
	(242.66, 82.58) --
	(243.09, 83.48) --
	(243.64, 84.55) --
	(244.97, 86.52) --
	(245.06, 86.62) --
	(246.75, 88.49) --
	(247.02, 88.75) --
	(248.99, 90.38) --
	(249.10, 90.45) --
	(250.96, 91.59) --
	(252.78, 92.42) --
	(252.92, 92.48) --
	(254.89, 93.00) --
	(256.86, 93.21) --
	(258.83, 93.09) --
	(260.79, 92.68) --
	(261.54, 92.42) --
	(262.76, 91.97) --
	(264.73, 90.99) --
	(265.62, 90.45) --
	(266.69, 89.74) --
	(268.35, 88.49) --
	(268.66, 88.22) --
	(270.45, 86.52) --
	(270.63, 86.32) --
	(272.04, 84.55) --
	(272.59, 83.66) --
	(273.21, 82.58) --
	(273.92, 80.62) --
	(274.21, 78.65) --
	(274.06, 76.68) --
	(273.51, 74.72) --
	(272.60, 72.75) --
	(272.59, 72.74) --
	(271.29, 70.78) --
	(270.63, 69.94) --
	(269.66, 68.82) --
	(268.66, 67.79) --
	(267.64, 66.85) --
	(266.69, 66.03) --
	(265.14, 64.88) --
	(264.73, 64.59) --
	(262.76, 63.45) --
	(261.51, 62.91) --
	(260.79, 62.61) --
	(258.83, 62.12) --
	(256.86, 62.00) --
	(254.89, 62.25) --
	(252.92, 62.89) --
	(252.87, 62.91) --
	(250.96, 63.94) --
	(249.61, 64.88) --
	(248.99, 65.35) --
	(247.34, 66.85) --
	(247.02, 67.17) --
	(245.58, 68.82) --
	(245.06, 69.50) --
	(244.16, 70.78) --
	(243.09, 72.68);

\path[draw=drawColor,line width= 1.1pt,line join=round] (233.25, 75.82) --
	(233.08, 76.68) --
	(233.06, 78.65) --
	(233.25, 79.72) --
	(233.41, 80.62) --
	(234.10, 82.58) --
	(235.09, 84.55) --
	(235.22, 84.75) --
	(236.35, 86.52) --
	(237.19, 87.63) --
	(237.83, 88.49) --
	(239.15, 90.06) --
	(239.49, 90.45) --
	(241.12, 92.20) --
	(241.34, 92.42) --
	(243.09, 94.11) --
	(243.39, 94.39) --
	(245.06, 95.84) --
	(245.71, 96.35) --
	(247.02, 97.37) --
	(248.44, 98.32) --
	(248.99, 98.69) --
	(250.96, 99.76) --
	(252.28,100.29) --
	(252.92,100.55) --
	(254.89,101.03) --
	(256.86,101.18) --
	(258.83,101.00) --
	(260.79,100.51) --
	(261.34,100.29) --
	(262.76, 99.74) --
	(264.73, 98.73) --
	(265.38, 98.32) --
	(266.69, 97.52) --
	(268.39, 96.35) --
	(268.66, 96.17) --
	(270.63, 94.68) --
	(271.00, 94.39) --
	(272.59, 93.08) --
	(273.37, 92.42) --
	(274.56, 91.36) --
	(275.54, 90.45) --
	(276.53, 89.47) --
	(277.49, 88.49) --
	(278.50, 87.32) --
	(279.17, 86.52) --
	(280.46, 84.64) --
	(280.52, 84.55) --
	(281.49, 82.58) --
	(282.04, 80.62) --
	(282.17, 78.65) --
	(281.89, 76.68) --
	(281.24, 74.72) --
	(280.46, 73.16) --
	(280.26, 72.75) --
	(279.01, 70.78) --
	(278.50, 70.10) --
	(277.53, 68.82) --
	(276.53, 67.62) --
	(275.88, 66.85) --
	(274.56, 65.39) --
	(274.09, 64.88) --
	(272.59, 63.32) --
	(272.19, 62.91) --
	(270.63, 61.37) --
	(270.17, 60.95) --
	(268.66, 59.51) --
	(268.04, 58.98) --
	(266.69, 57.76) --
	(265.76, 57.01) --
	(264.73, 56.13) --
	(263.22, 55.05) --
	(262.76, 54.68) --
	(260.79, 53.48) --
	(259.84, 53.08) --
	(258.83, 52.61) --
	(256.86, 52.16) --
	(254.89, 52.22) --
	(252.92, 52.80) --
	(252.41, 53.08) --
	(250.96, 53.83) --
	(249.29, 55.05) --
	(248.99, 55.26) --
	(247.02, 56.96) --
	(246.97, 57.01) --
	(245.06, 58.86) --
	(244.94, 58.98) --
	(243.09, 60.90) --
	(243.05, 60.95) --
	(241.26, 62.91) --
	(241.12, 63.07) --
	(239.56, 64.88) --
	(239.15, 65.38) --
	(237.97, 66.85) --
	(237.19, 67.90) --
	(236.51, 68.82) --
	(235.23, 70.78) --
	(235.22, 70.81) --
	(234.20, 72.75) --
	(233.47, 74.72) --
	(233.25, 75.82);

\path[draw=drawColor,line width= 1.1pt,line join=round] (249.72, 37.34) --
	(249.15, 39.31) --
	(248.99, 39.68) --
	(248.48, 41.28) --
	(247.63, 43.24) --
	(247.02, 44.35) --
	(246.63, 45.21) --
	(245.53, 47.18) --
	(245.06, 47.89) --
	(244.32, 49.15) --
	(243.09, 50.95) --
	(242.99, 51.11) --
	(241.59, 53.08) --
	(241.12, 53.68) --
	(240.09, 55.05) --
	(239.15, 56.19) --
	(238.49, 57.01) --
	(237.19, 58.55) --
	(236.81, 58.98) --
	(235.22, 60.77) --
	(235.06, 60.95) --
	(233.25, 62.89) --
	(233.23, 62.91) --
	(231.36, 64.88) --
	(231.29, 64.97) --
	(229.48, 66.85) --
	(229.32, 67.03) --
	(227.65, 68.82) --
	(227.35, 69.19) --
	(225.97, 70.78) --
	(225.39, 71.65) --
	(224.56, 72.75) --
	(223.61, 74.72) --
	(223.42, 75.69) --
	(223.20, 76.68) --
	(223.40, 78.65) --
	(223.42, 78.71) --
	(224.11, 80.62) --
	(225.29, 82.58) --
	(225.39, 82.71) --
	(226.74, 84.55) --
	(227.35, 85.22) --
	(228.42, 86.52) --
	(229.32, 87.46) --
	(230.22, 88.49) --
	(231.29, 89.57) --
	(232.10, 90.45) --
	(233.25, 91.61) --
	(234.03, 92.42) --
	(235.22, 93.61) --
	(235.98, 94.39) --
	(237.19, 95.59) --
	(237.96, 96.35) --
	(239.15, 97.54) --
	(239.96, 98.32) --
	(241.12, 99.47) --
	(241.99,100.29) --
	(243.09,101.35) --
	(244.09,102.26) --
	(245.06,103.17) --
	(246.29,104.22) --
	(247.02,104.89) --
	(248.68,106.19) --
	(248.99,106.45) --
	(250.96,107.81) --
	(251.63,108.16) --
	(252.92,108.89) --
	(254.89,109.58) --
	(256.86,109.83) --
	(258.83,109.60) --
	(260.79,108.92) --
	(262.13,108.16) --
	(262.76,107.83) --
	(264.73,106.47) --
	(265.05,106.19) --
	(266.69,104.92) --
	(267.49,104.22) --
	(268.66,103.27) --
	(269.80,102.26) --
	(270.63,101.56) --
	(272.07,100.29) --
	(272.59, 99.84) --
	(274.34, 98.32) --
	(274.56, 98.13) --
	(276.53, 96.43) --
	(276.62, 96.35) --
	(278.50, 94.74) --
	(278.90, 94.39) --
	(280.46, 93.02) --
	(281.17, 92.42) --
	(282.43, 91.27) --
	(283.35, 90.45) --
	(284.40, 89.42) --
	(285.36, 88.49) --
	(286.36, 87.35) --
	(287.12, 86.52) --
	(288.33, 84.78) --
	(288.50, 84.55) --
	(289.43, 82.58) --
	(289.86, 80.62) --
	(289.80, 78.65) --
	(289.28, 76.68) --
	(288.35, 74.72) --
	(288.33, 74.69) --
	(287.12, 72.75) --
	(286.36, 71.76) --
	(285.65, 70.78) --
	(284.40, 69.30) --
	(284.00, 68.82) --
	(282.43, 67.06) --
	(282.24, 66.85) --
	(280.46, 64.93) --
	(280.42, 64.88) --
	(278.57, 62.91) --
	(278.50, 62.84) --
	(276.72, 60.95) --
	(276.53, 60.74) --
	(274.89, 58.98) --
	(274.56, 58.61) --
	(273.11, 57.01) --
	(272.59, 56.41) --
	(271.38, 55.05) --
	(270.63, 54.11) --
	(269.73, 53.08) --
	(268.66, 51.69) --
	(268.17, 51.11) --
	(266.73, 49.15) --
	(266.69, 49.09) --
	(265.33, 47.18) --
	(264.73, 46.15) --
	(264.07, 45.21) --
	(262.98, 43.24) --
	(262.76, 42.73) --
	(261.97, 41.28) --
	(261.20, 39.31) --
	(260.79, 37.79) --
	(260.63, 37.34);

\path[draw=drawColor,line width= 1.1pt,line join=round] (241.05, 37.34) --
	(240.77, 39.31) --
	(240.35, 41.28) --
	(239.78, 43.24) --
	(239.15, 44.93) --
	(239.05, 45.21) --
	(238.19, 47.18) --
	(237.19, 49.05) --
	(237.14, 49.15) --
	(235.93, 51.11) --
	(235.22, 52.13) --
	(234.54, 53.08) --
	(233.25, 54.70) --
	(232.96, 55.05) --
	(231.29, 56.90) --
	(231.18, 57.01) --
	(229.32, 58.84) --
	(229.17, 58.98) --
	(227.35, 60.57) --
	(226.87, 60.95) --
	(225.39, 62.12) --
	(224.23, 62.91) --
	(223.42, 63.50) --
	(221.45, 64.72) --
	(221.13, 64.88) --
	(219.48, 65.82) --
	(217.52, 66.73) --
	(217.18, 66.85) --
	(215.55, 67.54);

\path[draw=drawColor,line width= 1.1pt,line join=round] (215.55, 84.99) --
	(217.52, 85.99) --
	(218.37, 86.52) --
	(219.48, 87.12) --
	(221.45, 88.40) --
	(221.56, 88.49) --
	(223.42, 89.75) --
	(224.33, 90.45) --
	(225.39, 91.22) --
	(226.87, 92.42) --
	(227.35, 92.79) --
	(229.23, 94.39) --
	(229.32, 94.46) --
	(231.29, 96.21) --
	(231.44, 96.35) --
	(233.25, 98.05) --
	(233.53, 98.32) --
	(235.22, 99.99) --
	(235.52,100.29) --
	(237.19,102.02) --
	(237.41,102.26) --
	(239.15,104.16) --
	(239.22,104.22) --
	(240.95,106.19) --
	(241.12,106.40) --
	(242.62,108.16) --
	(243.09,108.76) --
	(244.22,110.12) --
	(245.06,111.26) --
	(245.74,112.09) --
	(247.02,113.88) --
	(247.17,114.06) --
	(248.56,116.02);

\path[draw=drawColor,line width= 1.1pt,line join=round] (294.23, 88.54) --
	(292.26, 90.07) --
	(291.80, 90.45) --
	(290.30, 91.53) --
	(289.14, 92.42) --
	(288.33, 92.99) --
	(286.47, 94.39) --
	(286.36, 94.46) --
	(284.40, 95.96) --
	(283.90, 96.35) --
	(282.43, 97.52) --
	(281.47, 98.32) --
	(280.46, 99.17) --
	(279.19,100.29) --
	(278.50,100.92) --
	(277.07,102.26) --
	(276.53,102.79) --
	(275.10,104.22) --
	(274.56,104.82) --
	(273.29,106.19) --
	(272.59,107.03) --
	(271.62,108.16) --
	(270.63,109.46) --
	(270.08,110.12) --
	(268.70,112.09) --
	(268.66,112.16) --
	(267.39,114.06) --
	(266.69,115.33) --
	(266.24,116.02);

\path[draw=drawColor,line width= 1.1pt,line join=round] (268.53, 37.34) --
	(268.66, 38.10) --
	(268.85, 39.31) --
	(269.31, 41.28) --
	(269.94, 43.24) --
	(270.63, 44.91) --
	(270.74, 45.21) --
	(271.67, 47.18) --
	(272.59, 48.80) --
	(272.78, 49.15) --
	(274.02, 51.11) --
	(274.56, 51.87) --
	(275.41, 53.08) --
	(276.53, 54.52) --
	(276.94, 55.05) --
	(278.50, 56.90) --
	(278.59, 57.01) --
	(280.36, 58.98) --
	(280.46, 59.09) --
	(282.23, 60.95) --
	(282.43, 61.15) --
	(284.19, 62.91) --
	(284.40, 63.12) --
	(286.23, 64.88) --
	(286.36, 65.02) --
	(288.30, 66.85) --
	(288.33, 66.88) --
	(290.30, 68.73) --
	(290.39, 68.82) --
	(292.26, 70.62) --
	(292.45, 70.78) --
	(294.23, 72.59);

\path[draw=drawColor,line width= 1.1pt,line join=round] (215.55, 60.19) --
	(217.52, 59.55) --
	(218.88, 58.98) --
	(219.48, 58.73) --
	(221.45, 57.71) --
	(222.57, 57.01) --
	(223.42, 56.47) --
	(225.31, 55.05) --
	(225.39, 54.99) --
	(227.35, 53.19) --
	(227.47, 53.08) --
	(229.23, 51.11) --
	(229.32, 51.01) --
	(230.69, 49.15) --
	(231.29, 48.21) --
	(231.90, 47.18) --
	(232.89, 45.21) --
	(233.25, 44.33) --
	(233.69, 43.24) --
	(234.30, 41.28) --
	(234.76, 39.31) --
	(235.07, 37.34);

\path[draw=drawColor,line width= 1.1pt,line join=round] (215.55, 91.42) --
	(217.52, 92.23) --
	(217.90, 92.42) --
	(219.48, 93.17) --
	(221.45, 94.26) --
	(221.65, 94.39) --
	(223.42, 95.48) --
	(224.66, 96.35) --
	(225.39, 96.86) --
	(227.26, 98.32) --
	(227.35, 98.39) --
	(229.32,100.07) --
	(229.56,100.29) --
	(231.29,101.91) --
	(231.63,102.26) --
	(233.25,103.94) --
	(233.51,104.22) --
	(235.22,106.18) --
	(235.23,106.19) --
	(236.80,108.16) --
	(237.19,108.69) --
	(238.23,110.12) --
	(239.15,111.55) --
	(239.52,112.09) --
	(240.67,114.06) --
	(241.12,114.94) --
	(241.70,116.02);

\path[draw=drawColor,line width= 1.1pt,line join=round] (273.55, 37.34) --
	(273.85, 39.31) --
	(274.29, 41.28) --
	(274.56, 42.18) --
	(274.87, 43.24) --
	(275.61, 45.21) --
	(276.52, 47.18) --
	(276.53, 47.19) --
	(277.59, 49.15) --
	(278.50, 50.58) --
	(278.83, 51.11) --
	(280.24, 53.08) --
	(280.46, 53.36) --
	(281.81, 55.05) --
	(282.43, 55.75) --
	(283.56, 57.01) --
	(284.40, 57.89) --
	(285.48, 58.98) --
	(286.36, 59.84) --
	(287.56, 60.95) --
	(288.33, 61.65) --
	(289.82, 62.91) --
	(290.30, 63.33) --
	(292.24, 64.88) --
	(292.26, 64.90) --
	(294.23, 66.42);

\path[draw=drawColor,line width= 1.1pt,line join=round] (294.23, 95.26) --
	(292.26, 96.34) --
	(292.25, 96.35) --
	(290.30, 97.51) --
	(289.05, 98.32) --
	(288.33, 98.79) --
	(286.36,100.19) --
	(286.24,100.29) --
	(284.40,101.75) --
	(283.80,102.26) --
	(282.43,103.49) --
	(281.65,104.22) --
	(280.46,105.45) --
	(279.77,106.19) --
	(278.50,107.71) --
	(278.13,108.16) --
	(276.72,110.12) --
	(276.53,110.44) --
	(275.51,112.09) --
	(274.56,113.96) --
	(274.51,114.06) --
	(273.68,116.02);

\path[draw=drawColor,line width= 1.1pt,line join=round] (215.55, 53.99) --
	(217.52, 53.22) --
	(217.81, 53.08) --
	(219.48, 52.20) --
	(221.15, 51.11) --
	(221.45, 50.89) --
	(223.42, 49.21) --
	(223.48, 49.15) --
	(225.23, 47.18) --
	(225.39, 46.98) --
	(226.58, 45.21) --
	(227.35, 43.81) --
	(227.64, 43.24) --
	(228.44, 41.28) --
	(229.04, 39.31) --
	(229.32, 38.01) --
	(229.46, 37.34);
\definecolor{drawColor}{RGB}{0,0,0}

\path[draw=drawColor,draw opacity=0.33,line width= 1.1pt,line join=round] (227.35, 74.91) --
	(225.57, 76.68) --
	(227.35, 78.46) --
	(227.54, 78.65) --
	(229.32, 80.43) --
	(229.51, 80.62) --
	(231.29, 82.40) --
	(231.48, 82.58) --
	(233.25, 84.36) --
	(233.44, 84.55) --
	(235.22, 86.33) --
	(235.41, 86.52) --
	(237.19, 88.30) --
	(237.38, 88.49) --
	(239.15, 90.26) --
	(239.34, 90.45) --
	(241.12, 92.23) --
	(241.31, 92.42) --
	(243.09, 94.20) --
	(243.28, 94.39) --
	(245.06, 96.17) --
	(245.24, 96.35) --
	(247.02, 98.13) --
	(247.21, 98.32) --
	(248.99,100.10) --
	(249.18,100.29) --
	(250.96,102.07) --
	(251.15,102.26) --
	(252.92,104.03) --
	(253.11,104.22) --
	(254.89,106.00) --
	(256.67,104.22) --
	(256.86,104.03) --
	(258.64,102.26) --
	(258.83,102.07) --
	(260.60,100.29) --
	(260.79,100.10) --
	(262.57, 98.32) --
	(262.76, 98.13) --
	(264.54, 96.35) --
	(264.73, 96.17) --
	(266.50, 94.39) --
	(266.69, 94.20) --
	(268.47, 92.42) --
	(268.66, 92.23) --
	(270.44, 90.45) --
	(270.63, 90.26) --
	(272.41, 88.49) --
	(272.59, 88.30) --
	(274.37, 86.52) --
	(274.56, 86.33) --
	(276.34, 84.55) --
	(276.53, 84.36) --
	(278.31, 82.58) --
	(278.50, 82.40) --
	(280.27, 80.62) --
	(280.46, 80.43) --
	(282.24, 78.65) --
	(282.43, 78.46) --
	(284.21, 76.68) --
	(282.43, 74.91) --
	(282.24, 74.72) --
	(280.46, 72.94) --
	(280.27, 72.75) --
	(278.50, 70.97) --
	(278.31, 70.78) --
	(276.53, 69.00) --
	(276.34, 68.82) --
	(274.56, 67.04) --
	(274.37, 66.85) --
	(272.59, 65.07) --
	(272.41, 64.88) --
	(270.63, 63.10) --
	(270.44, 62.91) --
	(268.66, 61.14) --
	(268.47, 60.95) --
	(266.69, 59.17) --
	(266.50, 58.98) --
	(264.73, 57.20) --
	(264.54, 57.01) --
	(262.76, 55.23) --
	(262.57, 55.05) --
	(260.79, 53.27) --
	(260.60, 53.08) --
	(258.83, 51.30) --
	(258.64, 51.11) --
	(256.86, 49.33) --
	(256.67, 49.15) --
	(254.89, 47.37) --
	(253.11, 49.15) --
	(252.92, 49.33) --
	(251.15, 51.11) --
	(250.96, 51.30) --
	(249.18, 53.08) --
	(248.99, 53.27) --
	(247.21, 55.05) --
	(247.02, 55.23) --
	(245.24, 57.01) --
	(245.06, 57.20) --
	(243.28, 58.98) --
	(243.09, 59.17) --
	(241.31, 60.95) --
	(241.12, 61.14) --
	(239.34, 62.91) --
	(239.15, 63.10) --
	(237.38, 64.88) --
	(237.19, 65.07) --
	(235.41, 66.85) --
	(235.22, 67.04) --
	(233.44, 68.82) --
	(233.25, 69.00) --
	(231.48, 70.78) --
	(231.29, 70.97) --
	(229.51, 72.75) --
	(229.32, 72.94) --
	(227.54, 74.72) --
	(227.35, 74.91);

\path[draw=drawColor,draw opacity=0.33,line width= 1.1pt,line join=round] (252.77, 37.34) --
	(250.96, 39.15) --
	(250.80, 39.31) --
	(248.99, 41.12) --
	(248.83, 41.28) --
	(247.02, 43.09) --
	(246.87, 43.24) --
	(245.06, 45.05) --
	(244.90, 45.21) --
	(243.09, 47.02) --
	(242.93, 47.18) --
	(241.12, 48.99) --
	(240.96, 49.15) --
	(239.15, 50.95) --
	(239.00, 51.11) --
	(237.19, 52.92) --
	(237.03, 53.08) --
	(235.22, 54.89) --
	(235.06, 55.05) --
	(233.25, 56.86) --
	(233.10, 57.01) --
	(231.29, 58.82) --
	(231.13, 58.98) --
	(229.32, 60.79) --
	(229.16, 60.95) --
	(227.35, 62.76) --
	(227.19, 62.91) --
	(225.39, 64.72) --
	(225.23, 64.88) --
	(223.42, 66.69) --
	(223.26, 66.85) --
	(221.45, 68.66) --
	(221.29, 68.82) --
	(219.48, 70.63) --
	(219.33, 70.78) --
	(217.52, 72.59) --
	(217.36, 72.75) --
	(215.55, 74.56);

\path[draw=drawColor,draw opacity=0.33,line width= 1.1pt,line join=round] (215.55, 78.81) --
	(217.36, 80.62) --
	(217.52, 80.78) --
	(219.33, 82.58) --
	(219.48, 82.74) --
	(221.29, 84.55) --
	(221.45, 84.71) --
	(223.26, 86.52) --
	(223.42, 86.68) --
	(225.23, 88.49) --
	(225.39, 88.64) --
	(227.19, 90.45) --
	(227.35, 90.61) --
	(229.16, 92.42) --
	(229.32, 92.58) --
	(231.13, 94.39) --
	(231.29, 94.54) --
	(233.10, 96.35) --
	(233.25, 96.51) --
	(235.06, 98.32) --
	(235.22, 98.48) --
	(237.03,100.29) --
	(237.19,100.45) --
	(239.00,102.26) --
	(239.15,102.41) --
	(240.96,104.22) --
	(241.12,104.38) --
	(242.93,106.19) --
	(243.09,106.35) --
	(244.90,108.16) --
	(245.06,108.31) --
	(246.87,110.12) --
	(247.02,110.28) --
	(248.83,112.09) --
	(248.99,112.25) --
	(250.80,114.06) --
	(250.96,114.22) --
	(252.77,116.02);

\path[draw=drawColor,draw opacity=0.33,line width= 1.1pt,line join=round] (257.02, 37.34) --
	(258.83, 39.15) --
	(258.98, 39.31) --
	(260.79, 41.12) --
	(260.95, 41.28) --
	(262.76, 43.09) --
	(262.92, 43.24) --
	(264.73, 45.05) --
	(264.88, 45.21) --
	(266.69, 47.02) --
	(266.85, 47.18) --
	(268.66, 48.99) --
	(268.82, 49.15) --
	(270.63, 50.95) --
	(270.78, 51.11) --
	(272.59, 52.92) --
	(272.75, 53.08) --
	(274.56, 54.89) --
	(274.72, 55.05) --
	(276.53, 56.86) --
	(276.69, 57.01) --
	(278.50, 58.82) --
	(278.65, 58.98) --
	(280.46, 60.79) --
	(280.62, 60.95) --
	(282.43, 62.76) --
	(282.59, 62.91) --
	(284.40, 64.72) --
	(284.55, 64.88) --
	(286.36, 66.69) --
	(286.52, 66.85) --
	(288.33, 68.66) --
	(288.49, 68.82) --
	(290.30, 70.63) --
	(290.46, 70.78) --
	(292.26, 72.59) --
	(292.42, 72.75) --
	(294.23, 74.56);

\path[draw=drawColor,draw opacity=0.33,line width= 1.1pt,line join=round] (294.23, 78.81) --
	(292.42, 80.62) --
	(292.26, 80.78) --
	(290.46, 82.58) --
	(290.30, 82.74) --
	(288.49, 84.55) --
	(288.33, 84.71) --
	(286.52, 86.52) --
	(286.36, 86.68) --
	(284.55, 88.49) --
	(284.40, 88.64) --
	(282.59, 90.45) --
	(282.43, 90.61) --
	(280.62, 92.42) --
	(280.46, 92.58) --
	(278.65, 94.39) --
	(278.50, 94.54) --
	(276.69, 96.35) --
	(276.53, 96.51) --
	(274.72, 98.32) --
	(274.56, 98.48) --
	(272.75,100.29) --
	(272.59,100.45) --
	(270.78,102.26) --
	(270.63,102.41) --
	(268.82,104.22) --
	(268.66,104.38) --
	(266.85,106.19) --
	(266.69,106.35) --
	(264.88,108.16) --
	(264.73,108.31) --
	(262.92,110.12) --
	(262.76,110.28) --
	(260.95,112.09) --
	(260.79,112.25) --
	(258.98,114.06) --
	(258.83,114.22) --
	(257.02,116.02);

\path[draw=drawColor,draw opacity=0.33,line width= 1.1pt,line join=round] (243.45, 37.34) --
	(243.09, 37.70) --
	(241.48, 39.31) --
	(241.12, 39.67) --
	(239.51, 41.28) --
	(239.15, 41.64) --
	(237.55, 43.24) --
	(237.19, 43.60) --
	(235.58, 45.21) --
	(235.22, 45.57) --
	(233.61, 47.18) --
	(233.25, 47.54) --
	(231.65, 49.15) --
	(231.29, 49.51) --
	(229.68, 51.11) --
	(229.32, 51.47) --
	(227.71, 53.08) --
	(227.35, 53.44) --
	(225.75, 55.05) --
	(225.39, 55.41) --
	(223.78, 57.01) --
	(223.42, 57.37) --
	(221.81, 58.98) --
	(221.45, 59.34) --
	(219.84, 60.95) --
	(219.48, 61.31) --
	(217.88, 62.91) --
	(217.52, 63.27) --
	(215.91, 64.88) --
	(215.55, 65.24);

\path[draw=drawColor,draw opacity=0.33,line width= 1.1pt,line join=round] (215.55, 88.13) --
	(215.91, 88.49) --
	(217.52, 90.09) --
	(217.88, 90.45) --
	(219.48, 92.06) --
	(219.84, 92.42) --
	(221.45, 94.03) --
	(221.81, 94.39) --
	(223.42, 95.99) --
	(223.78, 96.35) --
	(225.39, 97.96) --
	(225.75, 98.32) --
	(227.35, 99.93) --
	(227.71,100.29) --
	(229.32,101.90) --
	(229.68,102.26) --
	(231.29,103.86) --
	(231.65,104.22) --
	(233.25,105.83) --
	(233.61,106.19) --
	(235.22,107.80) --
	(235.58,108.16) --
	(237.19,109.76) --
	(237.55,110.12) --
	(239.15,111.73) --
	(239.51,112.09) --
	(241.12,113.70) --
	(241.48,114.06) --
	(243.09,115.66) --
	(243.45,116.02);

\path[draw=drawColor,draw opacity=0.33,line width= 1.1pt,line join=round] (266.33, 37.34) --
	(266.69, 37.70) --
	(268.30, 39.31) --
	(268.66, 39.67) --
	(270.27, 41.28) --
	(270.63, 41.64) --
	(272.23, 43.24) --
	(272.59, 43.60) --
	(274.20, 45.21) --
	(274.56, 45.57) --
	(276.17, 47.18) --
	(276.53, 47.54) --
	(278.14, 49.15) --
	(278.50, 49.51) --
	(280.10, 51.11) --
	(280.46, 51.47) --
	(282.07, 53.08) --
	(282.43, 53.44) --
	(284.04, 55.05) --
	(284.40, 55.41) --
	(286.00, 57.01) --
	(286.36, 57.37) --
	(287.97, 58.98) --
	(288.33, 59.34) --
	(289.94, 60.95) --
	(290.30, 61.31) --
	(291.90, 62.91) --
	(292.26, 63.27) --
	(293.87, 64.88) --
	(294.23, 65.24);

\path[draw=drawColor,draw opacity=0.33,line width= 1.1pt,line join=round] (294.23, 88.13) --
	(293.87, 88.49) --
	(292.26, 90.09) --
	(291.90, 90.45) --
	(290.30, 92.06) --
	(289.94, 92.42) --
	(288.33, 94.03) --
	(287.97, 94.39) --
	(286.36, 95.99) --
	(286.00, 96.35) --
	(284.40, 97.96) --
	(284.04, 98.32) --
	(282.43, 99.93) --
	(282.07,100.29) --
	(280.46,101.90) --
	(280.10,102.26) --
	(278.50,103.86) --
	(278.14,104.22) --
	(276.53,105.83) --
	(276.17,106.19) --
	(274.56,107.80) --
	(274.20,108.16) --
	(272.59,109.76) --
	(272.23,110.12) --
	(270.63,111.73) --
	(270.27,112.09) --
	(268.66,113.70) --
	(268.30,114.06) --
	(266.69,115.66) --
	(266.33,116.02);

\path[draw=drawColor,draw opacity=0.33,line width= 1.1pt,line join=round] (235.59, 37.34) --
	(235.22, 37.71) --
	(233.62, 39.31) --
	(233.25, 39.68) --
	(231.66, 41.28) --
	(231.29, 41.65) --
	(229.69, 43.24) --
	(229.32, 43.61) --
	(227.72, 45.21) --
	(227.35, 45.58) --
	(225.76, 47.18) --
	(225.39, 47.55) --
	(223.79, 49.15) --
	(223.42, 49.52) --
	(221.82, 51.11) --
	(221.45, 51.48) --
	(219.85, 53.08) --
	(219.48, 53.45) --
	(217.89, 55.05) --
	(217.52, 55.42) --
	(215.92, 57.01) --
	(215.55, 57.38);

\path[draw=drawColor,draw opacity=0.33,line width= 1.1pt,line join=round] (215.55, 95.98) --
	(215.92, 96.35) --
	(217.52, 97.95) --
	(217.89, 98.32) --
	(219.48, 99.92) --
	(219.85,100.29) --
	(221.45,101.88) --
	(221.82,102.26) --
	(223.42,103.85) --
	(223.79,104.22) --
	(225.39,105.82) --
	(225.76,106.19) --
	(227.35,107.79) --
	(227.72,108.16) --
	(229.32,109.75) --
	(229.69,110.12) --
	(231.29,111.72) --
	(231.66,112.09) --
	(233.25,113.69) --
	(233.62,114.06) --
	(235.22,115.65) --
	(235.59,116.02);

\path[draw=drawColor,draw opacity=0.33,line width= 1.1pt,line join=round] (274.19, 37.34) --
	(274.56, 37.71) --
	(276.16, 39.31) --
	(276.53, 39.68) --
	(278.12, 41.28) --
	(278.50, 41.65) --
	(280.09, 43.24) --
	(280.46, 43.61) --
	(282.06, 45.21) --
	(282.43, 45.58) --
	(284.03, 47.18) --
	(284.40, 47.55) --
	(285.99, 49.15) --
	(286.36, 49.52) --
	(287.96, 51.11) --
	(288.33, 51.48) --
	(289.93, 53.08) --
	(290.30, 53.45) --
	(291.89, 55.05) --
	(292.26, 55.42) --
	(293.86, 57.01) --
	(294.23, 57.38);

\path[draw=drawColor,draw opacity=0.33,line width= 1.1pt,line join=round] (294.23, 95.98) --
	(293.86, 96.35) --
	(292.26, 97.95) --
	(291.89, 98.32) --
	(290.30, 99.92) --
	(289.93,100.29) --
	(288.33,101.88) --
	(287.96,102.26) --
	(286.36,103.85) --
	(285.99,104.22) --
	(284.40,105.82) --
	(284.03,106.19) --
	(282.43,107.79) --
	(282.06,108.16) --
	(280.46,109.75) --
	(280.09,110.12) --
	(278.50,111.72) --
	(278.12,112.09) --
	(276.53,113.69) --
	(276.16,114.06) --
	(274.56,115.65) --
	(274.19,116.02);

\path[draw=drawColor,draw opacity=0.33,line width= 1.1pt,line join=round] (228.67, 37.34) --
	(227.35, 38.66) --
	(226.70, 39.31) --
	(225.39, 40.63) --
	(224.74, 41.28) --
	(223.42, 42.60) --
	(222.77, 43.24) --
	(221.45, 44.56) --
	(220.80, 45.21) --
	(219.48, 46.53) --
	(218.84, 47.18) --
	(217.52, 48.50) --
	(216.87, 49.15) --
	(215.55, 50.46);

\path[draw=drawColor,draw opacity=0.33,line width= 1.1pt,line join=round] (215.55,102.90) --
	(216.87,104.22) --
	(217.52,104.87) --
	(218.84,106.19) --
	(219.48,106.84) --
	(220.80,108.16) --
	(221.45,108.81) --
	(222.77,110.12) --
	(223.42,110.77) --
	(224.74,112.09) --
	(225.39,112.74) --
	(226.70,114.06) --
	(227.35,114.71) --
	(228.67,116.02);

\path[draw=drawColor,draw opacity=0.33,line width= 1.1pt,line join=round] (281.11, 37.34) --
	(282.43, 38.66) --
	(283.08, 39.31) --
	(284.40, 40.63) --
	(285.05, 41.28) --
	(286.36, 42.60) --
	(287.01, 43.24) --
	(288.33, 44.56) --
	(288.98, 45.21) --
	(290.30, 46.53) --
	(290.95, 47.18) --
	(292.26, 48.50) --
	(292.91, 49.15) --
	(294.23, 50.46);

\path[draw=drawColor,draw opacity=0.33,line width= 1.1pt,line join=round] (294.23,102.90) --
	(292.91,104.22) --
	(292.26,104.87) --
	(290.95,106.19) --
	(290.30,106.84) --
	(288.98,108.16) --
	(288.33,108.81) --
	(287.01,110.12) --
	(286.36,110.77) --
	(285.05,112.09) --
	(284.40,112.74) --
	(283.08,114.06) --
	(282.43,114.71) --
	(281.11,116.02);

\path[draw=drawColor,draw opacity=0.33,line width= 1.1pt,line join=round] (222.41, 37.34) --
	(221.45, 38.30) --
	(220.45, 39.31) --
	(219.48, 40.27) --
	(218.48, 41.28) --
	(217.52, 42.24) --
	(216.51, 43.24) --
	(215.55, 44.21);

\path[draw=drawColor,draw opacity=0.33,line width= 1.1pt,line join=round] (215.55,109.16) --
	(216.51,110.12) --
	(217.52,111.13) --
	(218.48,112.09) --
	(219.48,113.10) --
	(220.45,114.06) --
	(221.45,115.06) --
	(222.41,116.02);

\path[draw=drawColor,draw opacity=0.33,line width= 1.1pt,line join=round] (287.37, 37.34) --
	(288.33, 38.30) --
	(289.34, 39.31) --
	(290.30, 40.27) --
	(291.30, 41.28) --
	(292.26, 42.24) --
	(293.27, 43.24) --
	(294.23, 44.21);

\path[draw=drawColor,draw opacity=0.33,line width= 1.1pt,line join=round] (294.23,109.16) --
	(293.27,110.12) --
	(292.26,111.13) --
	(291.30,112.09) --
	(290.30,113.10) --
	(289.34,114.06) --
	(288.33,115.06) --
	(287.37,116.02);

\path[draw=drawColor,draw opacity=0.33,line width= 1.1pt,line join=round] (216.66, 37.34) --
	(215.55, 38.45);

\path[draw=drawColor,draw opacity=0.33,line width= 1.1pt,line join=round] (215.55,114.92) --
	(216.66,116.02);

\path[draw=drawColor,draw opacity=0.33,line width= 1.1pt,line join=round] (293.13, 37.34) --
	(294.23, 38.45);

\path[draw=drawColor,draw opacity=0.33,line width= 1.1pt,line join=round] (294.23,114.92) --
	(293.13,116.02);
\definecolor{drawColor}{RGB}{0,0,0}

\path[draw=drawColor,line width= 1.1pt,line join=round] (215.55, 96.31) --
	(215.66, 96.35) --
	(217.52, 97.11) --
	(219.48, 98.07) --
	(219.93, 98.32) --
	(221.45, 99.19) --
	(223.13,100.29) --
	(223.42,100.48) --
	(225.39,101.96) --
	(225.74,102.26) --
	(227.35,103.65) --
	(227.96,104.22) --
	(229.32,105.58) --
	(229.90,106.19) --
	(231.29,107.79) --
	(231.59,108.16) --
	(233.08,110.12) --
	(233.25,110.37) --
	(234.39,112.09) --
	(235.22,113.52) --
	(235.53,114.06) --
	(236.52,116.02);

\path[draw=drawColor,line width= 1.1pt,line join=round] (277.85, 37.34) --
	(278.13, 39.31) --
	(278.50, 40.93) --
	(278.57, 41.28) --
	(279.16, 43.24) --
	(279.92, 45.21) --
	(280.46, 46.34) --
	(280.86, 47.18) --
	(281.99, 49.15) --
	(282.43, 49.82) --
	(283.30, 51.11) --
	(284.40, 52.54) --
	(284.82, 53.08) --
	(286.36, 54.84) --
	(286.56, 55.05) --
	(288.33, 56.84) --
	(288.52, 57.01) --
	(290.30, 58.62) --
	(290.73, 58.98) --
	(292.26, 60.22) --
	(293.24, 60.95) --
	(294.23, 61.69);

\path[draw=drawColor,line width= 1.1pt,line join=round] (294.23,100.79) --
	(292.26,101.85) --
	(291.60,102.26) --
	(290.30,103.09) --
	(288.71,104.22) --
	(288.33,104.51) --
	(286.36,106.18) --
	(286.35,106.19) --
	(284.42,108.16) --
	(284.40,108.19) --
	(282.84,110.12) --
	(282.43,110.72) --
	(281.54,112.09) --
	(280.49,114.06) --
	(280.46,114.13) --
	(279.68,116.02);

\path[draw=drawColor,line width= 1.1pt,line join=round] (223.36, 37.34) --
	(222.69, 39.31) --
	(221.75, 41.28) --
	(221.45, 41.78) --
	(220.42, 43.24) --
	(219.48, 44.32) --
	(218.55, 45.21) --
	(217.52, 46.05) --
	(215.75, 47.18) --
	(215.55, 47.29);

\path[draw=drawColor,line width= 1.1pt,line join=round] (215.55,100.77) --
	(217.52,101.65) --
	(218.66,102.26) --
	(219.48,102.71) --
	(221.45,103.97) --
	(221.80,104.22) --
	(223.42,105.46) --
	(224.27,106.19) --
	(225.39,107.22) --
	(226.31,108.16) --
	(227.35,109.31) --
	(228.04,110.12) --
	(229.32,111.84) --
	(229.50,112.09) --
	(230.73,114.06) --
	(231.29,115.09) --
	(231.78,116.02);

\path[draw=drawColor,line width= 1.1pt,line join=round] (281.86, 37.34) --
	(282.15, 39.31) --
	(282.43, 40.51) --
	(282.61, 41.28) --
	(283.24, 43.24) --
	(284.06, 45.21) --
	(284.40, 45.87) --
	(285.08, 47.18) --
	(286.31, 49.15) --
	(286.36, 49.22) --
	(287.78, 51.11) --
	(288.33, 51.76) --
	(289.51, 53.08) --
	(290.30, 53.88) --
	(291.53, 55.05) --
	(292.26, 55.69) --
	(293.89, 57.01) --
	(294.23, 57.28);

\path[draw=drawColor,line width= 1.1pt,line join=round] (294.23,106.54) --
	(292.26,107.84) --
	(291.85,108.16) --
	(290.30,109.49) --
	(289.64,110.12) --
	(288.33,111.61) --
	(287.95,112.09) --
	(286.65,114.06) --
	(286.36,114.62) --
	(285.69,116.02);

\path[draw=drawColor,line width= 1.1pt,line join=round] (215.55,105.27) --
	(217.32,106.19) --
	(217.52,106.30) --
	(219.48,107.58) --
	(220.25,108.16) --
	(221.45,109.14) --
	(222.52,110.12) --
	(223.42,111.06) --
	(224.32,112.09) --
	(225.39,113.47) --
	(225.80,114.06) --
	(227.00,116.02);

\path[draw=drawColor,line width= 1.1pt,line join=round] (285.83, 37.34) --
	(286.15, 39.31) --
	(286.36, 40.17) --
	(286.65, 41.28) --
	(287.35, 43.24) --
	(288.26, 45.21) --
	(288.33, 45.33) --
	(289.43, 47.18) --
	(290.30, 48.40) --
	(290.86, 49.15) --
	(292.26, 50.77) --
	(292.59, 51.11) --
	(294.23, 52.68);

\path[draw=drawColor,line width= 1.1pt,line join=round] (294.23,114.29) --
	(292.92,116.02);

\path[draw=drawColor,line width= 1.1pt,line join=round] (215.55,110.25) --
	(217.52,111.61) --
	(218.11,112.09) --
	(219.48,113.37) --
	(220.14,114.06) --
	(221.45,115.67) --
	(221.71,116.02);

\path[draw=drawColor,line width= 1.1pt,line join=round] (289.99, 37.34) --
	(290.30, 39.08) --
	(290.34, 39.31) --
	(290.93, 41.28) --
	(291.74, 43.24) --
	(292.26, 44.23) --
	(292.83, 45.21) --
	(294.22, 47.18) --
	(294.23, 47.19);
\definecolor{drawColor}{gray}{0.50}

\path[draw=drawColor,line width= 0.6pt,line join=round,line cap=round] (211.62, 33.41) rectangle (298.17,119.96);
\end{scope}
\begin{scope}
\path[clip] (298.17, 33.41) rectangle (384.72,119.96);
\definecolor{fillColor}{RGB}{255,255,255}

\path[fill=fillColor] (298.17, 33.41) rectangle (384.72,119.96);
\definecolor{drawColor}{RGB}{0,0,0}

\path[draw=drawColor,line width= 1.1pt,line join=round] (329.64, 73.33) --
	(329.15, 74.72) --
	(328.79, 76.68) --
	(328.82, 78.65) --
	(329.23, 80.62) --
	(329.64, 81.73) --
	(329.96, 82.58) --
	(330.97, 84.55) --
	(331.61, 85.60) --
	(332.19, 86.52) --
	(333.57, 88.47) --
	(333.58, 88.49) --
	(335.26, 90.45) --
	(335.54, 90.76) --
	(337.33, 92.42) --
	(337.51, 92.57) --
	(339.47, 93.80) --
	(341.20, 94.39) --
	(341.44, 94.46) --
	(343.41, 94.51) --
	(343.87, 94.39) --
	(345.37, 93.98) --
	(347.34, 92.99) --
	(348.17, 92.42) --
	(349.31, 91.62) --
	(350.72, 90.45) --
	(351.28, 89.98) --
	(352.88, 88.49) --
	(353.24, 88.13) --
	(354.81, 86.52) --
	(355.21, 86.07) --
	(356.50, 84.55) --
	(357.18, 83.60) --
	(357.88, 82.58) --
	(358.84, 80.62) --
	(359.14, 79.31) --
	(359.29, 78.65) --
	(359.23, 76.68) --
	(359.14, 76.37) --
	(358.69, 74.72) --
	(357.77, 72.75) --
	(357.18, 71.76) --
	(356.57, 70.78) --
	(355.21, 68.87) --
	(355.17, 68.82) --
	(353.56, 66.85) --
	(353.24, 66.47) --
	(351.77, 64.88) --
	(351.28, 64.35) --
	(349.71, 62.91) --
	(349.31, 62.55) --
	(347.34, 61.16) --
	(346.90, 60.95) --
	(345.37, 60.27) --
	(343.41, 59.93) --
	(341.44, 60.16) --
	(339.52, 60.95) --
	(339.47, 60.97) --
	(337.51, 62.39) --
	(336.95, 62.91) --
	(335.54, 64.32) --
	(335.05, 64.88) --
	(333.57, 66.69) --
	(333.45, 66.85) --
	(332.07, 68.82) --
	(331.61, 69.56) --
	(330.86, 70.78) --
	(329.85, 72.75) --
	(329.64, 73.33);

\path[draw=drawColor,line width= 1.1pt,line join=round] (321.77, 71.96) --
	(321.22, 72.75) --
	(320.32, 74.72) --
	(319.95, 76.68) --
	(320.16, 78.65) --
	(320.88, 80.62) --
	(321.77, 82.14) --
	(322.01, 82.58) --
	(323.38, 84.55) --
	(323.74, 84.99) --
	(324.89, 86.52) --
	(325.70, 87.53) --
	(326.44, 88.49) --
	(327.67, 90.02) --
	(328.01, 90.45) --
	(329.57, 92.42) --
	(329.64, 92.51) --
	(331.16, 94.39) --
	(331.61, 94.94) --
	(332.82, 96.35) --
	(333.57, 97.24) --
	(334.63, 98.32) --
	(335.54, 99.26) --
	(336.81,100.29) --
	(337.51,100.85) --
	(339.47,101.92) --
	(340.78,102.26) --
	(341.44,102.42) --
	(343.41,102.33) --
	(343.63,102.26) --
	(345.37,101.68) --
	(347.34,100.56) --
	(347.69,100.29) --
	(349.31, 99.05) --
	(350.10, 98.32) --
	(351.28, 97.28) --
	(352.23, 96.35) --
	(353.24, 95.39) --
	(354.26, 94.39) --
	(355.21, 93.48) --
	(356.31, 92.42) --
	(357.18, 91.60) --
	(358.41, 90.45) --
	(359.14, 89.76) --
	(360.55, 88.49) --
	(361.11, 87.93) --
	(362.64, 86.52) --
	(363.08, 86.04) --
	(364.52, 84.55) --
	(365.04, 83.85) --
	(366.02, 82.58) --
	(366.95, 80.62) --
	(367.01, 80.26) --
	(367.27, 78.65) --
	(367.01, 76.92) --
	(366.98, 76.68) --
	(366.18, 74.72) --
	(365.04, 72.81) --
	(365.01, 72.75) --
	(363.58, 70.78) --
	(363.08, 70.18) --
	(362.00, 68.82) --
	(361.11, 67.77) --
	(360.35, 66.85) --
	(359.14, 65.42) --
	(358.69, 64.88) --
	(357.18, 63.07) --
	(357.05, 62.91) --
	(355.38, 60.95) --
	(355.21, 60.74) --
	(353.66, 58.98) --
	(353.24, 58.50) --
	(351.82, 57.01) --
	(351.28, 56.45) --
	(349.68, 55.05) --
	(349.31, 54.73) --
	(347.34, 53.42) --
	(346.58, 53.08) --
	(345.37, 52.57) --
	(343.41, 52.20) --
	(341.44, 52.32) --
	(339.47, 52.92) --
	(339.19, 53.08) --
	(337.51, 54.03) --
	(336.21, 55.05) --
	(335.54, 55.59) --
	(334.10, 57.01) --
	(333.57, 57.55) --
	(332.31, 58.98) --
	(331.61, 59.79) --
	(330.64, 60.95) --
	(329.64, 62.15) --
	(329.01, 62.91) --
	(327.67, 64.53) --
	(327.37, 64.88) --
	(325.71, 66.85) --
	(325.70, 66.86) --
	(324.07, 68.82) --
	(323.74, 69.27) --
	(322.53, 70.78) --
	(321.77, 71.96);

\path[draw=drawColor,line width= 1.1pt,line join=round] (309.97, 74.37) --
	(309.73, 74.72) --
	(309.41, 76.68) --
	(309.97, 78.34) --
	(310.06, 78.65) --
	(311.50, 80.62) --
	(311.93, 81.01) --
	(313.47, 82.58) --
	(313.90, 82.93) --
	(315.70, 84.55) --
	(315.87, 84.68) --
	(317.84, 86.41) --
	(317.95, 86.52) --
	(319.80, 88.19) --
	(320.09, 88.49) --
	(321.77, 90.12) --
	(322.09, 90.45) --
	(323.74, 92.21) --
	(323.92, 92.42) --
	(325.62, 94.39) --
	(325.70, 94.49) --
	(327.22, 96.35) --
	(327.67, 96.95) --
	(328.74, 98.32) --
	(329.64, 99.54) --
	(330.22,100.29) --
	(331.61,102.15) --
	(331.69,102.26) --
	(333.28,104.22) --
	(333.57,104.60) --
	(335.06,106.19) --
	(335.54,106.71) --
	(337.31,108.16) --
	(337.51,108.32) --
	(339.47,109.35) --
	(341.44,109.78) --
	(343.41,109.61) --
	(345.37,108.85) --
	(346.41,108.16) --
	(347.34,107.56) --
	(348.87,106.19) --
	(349.31,105.82) --
	(350.82,104.22) --
	(351.28,103.77) --
	(352.61,102.26) --
	(353.24,101.58) --
	(354.37,100.29) --
	(355.21, 99.39) --
	(356.19, 98.32) --
	(357.18, 97.30) --
	(358.12, 96.35) --
	(359.14, 95.37) --
	(360.22, 94.39) --
	(361.11, 93.60) --
	(362.54, 92.42) --
	(363.08, 91.98) --
	(365.04, 90.48) --
	(365.09, 90.45) --
	(367.01, 89.03) --
	(367.83, 88.49) --
	(368.98, 87.61) --
	(370.53, 86.52) --
	(370.95, 86.16) --
	(372.88, 84.55) --
	(372.91, 84.51) --
	(374.56, 82.58) --
	(374.88, 81.84) --
	(375.40, 80.62) --
	(375.42, 78.65) --
	(374.88, 77.13) --
	(374.74, 76.68) --
	(373.51, 74.72) --
	(372.91, 74.03) --
	(371.88, 72.75) --
	(370.95, 71.80) --
	(370.02, 70.78) --
	(368.98, 69.77) --
	(368.06, 68.82) --
	(367.01, 67.79) --
	(366.10, 66.85) --
	(365.04, 65.78) --
	(364.21, 64.88) --
	(363.08, 63.67) --
	(362.41, 62.91) --
	(361.11, 61.42) --
	(360.70, 60.95) --
	(359.14, 59.05) --
	(359.09, 58.98) --
	(357.52, 57.01) --
	(357.18, 56.57) --
	(355.95, 55.05) --
	(355.21, 54.11) --
	(354.33, 53.08) --
	(353.24, 51.79) --
	(352.59, 51.11) --
	(351.28, 49.77) --
	(350.53, 49.15) --
	(349.31, 48.14) --
	(347.75, 47.18) --
	(347.34, 46.93) --
	(345.37, 46.16) --
	(343.41, 45.82) --
	(341.44, 45.90) --
	(339.47, 46.41) --
	(337.85, 47.18) --
	(337.51, 47.35) --
	(335.54, 48.72) --
	(335.06, 49.15) --
	(333.57, 50.50) --
	(333.00, 51.11) --
	(331.61, 52.62) --
	(331.22, 53.08) --
	(329.64, 54.96) --
	(329.57, 55.05) --
	(327.95, 57.01) --
	(327.67, 57.35) --
	(326.29, 58.98) --
	(325.70, 59.65) --
	(324.51, 60.95) --
	(323.74, 61.78) --
	(322.58, 62.91) --
	(321.77, 63.71) --
	(320.44, 64.88) --
	(319.80, 65.46) --
	(318.09, 66.85) --
	(317.84, 67.07) --
	(315.87, 68.61) --
	(315.57, 68.82) --
	(313.90, 70.18) --
	(313.08, 70.78) --
	(311.93, 71.92) --
	(311.01, 72.75) --
	(309.97, 74.37);

\path[draw=drawColor,line width= 1.1pt,line join=round] (302.10, 68.52) --
	(304.07, 67.84) --
	(306.03, 67.12) --
	(306.68, 66.85) --
	(308.00, 66.38) --
	(309.97, 65.59) --
	(311.46, 64.88) --
	(311.93, 64.68) --
	(313.90, 63.67) --
	(315.15, 62.91) --
	(315.87, 62.49) --
	(317.84, 61.14) --
	(318.08, 60.95) --
	(319.80, 59.56) --
	(320.45, 58.98) --
	(321.77, 57.73) --
	(322.46, 57.01) --
	(323.74, 55.63) --
	(324.24, 55.05) --
	(325.70, 53.28) --
	(325.86, 53.08) --
	(327.40, 51.11) --
	(327.67, 50.76) --
	(328.93, 49.15) --
	(329.64, 48.23) --
	(330.51, 47.18) --
	(331.61, 45.88) --
	(332.24, 45.21) --
	(333.57, 43.84) --
	(334.28, 43.24) --
	(335.54, 42.19) --
	(337.00, 41.28) --
	(337.51, 40.96) --
	(339.47, 40.12) --
	(341.44, 39.68) --
	(343.41, 39.61) --
	(345.37, 39.91) --
	(347.34, 40.59) --
	(348.57, 41.28) --
	(349.31, 41.68) --
	(351.28, 43.20) --
	(351.32, 43.24) --
	(353.24, 45.13) --
	(353.31, 45.21) --
	(354.97, 47.18) --
	(355.21, 47.47) --
	(356.46, 49.15) --
	(357.18, 50.09) --
	(357.89, 51.11) --
	(359.14, 52.85) --
	(359.30, 53.08) --
	(360.73, 55.05) --
	(361.11, 55.54) --
	(362.24, 57.01) --
	(363.08, 58.05) --
	(363.86, 58.98) --
	(365.04, 60.33) --
	(365.62, 60.95) --
	(367.01, 62.38) --
	(367.56, 62.91) --
	(368.98, 64.25) --
	(369.71, 64.88) --
	(370.95, 65.96) --
	(372.05, 66.85) --
	(372.91, 67.56) --
	(374.57, 68.82) --
	(374.88, 69.07) --
	(376.85, 70.55) --
	(377.19, 70.78) --
	(378.81, 72.05) --
	(379.79, 72.75) --
	(380.78, 73.61);

\path[draw=drawColor,line width= 1.1pt,line join=round] (302.10, 83.18) --
	(304.07, 84.10) --
	(304.97, 84.55) --
	(306.03, 85.00) --
	(308.00, 85.93) --
	(309.10, 86.52) --
	(309.97, 86.93) --
	(311.93, 88.01) --
	(312.69, 88.49) --
	(313.90, 89.21) --
	(315.71, 90.45) --
	(315.87, 90.56) --
	(317.84, 92.08) --
	(318.23, 92.42) --
	(319.80, 93.82) --
	(320.38, 94.39) --
	(321.77, 95.82) --
	(322.25, 96.35) --
	(323.74, 98.11) --
	(323.90, 98.32) --
	(325.39,100.29) --
	(325.70,100.74) --
	(326.77,102.26) --
	(327.67,103.66) --
	(328.05,104.22) --
	(329.29,106.19) --
	(329.64,106.78) --
	(330.53,108.16) --
	(331.61,109.89) --
	(331.77,110.12) --
	(333.13,112.09) --
	(333.57,112.75) --
	(334.69,114.06) --
	(335.54,115.08) --
	(336.64,116.02);

\path[draw=drawColor,line width= 1.1pt,line join=round] (380.78, 86.95) --
	(378.81, 87.89) --
	(377.63, 88.49) --
	(376.85, 88.80) --
	(374.88, 89.67) --
	(373.29, 90.45) --
	(372.91, 90.61) --
	(370.95, 91.58) --
	(369.46, 92.42) --
	(368.98, 92.68) --
	(367.01, 93.87) --
	(366.27, 94.39) --
	(365.04, 95.23) --
	(363.61, 96.35) --
	(363.08, 96.78) --
	(361.35, 98.32) --
	(361.11, 98.55) --
	(359.41,100.29) --
	(359.14,100.58) --
	(357.68,102.26) --
	(357.18,102.89) --
	(356.11,104.22) --
	(355.21,105.45) --
	(354.64,106.19) --
	(353.25,108.16) --
	(353.24,108.17) --
	(351.82,110.12) --
	(351.28,110.92) --
	(350.33,112.09) --
	(349.31,113.42) --
	(348.69,114.06) --
	(347.34,115.50) --
	(346.64,116.02);

\path[draw=drawColor,line width= 1.1pt,line join=round] (302.10, 62.25) --
	(304.07, 61.93) --
	(306.03, 61.51) --
	(308.00, 60.98) --
	(308.09, 60.95) --
	(309.97, 60.32) --
	(311.93, 59.50) --
	(312.94, 58.98) --
	(313.90, 58.48) --
	(315.87, 57.24) --
	(316.17, 57.01) --
	(317.84, 55.72) --
	(318.59, 55.05) --
	(319.80, 53.88) --
	(320.56, 53.08) --
	(321.77, 51.70) --
	(322.26, 51.11) --
	(323.74, 49.19) --
	(323.77, 49.15) --
	(325.17, 47.18) --
	(325.70, 46.39) --
	(326.53, 45.21) --
	(327.67, 43.53) --
	(327.88, 43.24) --
	(329.30, 41.28) --
	(329.64, 40.81) --
	(330.86, 39.31) --
	(331.61, 38.40) --
	(332.64, 37.34);

\path[draw=drawColor,line width= 1.1pt,line join=round] (302.10, 89.24) --
	(304.07, 89.76) --
	(306.03, 90.37) --
	(306.24, 90.45) --
	(308.00, 91.08) --
	(309.97, 91.93) --
	(310.94, 92.42) --
	(311.93, 92.93) --
	(313.90, 94.11) --
	(314.30, 94.39) --
	(315.87, 95.52) --
	(316.87, 96.35) --
	(317.84, 97.20) --
	(318.96, 98.32) --
	(319.80, 99.23) --
	(320.71,100.29) --
	(321.77,101.66) --
	(322.20,102.26) --
	(323.51,104.22) --
	(323.74,104.61) --
	(324.66,106.19) --
	(325.70,108.15) --
	(325.71,108.16) --
	(326.67,110.12) --
	(327.55,112.09) --
	(327.67,112.38) --
	(328.41,114.06) --
	(329.23,116.02);

\path[draw=drawColor,line width= 1.1pt,line join=round] (353.08, 37.34) --
	(353.24, 37.50) --
	(354.80, 39.31) --
	(355.21, 39.80) --
	(356.27, 41.28) --
	(357.18, 42.56) --
	(357.60, 43.24) --
	(358.85, 45.21) --
	(359.14, 45.67) --
	(360.05, 47.18) --
	(361.11, 48.89) --
	(361.26, 49.15) --
	(362.52, 51.11) --
	(363.08, 51.94) --
	(363.86, 53.08) --
	(365.04, 54.70) --
	(365.31, 55.05) --
	(366.92, 57.01) --
	(367.01, 57.11) --
	(368.76, 58.98) --
	(368.98, 59.20) --
	(370.86, 60.95) --
	(370.95, 61.03) --
	(372.91, 62.63) --
	(373.30, 62.91) --
	(374.88, 64.06) --
	(376.14, 64.88) --
	(376.85, 65.35) --
	(378.81, 66.53) --
	(379.42, 66.85) --
	(380.78, 67.62);

\path[draw=drawColor,line width= 1.1pt,line join=round] (380.78, 92.76) --
	(378.81, 93.18) --
	(376.85, 93.70) --
	(374.88, 94.33) --
	(374.74, 94.39) --
	(372.91, 95.07) --
	(370.95, 95.96) --
	(370.21, 96.35) --
	(368.98, 97.02) --
	(367.01, 98.28) --
	(366.96, 98.32) --
	(365.04, 99.80) --
	(364.49,100.29) --
	(363.08,101.62) --
	(362.46,102.26) --
	(361.11,103.80) --
	(360.76,104.22) --
	(359.30,106.19) --
	(359.14,106.42) --
	(358.01,108.16) --
	(357.18,109.58) --
	(356.85,110.12) --
	(355.78,112.09) --
	(355.21,113.26) --
	(354.79,114.06) --
	(353.83,116.02);

\path[draw=drawColor,line width= 1.1pt,line join=round] (302.10, 56.83) --
	(304.07, 56.67) --
	(306.03, 56.35) --
	(308.00, 55.87) --
	(309.97, 55.20) --
	(310.30, 55.05) --
	(311.93, 54.26) --
	(313.90, 53.08) --
	(313.90, 53.08) --
	(315.87, 51.52) --
	(316.31, 51.11) --
	(317.84, 49.55) --
	(318.19, 49.15) --
	(319.75, 47.18) --
	(319.80, 47.11) --
	(321.07, 45.21) --
	(321.77, 44.07) --
	(322.27, 43.24) --
	(323.36, 41.28) --
	(323.74, 40.54) --
	(324.39, 39.31) --
	(325.39, 37.34);

\path[draw=drawColor,line width= 1.1pt,line join=round] (302.10, 94.27) --
	(302.71, 94.39) --
	(304.07, 94.64) --
	(306.03, 95.15) --
	(308.00, 95.81) --
	(309.27, 96.35) --
	(309.97, 96.66) --
	(311.93, 97.75) --
	(312.80, 98.32) --
	(313.90, 99.11) --
	(315.30,100.29) --
	(315.87,100.82) --
	(317.21,102.26) --
	(317.84,103.01) --
	(318.74,104.22) --
	(319.80,105.86) --
	(320.01,106.19) --
	(321.04,108.16) --
	(321.77,109.80) --
	(321.91,110.12) --
	(322.62,112.09) --
	(323.22,114.06) --
	(323.70,116.02);
\definecolor{drawColor}{RGB}{0,0,0}

\path[draw=drawColor,draw opacity=0.33,line width= 1.1pt,line join=round] (313.90, 74.91) --
	(312.12, 76.68) --
	(313.90, 78.46) --
	(314.09, 78.65) --
	(315.87, 80.43) --
	(316.06, 80.62) --
	(317.84, 82.40) --
	(318.02, 82.58) --
	(319.80, 84.36) --
	(319.99, 84.55) --
	(321.77, 86.33) --
	(321.96, 86.52) --
	(323.74, 88.30) --
	(323.93, 88.49) --
	(325.70, 90.26) --
	(325.89, 90.45) --
	(327.67, 92.23) --
	(327.86, 92.42) --
	(329.64, 94.20) --
	(329.83, 94.39) --
	(331.61, 96.17) --
	(331.79, 96.35) --
	(333.57, 98.13) --
	(333.76, 98.32) --
	(335.54,100.10) --
	(335.73,100.29) --
	(337.51,102.07) --
	(337.69,102.26) --
	(339.47,104.03) --
	(339.66,104.22) --
	(341.44,106.00) --
	(343.22,104.22) --
	(343.41,104.03) --
	(345.19,102.26) --
	(345.37,102.07) --
	(347.15,100.29) --
	(347.34,100.10) --
	(349.12, 98.32) --
	(349.31, 98.13) --
	(351.09, 96.35) --
	(351.28, 96.17) --
	(353.05, 94.39) --
	(353.24, 94.20) --
	(355.02, 92.42) --
	(355.21, 92.23) --
	(356.99, 90.45) --
	(357.18, 90.26) --
	(358.96, 88.49) --
	(359.14, 88.30) --
	(360.92, 86.52) --
	(361.11, 86.33) --
	(362.89, 84.55) --
	(363.08, 84.36) --
	(364.86, 82.58) --
	(365.04, 82.40) --
	(366.82, 80.62) --
	(367.01, 80.43) --
	(368.79, 78.65) --
	(368.98, 78.46) --
	(370.76, 76.68) --
	(368.98, 74.91) --
	(368.79, 74.72) --
	(367.01, 72.94) --
	(366.82, 72.75) --
	(365.04, 70.97) --
	(364.86, 70.78) --
	(363.08, 69.00) --
	(362.89, 68.82) --
	(361.11, 67.04) --
	(360.92, 66.85) --
	(359.14, 65.07) --
	(358.96, 64.88) --
	(357.18, 63.10) --
	(356.99, 62.91) --
	(355.21, 61.14) --
	(355.02, 60.95) --
	(353.24, 59.17) --
	(353.05, 58.98) --
	(351.28, 57.20) --
	(351.09, 57.01) --
	(349.31, 55.23) --
	(349.12, 55.05) --
	(347.34, 53.27) --
	(347.15, 53.08) --
	(345.37, 51.30) --
	(345.19, 51.11) --
	(343.41, 49.33) --
	(343.22, 49.15) --
	(341.44, 47.37) --
	(339.66, 49.15) --
	(339.47, 49.33) --
	(337.69, 51.11) --
	(337.51, 51.30) --
	(335.73, 53.08) --
	(335.54, 53.27) --
	(333.76, 55.05) --
	(333.57, 55.23) --
	(331.79, 57.01) --
	(331.61, 57.20) --
	(329.83, 58.98) --
	(329.64, 59.17) --
	(327.86, 60.95) --
	(327.67, 61.14) --
	(325.89, 62.91) --
	(325.70, 63.10) --
	(323.93, 64.88) --
	(323.74, 65.07) --
	(321.96, 66.85) --
	(321.77, 67.04) --
	(319.99, 68.82) --
	(319.80, 69.00) --
	(318.02, 70.78) --
	(317.84, 70.97) --
	(316.06, 72.75) --
	(315.87, 72.94) --
	(314.09, 74.72) --
	(313.90, 74.91);

\path[draw=drawColor,draw opacity=0.33,line width= 1.1pt,line join=round] (339.32, 37.34) --
	(337.51, 39.15) --
	(337.35, 39.31) --
	(335.54, 41.12) --
	(335.38, 41.28) --
	(333.57, 43.09) --
	(333.41, 43.24) --
	(331.61, 45.05) --
	(331.45, 45.21) --
	(329.64, 47.02) --
	(329.48, 47.18) --
	(327.67, 48.99) --
	(327.51, 49.15) --
	(325.70, 50.95) --
	(325.55, 51.11) --
	(323.74, 52.92) --
	(323.58, 53.08) --
	(321.77, 54.89) --
	(321.61, 55.05) --
	(319.80, 56.86) --
	(319.65, 57.01) --
	(317.84, 58.82) --
	(317.68, 58.98) --
	(315.87, 60.79) --
	(315.71, 60.95) --
	(313.90, 62.76) --
	(313.74, 62.91) --
	(311.93, 64.72) --
	(311.78, 64.88) --
	(309.97, 66.69) --
	(309.81, 66.85) --
	(308.00, 68.66) --
	(307.84, 68.82) --
	(306.03, 70.63) --
	(305.88, 70.78) --
	(304.07, 72.59) --
	(303.91, 72.75) --
	(302.10, 74.56);

\path[draw=drawColor,draw opacity=0.33,line width= 1.1pt,line join=round] (302.10, 78.81) --
	(303.91, 80.62) --
	(304.07, 80.78) --
	(305.88, 82.58) --
	(306.03, 82.74) --
	(307.84, 84.55) --
	(308.00, 84.71) --
	(309.81, 86.52) --
	(309.97, 86.68) --
	(311.78, 88.49) --
	(311.93, 88.64) --
	(313.74, 90.45) --
	(313.90, 90.61) --
	(315.71, 92.42) --
	(315.87, 92.58) --
	(317.68, 94.39) --
	(317.84, 94.54) --
	(319.65, 96.35) --
	(319.80, 96.51) --
	(321.61, 98.32) --
	(321.77, 98.48) --
	(323.58,100.29) --
	(323.74,100.45) --
	(325.55,102.26) --
	(325.70,102.41) --
	(327.51,104.22) --
	(327.67,104.38) --
	(329.48,106.19) --
	(329.64,106.35) --
	(331.45,108.16) --
	(331.61,108.31) --
	(333.41,110.12) --
	(333.57,110.28) --
	(335.38,112.09) --
	(335.54,112.25) --
	(337.35,114.06) --
	(337.51,114.22) --
	(339.32,116.02);

\path[draw=drawColor,draw opacity=0.33,line width= 1.1pt,line join=round] (343.57, 37.34) --
	(345.37, 39.15) --
	(345.53, 39.31) --
	(347.34, 41.12) --
	(347.50, 41.28) --
	(349.31, 43.09) --
	(349.47, 43.24) --
	(351.28, 45.05) --
	(351.43, 45.21) --
	(353.24, 47.02) --
	(353.40, 47.18) --
	(355.21, 48.99) --
	(355.37, 49.15) --
	(357.18, 50.95) --
	(357.33, 51.11) --
	(359.14, 52.92) --
	(359.30, 53.08) --
	(361.11, 54.89) --
	(361.27, 55.05) --
	(363.08, 56.86) --
	(363.24, 57.01) --
	(365.04, 58.82) --
	(365.20, 58.98) --
	(367.01, 60.79) --
	(367.17, 60.95) --
	(368.98, 62.76) --
	(369.14, 62.91) --
	(370.95, 64.72) --
	(371.10, 64.88) --
	(372.91, 66.69) --
	(373.07, 66.85) --
	(374.88, 68.66) --
	(375.04, 68.82) --
	(376.85, 70.63) --
	(377.00, 70.78) --
	(378.81, 72.59) --
	(378.97, 72.75) --
	(380.78, 74.56);

\path[draw=drawColor,draw opacity=0.33,line width= 1.1pt,line join=round] (380.78, 78.81) --
	(378.97, 80.62) --
	(378.81, 80.78) --
	(377.00, 82.58) --
	(376.85, 82.74) --
	(375.04, 84.55) --
	(374.88, 84.71) --
	(373.07, 86.52) --
	(372.91, 86.68) --
	(371.10, 88.49) --
	(370.95, 88.64) --
	(369.14, 90.45) --
	(368.98, 90.61) --
	(367.17, 92.42) --
	(367.01, 92.58) --
	(365.20, 94.39) --
	(365.04, 94.54) --
	(363.24, 96.35) --
	(363.08, 96.51) --
	(361.27, 98.32) --
	(361.11, 98.48) --
	(359.30,100.29) --
	(359.14,100.45) --
	(357.33,102.26) --
	(357.18,102.41) --
	(355.37,104.22) --
	(355.21,104.38) --
	(353.40,106.19) --
	(353.24,106.35) --
	(351.43,108.16) --
	(351.28,108.31) --
	(349.47,110.12) --
	(349.31,110.28) --
	(347.50,112.09) --
	(347.34,112.25) --
	(345.53,114.06) --
	(345.37,114.22) --
	(343.57,116.02);

\path[draw=drawColor,draw opacity=0.33,line width= 1.1pt,line join=round] (330.00, 37.34) --
	(329.64, 37.70) --
	(328.03, 39.31) --
	(327.67, 39.67) --
	(326.06, 41.28) --
	(325.70, 41.64) --
	(324.10, 43.24) --
	(323.74, 43.60) --
	(322.13, 45.21) --
	(321.77, 45.57) --
	(320.16, 47.18) --
	(319.80, 47.54) --
	(318.20, 49.15) --
	(317.84, 49.51) --
	(316.23, 51.11) --
	(315.87, 51.47) --
	(314.26, 53.08) --
	(313.90, 53.44) --
	(312.29, 55.05) --
	(311.93, 55.41) --
	(310.33, 57.01) --
	(309.97, 57.37) --
	(308.36, 58.98) --
	(308.00, 59.34) --
	(306.39, 60.95) --
	(306.03, 61.31) --
	(304.43, 62.91) --
	(304.07, 63.27) --
	(302.46, 64.88) --
	(302.10, 65.24);

\path[draw=drawColor,draw opacity=0.33,line width= 1.1pt,line join=round] (302.10, 88.13) --
	(302.46, 88.49) --
	(304.07, 90.09) --
	(304.43, 90.45) --
	(306.03, 92.06) --
	(306.39, 92.42) --
	(308.00, 94.03) --
	(308.36, 94.39) --
	(309.97, 95.99) --
	(310.33, 96.35) --
	(311.93, 97.96) --
	(312.29, 98.32) --
	(313.90, 99.93) --
	(314.26,100.29) --
	(315.87,101.90) --
	(316.23,102.26) --
	(317.84,103.86) --
	(318.20,104.22) --
	(319.80,105.83) --
	(320.16,106.19) --
	(321.77,107.80) --
	(322.13,108.16) --
	(323.74,109.76) --
	(324.10,110.12) --
	(325.70,111.73) --
	(326.06,112.09) --
	(327.67,113.70) --
	(328.03,114.06) --
	(329.64,115.66) --
	(330.00,116.02);

\path[draw=drawColor,draw opacity=0.33,line width= 1.1pt,line join=round] (352.88, 37.34) --
	(353.24, 37.70) --
	(354.85, 39.31) --
	(355.21, 39.67) --
	(356.82, 41.28) --
	(357.18, 41.64) --
	(358.78, 43.24) --
	(359.14, 43.60) --
	(360.75, 45.21) --
	(361.11, 45.57) --
	(362.72, 47.18) --
	(363.08, 47.54) --
	(364.68, 49.15) --
	(365.04, 49.51) --
	(366.65, 51.11) --
	(367.01, 51.47) --
	(368.62, 53.08) --
	(368.98, 53.44) --
	(370.59, 55.05) --
	(370.95, 55.41) --
	(372.55, 57.01) --
	(372.91, 57.37) --
	(374.52, 58.98) --
	(374.88, 59.34) --
	(376.49, 60.95) --
	(376.85, 61.31) --
	(378.45, 62.91) --
	(378.81, 63.27) --
	(380.42, 64.88) --
	(380.78, 65.24);

\path[draw=drawColor,draw opacity=0.33,line width= 1.1pt,line join=round] (380.78, 88.13) --
	(380.42, 88.49) --
	(378.81, 90.09) --
	(378.45, 90.45) --
	(376.85, 92.06) --
	(376.49, 92.42) --
	(374.88, 94.03) --
	(374.52, 94.39) --
	(372.91, 95.99) --
	(372.55, 96.35) --
	(370.95, 97.96) --
	(370.59, 98.32) --
	(368.98, 99.93) --
	(368.62,100.29) --
	(367.01,101.90) --
	(366.65,102.26) --
	(365.04,103.86) --
	(364.68,104.22) --
	(363.08,105.83) --
	(362.72,106.19) --
	(361.11,107.80) --
	(360.75,108.16) --
	(359.14,109.76) --
	(358.78,110.12) --
	(357.18,111.73) --
	(356.82,112.09) --
	(355.21,113.70) --
	(354.85,114.06) --
	(353.24,115.66) --
	(352.88,116.02);

\path[draw=drawColor,draw opacity=0.33,line width= 1.1pt,line join=round] (322.14, 37.34) --
	(321.77, 37.71) --
	(320.17, 39.31) --
	(319.80, 39.68) --
	(318.21, 41.28) --
	(317.84, 41.65) --
	(316.24, 43.24) --
	(315.87, 43.61) --
	(314.27, 45.21) --
	(313.90, 45.58) --
	(312.31, 47.18) --
	(311.93, 47.55) --
	(310.34, 49.15) --
	(309.97, 49.52) --
	(308.37, 51.11) --
	(308.00, 51.48) --
	(306.40, 53.08) --
	(306.03, 53.45) --
	(304.44, 55.05) --
	(304.07, 55.42) --
	(302.47, 57.01) --
	(302.10, 57.38);

\path[draw=drawColor,draw opacity=0.33,line width= 1.1pt,line join=round] (302.10, 95.98) --
	(302.47, 96.35) --
	(304.07, 97.95) --
	(304.44, 98.32) --
	(306.03, 99.92) --
	(306.40,100.29) --
	(308.00,101.88) --
	(308.37,102.26) --
	(309.97,103.85) --
	(310.34,104.22) --
	(311.93,105.82) --
	(312.31,106.19) --
	(313.90,107.79) --
	(314.27,108.16) --
	(315.87,109.75) --
	(316.24,110.12) --
	(317.84,111.72) --
	(318.21,112.09) --
	(319.80,113.69) --
	(320.17,114.06) --
	(321.77,115.65) --
	(322.14,116.02);

\path[draw=drawColor,draw opacity=0.33,line width= 1.1pt,line join=round] (360.74, 37.34) --
	(361.11, 37.71) --
	(362.71, 39.31) --
	(363.08, 39.68) --
	(364.67, 41.28) --
	(365.04, 41.65) --
	(366.64, 43.24) --
	(367.01, 43.61) --
	(368.61, 45.21) --
	(368.98, 45.58) --
	(370.58, 47.18) --
	(370.95, 47.55) --
	(372.54, 49.15) --
	(372.91, 49.52) --
	(374.51, 51.11) --
	(374.88, 51.48) --
	(376.48, 53.08) --
	(376.85, 53.45) --
	(378.44, 55.05) --
	(378.81, 55.42) --
	(380.41, 57.01) --
	(380.78, 57.38);

\path[draw=drawColor,draw opacity=0.33,line width= 1.1pt,line join=round] (380.78, 95.98) --
	(380.41, 96.35) --
	(378.81, 97.95) --
	(378.44, 98.32) --
	(376.85, 99.92) --
	(376.48,100.29) --
	(374.88,101.88) --
	(374.51,102.26) --
	(372.91,103.85) --
	(372.54,104.22) --
	(370.95,105.82) --
	(370.58,106.19) --
	(368.98,107.79) --
	(368.61,108.16) --
	(367.01,109.75) --
	(366.64,110.12) --
	(365.04,111.72) --
	(364.67,112.09) --
	(363.08,113.69) --
	(362.71,114.06) --
	(361.11,115.65) --
	(360.74,116.02);

\path[draw=drawColor,draw opacity=0.33,line width= 1.1pt,line join=round] (315.22, 37.34) --
	(313.90, 38.66) --
	(313.25, 39.31) --
	(311.93, 40.63) --
	(311.29, 41.28) --
	(309.97, 42.60) --
	(309.32, 43.24) --
	(308.00, 44.56) --
	(307.35, 45.21) --
	(306.03, 46.53) --
	(305.38, 47.18) --
	(304.07, 48.50) --
	(303.42, 49.15) --
	(302.10, 50.46);

\path[draw=drawColor,draw opacity=0.33,line width= 1.1pt,line join=round] (302.10,102.90) --
	(303.42,104.22) --
	(304.07,104.87) --
	(305.38,106.19) --
	(306.03,106.84) --
	(307.35,108.16) --
	(308.00,108.81) --
	(309.32,110.12) --
	(309.97,110.77) --
	(311.29,112.09) --
	(311.93,112.74) --
	(313.25,114.06) --
	(313.90,114.71) --
	(315.22,116.02);

\path[draw=drawColor,draw opacity=0.33,line width= 1.1pt,line join=round] (367.66, 37.34) --
	(368.98, 38.66) --
	(369.63, 39.31) --
	(370.95, 40.63) --
	(371.60, 41.28) --
	(372.91, 42.60) --
	(373.56, 43.24) --
	(374.88, 44.56) --
	(375.53, 45.21) --
	(376.85, 46.53) --
	(377.50, 47.18) --
	(378.81, 48.50) --
	(379.46, 49.15) --
	(380.78, 50.46);

\path[draw=drawColor,draw opacity=0.33,line width= 1.1pt,line join=round] (380.78,102.90) --
	(379.46,104.22) --
	(378.81,104.87) --
	(377.50,106.19) --
	(376.85,106.84) --
	(375.53,108.16) --
	(374.88,108.81) --
	(373.56,110.12) --
	(372.91,110.77) --
	(371.60,112.09) --
	(370.95,112.74) --
	(369.63,114.06) --
	(368.98,114.71) --
	(367.66,116.02);

\path[draw=drawColor,draw opacity=0.33,line width= 1.1pt,line join=round] (308.96, 37.34) --
	(308.00, 38.30) --
	(306.99, 39.31) --
	(306.03, 40.27) --
	(305.03, 41.28) --
	(304.07, 42.24) --
	(303.06, 43.24) --
	(302.10, 44.21);

\path[draw=drawColor,draw opacity=0.33,line width= 1.1pt,line join=round] (302.10,109.16) --
	(303.06,110.12) --
	(304.07,111.13) --
	(305.03,112.09) --
	(306.03,113.10) --
	(306.99,114.06) --
	(308.00,115.06) --
	(308.96,116.02);

\path[draw=drawColor,draw opacity=0.33,line width= 1.1pt,line join=round] (373.92, 37.34) --
	(374.88, 38.30) --
	(375.89, 39.31) --
	(376.85, 40.27) --
	(377.85, 41.28) --
	(378.81, 42.24) --
	(379.82, 43.24) --
	(380.78, 44.21);

\path[draw=drawColor,draw opacity=0.33,line width= 1.1pt,line join=round] (380.78,109.16) --
	(379.82,110.12) --
	(378.81,111.13) --
	(377.85,112.09) --
	(376.85,113.10) --
	(375.89,114.06) --
	(374.88,115.06) --
	(373.92,116.02);

\path[draw=drawColor,draw opacity=0.33,line width= 1.1pt,line join=round] (303.21, 37.34) --
	(302.10, 38.45);

\path[draw=drawColor,draw opacity=0.33,line width= 1.1pt,line join=round] (302.10,114.92) --
	(303.21,116.02);

\path[draw=drawColor,draw opacity=0.33,line width= 1.1pt,line join=round] (379.67, 37.34) --
	(380.78, 38.45);

\path[draw=drawColor,draw opacity=0.33,line width= 1.1pt,line join=round] (380.78,114.92) --
	(379.67,116.02);
\definecolor{drawColor}{RGB}{0,0,0}

\path[draw=drawColor,line width= 1.1pt,line join=round] (380.78, 97.45) --
	(378.81, 97.69) --
	(376.85, 98.07) --
	(375.96, 98.32) --
	(374.88, 98.63) --
	(372.91, 99.36) --
	(370.96,100.29) --
	(370.95,100.30) --
	(368.98,101.52) --
	(367.98,102.26) --
	(367.01,103.04) --
	(365.76,104.22) --
	(365.04,104.98) --
	(364.03,106.19) --
	(363.08,107.49) --
	(362.62,108.16) --
	(361.47,110.12) --
	(361.11,110.84) --
	(360.50,112.09) --
	(359.70,114.06) --
	(359.14,115.69) --
	(359.03,116.02);

\path[draw=drawColor,line width= 1.1pt,line join=round] (360.45, 37.34) --
	(361.11, 38.85) --
	(361.30, 39.31) --
	(362.15, 41.28) --
	(363.03, 43.24) --
	(363.08, 43.35) --
	(363.96, 45.21) --
	(364.95, 47.18) --
	(365.04, 47.34) --
	(366.06, 49.15) --
	(367.01, 50.68) --
	(367.29, 51.11) --
	(368.70, 53.08) --
	(368.98, 53.43) --
	(370.35, 55.05) --
	(370.95, 55.69) --
	(372.30, 57.01) --
	(372.91, 57.57) --
	(374.65, 58.98) --
	(374.88, 59.16) --
	(376.85, 60.49) --
	(377.61, 60.95) --
	(378.81, 61.64) --
	(380.78, 62.64);

\path[draw=drawColor,line width= 1.1pt,line join=round] (302.10, 50.47) --
	(304.07, 50.49) --
	(306.03, 50.24) --
	(308.00, 49.70) --
	(309.31, 49.15) --
	(309.97, 48.82) --
	(311.93, 47.50) --
	(312.31, 47.18) --
	(313.90, 45.59) --
	(314.22, 45.21) --
	(315.62, 43.24) --
	(315.87, 42.80) --
	(316.65, 41.28) --
	(317.47, 39.31) --
	(317.84, 38.12) --
	(318.07, 37.34);

\path[draw=drawColor,line width= 1.1pt,line join=round] (302.10, 99.79) --
	(304.07,100.01) --
	(305.34,100.29) --
	(306.03,100.46) --
	(308.00,101.20) --
	(309.97,102.20) --
	(310.05,102.26) --
	(311.93,103.67) --
	(312.54,104.22) --
	(313.90,105.69) --
	(314.30,106.19) --
	(315.57,108.16) --
	(315.87,108.76) --
	(316.48,110.12) --
	(317.13,112.09) --
	(317.56,114.06) --
	(317.75,116.02);

\path[draw=drawColor,line width= 1.1pt,line join=round] (380.78,102.33) --
	(378.81,102.45) --
	(376.85,102.79) --
	(374.88,103.37) --
	(372.91,104.20) --
	(372.87,104.22) --
	(370.95,105.44) --
	(369.99,106.19) --
	(368.98,107.13) --
	(368.04,108.16) --
	(367.01,109.53) --
	(366.62,110.12) --
	(365.58,112.09) --
	(365.04,113.41) --
	(364.80,114.06) --
	(364.29,116.02);

\path[draw=drawColor,line width= 1.1pt,line join=round] (368.01, 37.34) --
	(368.19, 39.31) --
	(368.53, 41.28) --
	(368.98, 43.03) --
	(369.04, 43.24) --
	(369.76, 45.21) --
	(370.63, 47.18) --
	(370.95, 47.76) --
	(371.76, 49.15) --
	(372.91, 50.82) --
	(373.13, 51.11) --
	(374.88, 53.07) --
	(374.89, 53.08) --
	(376.85, 54.76) --
	(377.24, 55.05) --
	(378.81, 56.08) --
	(380.49, 57.01) --
	(380.78, 57.16);

\path[draw=drawColor,line width= 1.1pt,line join=round] (302.10,111.95) --
	(304.07,110.32) --
	(306.03,111.46) --
	(306.44,112.09) --
	(306.19,114.06) --
	(306.03,114.22) --
	(304.07,114.51) --
	(303.25,114.06) --
	(302.10,112.26);

\path[draw=drawColor,line width= 1.1pt,line join=round] (380.78,109.30) --
	(378.81,109.20) --
	(376.85,109.57) --
	(375.55,110.12) --
	(374.88,110.52) --
	(373.15,112.09) --
	(372.91,112.42) --
	(372.04,114.06) --
	(371.51,116.02);
\definecolor{drawColor}{gray}{0.50}

\path[draw=drawColor,line width= 0.6pt,line join=round,line cap=round] (298.17, 33.41) rectangle (384.72,119.96);
\end{scope}
\begin{scope}
\path[clip] (  0.00,  0.00) rectangle (505.89,144.54);
\definecolor{drawColor}{RGB}{0,0,0}

\node[text=drawColor,anchor=base east,inner sep=0pt, outer sep=0pt, scale=  0.87] at ( 33.12, 47.17) {-2};

\node[text=drawColor,anchor=base east,inner sep=0pt, outer sep=0pt, scale=  0.87] at ( 33.12, 60.28) {-1};

\node[text=drawColor,anchor=base east,inner sep=0pt, outer sep=0pt, scale=  0.87] at ( 33.12, 73.39) {0};

\node[text=drawColor,anchor=base east,inner sep=0pt, outer sep=0pt, scale=  0.87] at ( 33.12, 86.51) {1};

\node[text=drawColor,anchor=base east,inner sep=0pt, outer sep=0pt, scale=  0.87] at ( 33.12, 99.62) {2};
\end{scope}
\begin{scope}
\path[clip] (  0.00,  0.00) rectangle (505.89,144.54);
\definecolor{drawColor}{RGB}{0,0,0}

\path[draw=drawColor,line width= 0.6pt,line join=round] ( 35.52, 50.46) --
	( 38.52, 50.46);

\path[draw=drawColor,line width= 0.6pt,line join=round] ( 35.52, 63.57) --
	( 38.52, 63.57);

\path[draw=drawColor,line width= 0.6pt,line join=round] ( 35.52, 76.68) --
	( 38.52, 76.68);

\path[draw=drawColor,line width= 0.6pt,line join=round] ( 35.52, 89.80) --
	( 38.52, 89.80);

\path[draw=drawColor,line width= 0.6pt,line join=round] ( 35.52,102.91) --
	( 38.52,102.91);
\end{scope}
\begin{scope}
\path[clip] (  0.00,  0.00) rectangle (505.89,144.54);
\definecolor{drawColor}{RGB}{0,0,0}

\path[draw=drawColor,line width= 0.6pt,line join=round] ( 55.57, 30.41) --
	( 55.57, 33.41);

\path[draw=drawColor,line width= 0.6pt,line join=round] ( 68.68, 30.41) --
	( 68.68, 33.41);

\path[draw=drawColor,line width= 0.6pt,line join=round] ( 81.79, 30.41) --
	( 81.79, 33.41);

\path[draw=drawColor,line width= 0.6pt,line join=round] ( 94.91, 30.41) --
	( 94.91, 33.41);

\path[draw=drawColor,line width= 0.6pt,line join=round] (108.02, 30.41) --
	(108.02, 33.41);
\end{scope}
\begin{scope}
\path[clip] (  0.00,  0.00) rectangle (505.89,144.54);
\definecolor{drawColor}{RGB}{0,0,0}

\node[text=drawColor,anchor=base,inner sep=0pt, outer sep=0pt, scale=  0.87] at ( 55.57, 21.43) {-2};

\node[text=drawColor,anchor=base,inner sep=0pt, outer sep=0pt, scale=  0.87] at ( 68.68, 21.43) {-1};

\node[text=drawColor,anchor=base,inner sep=0pt, outer sep=0pt, scale=  0.87] at ( 81.79, 21.43) {0};

\node[text=drawColor,anchor=base,inner sep=0pt, outer sep=0pt, scale=  0.87] at ( 94.91, 21.43) {1};

\node[text=drawColor,anchor=base,inner sep=0pt, outer sep=0pt, scale=  0.87] at (108.02, 21.43) {2};
\end{scope}
\begin{scope}
\path[clip] (  0.00,  0.00) rectangle (505.89,144.54);
\definecolor{drawColor}{RGB}{0,0,0}

\path[draw=drawColor,line width= 0.6pt,line join=round] (142.11, 30.41) --
	(142.11, 33.41);

\path[draw=drawColor,line width= 0.6pt,line join=round] (155.23, 30.41) --
	(155.23, 33.41);

\path[draw=drawColor,line width= 0.6pt,line join=round] (168.34, 30.41) --
	(168.34, 33.41);

\path[draw=drawColor,line width= 0.6pt,line join=round] (181.46, 30.41) --
	(181.46, 33.41);

\path[draw=drawColor,line width= 0.6pt,line join=round] (194.57, 30.41) --
	(194.57, 33.41);
\end{scope}
\begin{scope}
\path[clip] (  0.00,  0.00) rectangle (505.89,144.54);
\definecolor{drawColor}{RGB}{0,0,0}

\node[text=drawColor,anchor=base,inner sep=0pt, outer sep=0pt, scale=  0.87] at (142.11, 21.43) {-2};

\node[text=drawColor,anchor=base,inner sep=0pt, outer sep=0pt, scale=  0.87] at (155.23, 21.43) {-1};

\node[text=drawColor,anchor=base,inner sep=0pt, outer sep=0pt, scale=  0.87] at (168.34, 21.43) {0};

\node[text=drawColor,anchor=base,inner sep=0pt, outer sep=0pt, scale=  0.87] at (181.46, 21.43) {1};

\node[text=drawColor,anchor=base,inner sep=0pt, outer sep=0pt, scale=  0.87] at (194.57, 21.43) {2};
\end{scope}
\begin{scope}
\path[clip] (  0.00,  0.00) rectangle (505.89,144.54);
\definecolor{drawColor}{RGB}{0,0,0}

\path[draw=drawColor,line width= 0.6pt,line join=round] (228.66, 30.41) --
	(228.66, 33.41);

\path[draw=drawColor,line width= 0.6pt,line join=round] (241.78, 30.41) --
	(241.78, 33.41);

\path[draw=drawColor,line width= 0.6pt,line join=round] (254.89, 30.41) --
	(254.89, 33.41);

\path[draw=drawColor,line width= 0.6pt,line join=round] (268.00, 30.41) --
	(268.00, 33.41);

\path[draw=drawColor,line width= 0.6pt,line join=round] (281.12, 30.41) --
	(281.12, 33.41);
\end{scope}
\begin{scope}
\path[clip] (  0.00,  0.00) rectangle (505.89,144.54);
\definecolor{drawColor}{RGB}{0,0,0}

\node[text=drawColor,anchor=base,inner sep=0pt, outer sep=0pt, scale=  0.87] at (228.66, 21.43) {-2};

\node[text=drawColor,anchor=base,inner sep=0pt, outer sep=0pt, scale=  0.87] at (241.78, 21.43) {-1};

\node[text=drawColor,anchor=base,inner sep=0pt, outer sep=0pt, scale=  0.87] at (254.89, 21.43) {0};

\node[text=drawColor,anchor=base,inner sep=0pt, outer sep=0pt, scale=  0.87] at (268.00, 21.43) {1};

\node[text=drawColor,anchor=base,inner sep=0pt, outer sep=0pt, scale=  0.87] at (281.12, 21.43) {2};
\end{scope}
\begin{scope}
\path[clip] (  0.00,  0.00) rectangle (505.89,144.54);
\definecolor{drawColor}{RGB}{0,0,0}

\path[draw=drawColor,line width= 0.6pt,line join=round] (315.21, 30.41) --
	(315.21, 33.41);

\path[draw=drawColor,line width= 0.6pt,line join=round] (328.33, 30.41) --
	(328.33, 33.41);

\path[draw=drawColor,line width= 0.6pt,line join=round] (341.44, 30.41) --
	(341.44, 33.41);

\path[draw=drawColor,line width= 0.6pt,line join=round] (354.55, 30.41) --
	(354.55, 33.41);

\path[draw=drawColor,line width= 0.6pt,line join=round] (367.67, 30.41) --
	(367.67, 33.41);
\end{scope}
\begin{scope}
\path[clip] (  0.00,  0.00) rectangle (505.89,144.54);
\definecolor{drawColor}{RGB}{0,0,0}

\node[text=drawColor,anchor=base,inner sep=0pt, outer sep=0pt, scale=  0.87] at (315.21, 21.43) {-2};

\node[text=drawColor,anchor=base,inner sep=0pt, outer sep=0pt, scale=  0.87] at (328.33, 21.43) {-1};

\node[text=drawColor,anchor=base,inner sep=0pt, outer sep=0pt, scale=  0.87] at (341.44, 21.43) {0};

\node[text=drawColor,anchor=base,inner sep=0pt, outer sep=0pt, scale=  0.87] at (354.55, 21.43) {1};

\node[text=drawColor,anchor=base,inner sep=0pt, outer sep=0pt, scale=  0.87] at (367.67, 21.43) {2};
\end{scope}
\begin{scope}
\path[clip] (  0.00,  0.00) rectangle (505.89,144.54);
\definecolor{drawColor}{RGB}{0,0,0}

\node[text=drawColor,anchor=base,inner sep=0pt, outer sep=0pt, scale=  1.09] at (211.62,  8.40) {feature 1};
\end{scope}
\begin{scope}
\path[clip] (  0.00,  0.00) rectangle (505.89,144.54);
\definecolor{drawColor}{RGB}{0,0,0}

\node[text=drawColor,rotate= 90.00,anchor=base,inner sep=0pt, outer sep=0pt, scale=  1.09] at ( 20.40, 76.68) {feature 2};
\end{scope}
\begin{scope}
\path[clip] (  0.00,  0.00) rectangle (505.89,144.54);
\definecolor{fillColor}{RGB}{255,255,255}

\path[fill=fillColor] (393.25, 68.93) rectangle (480.12,142.67);
\end{scope}
\begin{scope}
\path[clip] (  0.00,  0.00) rectangle (505.89,144.54);
\definecolor{drawColor}{RGB}{0,0,0}

\node[text=drawColor,anchor=base west,inner sep=0pt, outer sep=0pt, scale=  1.09] at (397.52,130.17) {label};
\end{scope}
\begin{scope}
\path[clip] (  0.00,  0.00) rectangle (505.89,144.54);
\definecolor{drawColor}{gray}{0.80}
\definecolor{fillColor}{RGB}{255,255,255}

\path[draw=drawColor,line width= 0.6pt,line join=round,line cap=round,fill=fillColor] (397.52, 99.70) rectangle (413.42,126.20);
\end{scope}
\begin{scope}
\path[clip] (  0.00,  0.00) rectangle (505.89,144.54);
\definecolor{drawColor}{RGB}{152,78,163}

\path[draw=drawColor,line width= 1.1pt,line join=round] (399.11,112.95) -- (411.83,112.95);
\end{scope}
\begin{scope}
\path[clip] (  0.00,  0.00) rectangle (505.89,144.54);
\definecolor{drawColor}{RGB}{152,78,163}

\path[draw=drawColor,line width= 1.1pt,line join=round] (399.11,112.95) -- (411.83,112.95);

\path[draw=drawColor,line width= 1.1pt,line join=round] (408.70,111.14) --
	(411.83,112.95) --
	(408.70,114.75);
\end{scope}
\begin{scope}
\path[clip] (  0.00,  0.00) rectangle (505.89,144.54);
\definecolor{drawColor}{gray}{0.80}
\definecolor{fillColor}{RGB}{255,255,255}

\path[draw=drawColor,line width= 0.6pt,line join=round,line cap=round,fill=fillColor] (397.52, 73.20) rectangle (413.42, 99.70);
\end{scope}
\begin{scope}
\path[clip] (  0.00,  0.00) rectangle (505.89,144.54);
\definecolor{drawColor}{RGB}{255,127,0}

\path[draw=drawColor,line width= 1.1pt,line join=round] (399.11, 86.45) -- (411.83, 86.45);
\end{scope}
\begin{scope}
\path[clip] (  0.00,  0.00) rectangle (505.89,144.54);
\definecolor{drawColor}{RGB}{255,127,0}

\path[draw=drawColor,line width= 1.1pt,line join=round] (399.11, 86.45) -- (411.83, 86.45);

\path[draw=drawColor,line width= 1.1pt,line join=round] (408.70, 84.64) --
	(411.83, 86.45) --
	(408.70, 88.25);
\end{scope}
\begin{scope}
\path[clip] (  0.00,  0.00) rectangle (505.89,144.54);
\definecolor{drawColor}{RGB}{0,0,0}

\node[text=drawColor,anchor=base west,inner sep=0pt, outer sep=0pt, scale=  0.87] at (415.41,114.84) {equality pair};

\node[text=drawColor,anchor=base west,inner sep=0pt, outer sep=0pt, scale=  0.87] at (415.41,104.47) {$y_i=0$};
\end{scope}
\begin{scope}
\path[clip] (  0.00,  0.00) rectangle (505.89,144.54);
\definecolor{drawColor}{RGB}{0,0,0}

\node[text=drawColor,anchor=base west,inner sep=0pt, outer sep=0pt, scale=  0.87] at (415.41, 88.34) {inequality pair};

\node[text=drawColor,anchor=base west,inner sep=0pt, outer sep=0pt, scale=  0.87] at (415.41, 77.97) {$y_i\in\{-1,1\}$};
\end{scope}
\begin{scope}
\path[clip] (  0.00,  0.00) rectangle (505.89,144.54);
\definecolor{fillColor}{RGB}{255,255,255}

\path[fill=fillColor] (393.25, 10.70) rectangle (487.58, 63.24);
\end{scope}
\begin{scope}
\path[clip] (  0.00,  0.00) rectangle (505.89,144.54);
\definecolor{drawColor}{RGB}{0,0,0}

\node[text=drawColor,anchor=base west,inner sep=0pt, outer sep=0pt, scale=  1.09] at (397.52, 50.74) {ranking function};
\end{scope}
\begin{scope}
\path[clip] (  0.00,  0.00) rectangle (505.89,144.54);
\definecolor{drawColor}{gray}{0.80}
\definecolor{fillColor}{RGB}{255,255,255}

\path[draw=drawColor,line width= 0.6pt,line join=round,line cap=round,fill=fillColor] (397.52, 30.87) rectangle (413.42, 46.77);
\end{scope}
\begin{scope}
\path[clip] (  0.00,  0.00) rectangle (505.89,144.54);
\definecolor{drawColor}{RGB}{0,0,0}

\path[draw=drawColor,line width= 1.1pt,line join=round] (399.11, 38.82) -- (411.83, 38.82);
\end{scope}
\begin{scope}
\path[clip] (  0.00,  0.00) rectangle (505.89,144.54);
\definecolor{drawColor}{gray}{0.80}
\definecolor{fillColor}{RGB}{255,255,255}

\path[draw=drawColor,line width= 0.6pt,line join=round,line cap=round,fill=fillColor] (397.52, 14.97) rectangle (413.42, 30.87);
\end{scope}
\begin{scope}
\path[clip] (  0.00,  0.00) rectangle (505.89,144.54);
\definecolor{drawColor}{RGB}{0,0,0}

\path[draw=drawColor,draw opacity=0.33,line width= 1.1pt,line join=round] (399.11, 22.92) -- (411.83, 22.92);
\end{scope}
\begin{scope}
\path[clip] (  0.00,  0.00) rectangle (505.89,144.54);
\definecolor{drawColor}{RGB}{0,0,0}

\node[text=drawColor,anchor=base west,inner sep=0pt, outer sep=0pt, scale=  0.87] at (415.41, 35.53) {learned};
\end{scope}
\begin{scope}
\path[clip] (  0.00,  0.00) rectangle (505.89,144.54);
\definecolor{drawColor}{RGB}{0,0,0}

\node[text=drawColor,anchor=base west,inner sep=0pt, outer sep=0pt, scale=  0.87] at (415.41, 19.63) {truth};
\end{scope}
\end{tikzpicture}

  \vskip -1cm
  \caption{Application to simulated patterns $x,x'\in\RR^2$, where the
    ranking functions $r(x)=||x||^2$ are squared norms (panels from
    left to right). \textbf{Top}: the training data are $n=100$ pairs,
    half equality (black segments indicate two points of equal rank),
    and half inequality (red arrows point to the higher
    rank). \textbf{Middle and bottom}: level curves of the ranking
    functions learned by SVMrank and SVMcompare. SVMrank ignores the
    equality pairs, so in general SVMcompare recovers the latent
    pattern better.}
  \label{fig:norm-level-curves}
\end{figure*}

\begin{figure*}[b!]
  % Created by tikzDevice version 0.10.1 on 2018-01-31 09:59:17
% !TEX encoding = UTF-8 Unicode
\begin{tikzpicture}[x=1pt,y=1pt]
\definecolor{fillColor}{RGB}{255,255,255}
\path[use as bounding box,fill=fillColor,fill opacity=0.00] (0,0) rectangle (505.89,144.54);
\begin{scope}
\path[clip] (  0.00,  0.00) rectangle (505.89,144.54);
\definecolor{drawColor}{RGB}{255,255,255}
\definecolor{fillColor}{RGB}{255,255,255}

\path[draw=drawColor,line width= 0.6pt,line join=round,line cap=round,fill=fillColor] ( -0.00, -0.00) rectangle (505.89,144.54);
\end{scope}
\begin{scope}
\path[clip] ( 49.29,119.96) rectangle (142.36,138.54);
\definecolor{drawColor}{gray}{0.50}
\definecolor{fillColor}{gray}{0.80}

\path[draw=drawColor,line width= 0.2pt,line join=round,line cap=round,fill=fillColor] ( 49.29,119.96) rectangle (142.36,138.54);
\definecolor{drawColor}{gray}{0.10}

\node[text=drawColor,anchor=base,inner sep=0pt, outer sep=0pt, scale=  0.87] at ( 95.82,125.96) {$r(\mathbf x) = ||\mathbf x||_\infty^2$};
\end{scope}
\begin{scope}
\path[clip] (142.36,119.96) rectangle (235.42,138.54);
\definecolor{drawColor}{gray}{0.50}
\definecolor{fillColor}{gray}{0.80}

\path[draw=drawColor,line width= 0.2pt,line join=round,line cap=round,fill=fillColor] (142.36,119.96) rectangle (235.42,138.54);
\definecolor{drawColor}{gray}{0.10}

\node[text=drawColor,anchor=base,inner sep=0pt, outer sep=0pt, scale=  0.87] at (188.89,125.96) {$r(\mathbf x) = ||\mathbf x||_1^2$};
\end{scope}
\begin{scope}
\path[clip] (235.42,119.96) rectangle (328.49,138.54);
\definecolor{drawColor}{gray}{0.50}
\definecolor{fillColor}{gray}{0.80}

\path[draw=drawColor,line width= 0.2pt,line join=round,line cap=round,fill=fillColor] (235.42,119.96) rectangle (328.49,138.54);
\definecolor{drawColor}{gray}{0.10}

\node[text=drawColor,anchor=base,inner sep=0pt, outer sep=0pt, scale=  0.87] at (281.96,125.96) {$r(\mathbf x) = ||\mathbf x||_2^2$};
\end{scope}
\begin{scope}
\path[clip] (328.49,119.96) rectangle (421.56,138.54);
\definecolor{drawColor}{gray}{0.50}
\definecolor{fillColor}{gray}{0.80}

\path[draw=drawColor,line width= 0.2pt,line join=round,line cap=round,fill=fillColor] (328.49,119.96) rectangle (421.56,138.54);
\definecolor{drawColor}{gray}{0.10}

\node[text=drawColor,anchor=base,inner sep=0pt, outer sep=0pt, scale=  0.87] at (375.02,125.96) {sushi};
\end{scope}
\begin{scope}
\path[clip] ( 49.29, 33.41) rectangle (142.36,119.96);
\definecolor{fillColor}{RGB}{255,255,255}

\path[fill=fillColor] ( 49.29, 33.41) rectangle (142.36,119.96);
\definecolor{fillColor}{RGB}{135,206,235}

\path[fill=fillColor,fill opacity=0.50] ( 53.52, 78.24) --
	( 59.16, 72.37) --
	( 70.44, 65.55) --
	( 93.00, 58.37) --
	(138.13, 60.17) --
	(138.13, 57.22) --
	( 93.00, 55.04) --
	( 70.44, 58.40) --
	( 59.16, 66.28) --
	( 53.52, 57.23) --
	cycle;
\definecolor{fillColor}{RGB}{0,0,255}

\path[fill=fillColor,fill opacity=0.50] ( 53.52, 87.73) --
	( 59.16, 64.93) --
	( 70.44, 55.15) --
	( 93.00, 56.57) --
	(138.13, 56.43) --
	(138.13, 52.12) --
	( 93.00, 51.28) --
	( 70.44, 51.71) --
	( 59.16, 55.44) --
	( 53.52, 62.04) --
	cycle;
\definecolor{fillColor}{RGB}{0,0,0}

\path[fill=fillColor,fill opacity=0.50] ( 53.52, 76.49) --
	( 59.16, 67.62) --
	( 70.44, 58.69) --
	( 93.00, 51.94) --
	(138.13, 50.13) --
	(138.13, 47.19) --
	( 93.00, 46.38) --
	( 70.44, 54.13) --
	( 59.16, 49.57) --
	( 53.52, 58.98) --
	cycle;
\definecolor{fillColor}{RGB}{204,204,204}

\path[fill=fillColor,fill opacity=0.50] ( 53.52, 50.45) --
	( 59.16, 49.71) --
	( 70.44, 44.27) --
	( 93.00, 44.96) --
	(138.13, 45.57) --
	(138.13, 43.22) --
	( 93.00, 41.24) --
	( 70.44, 40.74) --
	( 59.16, 42.85) --
	( 53.52, 37.34) --
	cycle;
\definecolor{drawColor}{RGB}{135,206,235}

\path[draw=drawColor,line width= 1.7pt,line join=round] ( 53.52, 67.73) --
	( 59.16, 69.32) --
	( 70.44, 61.97) --
	( 93.00, 56.71) --
	(138.13, 58.70);
\definecolor{drawColor}{RGB}{0,0,255}

\path[draw=drawColor,line width= 1.7pt,line join=round] ( 53.52, 74.89) --
	( 59.16, 60.19) --
	( 70.44, 53.43) --
	( 93.00, 53.93) --
	(138.13, 54.28);
\definecolor{drawColor}{RGB}{0,0,0}

\path[draw=drawColor,line width= 1.7pt,line join=round] ( 53.52, 67.73) --
	( 59.16, 58.60) --
	( 70.44, 56.41) --
	( 93.00, 49.16) --
	(138.13, 48.66);
\definecolor{drawColor}{gray}{0.80}

\path[draw=drawColor,line width= 1.7pt,line join=round] ( 53.52, 43.90) --
	( 59.16, 46.28) --
	( 70.44, 42.51) --
	( 93.00, 43.10) --
	(138.13, 44.39);
\definecolor{drawColor}{gray}{0.50}

\path[draw=drawColor,line width= 0.6pt,line join=round,line cap=round] ( 49.29, 33.41) rectangle (142.36,119.96);
\end{scope}
\begin{scope}
\path[clip] (142.36, 33.41) rectangle (235.42,119.96);
\definecolor{fillColor}{RGB}{255,255,255}

\path[fill=fillColor] (142.36, 33.41) rectangle (235.42,119.96);
\definecolor{drawColor}{RGB}{0,0,0}

\path[draw=drawColor,line width= 2.3pt,line join=round] (152.23, 33.41) -- (152.23,119.96);
\definecolor{fillColor}{RGB}{135,206,235}

\path[fill=fillColor,fill opacity=0.50] (146.59, 78.72) --
	(152.23, 71.07) --
	(163.51, 67.62) --
	(186.07, 65.89) --
	(231.19, 60.98) --
	(231.19, 53.43) --
	(186.07, 60.64) --
	(163.51, 57.12) --
	(152.23, 57.25) --
	(146.59, 61.51) --
	cycle;
\definecolor{fillColor}{RGB}{0,0,255}

\path[fill=fillColor,fill opacity=0.50] (146.59, 88.70) --
	(152.23, 60.71) --
	(163.51, 59.98) --
	(186.07, 56.96) --
	(231.19, 54.63) --
	(231.19, 51.74) --
	(186.07, 50.70) --
	(163.51, 55.23) --
	(152.23, 58.87) --
	(146.59, 69.02) --
	cycle;
\definecolor{fillColor}{RGB}{0,0,0}

\path[fill=fillColor,fill opacity=0.50] (146.59, 72.50) --
	(152.23, 65.06) --
	(163.51, 58.62) --
	(186.07, 55.14) --
	(231.19, 50.75) --
	(231.19, 47.37) --
	(186.07, 45.37) --
	(163.51, 52.62) --
	(152.23, 54.52) --
	(146.59, 62.97) --
	cycle;
\definecolor{fillColor}{RGB}{204,204,204}

\path[fill=fillColor,fill opacity=0.50] (146.59, 49.19) --
	(152.23, 48.53) --
	(163.51, 45.76) --
	(186.07, 43.78) --
	(231.19, 43.84) --
	(231.19, 41.77) --
	(186.07, 43.02) --
	(163.51, 40.05) --
	(152.23, 42.44) --
	(146.59, 40.20) --
	cycle;
\definecolor{drawColor}{RGB}{135,206,235}

\path[draw=drawColor,line width= 1.7pt,line join=round] (146.59, 70.12) --
	(152.23, 64.16) --
	(163.51, 62.37) --
	(186.07, 63.26) --
	(231.19, 57.21);
\definecolor{drawColor}{RGB}{0,0,255}

\path[draw=drawColor,line width= 1.7pt,line join=round] (146.59, 78.86) --
	(152.23, 59.79) --
	(163.51, 57.60) --
	(186.07, 53.83) --
	(231.19, 53.18);
\definecolor{drawColor}{RGB}{0,0,0}

\path[draw=drawColor,line width= 1.7pt,line join=round] (146.59, 67.73) --
	(152.23, 59.79) --
	(163.51, 55.62) --
	(186.07, 50.25) --
	(231.19, 49.06);
\definecolor{drawColor}{gray}{0.80}

\path[draw=drawColor,line width= 1.7pt,line join=round] (146.59, 44.69) --
	(152.23, 45.48) --
	(163.51, 42.90) --
	(186.07, 43.40) --
	(231.19, 42.80);
\definecolor{drawColor}{gray}{0.50}

\path[draw=drawColor,line width= 0.6pt,line join=round,line cap=round] (142.36, 33.41) rectangle (235.42,119.96);
\end{scope}
\begin{scope}
\path[clip] (235.42, 33.41) rectangle (328.49,119.96);
\definecolor{fillColor}{RGB}{255,255,255}

\path[fill=fillColor] (235.42, 33.41) rectangle (328.49,119.96);
\definecolor{fillColor}{RGB}{135,206,235}

\path[fill=fillColor,fill opacity=0.50] (239.65, 76.86) --
	(245.29, 69.34) --
	(256.57, 52.84) --
	(279.14, 51.81) --
	(324.26, 52.40) --
	(324.26, 51.28) --
	(279.14, 47.30) --
	(256.57, 48.86) --
	(245.29, 43.08) --
	(239.65, 58.60) --
	cycle;
\definecolor{fillColor}{RGB}{0,0,255}

\path[fill=fillColor,fill opacity=0.50] (239.65, 74.21) --
	(245.29, 61.56) --
	(256.57, 50.71) --
	(279.14, 53.97) --
	(324.26, 54.28) --
	(324.26, 49.90) --
	(279.14, 47.73) --
	(256.57, 47.41) --
	(245.29, 49.28) --
	(239.65, 50.14) --
	cycle;
\definecolor{fillColor}{RGB}{0,0,0}

\path[fill=fillColor,fill opacity=0.50] (239.65, 62.01) --
	(245.29, 53.88) --
	(256.57, 53.56) --
	(279.14, 46.21) --
	(324.26, 47.16) --
	(324.26, 44.90) --
	(279.14, 41.98) --
	(256.57, 44.17) --
	(245.29, 41.07) --
	(239.65, 46.44) --
	cycle;
\definecolor{fillColor}{RGB}{204,204,204}

\path[fill=fillColor,fill opacity=0.50] (239.65, 54.09) --
	(245.29, 51.88) --
	(256.57, 46.48) --
	(279.14, 45.78) --
	(324.26, 46.13) --
	(324.26, 43.85) --
	(279.14, 39.63) --
	(256.57, 42.51) --
	(245.29, 39.88) --
	(239.65, 43.24) --
	cycle;
\definecolor{drawColor}{RGB}{135,206,235}

\path[draw=drawColor,line width= 1.7pt,line join=round] (239.65, 67.73) --
	(245.29, 56.21) --
	(256.57, 50.85) --
	(279.14, 49.56) --
	(324.26, 51.84);
\definecolor{drawColor}{RGB}{0,0,255}

\path[draw=drawColor,line width= 1.7pt,line join=round] (239.65, 62.17) --
	(245.29, 55.42) --
	(256.57, 49.06) --
	(279.14, 50.85) --
	(324.26, 52.09);
\definecolor{drawColor}{RGB}{0,0,0}

\path[draw=drawColor,line width= 1.7pt,line join=round] (239.65, 54.23) --
	(245.29, 47.47) --
	(256.57, 48.86) --
	(279.14, 44.09) --
	(324.26, 46.03);
\definecolor{drawColor}{gray}{0.80}

\path[draw=drawColor,line width= 1.7pt,line join=round] (239.65, 48.66) --
	(245.29, 45.88) --
	(256.57, 44.49) --
	(279.14, 42.70) --
	(324.26, 44.99);
\definecolor{drawColor}{gray}{0.50}

\path[draw=drawColor,line width= 0.6pt,line join=round,line cap=round] (235.42, 33.41) rectangle (328.49,119.96);
\end{scope}
\begin{scope}
\path[clip] (328.49, 33.41) rectangle (421.56,119.96);
\definecolor{fillColor}{RGB}{255,255,255}

\path[fill=fillColor] (328.49, 33.41) rectangle (421.56,119.96);
\definecolor{fillColor}{RGB}{135,206,235}

\path[fill=fillColor,fill opacity=0.50] (332.72,111.26) --
	(338.36,101.28) --
	(349.64,107.14) --
	(372.20,104.22) --
	(417.33,105.24) --
	(417.33, 97.37) --
	(372.20, 96.00) --
	(349.64, 91.90) --
	(338.36, 91.40) --
	(332.72, 92.54) --
	cycle;
\definecolor{fillColor}{RGB}{0,0,255}

\path[fill=fillColor,fill opacity=0.50] (332.72,116.02) --
	(338.36,102.35) --
	(349.64,104.15) --
	(372.20,100.43) --
	(417.33, 97.54) --
	(417.33, 96.13) --
	(372.20, 96.62) --
	(349.64, 92.51) --
	(338.36, 89.54) --
	(332.72, 89.37) --
	cycle;
\definecolor{fillColor}{RGB}{0,0,0}

\path[fill=fillColor,fill opacity=0.50] (332.72,113.04) --
	(338.36, 96.04) --
	(349.64, 97.28) --
	(372.20, 95.07) --
	(417.33, 93.17) --
	(417.33, 91.76) --
	(372.20, 87.08) --
	(349.64, 82.29) --
	(338.36, 89.49) --
	(332.72, 95.53) --
	cycle;
\definecolor{drawColor}{RGB}{135,206,235}

\path[draw=drawColor,line width= 1.7pt,line join=round] (332.72,101.90) --
	(338.36, 96.34) --
	(349.64, 99.52) --
	(372.20,100.11) --
	(417.33,101.31);
\definecolor{drawColor}{RGB}{0,0,255}

\path[draw=drawColor,line width= 1.7pt,line join=round] (332.72,102.70) --
	(338.36, 95.94) --
	(349.64, 98.33) --
	(372.20, 98.52) --
	(417.33, 96.84);
\definecolor{drawColor}{RGB}{0,0,0}

\path[draw=drawColor,line width= 1.7pt,line join=round] (332.72,104.29) --
	(338.36, 92.76) --
	(349.64, 89.78) --
	(372.20, 91.08) --
	(417.33, 92.47);
\definecolor{drawColor}{gray}{0.50}

\path[draw=drawColor,line width= 0.6pt,line join=round,line cap=round] (328.49, 33.41) rectangle (421.56,119.96);
\end{scope}
\begin{scope}
\path[clip] (  0.00,  0.00) rectangle (505.89,144.54);
\definecolor{drawColor}{RGB}{0,0,0}

\node[text=drawColor,anchor=base east,inner sep=0pt, outer sep=0pt, scale=  0.87] at ( 43.89, 44.58) {10};

\node[text=drawColor,anchor=base east,inner sep=0pt, outer sep=0pt, scale=  0.87] at ( 43.89, 60.47) {20};

\node[text=drawColor,anchor=base east,inner sep=0pt, outer sep=0pt, scale=  0.87] at ( 43.89, 76.36) {30};

\node[text=drawColor,anchor=base east,inner sep=0pt, outer sep=0pt, scale=  0.87] at ( 43.89, 92.25) {40};

\node[text=drawColor,anchor=base east,inner sep=0pt, outer sep=0pt, scale=  0.87] at ( 43.89,108.15) {50};
\end{scope}
\begin{scope}
\path[clip] (  0.00,  0.00) rectangle (505.89,144.54);
\definecolor{drawColor}{RGB}{0,0,0}

\path[draw=drawColor,line width= 0.6pt,line join=round] ( 46.29, 47.87) --
	( 49.29, 47.87);

\path[draw=drawColor,line width= 0.6pt,line join=round] ( 46.29, 63.76) --
	( 49.29, 63.76);

\path[draw=drawColor,line width= 0.6pt,line join=round] ( 46.29, 79.65) --
	( 49.29, 79.65);

\path[draw=drawColor,line width= 0.6pt,line join=round] ( 46.29, 95.55) --
	( 49.29, 95.55);

\path[draw=drawColor,line width= 0.6pt,line join=round] ( 46.29,111.44) --
	( 49.29,111.44);
\end{scope}
\begin{scope}
\path[clip] (  0.00,  0.00) rectangle (505.89,144.54);
\definecolor{drawColor}{RGB}{0,0,0}

\path[draw=drawColor,line width= 0.6pt,line join=round] ( 70.44, 30.41) --
	( 70.44, 33.41);

\path[draw=drawColor,line width= 0.6pt,line join=round] ( 93.00, 30.41) --
	( 93.00, 33.41);

\path[draw=drawColor,line width= 0.6pt,line join=round] (115.56, 30.41) --
	(115.56, 33.41);

\path[draw=drawColor,line width= 0.6pt,line join=round] (138.13, 30.41) --
	(138.13, 33.41);
\end{scope}
\begin{scope}
\path[clip] (  0.00,  0.00) rectangle (505.89,144.54);
\definecolor{drawColor}{RGB}{0,0,0}

\node[text=drawColor,anchor=base,inner sep=0pt, outer sep=0pt, scale=  0.87] at ( 70.44, 21.43) {200};

\node[text=drawColor,anchor=base,inner sep=0pt, outer sep=0pt, scale=  0.87] at ( 93.00, 21.43) {400};

\node[text=drawColor,anchor=base,inner sep=0pt, outer sep=0pt, scale=  0.87] at (115.56, 21.43) {600};

\node[text=drawColor,anchor=base,inner sep=0pt, outer sep=0pt, scale=  0.87] at (138.13, 21.43) {800};
\end{scope}
\begin{scope}
\path[clip] (  0.00,  0.00) rectangle (505.89,144.54);
\definecolor{drawColor}{RGB}{0,0,0}

\path[draw=drawColor,line width= 0.6pt,line join=round] (163.51, 30.41) --
	(163.51, 33.41);

\path[draw=drawColor,line width= 0.6pt,line join=round] (186.07, 30.41) --
	(186.07, 33.41);

\path[draw=drawColor,line width= 0.6pt,line join=round] (208.63, 30.41) --
	(208.63, 33.41);

\path[draw=drawColor,line width= 0.6pt,line join=round] (231.19, 30.41) --
	(231.19, 33.41);
\end{scope}
\begin{scope}
\path[clip] (  0.00,  0.00) rectangle (505.89,144.54);
\definecolor{drawColor}{RGB}{0,0,0}

\node[text=drawColor,anchor=base,inner sep=0pt, outer sep=0pt, scale=  0.87] at (163.51, 21.43) {200};

\node[text=drawColor,anchor=base,inner sep=0pt, outer sep=0pt, scale=  0.87] at (186.07, 21.43) {400};

\node[text=drawColor,anchor=base,inner sep=0pt, outer sep=0pt, scale=  0.87] at (208.63, 21.43) {600};

\node[text=drawColor,anchor=base,inner sep=0pt, outer sep=0pt, scale=  0.87] at (231.19, 21.43) {800};
\end{scope}
\begin{scope}
\path[clip] (  0.00,  0.00) rectangle (505.89,144.54);
\definecolor{drawColor}{RGB}{0,0,0}

\path[draw=drawColor,line width= 0.6pt,line join=round] (256.57, 30.41) --
	(256.57, 33.41);

\path[draw=drawColor,line width= 0.6pt,line join=round] (279.14, 30.41) --
	(279.14, 33.41);

\path[draw=drawColor,line width= 0.6pt,line join=round] (301.70, 30.41) --
	(301.70, 33.41);

\path[draw=drawColor,line width= 0.6pt,line join=round] (324.26, 30.41) --
	(324.26, 33.41);
\end{scope}
\begin{scope}
\path[clip] (  0.00,  0.00) rectangle (505.89,144.54);
\definecolor{drawColor}{RGB}{0,0,0}

\node[text=drawColor,anchor=base,inner sep=0pt, outer sep=0pt, scale=  0.87] at (256.57, 21.43) {200};

\node[text=drawColor,anchor=base,inner sep=0pt, outer sep=0pt, scale=  0.87] at (279.14, 21.43) {400};

\node[text=drawColor,anchor=base,inner sep=0pt, outer sep=0pt, scale=  0.87] at (301.70, 21.43) {600};

\node[text=drawColor,anchor=base,inner sep=0pt, outer sep=0pt, scale=  0.87] at (324.26, 21.43) {800};
\end{scope}
\begin{scope}
\path[clip] (  0.00,  0.00) rectangle (505.89,144.54);
\definecolor{drawColor}{RGB}{0,0,0}

\path[draw=drawColor,line width= 0.6pt,line join=round] (349.64, 30.41) --
	(349.64, 33.41);

\path[draw=drawColor,line width= 0.6pt,line join=round] (372.20, 30.41) --
	(372.20, 33.41);

\path[draw=drawColor,line width= 0.6pt,line join=round] (394.76, 30.41) --
	(394.76, 33.41);

\path[draw=drawColor,line width= 0.6pt,line join=round] (417.33, 30.41) --
	(417.33, 33.41);
\end{scope}
\begin{scope}
\path[clip] (  0.00,  0.00) rectangle (505.89,144.54);
\definecolor{drawColor}{RGB}{0,0,0}

\node[text=drawColor,anchor=base,inner sep=0pt, outer sep=0pt, scale=  0.87] at (349.64, 21.43) {200};

\node[text=drawColor,anchor=base,inner sep=0pt, outer sep=0pt, scale=  0.87] at (372.20, 21.43) {400};

\node[text=drawColor,anchor=base,inner sep=0pt, outer sep=0pt, scale=  0.87] at (394.76, 21.43) {600};

\node[text=drawColor,anchor=base,inner sep=0pt, outer sep=0pt, scale=  0.87] at (417.33, 21.43) {800};
\end{scope}
\begin{scope}
\path[clip] (  0.00,  0.00) rectangle (505.89,144.54);
\definecolor{drawColor}{RGB}{0,0,0}

\node[text=drawColor,anchor=base,inner sep=0pt, outer sep=0pt, scale=  1.09] at (235.42,  8.40) {$n=$ number of labeled pairs, half equality and half inequality};
\end{scope}
\begin{scope}
\path[clip] (  0.00,  0.00) rectangle (505.89,144.54);
\definecolor{drawColor}{RGB}{0,0,0}

\node[text=drawColor,rotate= 90.00,anchor=base,inner sep=0pt, outer sep=0pt, scale=  1.09] at ( 16.63, 76.68) {percent incorrectly};

\node[text=drawColor,rotate= 90.00,anchor=base,inner sep=0pt, outer sep=0pt, scale=  1.09] at ( 29.59, 76.68) {predicted test pairs};
\end{scope}
\begin{scope}
\path[clip] (  0.00,  0.00) rectangle (505.89,144.54);
\definecolor{fillColor}{RGB}{255,255,255}

\path[fill=fillColor] (430.09, 34.52) rectangle (491.35,118.85);
\end{scope}
\begin{scope}
\path[clip] (  0.00,  0.00) rectangle (505.89,144.54);
\definecolor{drawColor}{RGB}{0,0,0}

\node[text=drawColor,anchor=base west,inner sep=0pt, outer sep=0pt, scale=  1.09] at (434.36,106.36) {function};
\end{scope}
\begin{scope}
\path[clip] (  0.00,  0.00) rectangle (505.89,144.54);
\definecolor{drawColor}{gray}{0.80}
\definecolor{fillColor}{RGB}{255,255,255}

\path[draw=drawColor,line width= 0.6pt,line join=round,line cap=round,fill=fillColor] (434.36, 86.48) rectangle (450.26,102.38);
\end{scope}
\begin{scope}
\path[clip] (  0.00,  0.00) rectangle (505.89,144.54);
\definecolor{fillColor}{RGB}{135,206,235}

\path[fill=fillColor,fill opacity=0.50] (435.07, 87.19) rectangle (449.55,101.67);
\end{scope}
\begin{scope}
\path[clip] (  0.00,  0.00) rectangle (505.89,144.54);
\definecolor{drawColor}{RGB}{135,206,235}

\path[draw=drawColor,line width= 1.7pt,line join=round] (435.95, 94.43) -- (448.67, 94.43);
\end{scope}
\begin{scope}
\path[clip] (  0.00,  0.00) rectangle (505.89,144.54);
\definecolor{drawColor}{gray}{0.80}
\definecolor{fillColor}{RGB}{255,255,255}

\path[draw=drawColor,line width= 0.6pt,line join=round,line cap=round,fill=fillColor] (434.36, 70.58) rectangle (450.26, 86.48);
\end{scope}
\begin{scope}
\path[clip] (  0.00,  0.00) rectangle (505.89,144.54);
\definecolor{fillColor}{RGB}{0,0,255}

\path[fill=fillColor,fill opacity=0.50] (435.07, 71.29) rectangle (449.55, 85.77);
\end{scope}
\begin{scope}
\path[clip] (  0.00,  0.00) rectangle (505.89,144.54);
\definecolor{drawColor}{RGB}{0,0,255}

\path[draw=drawColor,line width= 1.7pt,line join=round] (435.95, 78.53) -- (448.67, 78.53);
\end{scope}
\begin{scope}
\path[clip] (  0.00,  0.00) rectangle (505.89,144.54);
\definecolor{drawColor}{gray}{0.80}
\definecolor{fillColor}{RGB}{255,255,255}

\path[draw=drawColor,line width= 0.6pt,line join=round,line cap=round,fill=fillColor] (434.36, 54.68) rectangle (450.26, 70.58);
\end{scope}
\begin{scope}
\path[clip] (  0.00,  0.00) rectangle (505.89,144.54);
\definecolor{fillColor}{RGB}{0,0,0}

\path[fill=fillColor,fill opacity=0.50] (435.07, 55.39) rectangle (449.55, 69.87);
\end{scope}
\begin{scope}
\path[clip] (  0.00,  0.00) rectangle (505.89,144.54);
\definecolor{drawColor}{RGB}{0,0,0}

\path[draw=drawColor,line width= 1.7pt,line join=round] (435.95, 62.63) -- (448.67, 62.63);
\end{scope}
\begin{scope}
\path[clip] (  0.00,  0.00) rectangle (505.89,144.54);
\definecolor{drawColor}{gray}{0.80}
\definecolor{fillColor}{RGB}{255,255,255}

\path[draw=drawColor,line width= 0.6pt,line join=round,line cap=round,fill=fillColor] (434.36, 38.78) rectangle (450.26, 54.68);
\end{scope}
\begin{scope}
\path[clip] (  0.00,  0.00) rectangle (505.89,144.54);
\definecolor{fillColor}{RGB}{204,204,204}

\path[fill=fillColor,fill opacity=0.50] (435.07, 39.50) rectangle (449.55, 53.97);
\end{scope}
\begin{scope}
\path[clip] (  0.00,  0.00) rectangle (505.89,144.54);
\definecolor{drawColor}{gray}{0.80}

\path[draw=drawColor,line width= 1.7pt,line join=round] (435.95, 46.73) -- (448.67, 46.73);
\end{scope}
\begin{scope}
\path[clip] (  0.00,  0.00) rectangle (505.89,144.54);
\definecolor{drawColor}{RGB}{0,0,0}

\node[text=drawColor,anchor=base west,inner sep=0pt, outer sep=0pt, scale=  0.87] at (452.25, 91.14) {rank};
\end{scope}
\begin{scope}
\path[clip] (  0.00,  0.00) rectangle (505.89,144.54);
\definecolor{drawColor}{RGB}{0,0,0}

\node[text=drawColor,anchor=base west,inner sep=0pt, outer sep=0pt, scale=  0.87] at (452.25, 75.24) {rank2};
\end{scope}
\begin{scope}
\path[clip] (  0.00,  0.00) rectangle (505.89,144.54);
\definecolor{drawColor}{RGB}{0,0,0}

\node[text=drawColor,anchor=base west,inner sep=0pt, outer sep=0pt, scale=  0.87] at (452.25, 59.34) {compare};
\end{scope}
\begin{scope}
\path[clip] (  0.00,  0.00) rectangle (505.89,144.54);
\definecolor{drawColor}{RGB}{0,0,0}

\node[text=drawColor,anchor=base west,inner sep=0pt, outer sep=0pt, scale=  0.87] at (452.25, 43.44) {truth};
\end{scope}
\end{tikzpicture}

  \vskip -1cm
  \caption{Test error of SVMrank and SVMcompare (the Bayes error rate
    of the latent $r$ is shown for comparison). We plot mean and
    standard deviation of the prediction error across 4 randomly
    chosen data sets, as a function of training set size $n$ (a
    vertical line shows the data set size $n=100$ which was used in
    Figure~\ref{fig:norm-level-curves}). It is clear that in general
    SVMcompare makes better predictions on the test data.}
  \label{fig:simulation-samples}
\end{figure*}

\section{Conclusions and future work}
\label{sec:conclusions}

We discussed an extension of SVM to comparison problems. Our results
highlighted the importance of directly modeling the equality pairs
($y_i=0$), and it will be interesting to see if the same results are
observed in learning to rank data sets. For scaling to very large data
sets, it will be interesting to try algorithms based on smooth
discriminative loss functions, such as stochastic gradient descent
with a logistic loss.

\textbf{Acknowledgements}: TDH was funded by KAKENHI 23120004, SS by a
MEXT scholarship, and MS by KAKENHI 26700022. Thanks to Simon
Lacoste-Julien and Hang Li for helpful discussions.

\bibliographystyle{abbrvnat}
\bibliography{refs}

\end{document}

